\section{Constructors and Destructors}
Whenever a class is instantiated, the constructor is used.  This is analogous to python's \texttt{\_\_init\_\_()} function, in that it is a set of commands that are run the moment the class object is created.  Conversely, a destructor is a set of commands that are executed whenever the object is destroyed.  This includes any event where the memory is released, whether from a specific command inside the code, or when the program itself finishes execution and exits.

By convention, constructors and destructors are named the same as their class, with the special addition of the \texttt{~} before the destructor's name.  See the example below.

\begin{minted}{c++}
class Square
{
  private:
    double edge;
  public:
    Square(double); // constructor
    ~Square();      // destructor
};
Square::Square(double length)
{
    edge = length;
}
Square::~Square()
{
    std::cout << "Goodbye, Square!" << endl;
}
\end{minted}
