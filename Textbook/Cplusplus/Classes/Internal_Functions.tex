\section{Internal Functions}
Internal functions are simply functions that act on variables inside the class.  An internal function may access internal variables in a class as well as take arguments like any other C++ function.  If we expand the example above, we can add in a function for the area and perimeter of the square class as callable functions.  Also included is a small \texttt{main()} function to show how these functions may be called.

\begin{minted}{c++}
#include <iostream>
class Square
{
  private:
    double edge;
  public:
    Square(double); 
    ~Square();
    double area();
    double perimeter();
};
Square::Square(double length)
{
    edge = length;
}
Square::~Square()
{
    std::cout << "Goodbye, Square!" << std::endl;
}
double Square::area()
{
    return edge * edge;
}
double Square::perimeter()
{
    return edge * 4;
}
int main()
{
    Square my_square (5.2); 
    /* The class definition "Square" is used like a datatype, 
    like "int" or "double", then a variable name */
    cout << "Area of the square is " << my_square.area() << endl;
    cout << "Perimeter of the square is " << my_square.perimeter() << endl;
    return 0;
}
\end{minted}
