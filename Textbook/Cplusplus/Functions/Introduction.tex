\begin{Goals}
    \begin{enumerate}
        \item Understanding of function datatypes
        \item Understanding the void function
        \item Differentiate between function declarations and definitions
        \item Identify and manage function arguments and returns
        \item Understand pass-by-value and pass-by-reference, and when to use which
    \end{enumerate}
    \end{Goals}
    
    There are a few vocabulary terms we'll need to establish right away with functions, starting with \textbf{function}.
    
    In C++, a \textbf{function} is a set of instructions inside a scope that takes arguments as inputs, performs a task on them, and returns something back to wherever the function was called from.  A \textbf{function call} is the point in a program where the function you wrote elsewhere is being executed.  A function \textbf{definition} is where the instructions of the function are actually defined, and includes the return type, function name, arguments list (including data types for each argument), and the instructions within the scope of the function.  A function \textbf{declaration} is a truncated form of a definition, which doesn't include any of the instructions, and is meant more to alert the compiler that a function exists with those parameters and is simply defined elsewhere.  Without declarations, every function \textbf{MUST} be defined before they are called, and this can become difficult in more complex codebases.
    
    