\section{Basic Requirements for Coding}
\paragraph{} At a minimum, you need exactly two things to become a programmer.
\begin{enumerate}
    \item \textit{A way to write the code.}
    \item \textit{A way to run the code.}
\end{enumerate}

This may seem insanely simplistic (because it very much is), but at the absolute core of the process, programming requires that you be able to write a program and then execute said program.
The writing can be done with something as basic as a text editor like \texttt{vi} or \texttt{Notepad}, or more complicated programs designed to provide assistance and organization.
Running the code can likewise be as simple as executing a script in a console or terminal environment or compiling the code into an executable file.

There are a number of things that can be added to these two basic requirements to make your coding work considerably easier.
They range from language-specific tools like python notebooks up to larger setups called IDEs.
Other tools can provide additional support beyond the coding phase, such as continuous integration.