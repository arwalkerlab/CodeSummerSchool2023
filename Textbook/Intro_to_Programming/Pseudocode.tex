%%%%%%%%%%%%%%%%%%%%%%%%%%%%%%%%%%%%%%%%%%%%%%%%%%%%%%%%%%%%%%%%%%%%%%%%%%%%%%%%%%%%%%%%%%%%%%%%%%%%%%%%%%%%%%%%%%%%%
\section{Pseudocode}
Other names for pseudocode include ``algorithm development'', ``project management'', ``outlining'', ``planning'', ``thinking'', and ``jotting that down so I don't forget it later''.
Pseudocoding is effectively just writing out the stages of your program in plain language (\emph{not code}) to ensure a clear understanding of the problem you're trying to solve. 
Often, programmers will begin pseudocode with a very simple set of steps they think of the problem. 
Each step can be explained in more and more detail as a set of smaller, more manageable steps, until eventually you wind up with a complete list of steps that can be converted to computer commands. 
Pseudocode can also provide some insight into ways the code might be optimized, such as by revealing opportunities to parallelise the execution, or by revealing regions where something being calculated can just be saved for reuse rather than recalculated again later.

One of the most valuable skills a programmer can develop is the ability to think like a computer.
This means learning to break down larger problems and complex behaviors into smaller and smaller pieces until it becomes a collection of tiny calculations such as a sequence of additions and subtractions, or comparisons between two values.

A good habit to develop whenever beginning a new project is to first outline the expected flow of the code. 
Some people use a whiteboard, or scratch paper, or just a blank text document on their computer - it doesn't matter how you do it, just that you plan it out somehow before jumping into the code.

\hypertarget{example-project---brownian-motion}{%
\paragraph{Example Project - Brownian
Motion}\label{example-project---brownian-motion}}

Write a program that places an arbitrary number of particles in a box of
some other arbitrary size, then moves them around randomly by assigning
a random x, y, and z component of their motion vector between 0 and 1.

Remember to just write in pseudocode. 
You can make it as complex as you like, but
it should just be plain text. Try to think through the steps of the
problem like a computer might.

% Once you've completed your pseudocode, keep it handy - you can use it
% for tomorrow's project!
