\paragraph{}This Code Summer School will cover the basics of Python and C++ as well as some best practices and good habits to develop early as programmers.  Whether you intend to pursue programming professionally or merely keep it as a skill in your back pocket, these lessons will hopefully serve as a good foundation.  I anticipate each day will be a couple of hours of lecture and practice, with Q\&A mixed in as we go.  I plan to keep it fairly informal, so questions and interruptions are fine and welcome.  Also, feedback is much appreciated as it will help to refine this Summer School for following years.

\begin{center}\rule{0.5\linewidth}{0.5pt}\end{center}

``Programming'' is such a hugely expansive term,
covering a staggering number of techniques, languages, processes,
hardware configurations, purposes, applications, scales, and so forth.

\emph{So what is programming, really?}

The short version is that programming is a way of making a computer
(another broad term!) perform tasks. These tasks can be as simple as the
addition of two numbers, or as complex as calculating the excited state
energy of a fluorescent nucleotide as it moves through a solution of
water and sodium chloride ions. As the tasks become more complex, the
code becomes more complex as well. Small tasks may be manageable with
tiny scripts or single-file programs, while larger ones may require the
inclusion of other tools, multiple files, and more complicated design
principles.

For the purposes of this Summer School, we'll mostly be sticking to the
introductory/entry-level stuff, but remember that the actual limits of
your programs are only in the capacity of your hardware and the breadth
of your imagination.

There are some rules to programming that should really be learned early
on, if only because knowing them early will make the rest considerably
easier. For many experienced programmers, the actual writing of code is
a smaller portion of the overall process than one might expect.




