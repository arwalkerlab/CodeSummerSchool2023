\section{Quality Control}
\paragraph{}  The term "quality control" has some fairly broad implications.
Depending on the specific project, it might refer to different aspects of programming, from how well the code is documented to how close to reality the results of some internal calculation get.
Programs, especially large ones with multiple contributors, often need lots of careful attention to quality control.
This is most important when programming for a company or organization that relies on the quality of the code to make everything else work.

Quality control also means ensuring that you (or your colleagues) are not committing some of the more common "tricks" that many programmers will boast about behind the anonymity of the internet, such as "I've included a sleep function in the code so that whenever my boss complains that the program is too slow, I can reduce the value of that function by one and then tell him I optimized the code a bit and got a speedup!"
While that sounds very clever on the face of it, and like a great way to make yourself look good to your employers, the discovery of such things will usually lead to getting fired, and it also shows other programmers that your goal is to \textbf{seem} good, rather than actually \textbf{be} good.
Don't do stuff like that, and discourage others from that behavior as well.