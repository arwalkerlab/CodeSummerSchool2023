\section{Data and Variable Types}
Different types of information can be stored for use by the computer. 
Some programming languages are fairly lenient about variable types and are flexible with what types of data are being provided in a given variable. 
Others are a bit more strict, and require explicit type declarations/definitions for each variable to ensure proper memory allocation. 
\emph{If only some of that made sense, you're in the right place.}

C++ requires that every variable have a clearly defined type, which the compiler uses to ensure the program allocates the correct amount of memory at runtime.
Python doesn't generally require \emph{explicit} variable type declarations (with some exceptions that will come later as we get into more advanced programming). 
However, it is still useful to know what kinds of data there is, what can be done with it, and how it's stored.

First, let's explore data types like \texttt{int}, \texttt{float},
\texttt{list}, \texttt{tuple}, and \texttt{string}.

    \begin{tcolorbox}[breakable, size=fbox, boxrule=1pt, pad at break*=1mm,colback=cellbackground, colframe=cellborder]
\prompt{In}{incolor}{1}{\boxspacing}
\begin{Verbatim}[commandchars=\\\{\}]
\PY{n}{my\PYZus{}int}    \PY{o}{=} \PY{l+m+mi}{2}
\PY{n}{my\PYZus{}float}  \PY{o}{=} \PY{l+m+mf}{3.1415}
\PY{n}{my\PYZus{}list}   \PY{o}{=} \PY{p}{[}\PY{l+m+mi}{1}\PY{p}{,}\PY{l+m+mf}{3.1415}\PY{p}{,}\PY{l+s+s2}{\PYZdq{}}\PY{l+s+s2}{Hello World!}\PY{l+s+s2}{\PYZdq{}}\PY{p}{,}\PY{l+s+s2}{\PYZdq{}}\PY{l+s+s2}{pizza}\PY{l+s+s2}{\PYZdq{}}\PY{p}{]}
\PY{n}{my\PYZus{}tuple}  \PY{o}{=} \PY{p}{(}\PY{l+m+mi}{5}\PY{p}{,}\PY{l+m+mi}{6}\PY{p}{,}\PY{l+m+mi}{7}\PY{p}{,}\PY{l+m+mi}{8}\PY{p}{,}\PY{l+m+mi}{9}\PY{p}{)}
\PY{n}{my\PYZus{}string} \PY{o}{=} \PY{l+s+s2}{\PYZdq{}}\PY{l+s+s2}{Hello World!}\PY{l+s+s2}{\PYZdq{}}
\end{Verbatim}
\end{tcolorbox}

    These examples are fairly simple.

\begin{itemize}
\tightlist
\item
  \texttt{my\_int} is an integer, and gets treated like one. Integers
  are useful for things like indexes, counters, and so forth.
\item
  \texttt{my\_float} is a float (often called a ``double'' in other
  programming languages), and are regular numbers including decimals.
\item
  \texttt{my\_list} is a list of values enclosed in square brackets.
  Lists are indexed from zero, which means the first item in a list is
  ``item 0''. Lists are great ways to keep collections of data organized
  and in order, and you can extract individual values simply by
  including the index with the variable name: \texttt{my\_list{[}3{]}}
  will return ``pizza''. You can also get values from the end of a list
  with negative indices. my\_list\texttt{{[}-1{]}} will return ``pizza''
  because it's the last value.
\item
  \texttt{my\_tuple} is similar to a list, except that it is a little
  more difficult to pull individual values from it. Tuples are useful
  when you need to maintain groups of values together in relation to
  each other, such as with (x,y,z) coordinates.
\item
  \texttt{my\_string} is a list of characters including letters,
  numbers, punctuation, whitespace (tabs, spaces, line breaks, etc.).
  The contents of a string do not include the quotation marks on either
  side. Strings can include quotes using \emph{escapes} like
  \texttt{\textbackslash{}"} or
  \texttt{\textbackslash{}\textbackslash{}} to include a backslash.
\end{itemize}

Variable manipulation comes in many forms and depends on the type of
data contained within. Better understanding of how data types work can
allow you to do some interesting things, like taking a ``slice'' of a
string like you would from a list.

Consider the examples below.

    \begin{tcolorbox}[breakable, size=fbox, boxrule=1pt, pad at break*=1mm,colback=cellbackground, colframe=cellborder]
\prompt{In}{incolor}{2}{\boxspacing}
\begin{Verbatim}[commandchars=\\\{\}]
\PY{n}{my\PYZus{}list} \PY{o}{=} \PY{p}{[}\PY{l+m+mi}{0}\PY{p}{,}\PY{l+m+mi}{1}\PY{p}{,}\PY{l+m+mi}{2}\PY{p}{,}\PY{l+m+mi}{3}\PY{p}{,}\PY{l+m+mi}{4}\PY{p}{,}\PY{l+m+mi}{5}\PY{p}{,}\PY{l+m+mi}{6}\PY{p}{,}\PY{l+m+mi}{7}\PY{p}{,}\PY{l+m+mi}{8}\PY{p}{,}\PY{l+m+mi}{9}\PY{p}{,}\PY{l+m+mi}{10}\PY{p}{,}\PY{l+m+mi}{11}\PY{p}{,}\PY{l+m+mi}{12}\PY{p}{,}\PY{l+m+mi}{13}\PY{p}{,}\PY{l+m+mi}{14}\PY{p}{,}\PY{l+m+mi}{15}\PY{p}{,}\PY{l+m+mi}{16}\PY{p}{,}\PY{l+m+mi}{17}\PY{p}{,}\PY{l+m+mi}{18}\PY{p}{,}\PY{l+m+mi}{19}\PY{p}{,}\PY{l+m+mi}{20}\PY{p}{,}\PY{l+m+mi}{21}\PY{p}{,}\PY{l+m+mi}{22}\PY{p}{,}\PY{l+m+mi}{23}\PY{p}{,}\PY{l+m+mi}{24}\PY{p}{,}\PY{l+m+mi}{25}\PY{p}{]}
\PY{c+c1}{\PYZsh{} my\PYZus{}list has a length of 26 individual values}
\PY{n+nb}{len}\PY{p}{(}\PY{n}{my\PYZus{}list}\PY{p}{)}
\end{Verbatim}
\end{tcolorbox}

            \begin{tcolorbox}[breakable, size=fbox, boxrule=.5pt, pad at break*=1mm, opacityfill=0]
\prompt{Out}{outcolor}{2}{\boxspacing}
\begin{Verbatim}[commandchars=\\\{\}]
26
\end{Verbatim}
\end{tcolorbox}
        
    \begin{tcolorbox}[breakable, size=fbox, boxrule=1pt, pad at break*=1mm,colback=cellbackground, colframe=cellborder]
\prompt{In}{incolor}{3}{\boxspacing}
\begin{Verbatim}[commandchars=\\\{\}]
\PY{n}{my\PYZus{}string} \PY{o}{=} \PY{l+s+s2}{\PYZdq{}}\PY{l+s+s2}{Once more into the breach!}\PY{l+s+s2}{\PYZdq{}}
\PY{c+c1}{\PYZsh{} my\PYZus{}string has a length of 26 characters including whitespace and punctuation.}
\PY{n+nb}{len}\PY{p}{(}\PY{n}{my\PYZus{}string}\PY{p}{)}
\end{Verbatim}
\end{tcolorbox}

            \begin{tcolorbox}[breakable, size=fbox, boxrule=.5pt, pad at break*=1mm, opacityfill=0]
\prompt{Out}{outcolor}{3}{\boxspacing}
\begin{Verbatim}[commandchars=\\\{\}]
26
\end{Verbatim}
\end{tcolorbox}
        
    \hypertarget{slicing-lists}{%
\subsubsection{Slicing Lists}\label{slicing-lists}}

One common use for lists is ``slicing'', where you can get a small
subsection of the list. Let's say you wanted just the first five
elements in \texttt{my\_list}. You would use a slice. Slices are
generated similar to how an individual element is called from a list,
from inside square brackets. However, we can put a \texttt{:} between
the starting and ending indices to get everything between. We can also
use an empty space to indicate ``everything''. Check out the examples
below.

    \begin{tcolorbox}[breakable, size=fbox, boxrule=1pt, pad at break*=1mm,colback=cellbackground, colframe=cellborder]
\prompt{In}{incolor}{4}{\boxspacing}
\begin{Verbatim}[commandchars=\\\{\}]
\PY{n}{my\PYZus{}list}\PY{p}{[}\PY{p}{:}\PY{l+m+mi}{5}\PY{p}{]}
\end{Verbatim}
\end{tcolorbox}

            \begin{tcolorbox}[breakable, size=fbox, boxrule=.5pt, pad at break*=1mm, opacityfill=0]
\prompt{Out}{outcolor}{4}{\boxspacing}
\begin{Verbatim}[commandchars=\\\{\}]
[0, 1, 2, 3, 4]
\end{Verbatim}
\end{tcolorbox}
        
    \begin{tcolorbox}[breakable, size=fbox, boxrule=1pt, pad at break*=1mm,colback=cellbackground, colframe=cellborder]
\prompt{In}{incolor}{5}{\boxspacing}
\begin{Verbatim}[commandchars=\\\{\}]
\PY{n}{my\PYZus{}list}\PY{p}{[}\PY{l+m+mi}{5}\PY{p}{:}\PY{l+m+mi}{10}\PY{p}{]}
\end{Verbatim}
\end{tcolorbox}

            \begin{tcolorbox}[breakable, size=fbox, boxrule=.5pt, pad at break*=1mm, opacityfill=0]
\prompt{Out}{outcolor}{5}{\boxspacing}
\begin{Verbatim}[commandchars=\\\{\}]
[5, 6, 7, 8, 9]
\end{Verbatim}
\end{tcolorbox}
        
    Note how the two results are different. The ending in the first cell is
the same as the beginning of the second cell, but we don't actually get
``5'' in the results in the first cell. Slices go ``up to'' the ending
value, but don't include it. Keep this in mind when working with slices.
We can also combine other tricks from list manipulations, like using
negative indices to go backwards from the end.

In the next cell, we'll get the last seven elements from the list.

    \begin{tcolorbox}[breakable, size=fbox, boxrule=1pt, pad at break*=1mm,colback=cellbackground, colframe=cellborder]
\prompt{In}{incolor}{6}{\boxspacing}
\begin{Verbatim}[commandchars=\\\{\}]
\PY{n}{my\PYZus{}list}\PY{p}{[}\PY{o}{\PYZhy{}}\PY{l+m+mi}{7}\PY{p}{:}\PY{p}{]}
\end{Verbatim}
\end{tcolorbox}

            \begin{tcolorbox}[breakable, size=fbox, boxrule=.5pt, pad at break*=1mm, opacityfill=0]
\prompt{Out}{outcolor}{6}{\boxspacing}
\begin{Verbatim}[commandchars=\\\{\}]
[19, 20, 21, 22, 23, 24, 25]
\end{Verbatim}
\end{tcolorbox}
        
    What if we wanted every third element in the list?

    \begin{tcolorbox}[breakable, size=fbox, boxrule=1pt, pad at break*=1mm,colback=cellbackground, colframe=cellborder]
\prompt{In}{incolor}{7}{\boxspacing}
\begin{Verbatim}[commandchars=\\\{\}]
\PY{n}{my\PYZus{}list}\PY{p}{[}\PY{p}{:}\PY{p}{:}\PY{l+m+mi}{3}\PY{p}{]}
\end{Verbatim}
\end{tcolorbox}

            \begin{tcolorbox}[breakable, size=fbox, boxrule=.5pt, pad at break*=1mm, opacityfill=0]
\prompt{Out}{outcolor}{7}{\boxspacing}
\begin{Verbatim}[commandchars=\\\{\}]
[0, 3, 6, 9, 12, 15, 18, 21, 24]
\end{Verbatim}
\end{tcolorbox}
        
    The second \texttt{:} indicates a ``stride''. This is useful when you
have data that is strangely shaped (such as a long list of values that
correspond to x,y,z coordinates, but aren't in a (3,n) shaped list.

Now let's combine these. We'll get every other element starting from the
tenth and going up to the twentieth.

    \begin{tcolorbox}[breakable, size=fbox, boxrule=1pt, pad at break*=1mm,colback=cellbackground, colframe=cellborder]
\prompt{In}{incolor}{8}{\boxspacing}
\begin{Verbatim}[commandchars=\\\{\}]
\PY{n}{my\PYZus{}list}\PY{p}{[}\PY{l+m+mi}{10}\PY{p}{:}\PY{l+m+mi}{20}\PY{p}{:}\PY{l+m+mi}{2}\PY{p}{]}
\end{Verbatim}
\end{tcolorbox}

            \begin{tcolorbox}[breakable, size=fbox, boxrule=.5pt, pad at break*=1mm, opacityfill=0]
\prompt{Out}{outcolor}{8}{\boxspacing}
\begin{Verbatim}[commandchars=\\\{\}]
[10, 12, 14, 16, 18]
\end{Verbatim}
\end{tcolorbox}
        
    Now let's look at strings. Strings are just lists of letters, numbers,
and any other characters you can think of. With this in mind, we can do
things to strings that we have done to lists.

    \begin{tcolorbox}[breakable, size=fbox, boxrule=1pt, pad at break*=1mm,colback=cellbackground, colframe=cellborder]
\prompt{In}{incolor}{9}{\boxspacing}
\begin{Verbatim}[commandchars=\\\{\}]
\PY{n}{my\PYZus{}string}
\end{Verbatim}
\end{tcolorbox}

            \begin{tcolorbox}[breakable, size=fbox, boxrule=.5pt, pad at break*=1mm, opacityfill=0]
\prompt{Out}{outcolor}{9}{\boxspacing}
\begin{Verbatim}[commandchars=\\\{\}]
'Once more into the breach!'
\end{Verbatim}
\end{tcolorbox}
        
    \begin{tcolorbox}[breakable, size=fbox, boxrule=1pt, pad at break*=1mm,colback=cellbackground, colframe=cellborder]
\prompt{In}{incolor}{10}{\boxspacing}
\begin{Verbatim}[commandchars=\\\{\}]
\PY{n}{my\PYZus{}string}\PY{p}{[}\PY{p}{:}\PY{l+m+mi}{5}\PY{p}{]}
\end{Verbatim}
\end{tcolorbox}

            \begin{tcolorbox}[breakable, size=fbox, boxrule=.5pt, pad at break*=1mm, opacityfill=0]
\prompt{Out}{outcolor}{10}{\boxspacing}
\begin{Verbatim}[commandchars=\\\{\}]
'Once '
\end{Verbatim}
\end{tcolorbox}
        
    \begin{tcolorbox}[breakable, size=fbox, boxrule=1pt, pad at break*=1mm,colback=cellbackground, colframe=cellborder]
\prompt{In}{incolor}{11}{\boxspacing}
\begin{Verbatim}[commandchars=\\\{\}]
\PY{n}{my\PYZus{}string}\PY{p}{[}\PY{l+m+mi}{5}\PY{p}{:}\PY{l+m+mi}{10}\PY{p}{]}
\end{Verbatim}
\end{tcolorbox}

            \begin{tcolorbox}[breakable, size=fbox, boxrule=.5pt, pad at break*=1mm, opacityfill=0]
\prompt{Out}{outcolor}{11}{\boxspacing}
\begin{Verbatim}[commandchars=\\\{\}]
'more '
\end{Verbatim}
\end{tcolorbox}
        
    \begin{tcolorbox}[breakable, size=fbox, boxrule=1pt, pad at break*=1mm,colback=cellbackground, colframe=cellborder]
\prompt{In}{incolor}{12}{\boxspacing}
\begin{Verbatim}[commandchars=\\\{\}]
\PY{n}{my\PYZus{}string}\PY{p}{[}\PY{o}{\PYZhy{}}\PY{l+m+mi}{7}\PY{p}{:}\PY{p}{]}
\end{Verbatim}
\end{tcolorbox}

            \begin{tcolorbox}[breakable, size=fbox, boxrule=.5pt, pad at break*=1mm, opacityfill=0]
\prompt{Out}{outcolor}{12}{\boxspacing}
\begin{Verbatim}[commandchars=\\\{\}]
'breach!'
\end{Verbatim}
\end{tcolorbox}
        
    \begin{tcolorbox}[breakable, size=fbox, boxrule=1pt, pad at break*=1mm,colback=cellbackground, colframe=cellborder]
\prompt{In}{incolor}{13}{\boxspacing}
\begin{Verbatim}[commandchars=\\\{\}]
\PY{n}{my\PYZus{}string}\PY{p}{[}\PY{p}{:}\PY{p}{:}\PY{l+m+mi}{3}\PY{p}{]}
\end{Verbatim}
\end{tcolorbox}

            \begin{tcolorbox}[breakable, size=fbox, boxrule=.5pt, pad at break*=1mm, opacityfill=0]
\prompt{Out}{outcolor}{13}{\boxspacing}
\begin{Verbatim}[commandchars=\\\{\}]
'Oeo tt eh'
\end{Verbatim}
\end{tcolorbox}
        
    \begin{tcolorbox}[breakable, size=fbox, boxrule=1pt, pad at break*=1mm,colback=cellbackground, colframe=cellborder]
\prompt{In}{incolor}{14}{\boxspacing}
\begin{Verbatim}[commandchars=\\\{\}]
\PY{n}{my\PYZus{}string}\PY{p}{[}\PY{l+m+mi}{10}\PY{p}{:}\PY{l+m+mi}{20}\PY{p}{:}\PY{l+m+mi}{2}\PY{p}{]}
\end{Verbatim}
\end{tcolorbox}

            \begin{tcolorbox}[breakable, size=fbox, boxrule=.5pt, pad at break*=1mm, opacityfill=0]
\prompt{Out}{outcolor}{14}{\boxspacing}
\begin{Verbatim}[commandchars=\\\{\}]
'it h '
\end{Verbatim}
\end{tcolorbox}
        
    \ldots{} some functions are more useful than others, but you get the
idea!

Now let's look at integers and floats. In some programming languages,
the difference between these two can be pretty severe. For example, in
C++, dividing a double by an integer will give you a truncated integer,
which means you can lose some of the information in your data if you're
not careful. Thankfully, Python is a little more forgiving.

Normal division works like we might intuitively expect, where a float
divided by an integer can be a float, and is therefore assumed to be.

    \begin{tcolorbox}[breakable, size=fbox, boxrule=1pt, pad at break*=1mm,colback=cellbackground, colframe=cellborder]
\prompt{In}{incolor}{15}{\boxspacing}
\begin{Verbatim}[commandchars=\\\{\}]
\PY{n}{my\PYZus{}float}\PY{o}{/}\PY{n}{my\PYZus{}int}
\end{Verbatim}
\end{tcolorbox}

            \begin{tcolorbox}[breakable, size=fbox, boxrule=.5pt, pad at break*=1mm, opacityfill=0]
\prompt{Out}{outcolor}{15}{\boxspacing}
\begin{Verbatim}[commandchars=\\\{\}]
1.57075
\end{Verbatim}
\end{tcolorbox}
        
    We can also force the division to return an integer value (which is
useful in some situations)

In the example above, we got a value of 1.57075. If we were to round
this using conventional methods, we'd get 2 However, forcing integer
division with the \texttt{//} below gives us a truncated (not rounded)
value of 1.0. This is also slightly deceptive, as the \texttt{.0}
implies the value is a float,even though the result is a whole number.
This is important to be aware of when doing mathematical work in python.
Truncation just removes everything after the decimal point, while
rounding actually considers the value beforehand.

    \begin{tcolorbox}[breakable, size=fbox, boxrule=1pt, pad at break*=1mm,colback=cellbackground, colframe=cellborder]
\prompt{In}{incolor}{16}{\boxspacing}
\begin{Verbatim}[commandchars=\\\{\}]
\PY{n}{my\PYZus{}float}\PY{o}{/}\PY{o}{/}\PY{n}{my\PYZus{}int}
\end{Verbatim}
\end{tcolorbox}

            \begin{tcolorbox}[breakable, size=fbox, boxrule=.5pt, pad at break*=1mm, opacityfill=0]
\prompt{Out}{outcolor}{16}{\boxspacing}
\begin{Verbatim}[commandchars=\\\{\}]
1.0
\end{Verbatim}
\end{tcolorbox}
        
    \begin{tcolorbox}[breakable, size=fbox, boxrule=1pt, pad at break*=1mm,colback=cellbackground, colframe=cellborder]
\prompt{In}{incolor}{17}{\boxspacing}
\begin{Verbatim}[commandchars=\\\{\}]
\PY{n+nb}{round}\PY{p}{(}\PY{n}{my\PYZus{}float}\PY{o}{/}\PY{n}{my\PYZus{}int}\PY{p}{)}
\end{Verbatim}
\end{tcolorbox}

            \begin{tcolorbox}[breakable, size=fbox, boxrule=.5pt, pad at break*=1mm, opacityfill=0]
\prompt{Out}{outcolor}{17}{\boxspacing}
\begin{Verbatim}[commandchars=\\\{\}]
2
\end{Verbatim}
\end{tcolorbox}
        
    \hypertarget{math-with-variables}{%
\subsubsection{Math with Variables}\label{math-with-variables}}

Math can get incredibly complex, so it's important to remember your
Order of Operations (PEMDAS) - Parentheses, Exponents, Multiplication,
Division, Addition, and Subtraction.

However, in python it's a little different. Parentheses are solved
first, then exponents, until everything in a given equation is reduced
down to a series of terms separated by \texttt{+},\texttt{-},\texttt{*},
and \texttt{/}. Then, the values are processed left-to-right.

    \begin{tcolorbox}[breakable, size=fbox, boxrule=1pt, pad at break*=1mm,colback=cellbackground, colframe=cellborder]
\prompt{In}{incolor}{18}{\boxspacing}
\begin{Verbatim}[commandchars=\\\{\}]
\PY{l+m+mi}{1} \PY{o}{+} \PY{l+m+mi}{2} \PY{o}{\PYZhy{}} \PY{l+m+mi}{3} \PY{o}{*} \PY{l+m+mi}{4} \PY{o}{/} \PY{l+m+mi}{5}
\end{Verbatim}
\end{tcolorbox}

            \begin{tcolorbox}[breakable, size=fbox, boxrule=.5pt, pad at break*=1mm, opacityfill=0]
\prompt{Out}{outcolor}{18}{\boxspacing}
\begin{Verbatim}[commandchars=\\\{\}]
0.6000000000000001
\end{Verbatim}
\end{tcolorbox}
        
    \begin{tcolorbox}[breakable, size=fbox, boxrule=1pt, pad at break*=1mm,colback=cellbackground, colframe=cellborder]
\prompt{In}{incolor}{19}{\boxspacing}
\begin{Verbatim}[commandchars=\\\{\}]
\PY{p}{(}\PY{l+m+mi}{1} \PY{o}{+} \PY{l+m+mi}{2}\PY{p}{)} \PY{o}{\PYZhy{}} \PY{l+m+mi}{3} \PY{o}{*} \PY{l+m+mi}{4} \PY{o}{/} \PY{l+m+mi}{5}
\end{Verbatim}
\end{tcolorbox}

            \begin{tcolorbox}[breakable, size=fbox, boxrule=.5pt, pad at break*=1mm, opacityfill=0]
\prompt{Out}{outcolor}{19}{\boxspacing}
\begin{Verbatim}[commandchars=\\\{\}]
0.6000000000000001
\end{Verbatim}
\end{tcolorbox}
        
    \begin{tcolorbox}[breakable, size=fbox, boxrule=1pt, pad at break*=1mm,colback=cellbackground, colframe=cellborder]
\prompt{In}{incolor}{20}{\boxspacing}
\begin{Verbatim}[commandchars=\\\{\}]
\PY{p}{(}\PY{l+m+mi}{1} \PY{o}{+} \PY{l+m+mi}{2} \PY{o}{\PYZhy{}} \PY{l+m+mi}{3} \PY{o}{*} \PY{l+m+mi}{4}\PY{p}{)} \PY{o}{/} \PY{l+m+mi}{5}
\end{Verbatim}
\end{tcolorbox}

            \begin{tcolorbox}[breakable, size=fbox, boxrule=.5pt, pad at break*=1mm, opacityfill=0]
\prompt{Out}{outcolor}{20}{\boxspacing}
\begin{Verbatim}[commandchars=\\\{\}]
-1.8
\end{Verbatim}
\end{tcolorbox}
        
    \begin{tcolorbox}[breakable, size=fbox, boxrule=1pt, pad at break*=1mm,colback=cellbackground, colframe=cellborder]
\prompt{In}{incolor}{21}{\boxspacing}
\begin{Verbatim}[commandchars=\\\{\}]
\PY{p}{(}\PY{l+m+mi}{1} \PY{o}{+} \PY{l+m+mi}{2} \PY{o}{\PYZhy{}} \PY{l+m+mi}{3}\PY{p}{)} \PY{o}{*} \PY{l+m+mi}{4} \PY{o}{/} \PY{l+m+mi}{5}
\end{Verbatim}
\end{tcolorbox}

            \begin{tcolorbox}[breakable, size=fbox, boxrule=.5pt, pad at break*=1mm, opacityfill=0]
\prompt{Out}{outcolor}{21}{\boxspacing}
\begin{Verbatim}[commandchars=\\\{\}]
0.0
\end{Verbatim}
\end{tcolorbox}
        
    These are just a few examples of how order of operations affects the
results. With this in mind, you can see why it's very important to keep
track of what you're doing in a complex mathematical function. The next
cell has a complex equation in a single line, then the same equation
separated into more easily-managed terms.

    \begin{tcolorbox}[breakable, size=fbox, boxrule=1pt, pad at break*=1mm,colback=cellbackground, colframe=cellborder]
\prompt{In}{incolor}{22}{\boxspacing}
\begin{Verbatim}[commandchars=\\\{\}]
\PY{n}{x}\PY{o}{=}\PY{l+m+mi}{3}
\PY{n}{y}\PY{o}{=}\PY{l+m+mi}{5}
\PY{n}{z}\PY{o}{=}\PY{l+m+mi}{7}
\PY{n}{answer} \PY{o}{=}  \PY{p}{(}\PY{n}{x}\PY{o}{*}\PY{o}{*}\PY{p}{(}\PY{n}{y}\PY{o}{/}\PY{n}{z}\PY{p}{)}\PY{o}{\PYZhy{}}\PY{n}{x}\PY{o}{/}\PY{p}{(}\PY{p}{(}\PY{n}{y}\PY{o}{+}\PY{l+m+mi}{2}\PY{p}{)}\PY{o}{*}\PY{n}{z}\PY{p}{)}\PY{o}{\PYZhy{}}\PY{n}{x}\PY{p}{)}\PY{o}{/}\PY{p}{(}\PY{n}{y}\PY{o}{*}\PY{n}{z}\PY{p}{)}\PY{o}{*}\PY{n}{x}

\PY{n+nb}{print}\PY{p}{(}\PY{n}{answer}\PY{p}{)}
\end{Verbatim}
\end{tcolorbox}

    \begin{Verbatim}[commandchars=\\\{\}]
-0.07452211053138932
    \end{Verbatim}

    Not only is that difficult to read, but it's also harder to see where
errors might be arising. So we can rewrite it and create additional
variables to hold small chunks

    \begin{tcolorbox}[breakable, size=fbox, boxrule=1pt, pad at break*=1mm,colback=cellbackground, colframe=cellborder]
\prompt{In}{incolor}{23}{\boxspacing}
\begin{Verbatim}[commandchars=\\\{\}]
\PY{n}{x}\PY{o}{=}\PY{l+m+mi}{3}
\PY{n}{y}\PY{o}{=}\PY{l+m+mi}{5}
\PY{n}{z}\PY{o}{=}\PY{l+m+mi}{7}

\PY{c+c1}{\PYZsh{} (x**(y/z)\PYZhy{}x/((y+2)*z)\PYZhy{}x)/(y*z)*x}
\PY{n}{p} \PY{o}{=} \PY{n}{y}\PY{o}{/}\PY{n}{z}
\PY{c+c1}{\PYZsh{} (x**p\PYZhy{}x/((y+2)*z)\PYZhy{}x)/(y*z)*x}
\PY{n}{q} \PY{o}{=} \PY{n}{x}\PY{o}{*}\PY{o}{*}\PY{n}{p}
\PY{c+c1}{\PYZsh{} (q\PYZhy{}x/((y+2)*z)\PYZhy{}x)/(y*z)*x}
\PY{n}{r} \PY{o}{=} \PY{n}{y}\PY{o}{+}\PY{l+m+mi}{2}
\PY{c+c1}{\PYZsh{} (q\PYZhy{}x/(r*z)\PYZhy{}x)/(y*z)*x}
\PY{n}{s} \PY{o}{=} \PY{n}{r}\PY{o}{*}\PY{n}{z}
\PY{c+c1}{\PYZsh{} (q\PYZhy{}x/s\PYZhy{}x)/(y*z)*x}
\PY{n}{t} \PY{o}{=} \PY{n}{y}\PY{o}{*}\PY{n}{z}
\PY{c+c1}{\PYZsh{} (q\PYZhy{}x/s\PYZhy{}x)/t*x}
\PY{n}{u} \PY{o}{=} \PY{n}{x}\PY{o}{/}\PY{n}{s}
\PY{c+c1}{\PYZsh{} (q\PYZhy{}u\PYZhy{}x)/t*x}
\PY{n}{v} \PY{o}{=} \PY{n}{q}\PY{o}{\PYZhy{}}\PY{n}{u}\PY{o}{\PYZhy{}}\PY{n}{x}
\PY{c+c1}{\PYZsh{} v/t*x}
\PY{n}{w} \PY{o}{=} \PY{n}{v}\PY{o}{/}\PY{n}{t}
\PY{c+c1}{\PYZsh{} w*x}
\PY{n}{answer} \PY{o}{=} \PY{n}{w}\PY{o}{*}\PY{n}{x}

\PY{n+nb}{print}\PY{p}{(}\PY{n}{answer}\PY{p}{)}
\end{Verbatim}
\end{tcolorbox}

    \begin{Verbatim}[commandchars=\\\{\}]
-0.07452211053138932
    \end{Verbatim}

    This may seem overengineered, but breaking down the individual terms is
helpful in both programming and math, especially when it reveals certain
trends, or even ways to rearrange an equation to reduce the overall
number of calculations being performed. This kind of breakdown can also
be useful when you begin building larger, more complicated functions,
even up to the point of creating entire programs or modules.

\hypertarget{booleans}{%
\subsubsection{Booleans}\label{booleans}}

Booleans are simply variables that are either \texttt{True} or
\texttt{False}. They can also be interpreted as \texttt{1} and
\texttt{0}. Booleans get used all the time in programming, though we may
not be constantly aware of them.

For example, whenever we compare two numbers, the comparison creates a
boolean

    \begin{tcolorbox}[breakable, size=fbox, boxrule=1pt, pad at break*=1mm,colback=cellbackground, colframe=cellborder]
\prompt{In}{incolor}{24}{\boxspacing}
\begin{Verbatim}[commandchars=\\\{\}]
\PY{l+m+mi}{3}\PY{o}{\PYZlt{}}\PY{l+m+mi}{5}
\end{Verbatim}
\end{tcolorbox}

            \begin{tcolorbox}[breakable, size=fbox, boxrule=.5pt, pad at break*=1mm, opacityfill=0]
\prompt{Out}{outcolor}{24}{\boxspacing}
\begin{Verbatim}[commandchars=\\\{\}]
True
\end{Verbatim}
\end{tcolorbox}
        
    \begin{tcolorbox}[breakable, size=fbox, boxrule=1pt, pad at break*=1mm,colback=cellbackground, colframe=cellborder]
\prompt{In}{incolor}{25}{\boxspacing}
\begin{Verbatim}[commandchars=\\\{\}]
\PY{l+m+mi}{3}\PY{o}{\PYZgt{}}\PY{l+m+mi}{5}
\end{Verbatim}
\end{tcolorbox}

            \begin{tcolorbox}[breakable, size=fbox, boxrule=.5pt, pad at break*=1mm, opacityfill=0]
\prompt{Out}{outcolor}{25}{\boxspacing}
\begin{Verbatim}[commandchars=\\\{\}]
False
\end{Verbatim}
\end{tcolorbox}
        
    \begin{tcolorbox}[breakable, size=fbox, boxrule=1pt, pad at break*=1mm,colback=cellbackground, colframe=cellborder]
\prompt{In}{incolor}{26}{\boxspacing}
\begin{Verbatim}[commandchars=\\\{\}]
\PY{l+m+mi}{3} \PY{o}{==} \PY{l+m+mi}{5}
\end{Verbatim}
\end{tcolorbox}

            \begin{tcolorbox}[breakable, size=fbox, boxrule=.5pt, pad at break*=1mm, opacityfill=0]
\prompt{Out}{outcolor}{26}{\boxspacing}
\begin{Verbatim}[commandchars=\\\{\}]
False
\end{Verbatim}
\end{tcolorbox}
        
    We can see that the responses for the different comparisons are correct.
\(3 < 5\) is true, while \(3 > 5\) and \(3 == 5\) are both false.
Incidentally, the \texttt{==} is intentional. In Python and C++,
\texttt{=} \emph{assigns} a value, while \texttt{==} \emph{compares} two
values.

Booleans get used constantly in things like ``if-else statements'' or
``while loops''.

\hypertarget{dictionaries}{%
\subsubsection{Dictionaries}\label{dictionaries}}

Another python data type is the \texttt{dictionary} (or \texttt{dict} as
it's written in python). The dictionary is a very useful datatype, as it
can be used to store many different pieces of information in their own
types.

    \begin{tcolorbox}[breakable, size=fbox, boxrule=1pt, pad at break*=1mm,colback=cellbackground, colframe=cellborder]
\prompt{In}{incolor}{27}{\boxspacing}
\begin{Verbatim}[commandchars=\\\{\}]
\PY{c+c1}{\PYZsh{} A dictionary is denoted by \PYZob{} \PYZcb{} }
\PY{n}{my\PYZus{}dictionary} \PY{o}{=} \PY{p}{\PYZob{}}\PY{p}{\PYZcb{}}

\PY{c+c1}{\PYZsh{} At this point, \PYZdq{}my\PYZus{}dictionary\PYZdq{} is an empty dictionary with no keys or values assigned.}
\PY{c+c1}{\PYZsh{} We can assign a key/value pair like this}

\PY{n}{my\PYZus{}dictionary}\PY{p}{[}\PY{l+s+s2}{\PYZdq{}}\PY{l+s+s2}{Name}\PY{l+s+s2}{\PYZdq{}}\PY{p}{]} \PY{o}{=} \PY{l+s+s2}{\PYZdq{}}\PY{l+s+s2}{Mark}\PY{l+s+s2}{\PYZdq{}}
\PY{n}{my\PYZus{}dictionary}\PY{p}{[}\PY{l+s+s2}{\PYZdq{}}\PY{l+s+s2}{Age}\PY{l+s+s2}{\PYZdq{}}\PY{p}{]}  \PY{o}{=} \PY{l+m+mi}{37}
\PY{n}{my\PYZus{}dictionary}\PY{p}{[}\PY{l+s+s2}{\PYZdq{}}\PY{l+s+s2}{Job}\PY{l+s+s2}{\PYZdq{}}\PY{p}{]} \PY{o}{=} \PY{l+s+s2}{\PYZdq{}}\PY{l+s+s2}{Postdoc}\PY{l+s+s2}{\PYZdq{}}

\PY{c+c1}{\PYZsh{} Now we can recall any of the values held in the dictionary by using the [key]. Keep in mind, if a key already exists, the previous value will be overwritten.}

\PY{c+c1}{\PYZsh{} If you have a dictionary with keys that you don\PYZsq{}t know, you can get them like this:}
\PY{n}{key\PYZus{}list} \PY{o}{=} \PY{p}{[}\PY{n}{key} \PY{k}{for} \PY{n}{key} \PY{o+ow}{in} \PY{n}{my\PYZus{}dictionary}\PY{o}{.}\PY{n}{keys}\PY{p}{(}\PY{p}{)}\PY{p}{]}
\PY{n+nb}{print}\PY{p}{(}\PY{n}{key\PYZus{}list}\PY{p}{)}
\PY{c+c1}{\PYZsh{} This might look strange, but it\PYZsq{}s done this way because my\PYZus{}dictionary.keys() is a function call that returns an iterative set of single values, rather than the entire list.}

\PY{c+c1}{\PYZsh{} You can also iterate through all the keys and values together.}
\PY{k}{for} \PY{n}{key}\PY{p}{,}\PY{n}{value} \PY{o+ow}{in} \PY{n}{my\PYZus{}dictionary}\PY{o}{.}\PY{n}{items}\PY{p}{(}\PY{p}{)}\PY{p}{:}
    \PY{n+nb}{print}\PY{p}{(}\PY{n}{key}\PY{p}{,}\PY{l+s+s2}{\PYZdq{}}\PY{l+s+s2}{=}\PY{l+s+s2}{\PYZdq{}}\PY{p}{,}\PY{n}{value}\PY{p}{)}
\end{Verbatim}
\end{tcolorbox}

    \begin{Verbatim}[commandchars=\\\{\}]
['Name', 'Age', 'Job']
Name = Mark
Age = 37
Job = Postdoc
    \end{Verbatim}

    \begin{tcolorbox}[breakable, size=fbox, boxrule=1pt, pad at break*=1mm,colback=cellbackground, colframe=cellborder]
\prompt{In}{incolor}{28}{\boxspacing}
\begin{Verbatim}[commandchars=\\\{\}]
\PY{c+c1}{\PYZsh{} In a more relevant example to the lab (and demonstration of dictionary initialization with keys and values):}

\PY{n}{variant\PYZus{}prmtops} \PY{o}{=} \PY{p}{\PYZob{}}\PY{l+s+s2}{\PYZdq{}}\PY{l+s+s2}{WT}\PY{l+s+s2}{\PYZdq{}}\PY{p}{:}\PY{l+s+s2}{\PYZdq{}}\PY{l+s+s2}{A3H\PYZus{}WT.prmtop}\PY{l+s+s2}{\PYZdq{}}\PY{p}{,}
                   \PY{l+s+s2}{\PYZdq{}}\PY{l+s+s2}{K121E}\PY{l+s+s2}{\PYZdq{}}\PY{p}{:}\PY{l+s+s2}{\PYZdq{}}\PY{l+s+s2}{A3H\PYZus{}K121E.prmtop}\PY{l+s+s2}{\PYZdq{}}\PY{p}{,}
                   \PY{l+s+s2}{\PYZdq{}}\PY{l+s+s2}{K117E}\PY{l+s+s2}{\PYZdq{}}\PY{p}{:}\PY{l+s+s2}{\PYZdq{}}\PY{l+s+s2}{A3H\PYZus{}K117E.prmtop}\PY{l+s+s2}{\PYZdq{}}\PY{p}{,}
                   \PY{l+s+s2}{\PYZdq{}}\PY{l+s+s2}{R124D}\PY{l+s+s2}{\PYZdq{}}\PY{p}{:}\PY{l+s+s2}{\PYZdq{}}\PY{l+s+s2}{A3H\PYZus{}R124D.prmtop}\PY{l+s+s2}{\PYZdq{}}\PY{p}{\PYZcb{}}

\PY{c+c1}{\PYZsh{} Now I have a list of filenames stored, and I can recall them anytime with this}
\PY{n+nb}{print}\PY{p}{(}\PY{n}{variant\PYZus{}prmtops}\PY{p}{[}\PY{l+s+s2}{\PYZdq{}}\PY{l+s+s2}{K121E}\PY{l+s+s2}{\PYZdq{}}\PY{p}{]}\PY{p}{)}

\PY{c+c1}{\PYZsh{} We can also have dictionaries inside dictionaries, which can be useful for bigger datasets.}

\PY{n}{full\PYZus{}systems} \PY{o}{=} \PY{p}{\PYZob{}}
\PY{l+s+s2}{\PYZdq{}}\PY{l+s+s2}{WT}\PY{l+s+s2}{\PYZdq{}} \PY{p}{:} \PY{p}{\PYZob{}}\PY{l+s+s2}{\PYZdq{}}\PY{l+s+s2}{prmtop}\PY{l+s+s2}{\PYZdq{}}\PY{p}{:}\PY{l+s+s2}{\PYZdq{}}\PY{l+s+s2}{WT.prmtop}\PY{l+s+s2}{\PYZdq{}}\PY{p}{,}\PY{l+s+s2}{\PYZdq{}}\PY{l+s+s2}{trajectory}\PY{l+s+s2}{\PYZdq{}}\PY{p}{:}\PY{l+s+s2}{\PYZdq{}}\PY{l+s+s2}{WT\PYZus{}100ns.dcd}\PY{l+s+s2}{\PYZdq{}}\PY{p}{,}\PY{l+s+s2}{\PYZdq{}}\PY{l+s+s2}{num\PYZus{}residues}\PY{l+s+s2}{\PYZdq{}}\PY{p}{:}\PY{l+m+mi}{180}\PY{p}{,}\PY{l+s+s2}{\PYZdq{}}\PY{l+s+s2}{duration}\PY{l+s+s2}{\PYZdq{}}\PY{p}{:}\PY{l+m+mi}{100}\PY{p}{\PYZcb{}}\PY{p}{,}
\PY{l+s+s2}{\PYZdq{}}\PY{l+s+s2}{K121E}\PY{l+s+s2}{\PYZdq{}} \PY{p}{:} \PY{p}{\PYZob{}}\PY{l+s+s2}{\PYZdq{}}\PY{l+s+s2}{prmtop}\PY{l+s+s2}{\PYZdq{}}\PY{p}{:}\PY{l+s+s2}{\PYZdq{}}\PY{l+s+s2}{K121E.prmtop}\PY{l+s+s2}{\PYZdq{}}\PY{p}{,}\PY{l+s+s2}{\PYZdq{}}\PY{l+s+s2}{trajectory}\PY{l+s+s2}{\PYZdq{}}\PY{p}{:}\PY{l+s+s2}{\PYZdq{}}\PY{l+s+s2}{K121E\PYZus{}150ns.dcd}\PY{l+s+s2}{\PYZdq{}}\PY{p}{,}\PY{l+s+s2}{\PYZdq{}}\PY{l+s+s2}{num\PYZus{}residues}\PY{l+s+s2}{\PYZdq{}}\PY{p}{:}\PY{l+m+mi}{180}\PY{p}{,}\PY{l+s+s2}{\PYZdq{}}\PY{l+s+s2}{duration}\PY{l+s+s2}{\PYZdq{}}\PY{p}{:}\PY{l+m+mi}{150}\PY{p}{\PYZcb{}}\PY{p}{,}
\PY{l+s+s2}{\PYZdq{}}\PY{l+s+s2}{K117E}\PY{l+s+s2}{\PYZdq{}} \PY{p}{:} \PY{p}{\PYZob{}}\PY{l+s+s2}{\PYZdq{}}\PY{l+s+s2}{prmtop}\PY{l+s+s2}{\PYZdq{}}\PY{p}{:}\PY{l+s+s2}{\PYZdq{}}\PY{l+s+s2}{K117E.prmtop}\PY{l+s+s2}{\PYZdq{}}\PY{p}{,}\PY{l+s+s2}{\PYZdq{}}\PY{l+s+s2}{trajectory}\PY{l+s+s2}{\PYZdq{}}\PY{p}{:}\PY{l+s+s2}{\PYZdq{}}\PY{l+s+s2}{K117E\PYZus{}200ns.dcd}\PY{l+s+s2}{\PYZdq{}}\PY{p}{,}\PY{l+s+s2}{\PYZdq{}}\PY{l+s+s2}{num\PYZus{}residues}\PY{l+s+s2}{\PYZdq{}}\PY{p}{:}\PY{l+m+mi}{180}\PY{p}{,}\PY{l+s+s2}{\PYZdq{}}\PY{l+s+s2}{duration}\PY{l+s+s2}{\PYZdq{}}\PY{p}{:}\PY{l+m+mi}{200}\PY{p}{\PYZcb{}} 
\PY{p}{\PYZcb{}}

\PY{n+nb}{print}\PY{p}{(}\PY{n}{full\PYZus{}systems}\PY{p}{[}\PY{l+s+s2}{\PYZdq{}}\PY{l+s+s2}{WT}\PY{l+s+s2}{\PYZdq{}}\PY{p}{]}\PY{p}{)} \PY{c+c1}{\PYZsh{}\PYZsh{} This prints the entire dictionary}

\PY{n+nb}{print}\PY{p}{(}\PY{n}{full\PYZus{}systems}\PY{p}{[}\PY{l+s+s2}{\PYZdq{}}\PY{l+s+s2}{WT}\PY{l+s+s2}{\PYZdq{}}\PY{p}{]}\PY{p}{[}\PY{l+s+s2}{\PYZdq{}}\PY{l+s+s2}{prmtop}\PY{l+s+s2}{\PYZdq{}}\PY{p}{]}\PY{p}{)}

\PY{c+c1}{\PYZsh{} You can also store larger datasets inside dictionaries this way.  For example, let\PYZsq{}s say you have a dataset for the RMSD of an MD trajectory called \PYZdq{}rmsd\PYZdq{}, and one for correlated motion called \PYZdq{}correl\PYZdq{}}
\PY{k+kn}{import} \PY{n+nn}{numpy} \PY{k}{as} \PY{n+nn}{np}
\PY{n}{rmsd} \PY{o}{=} \PY{n}{np}\PY{o}{.}\PY{n}{random}\PY{o}{.}\PY{n}{rand}\PY{p}{(}\PY{l+m+mi}{100}\PY{p}{)}
\PY{n}{correl} \PY{o}{=} \PY{n}{np}\PY{o}{.}\PY{n}{random}\PY{o}{.}\PY{n}{rand}\PY{p}{(}\PY{l+m+mi}{50}\PY{p}{,}\PY{l+m+mi}{50}\PY{p}{)}

\PY{n}{WT\PYZus{}analyses} \PY{o}{=} \PY{p}{\PYZob{}}\PY{l+s+s2}{\PYZdq{}}\PY{l+s+s2}{RMSD}\PY{l+s+s2}{\PYZdq{}}\PY{p}{:}\PY{n}{rmsd}\PY{p}{,}\PY{l+s+s2}{\PYZdq{}}\PY{l+s+s2}{correl}\PY{l+s+s2}{\PYZdq{}}\PY{p}{:}\PY{n}{correl}\PY{p}{\PYZcb{}}
\end{Verbatim}
\end{tcolorbox}

    \begin{Verbatim}[commandchars=\\\{\}]
A3H\_K121E.prmtop
\{'prmtop': 'WT.prmtop', 'trajectory': 'WT\_100ns.dcd', 'num\_residues': 180,
'duration': 100\}
WT.prmtop
    \end{Verbatim}

    \begin{tcolorbox}[breakable, size=fbox, boxrule=1pt, pad at break*=1mm,colback=cellbackground, colframe=cellborder]
\prompt{In}{incolor}{29}{\boxspacing}
\begin{Verbatim}[commandchars=\\\{\}]
\PY{n}{WT\PYZus{}analyses}\PY{p}{[}\PY{l+s+s2}{\PYZdq{}}\PY{l+s+s2}{correl}\PY{l+s+s2}{\PYZdq{}}\PY{p}{]}
\end{Verbatim}
\end{tcolorbox}

            \begin{tcolorbox}[breakable, size=fbox, boxrule=.5pt, pad at break*=1mm, opacityfill=0]
\prompt{Out}{outcolor}{29}{\boxspacing}
\begin{Verbatim}[commandchars=\\\{\}]
array([[0.07872392, 0.42709779, 0.73786009, {\ldots}, 0.14944642, 0.19055872,
        0.45752917],
       [0.14328318, 0.1483461 , 0.63825779, {\ldots}, 0.92527165, 0.9000854 ,
        0.16021734],
       [0.50201692, 0.75404792, 0.46468422, {\ldots}, 0.44324784, 0.74848111,
        0.94918908],
       {\ldots},
       [0.64425718, 0.47652132, 0.69848477, {\ldots}, 0.19183174, 0.42951161,
        0.73094065],
       [0.2925301 , 0.79944765, 0.88059451, {\ldots}, 0.65282398, 0.82365397,
        0.56632012],
       [0.68523568, 0.71003483, 0.5788092 , {\ldots}, 0.22053449, 0.22295812,
        0.29099571]])
\end{Verbatim}
\end{tcolorbox}
        
    \begin{tcolorbox}[breakable, size=fbox, boxrule=1pt, pad at break*=1mm,colback=cellbackground, colframe=cellborder]
\prompt{In}{incolor}{30}{\boxspacing}
\begin{Verbatim}[commandchars=\\\{\}]
\PY{n}{WT\PYZus{}analyses}\PY{p}{[}\PY{l+s+s2}{\PYZdq{}}\PY{l+s+s2}{RMSD}\PY{l+s+s2}{\PYZdq{}}\PY{p}{]}
\end{Verbatim}
\end{tcolorbox}

            \begin{tcolorbox}[breakable, size=fbox, boxrule=.5pt, pad at break*=1mm, opacityfill=0]
\prompt{Out}{outcolor}{30}{\boxspacing}
\begin{Verbatim}[commandchars=\\\{\}]
array([0.77547714, 0.86825809, 0.73361393, 0.53601816, 0.82347301,
       0.77017413, 0.96120335, 0.41538613, 0.59034859, 0.04939984,
       0.11903879, 0.20902424, 0.76954975, 0.20189964, 0.31954835,
       0.51617544, 0.71262053, 0.44747436, 0.17991915, 0.95935479,
       0.02334348, 0.37246611, 0.37715923, 0.31250376, 0.83167922,
       0.7179388 , 0.26188751, 0.10515804, 0.62818762, 0.82602609,
       0.98213736, 0.22903547, 0.72848045, 0.45872938, 0.26119027,
       0.05973667, 0.65432271, 0.86798405, 0.66082425, 0.0277142 ,
       0.45905896, 0.74384669, 0.29576668, 0.89319424, 0.14992499,
       0.1549802 , 0.89640709, 0.49811178, 0.07505158, 0.85436875,
       0.21030718, 0.13886558, 0.0993692 , 0.04939456, 0.71672669,
       0.00323682, 0.60787663, 0.43602982, 0.31186697, 0.02260092,
       0.41310232, 0.56875889, 0.27089672, 0.24635137, 0.25593297,
       0.65855327, 0.7809256 , 0.37840985, 0.18978786, 0.50569471,
       0.36361048, 0.32712705, 0.5499439 , 0.71682757, 0.01187723,
       0.74752752, 0.36253046, 0.4534361 , 0.75456511, 0.1084834 ,
       0.80284826, 0.66146861, 0.70693715, 0.61945641, 0.04079063,
       0.93344407, 0.71655886, 0.34367994, 0.41819506, 0.66072591,
       0.32044302, 0.3986235 , 0.87465149, 0.94527801, 0.72854001,
       0.16153919, 0.59174719, 0.56051911, 0.26658   , 0.14256879])
\end{Verbatim}
\end{tcolorbox}
