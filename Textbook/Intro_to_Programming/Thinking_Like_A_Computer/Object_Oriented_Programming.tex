\section{Object-Oriented Programming}
\paragraph{} Object-oriented programming (OOP) looks at code based around "objects" rather than logical functions.
Beyond basic data structures like we see with variable types or functions, objects allow more complex arrangements of information and operations.  
For example, a geometry program might have a "circle" object and a "square" object, each with their own unique sets of information.
"Circle" may have a radius variable and include functions to provide circumference and area, while "Square" might have similar variables and functions, yet with different internal code.
Objects are a useful way to identify larger and more complicated constructs with relative ease.

As we will see in later chapters, objects can be a wide array of things, such as a graphical figure in the Python library "matplotlib", a table in "pandas", or an entire molecule with atom objects inside.
The use of objects can help to streamline larger processes and maintain some readability and control in your code.