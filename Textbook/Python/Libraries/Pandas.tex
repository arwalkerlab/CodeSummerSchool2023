\section{Library - Pandas}
Pandas is a wonderful data organization and processing library. It works
off of a single major underlying structure, the \texttt{DataFrame}. The
\texttt{DataFrame} is an object that can be described as a ``dictionary
of dictionaries''. Similar to how previous examples have shown that you
can have nested dictionary datatypes, Pandas takes it a step further and
provides tools to organize, manipulate, combine, compare, and store
these multi-layer dictionaries. Pandas also includes built-in Excel
Spreadsheet creation functionality, meaning that you can transform your
data into more complex structures in Excel, making it easier to share
with collaborators.

Let's look at an example of a simple dataframe with just a few rows and
columns.

    \begin{tcolorbox}[breakable, size=fbox, boxrule=1pt, pad at break*=1mm,colback=cellbackground, colframe=cellborder]
\prompt{In}{incolor}{2}{\boxspacing}
\begin{Verbatim}[commandchars=\\\{\}]
\PY{k+kn}{import} \PY{n+nn}{pandas} \PY{k}{as} \PY{n+nn}{pd}

\PY{n}{df} \PY{o}{=} \PY{n}{pd}\PY{o}{.}\PY{n}{DataFrame}\PY{p}{(}\PY{p}{[}\PY{p}{\PYZob{}}\PY{l+s+s2}{\PYZdq{}}\PY{l+s+s2}{First Name}\PY{l+s+s2}{\PYZdq{}}\PY{p}{:}\PY{l+s+s2}{\PYZdq{}}\PY{l+s+s2}{Zee}\PY{l+s+s2}{\PYZdq{}}\PY{p}{,}\PY{l+s+s2}{\PYZdq{}}\PY{l+s+s2}{Age}\PY{l+s+s2}{\PYZdq{}}\PY{p}{:}\PY{l+m+mi}{40}\PY{p}{,}\PY{l+s+s2}{\PYZdq{}}\PY{l+s+s2}{Team}\PY{l+s+s2}{\PYZdq{}}\PY{p}{:}\PY{l+s+s2}{\PYZdq{}}\PY{l+s+s2}{Blue}\PY{l+s+s2}{\PYZdq{}}\PY{p}{\PYZcb{}}\PY{p}{,}
                   \PY{p}{\PYZob{}}\PY{l+s+s2}{\PYZdq{}}\PY{l+s+s2}{First Name}\PY{l+s+s2}{\PYZdq{}}\PY{p}{:}\PY{l+s+s2}{\PYZdq{}}\PY{l+s+s2}{Charlotte}\PY{l+s+s2}{\PYZdq{}}\PY{p}{,}\PY{l+s+s2}{\PYZdq{}}\PY{l+s+s2}{Age}\PY{l+s+s2}{\PYZdq{}}\PY{p}{:}\PY{l+m+mi}{45}\PY{p}{,}\PY{l+s+s2}{\PYZdq{}}\PY{l+s+s2}{Team}\PY{l+s+s2}{\PYZdq{}}\PY{p}{:}\PY{l+s+s2}{\PYZdq{}}\PY{l+s+s2}{Red}\PY{l+s+s2}{\PYZdq{}}\PY{p}{\PYZcb{}}\PY{p}{,}
                   \PY{p}{\PYZob{}}\PY{l+s+s2}{\PYZdq{}}\PY{l+s+s2}{First Name}\PY{l+s+s2}{\PYZdq{}}\PY{p}{:}\PY{l+s+s2}{\PYZdq{}}\PY{l+s+s2}{Wilbur}\PY{l+s+s2}{\PYZdq{}}\PY{p}{,}\PY{l+s+s2}{\PYZdq{}}\PY{l+s+s2}{Age}\PY{l+s+s2}{\PYZdq{}}\PY{p}{:}\PY{l+m+mi}{50}\PY{p}{,}\PY{l+s+s2}{\PYZdq{}}\PY{l+s+s2}{Team}\PY{l+s+s2}{\PYZdq{}}\PY{p}{:}\PY{l+s+s2}{\PYZdq{}}\PY{l+s+s2}{Green}\PY{l+s+s2}{\PYZdq{}}\PY{p}{\PYZcb{}}\PY{p}{]}\PY{p}{)}
\end{Verbatim}
\end{tcolorbox}

    \begin{tcolorbox}[breakable, size=fbox, boxrule=1pt, pad at break*=1mm,colback=cellbackground, colframe=cellborder]
\prompt{In}{incolor}{3}{\boxspacing}
\begin{Verbatim}[commandchars=\\\{\}]
\PY{n}{display}\PY{p}{(}\PY{n}{df}\PY{p}{)}
\end{Verbatim}
\end{tcolorbox}

    
    \begin{Verbatim}[commandchars=\\\{\}]
  First Name  Age   Team
0        Zee   40   Blue
1  Charlotte   45    Red
2     Wilbur   50  Green
    \end{Verbatim}

    
    \begin{tcolorbox}[breakable, size=fbox, boxrule=1pt, pad at break*=1mm,colback=cellbackground, colframe=cellborder]
\prompt{In}{incolor}{4}{\boxspacing}
\begin{Verbatim}[commandchars=\\\{\}]
\PY{n+nb}{print}\PY{p}{(}\PY{n}{df}\PY{p}{)}
\end{Verbatim}
\end{tcolorbox}

    \begin{Verbatim}[commandchars=\\\{\}]
  First Name  Age   Team
0        Zee   40   Blue
1  Charlotte   45    Red
2     Wilbur   50  Green
    \end{Verbatim}

    Notice the difference between \texttt{display} and \texttt{print} in the
example above. It should be pointed out that \texttt{display} is
functional in notebook environments only, not in terminals. Trying to
use \texttt{display} in a terminal will result in an error.

You can add to existing dataframes by appending new rows. Each row will
be made as a dictionary beforehand.

    \begin{tcolorbox}[breakable, size=fbox, boxrule=1pt, pad at break*=1mm,colback=cellbackground, colframe=cellborder]
\prompt{In}{incolor}{5}{\boxspacing}
\begin{Verbatim}[commandchars=\\\{\}]
\PY{n}{newrow} \PY{o}{=} \PY{p}{\PYZob{}}\PY{l+s+s2}{\PYZdq{}}\PY{l+s+s2}{First Name}\PY{l+s+s2}{\PYZdq{}}\PY{p}{:}\PY{l+s+s2}{\PYZdq{}}\PY{l+s+s2}{Iroh}\PY{l+s+s2}{\PYZdq{}}\PY{p}{,}\PY{l+s+s2}{\PYZdq{}}\PY{l+s+s2}{Age}\PY{l+s+s2}{\PYZdq{}}\PY{p}{:}\PY{l+m+mi}{99}\PY{p}{,}\PY{l+s+s2}{\PYZdq{}}\PY{l+s+s2}{Team}\PY{l+s+s2}{\PYZdq{}}\PY{p}{:}\PY{l+s+s2}{\PYZdq{}}\PY{l+s+s2}{Red}\PY{l+s+s2}{\PYZdq{}}\PY{p}{\PYZcb{}}
\PY{n}{df} \PY{o}{=} \PY{n}{df}\PY{o}{.}\PY{n}{append}\PY{p}{(}\PY{n}{newrow}\PY{p}{,}\PY{n}{ignore\PYZus{}index}\PY{o}{=}\PY{k+kc}{True}\PY{p}{)}
\end{Verbatim}
\end{tcolorbox}

    \begin{tcolorbox}[breakable, size=fbox, boxrule=1pt, pad at break*=1mm,colback=cellbackground, colframe=cellborder]
\prompt{In}{incolor}{6}{\boxspacing}
\begin{Verbatim}[commandchars=\\\{\}]
\PY{n}{display}\PY{p}{(}\PY{n}{df}\PY{p}{)}
\end{Verbatim}
\end{tcolorbox}

    
    \begin{Verbatim}[commandchars=\\\{\}]
  First Name  Age   Team
0        Zee   40   Blue
1  Charlotte   45    Red
2     Wilbur   50  Green
3       Iroh   99    Red
    \end{Verbatim}

    
    Notice that we had to reassign the dataframe when we added the new row.
This is because all of the internal dataframe functions return a
\emph{new} dataframe object. This method ensures that dataframes are not
overwritten on accident, and that data is not lost without intent. Also,
we now have two rows that have a shared value in a column. We can sort
the rows by specific columns, which can be used to group things
together. We can also group by multiple columns to have improved
organization.

    \begin{tcolorbox}[breakable, size=fbox, boxrule=1pt, pad at break*=1mm,colback=cellbackground, colframe=cellborder]
\prompt{In}{incolor}{7}{\boxspacing}
\begin{Verbatim}[commandchars=\\\{\}]
\PY{n}{df2} \PY{o}{=} \PY{n}{df}\PY{o}{.}\PY{n}{sort\PYZus{}values}\PY{p}{(}\PY{l+s+s2}{\PYZdq{}}\PY{l+s+s2}{Team}\PY{l+s+s2}{\PYZdq{}}\PY{p}{,}\PY{n}{ascending}\PY{o}{=}\PY{k+kc}{True}\PY{p}{)}
\PY{n}{display}\PY{p}{(}\PY{n}{df2}\PY{p}{)}
\end{Verbatim}
\end{tcolorbox}

    
    \begin{Verbatim}[commandchars=\\\{\}]
  First Name  Age   Team
0        Zee   40   Blue
2     Wilbur   50  Green
1  Charlotte   45    Red
3       Iroh   99    Red
    \end{Verbatim}

    
    \begin{tcolorbox}[breakable, size=fbox, boxrule=1pt, pad at break*=1mm,colback=cellbackground, colframe=cellborder]
\prompt{In}{incolor}{8}{\boxspacing}
\begin{Verbatim}[commandchars=\\\{\}]
\PY{n}{df3} \PY{o}{=} \PY{n}{df}\PY{o}{.}\PY{n}{sort\PYZus{}values}\PY{p}{(}\PY{p}{[}\PY{l+s+s2}{\PYZdq{}}\PY{l+s+s2}{Team}\PY{l+s+s2}{\PYZdq{}}\PY{p}{,}\PY{l+s+s2}{\PYZdq{}}\PY{l+s+s2}{Age}\PY{l+s+s2}{\PYZdq{}}\PY{p}{]}\PY{p}{,}\PY{n}{ascending}\PY{o}{=}\PY{p}{[}\PY{k+kc}{True}\PY{p}{,}\PY{k+kc}{False}\PY{p}{]}\PY{p}{)}
\PY{n}{display}\PY{p}{(}\PY{n}{df3}\PY{p}{)}
\end{Verbatim}
\end{tcolorbox}

    
    \begin{Verbatim}[commandchars=\\\{\}]
  First Name  Age   Team
0        Zee   40   Blue
2     Wilbur   50  Green
3       Iroh   99    Red
1  Charlotte   45    Red
    \end{Verbatim}

    
    In the first cell, we sort only by the column ``Team'', and put the
results in ascending order. Pandas uses alphabetical sorting unless all
cells in a column are \emph{entirely} numerical.

In the second cell, we sorted by a list, which means the whole dataframe
is sorted by the first element, and any rows that have the same value in
that first element are then sorted by the second element, and so on.
Additionally, we use a second list for the ascending values to set the
ascending/descending state for each element being sorted.

What if we wanted just the values in one column? We can call the
dataframe like we would a dictionary, using the square brackets and a
column name.

    \begin{tcolorbox}[breakable, size=fbox, boxrule=1pt, pad at break*=1mm,colback=cellbackground, colframe=cellborder]
\prompt{In}{incolor}{9}{\boxspacing}
\begin{Verbatim}[commandchars=\\\{\}]
\PY{n+nb}{print}\PY{p}{(}\PY{n}{df}\PY{p}{[}\PY{l+s+s2}{\PYZdq{}}\PY{l+s+s2}{Age}\PY{l+s+s2}{\PYZdq{}}\PY{p}{]}\PY{p}{)}
\end{Verbatim}
\end{tcolorbox}

    \begin{Verbatim}[commandchars=\\\{\}]
0    40
1    45
2    50
3    99
Name: Age, dtype: int64
    \end{Verbatim}

    \begin{tcolorbox}[breakable, size=fbox, boxrule=1pt, pad at break*=1mm,colback=cellbackground, colframe=cellborder]
\prompt{In}{incolor}{10}{\boxspacing}
\begin{Verbatim}[commandchars=\\\{\}]
\PY{n+nb}{print}\PY{p}{(}\PY{n}{df}\PY{p}{[}\PY{l+s+s2}{\PYZdq{}}\PY{l+s+s2}{Age}\PY{l+s+s2}{\PYZdq{}}\PY{p}{]}\PY{o}{.}\PY{n}{values}\PY{p}{)}
\end{Verbatim}
\end{tcolorbox}

    \begin{Verbatim}[commandchars=\\\{\}]
[40 45 50 99]
    \end{Verbatim}

    We can also call individual rows with the \texttt{.iloc} function. In
the example below, \texttt{.iloc(0)} indicates we're iterating on the
0th axis, which is down the rows. The \texttt{{[}2{]}} indicates we want
the data at index 2 in that list of rows. In the original \texttt{df}
dataframe object, that is the third row.

    \begin{tcolorbox}[breakable, size=fbox, boxrule=1pt, pad at break*=1mm,colback=cellbackground, colframe=cellborder]
\prompt{In}{incolor}{12}{\boxspacing}
\begin{Verbatim}[commandchars=\\\{\}]
\PY{n+nb}{print}\PY{p}{(}\PY{n}{df}\PY{o}{.}\PY{n}{iloc}\PY{p}{(}\PY{l+m+mi}{0}\PY{p}{)}\PY{p}{[}\PY{l+m+mi}{2}\PY{p}{]}\PY{p}{)}
\end{Verbatim}
\end{tcolorbox}

    \begin{Verbatim}[commandchars=\\\{\}]
First Name    Wilbur
Age               50
Team           Green
Name: 2, dtype: object
    \end{Verbatim}

    Let's consider an example that might be more relevant to the lab. Mark
is currently working on a Automated Fluorescent Nucleotide Workflow
(that needs a better name with a fun acronym). Over time, this workflow
is intended to produce large amounts of data for hundreds - even
thousands - of molecules. We want to use this data in a machine learning
model to predict possible new fluorescent nucleotides.

We can use Pandas to organize this data into a more easily managed form.

What are the different things to consider for each of the molecules? -
Sugar type (Ribose / Deoxyribose) - Nucleobase (A/C/G/T/U) - Connection
point on the base (C6,C5, etc.) - Tag structure and connection point. -
Absorption Wavelength - Emission Wavelength - Quantum Yield

Let's assume that we don't have all that information for every single
molecule just yet. In fact, that's kind of the point of machine
learning, filling in the gaps of data.

    \begin{tcolorbox}[breakable, size=fbox, boxrule=1pt, pad at break*=1mm,colback=cellbackground, colframe=cellborder]
\prompt{In}{incolor}{54}{\boxspacing}
\begin{Verbatim}[commandchars=\\\{\}]
\PY{c+c1}{\PYZsh{}\PYZsh{} A list of all the columns we want in our dataframe, corresponding to values we want to keep track of for each system.}
\PY{n}{df\PYZus{}columns} \PY{o}{=} \PY{p}{[}\PY{l+s+s2}{\PYZdq{}}\PY{l+s+s2}{Sugar}\PY{l+s+s2}{\PYZdq{}}\PY{p}{,}\PY{l+s+s2}{\PYZdq{}}\PY{l+s+s2}{Base}\PY{l+s+s2}{\PYZdq{}}\PY{p}{,}\PY{l+s+s2}{\PYZdq{}}\PY{l+s+s2}{ConnectionPoint}\PY{l+s+s2}{\PYZdq{}}\PY{p}{,}\PY{l+s+s2}{\PYZdq{}}\PY{l+s+s2}{FluorescentTag}\PY{l+s+s2}{\PYZdq{}}\PY{p}{, }\PY{l+s+s2}{\PYZdq{}}\PY{l+s+s2}{AbsorptionWavelength}\PY{l+s+s2}{\PYZdq{}}\PY{p}{,}\PY{l+s+s2}{\PYZdq{}}\PY{l+s+s2}{EmissionWavelength}\PY{l+s+s2}{\PYZdq{}}\PY{p}{,}\PY{l+s+s2}{\PYZdq{}}\PY{l+s+s2}{QuantumYield}\PY{l+s+s2}{\PYZdq{}}\PY{p}{]}

\PY{c+c1}{\PYZsh{}\PYZsh{} Initialize the dataframe without data, just column names}
\PY{n}{nucleotides} \PY{o}{=} \PY{n}{pd}\PY{o}{.}\PY{n}{DataFrame}\PY{p}{(}\PY{n}{columns}\PY{o}{=}\PY{n}{df\PYZus{}columns}\PY{p}{)}

\PY{n}{display}\PY{p}{(}\PY{n}{nucleotides}\PY{p}{)}
\end{Verbatim}
\end{tcolorbox}

    
    \begin{Verbatim}[commandchars=\\\{\}]
Empty DataFrame
Columns: [Sugar, Base, ConnectionPoint, FluorescentTag, AbsorptionWavelength, EmissionWavelength, QuantumYield]
Index: []
    \end{Verbatim}

    
    I'm including a function below to make my life easier for the example
process, but if we haven't talked about functions yet, you can ignore it
for now.

    \begin{tcolorbox}[breakable, size=fbox, boxrule=1pt, pad at break*=1mm,colback=cellbackground, colframe=cellborder]
\prompt{In}{incolor}{55}{\boxspacing}
\begin{Verbatim}[commandchars=\\\{\}]
\PY{k}{def} \PY{n+nf}{add\PYZus{}molecule}\PY{p}{(}\PY{n}{df}\PY{p}{,}\PY{o}{*}\PY{o}{*}\PY{n}{kwargs}\PY{p}{)}\PY{p}{:}
    \PY{n}{new\PYZus{}molecule} \PY{o}{=} \PY{p}{\PYZob{}}\PY{p}{\PYZcb{}}
    \PY{k}{for} \PY{n}{key}\PY{p}{,}\PY{n}{val} \PY{o+ow}{in} \PY{n}{kwargs}\PY{o}{.}\PY{n}{items}\PY{p}{(}\PY{p}{)}\PY{p}{:}
        \PY{n}{new\PYZus{}molecule}\PY{p}{[}\PY{n}{key}\PY{p}{]} \PY{o}{=} \PY{n}{val}
    \PY{n}{df} \PY{o}{=} \PY{n}{df}\PY{o}{.}\PY{n}{append}\PY{p}{(}\PY{n}{new\PYZus{}molecule}\PY{p}{,}\PY{n}{ignore\PYZus{}index}\PY{o}{=}\PY{k+kc}{True}\PY{p}{)}
    \PY{k}{return} \PY{n}{df}
\end{Verbatim}
\end{tcolorbox}

    \begin{tcolorbox}[breakable, size=fbox, boxrule=1pt, pad at break*=1mm,colback=cellbackground, colframe=cellborder]
\prompt{In}{incolor}{56}{\boxspacing}
\begin{Verbatim}[commandchars=\\\{\}]
\PY{n}{nucleotides} \PY{o}{=} \PY{n}{add\PYZus{}molecule}\PY{p}{(}\PY{n}{nucleotides}\PY{p}{,}\PY{n}{Sugar}\PY{o}{=}\PY{l+s+s2}{\PYZdq{}}\PY{l+s+s2}{Ribose}\PY{l+s+s2}{\PYZdq{}}\PY{p}{,}\PY{n}{Base}\PY{o}{=}\PY{l+s+s2}{\PYZdq{}}\PY{l+s+s2}{C}\PY{l+s+s2}{\PYZdq{}}\PY{p}{,}\PY{n}{ConnectionPoint}\PY{o}{=}\PY{l+s+s2}{\PYZdq{}}\PY{l+s+s2}{C5}\PY{l+s+s2}{\PYZdq{}}\PY{p}{, }\PY{n}{AbsorptionWavelength}\PY{o}{=}\PY{l+m+mi}{380}\PY{p}{,}\PY{n}{EmissionWavelength}\PY{o}{=}\PY{l+m+mi}{415}\PY{p}{,}\PY{n}{QuantumYield}\PY{o}{=}\PY{l+m+mf}{.93}\PY{p}{)}
\PY{n}{nucleotides} \PY{o}{=} \PY{n}{add\PYZus{}molecule}\PY{p}{(}\PY{n}{nucleotides}\PY{p}{,}\PY{n}{Sugar}\PY{o}{=}\PY{l+s+s2}{\PYZdq{}}\PY{l+s+s2}{Ribose}\PY{l+s+s2}{\PYZdq{}}\PY{p}{,}\PY{n}{Base}\PY{o}{=}\PY{l+s+s2}{\PYZdq{}}\PY{l+s+s2}{C}\PY{l+s+s2}{\PYZdq{}}\PY{p}{,}\PY{n}{FluorescentTag}\PY{o}{=}\PY{l+s+s2}{\PYZdq{}}\PY{l+s+s2}{Perylene}\PY{l+s+s2}{\PYZdq{}}\PY{p}{, }\PY{n}{ConnectionPoint}\PY{o}{=}\PY{l+s+s2}{\PYZdq{}}\PY{l+s+s2}{C6}\PY{l+s+s2}{\PYZdq{}}\PY{p}{,}\PY{n}{EmissionWavelength}\PY{o}{=}\PY{l+m+mi}{465}\PY{p}{,}\PY{n}{QuantumYield}\PY{o}{=}\PY{l+m+mf}{.96}\PY{p}{)}
\PY{n}{nucleotides} \PY{o}{=} \PY{n}{add\PYZus{}molecule}\PY{p}{(}\PY{n}{nucleotides}\PY{p}{,}\PY{n}{Sugar}\PY{o}{=}\PY{l+s+s2}{\PYZdq{}}\PY{l+s+s2}{Deoxyribose}\PY{l+s+s2}{\PYZdq{}}\PY{p}{,}\PY{n}{Base}\PY{o}{=}\PY{l+s+s2}{\PYZdq{}}\PY{l+s+s2}{C}\PY{l+s+s2}{\PYZdq{}}\PY{p}{,}\PY{n}{FluorescentTag}\PY{o}{=}\PY{l+s+s2}{\PYZdq{}}\PY{l+s+s2}{Benzopyrene}\PY{l+s+s2}{\PYZdq{}}\PY{p}{, }\PY{n}{ConnectionPoint}\PY{o}{=}\PY{l+s+s2}{\PYZdq{}}\PY{l+s+s2}{C5}\PY{l+s+s2}{\PYZdq{}}\PY{p}{,}\PY{n}{AbsorptionWavelength}\PY{o}{=}\PY{l+m+mi}{360}\PY{p}{,}\PY{n}{EmissionWavelength}\PY{o}{=}\PY{l+m+mi}{395}\PY{p}{)}
\PY{n}{nucleotides} \PY{o}{=} \PY{n}{add\PYZus{}molecule}\PY{p}{(}\PY{n}{nucleotides}\PY{p}{,}\PY{n}{Sugar}\PY{o}{=}\PY{l+s+s2}{\PYZdq{}}\PY{l+s+s2}{Deoxyribose}\PY{l+s+s2}{\PYZdq{}}\PY{p}{,}\PY{n}{Base}\PY{o}{=}\PY{l+s+s2}{\PYZdq{}}\PY{l+s+s2}{C}\PY{l+s+s2}{\PYZdq{}}\PY{p}{,}\PY{n}{FluorescentTag}\PY{o}{=}\PY{l+s+s2}{\PYZdq{}}\PY{l+s+s2}{Furan}\PY{l+s+s2}{\PYZdq{}}\PY{p}{, }\PY{n}{ConnectionPoint}\PY{o}{=}\PY{l+s+s2}{\PYZdq{}}\PY{l+s+s2}{C6}\PY{l+s+s2}{\PYZdq{}}\PY{p}{,}\PY{n}{AbsorptionWavelength}\PY{o}{=}\PY{l+m+mi}{390}\PY{p}{,}\PY{n}{QuantumYield}\PY{o}{=}\PY{l+m+mf}{.75}\PY{p}{)}

\PY{n}{display}\PY{p}{(}\PY{n}{nucleotides}\PY{p}{)}
\end{Verbatim}
\end{tcolorbox}

    
    \begin{Verbatim}[commandchars=\\\{\}]
         Sugar Base ConnectionPoint FluorescentTag AbsorptionWavelength  \textbackslash{}
0       Ribose    C              C5            NaN                  380   
1       Ribose    C              C6       Perylene                  NaN   
2  Deoxyribose    C              C5    Benzopyrene                  360   
3  Deoxyribose    C              C6          Furan                  390   

  EmissionWavelength  QuantumYield  
0                415          0.93  
1                465          0.96  
2                395           NaN  
3                NaN          0.75  
    \end{Verbatim}

    
    In the above cell, we can see that empty values are printed as
\texttt{NaN}. We can clear those out and replace them with empty cells
to make it easier to see where the data is missing.

    \begin{tcolorbox}[breakable, size=fbox, boxrule=1pt, pad at break*=1mm,colback=cellbackground, colframe=cellborder]
\prompt{In}{incolor}{57}{\boxspacing}
\begin{Verbatim}[commandchars=\\\{\}]
\PY{n}{display}\PY{p}{(}\PY{n}{nucleotides}\PY{o}{.}\PY{n}{fillna}\PY{p}{(}\PY{l+s+s1}{\PYZsq{}}\PY{l+s+s1}{\PYZsq{}}\PY{p}{)}\PY{p}{)}
\end{Verbatim}
\end{tcolorbox}

    
    \begin{Verbatim}[commandchars=\\\{\}]
         Sugar Base ConnectionPoint FluorescentTag AbsorptionWavelength  \textbackslash{}
0       Ribose    C              C5                                 380   
1       Ribose    C              C6       Perylene                        
2  Deoxyribose    C              C5    Benzopyrene                  360   
3  Deoxyribose    C              C6          Furan                  390   

  EmissionWavelength QuantumYield  
0                415         0.93  
1                465         0.96  
2                395               
3                            0.75  
    \end{Verbatim}

    
    \begin{tcolorbox}[breakable, size=fbox, boxrule=1pt, pad at break*=1mm,colback=cellbackground, colframe=cellborder]
\prompt{In}{incolor}{58}{\boxspacing}
\begin{Verbatim}[commandchars=\\\{\}]
\PY{n}{display}\PY{p}{(}\PY{n}{nucleotides}\PY{o}{.}\PY{n}{sort\PYZus{}values}\PY{p}{(}\PY{p}{[}\PY{l+s+s2}{\PYZdq{}}\PY{l+s+s2}{AbsorptionWavelength}\PY{l+s+s2}{\PYZdq{}}\PY{p}{,}\PY{l+s+s2}{\PYZdq{}}\PY{l+s+s2}{EmissionWavelength}\PY{l+s+s2}{\PYZdq{}}\PY{p}{,}\PY{l+s+s2}{\PYZdq{}}\PY{l+s+s2}{QuantumYield}\PY{l+s+s2}{\PYZdq{}}\PY{p}{]}\PY{p}{)}\PY{o}{.}\PY{n}{fillna}\PY{p}{(}\PY{l+s+s2}{\PYZdq{}}\PY{l+s+s2}{\PYZdq{}}\PY{p}{)}\PY{p}{)}
\end{Verbatim}
\end{tcolorbox}

    
    \begin{Verbatim}[commandchars=\\\{\}]
         Sugar Base ConnectionPoint FluorescentTag AbsorptionWavelength  \textbackslash{}
2  Deoxyribose    C              C5    Benzopyrene                  360   
0       Ribose    C              C5                                 380   
3  Deoxyribose    C              C6          Furan                  390   
1       Ribose    C              C6       Perylene                        

  EmissionWavelength QuantumYield  
2                395               
0                415         0.93  
3                            0.75  
1                465         0.96  
    \end{Verbatim}

    
    Let's generate some random data to mess with.

    \begin{tcolorbox}[breakable, size=fbox, boxrule=1pt, pad at break*=1mm,colback=cellbackground, colframe=cellborder]
\prompt{In}{incolor}{71}{\boxspacing}
\begin{Verbatim}[commandchars=\\\{\}]
\PY{k+kn}{import} \PY{n+nn}{numpy} \PY{k}{as} \PY{n+nn}{np}
\PY{n}{base\PYZus{}list}\PY{o}{=}\PY{p}{[}\PY{l+s+s2}{\PYZdq{}}\PY{l+s+s2}{A\PYZus{}C2}\PY{l+s+s2}{\PYZdq{}}\PY{p}{,}\PY{l+s+s2}{\PYZdq{}}\PY{l+s+s2}{A\PYZus{}C8}\PY{l+s+s2}{\PYZdq{}}\PY{p}{,}\PY{l+s+s2}{\PYZdq{}}\PY{l+s+s2}{C\PYZus{}C5}\PY{l+s+s2}{\PYZdq{}}\PY{p}{,}\PY{l+s+s2}{\PYZdq{}}\PY{l+s+s2}{C\PYZus{}C6}\PY{l+s+s2}{\PYZdq{}}\PY{p}{,}\PY{l+s+s2}{\PYZdq{}}\PY{l+s+s2}{G\PYZus{}C2}\PY{l+s+s2}{\PYZdq{}}\PY{p}{,}\PY{l+s+s2}{\PYZdq{}}\PY{l+s+s2}{G\PYZus{}C8}\PY{l+s+s2}{\PYZdq{}}\PY{p}{,}\PY{l+s+s2}{\PYZdq{}}\PY{l+s+s2}{G\PYZus{}N7}\PY{l+s+s2}{\PYZdq{}}\PY{p}{,}\PY{l+s+s2}{\PYZdq{}}\PY{l+s+s2}{U\PYZus{}C6}\PY{l+s+s2}{\PYZdq{}}\PY{p}{,}\PY{l+s+s2}{\PYZdq{}}\PY{l+s+s2}{U\PYZus{}C5}\PY{l+s+s2}{\PYZdq{}}\PY{p}{,}\PY{l+s+s2}{\PYZdq{}}\PY{l+s+s2}{T\PYZus{}C6}\PY{l+s+s2}{\PYZdq{}}\PY{p}{]}
\PY{n}{sugar\PYZus{}list}\PY{o}{=}\PY{p}{[}\PY{l+s+s2}{\PYZdq{}}\PY{l+s+s2}{Ribose}\PY{l+s+s2}{\PYZdq{}}\PY{p}{,}\PY{l+s+s2}{\PYZdq{}}\PY{l+s+s2}{Deoxyribose}\PY{l+s+s2}{\PYZdq{}}\PY{p}{]}
\PY{n}{tag\PYZus{}list} \PY{o}{=} \PY{p}{[}\PY{l+s+s2}{\PYZdq{}}\PY{l+s+s2}{Perylene}\PY{l+s+s2}{\PYZdq{}}\PY{p}{,}\PY{l+s+s2}{\PYZdq{}}\PY{l+s+s2}{Benzopyrene}\PY{l+s+s2}{\PYZdq{}}\PY{p}{,}\PY{l+s+s2}{\PYZdq{}}\PY{l+s+s2}{Furan}\PY{l+s+s2}{\PYZdq{}}\PY{p}{,}\PY{l+s+s2}{\PYZdq{}}\PY{l+s+s2}{Naphthalene}\PY{l+s+s2}{\PYZdq{}}\PY{p}{,}\PY{l+s+s2}{\PYZdq{}}\PY{l+s+s2}{Beta Carotene}\PY{l+s+s2}{\PYZdq{}}\PY{p}{,}\PY{l+s+s2}{\PYZdq{}}\PY{l+s+s2}{Imidazole}\PY{l+s+s2}{\PYZdq{}}\PY{p}{]}
\PY{n}{nucleotides} \PY{o}{=} \PY{n}{pd}\PY{o}{.}\PY{n}{DataFrame}\PY{p}{(}\PY{n}{columns}\PY{o}{=}\PY{n}{df\PYZus{}columns}\PY{p}{)}
\PY{k}{for} \PY{n}{base} \PY{o+ow}{in} \PY{n}{base\PYZus{}list}\PY{p}{:}
    \PY{k}{for} \PY{n}{sugar} \PY{o+ow}{in} \PY{n}{sugar\PYZus{}list}\PY{p}{:}
        \PY{k}{for} \PY{n}{tag} \PY{o+ow}{in} \PY{n}{tag\PYZus{}list}\PY{p}{:}
            \PY{n}{abs\PYZus{}wl} \PY{o}{=} \PY{n}{np}\PY{o}{.}\PY{n}{random}\PY{o}{.}\PY{n}{randint}\PY{p}{(}\PY{l+m+mi}{355}\PY{p}{,}\PY{l+m+mi}{800}\PY{p}{)}
            \PY{n}{emi\PYZus{}wl} \PY{o}{=} \PY{n}{np}\PY{o}{.}\PY{n}{random}\PY{o}{.}\PY{n}{randint}\PY{p}{(}\PY{n}{abs\PYZus{}wl}\PY{p}{,}\PY{l+m+mi}{850}\PY{p}{)}
            \PY{n}{nucleotides} \PY{o}{=} \PY{n}{add\PYZus{}molecule}\PY{p}{(}\PY{n}{nucleotides}\PY{p}{,}\PY{n}{Sugar}\PY{o}{=}\PY{n}{sugar}\PY{p}{,}\PY{n}{Base}\PY{o}{=}\PY{n}{base}\PY{o}{.}\PY{n}{split}\PY{p}{(}\PY{l+s+s2}{\PYZdq{}}\PY{l+s+s2}{\PYZus{}}\PY{l+s+s2}{\PYZdq{}}\PY{p}{)}\PY{p}{[}\PY{l+m+mi}{0}\PY{p}{]}\PY{p}{,}\PY{n}{ConnectionPoint}\PY{o}{=}\PY{n}{base}\PY{o}{.}\PY{n}{split}\PY{p}{(}\PY{l+s+s2}{\PYZdq{}}\PY{l+s+s2}{\PYZus{}}\PY{l+s+s2}{\PYZdq{}}\PY{p}{)}\PY{p}{[}\PY{l+m+mi}{1}\PY{p}{]}\PY{p}{,}\PY{n}{FluorescentTag}\PY{o}{=}\PY{n}{tag}\PY{p}{,}\PY{n}{AbsorptionWavelength}\PY{o}{=}\PY{n}{abs\PYZus{}wl}\PY{p}{,}\PY{n}{EmissionWavelength}\PY{o}{=}\PY{n}{emi\PYZus{}wl}\PY{p}{,}\PY{n}{QuantumYield}\PY{o}{=}\PY{n+nb}{round}\PY{p}{(}\PY{n}{np}\PY{o}{.}\PY{n}{random}\PY{o}{.}\PY{n}{rand}\PY{p}{(}\PY{p}{)}\PY{p}{,}\PY{l+m+mi}{2}\PY{p}{)}\PY{p}{)}
\end{Verbatim}
\end{tcolorbox}

    \begin{tcolorbox}[breakable, size=fbox, boxrule=1pt, pad at break*=1mm,colback=cellbackground, colframe=cellborder]
\prompt{In}{incolor}{72}{\boxspacing}
\begin{Verbatim}[commandchars=\\\{\}]
\PY{n}{display}\PY{p}{(}\PY{n}{nucleotides}\PY{o}{.}\PY{n}{fillna}\PY{p}{(}\PY{l+s+s2}{\PYZdq{}}\PY{l+s+s2}{\PYZdq{}}\PY{p}{)}\PY{p}{)}
\end{Verbatim}
\end{tcolorbox}

    
    \begin{Verbatim}[commandchars=\\\{\}]
           Sugar Base ConnectionPoint FluorescentTag  AbsorptionWavelength  \textbackslash{}
0         Ribose    A              C2       Perylene                   561   
1         Ribose    A              C2    Benzopyrene                   664   
2         Ribose    A              C2          Furan                   604   
3         Ribose    A              C2    Naphthalene                   424   
4         Ribose    A              C2  Beta Carotene                   774   
..           {\ldots}  {\ldots}             {\ldots}            {\ldots}                   {\ldots}   
115  Deoxyribose    T              C6    Benzopyrene                   602   
116  Deoxyribose    T              C6          Furan                   531   
117  Deoxyribose    T              C6    Naphthalene                   489   
118  Deoxyribose    T              C6  Beta Carotene                   697   
119  Deoxyribose    T              C6      Imidazole                   389   

     EmissionWavelength  QuantumYield  
0                   828          0.65  
1                   771          0.55  
2                   737          0.09  
3                   734          0.61  
4                   791          0.01  
..                  {\ldots}           {\ldots}  
115                 808          0.38  
116                 776          0.91  
117                 639          0.68  
118                 757          0.11  
119                 529          0.09  

[120 rows x 7 columns]
    \end{Verbatim}

    
    Note here that the full dataframe is not shown, but rather displays only
the first five and last five rows. The dimensions of the dataframe are
given below it. In this case, it's 120 rows and 7 columns.

What if we wanted to sort the data by Quantum Yield?

    \begin{tcolorbox}[breakable, size=fbox, boxrule=1pt, pad at break*=1mm,colback=cellbackground, colframe=cellborder]
\prompt{In}{incolor}{73}{\boxspacing}
\begin{Verbatim}[commandchars=\\\{\}]
\PY{n}{nucleotides}\PY{o}{.}\PY{n}{sort\PYZus{}values}\PY{p}{(}\PY{l+s+s2}{\PYZdq{}}\PY{l+s+s2}{QuantumYield}\PY{l+s+s2}{\PYZdq{}}\PY{p}{,}\PY{n}{ascending}\PY{o}{=}\PY{k+kc}{False}\PY{p}{)}
\end{Verbatim}
\end{tcolorbox}

            \begin{tcolorbox}[breakable, size=fbox, boxrule=.5pt, pad at break*=1mm, opacityfill=0]
\prompt{Out}{outcolor}{73}{\boxspacing}
\begin{Verbatim}[commandchars=\\\{\}]
           Sugar Base ConnectionPoint FluorescentTag AbsorptionWavelength  \textbackslash{}
83   Deoxyribose    G              N7      Imidazole                  587
92   Deoxyribose    U              C6          Furan                  724
21   Deoxyribose    A              C8    Naphthalene                  513
70   Deoxyribose    G              C8  Beta Carotene                  643
29        Ribose    C              C5      Imidazole                  450
..           {\ldots}  {\ldots}             {\ldots}            {\ldots}                  {\ldots}
44   Deoxyribose    C              C6          Furan                  588
89        Ribose    U              C6      Imidazole                  415
26        Ribose    C              C5          Furan                  761
110       Ribose    T              C6          Furan                  419
4         Ribose    A              C2  Beta Carotene                  774

    EmissionWavelength  QuantumYield
83                 740          1.00
92                 846          0.99
21                 765          0.98
70                 758          0.96
29                 500          0.95
..                 {\ldots}           {\ldots}
44                 828          0.06
89                 707          0.03
26                 797          0.01
110                472          0.01
4                  791          0.01

[120 rows x 7 columns]
\end{Verbatim}
\end{tcolorbox}
        
    Keep in mind that the above result did not sort the dataframe in itself,
it just presented the results of a sorting algorithm applied to it. Note
the indices on the left side and how they're out of order.

Now, what if we wanted to plot some of the data from our dataframe?

If our dataframe is simple enough, we can just call the \texttt{plot()}
function directly from it, which returns a \texttt{matplotlib\ Axes}
object.

    \begin{tcolorbox}[breakable, size=fbox, boxrule=1pt, pad at break*=1mm,colback=cellbackground, colframe=cellborder]
\prompt{In}{incolor}{83}{\boxspacing}
\begin{Verbatim}[commandchars=\\\{\}]
\PY{k+kn}{import} \PY{n+nn}{matplotlib}\PY{n+nn}{.}\PY{n+nn}{pyplot} \PY{k}{as} \PY{n+nn}{plt}
\PY{n}{fig} \PY{o}{=} \PY{n}{plt}\PY{o}{.}\PY{n}{figure}\PY{p}{(}\PY{n}{figsize}\PY{o}{=}\PY{p}{[}\PY{l+m+mi}{5}\PY{p}{,}\PY{l+m+mi}{4}\PY{p}{]}\PY{p}{,}\PY{n}{dpi}\PY{o}{=}\PY{l+m+mi}{300}\PY{p}{)}
\PY{n}{ax} \PY{o}{=} \PY{n}{nucleotides}\PY{o}{.}\PY{n}{plot}\PY{p}{(}\PY{p}{)}
\end{Verbatim}
\end{tcolorbox}

    
    \begin{Verbatim}[commandchars=\\\{\}]
<Figure size 1500x1200 with 0 Axes>
    \end{Verbatim}

    
    \begin{center}
    \adjustimage{max size={0.9\linewidth}{0.9\paperheight}}{output_29_1.png}
    \end{center}
    { \hspace*{\fill} \\}
    
    We can also simply call columns of our data as though they were any
other array of numbers (assuming the column is, in fact, strictly
numerical).

    \begin{tcolorbox}[breakable, size=fbox, boxrule=1pt, pad at break*=1mm,colback=cellbackground, colframe=cellborder]
\prompt{In}{incolor}{85}{\boxspacing}
\begin{Verbatim}[commandchars=\\\{\}]
\PY{k+kn}{import} \PY{n+nn}{matplotlib}\PY{n+nn}{.}\PY{n+nn}{pyplot} \PY{k}{as} \PY{n+nn}{plt}
\PY{n}{fig} \PY{o}{=} \PY{n}{plt}\PY{o}{.}\PY{n}{figure}\PY{p}{(}\PY{n}{figsize}\PY{o}{=}\PY{p}{[}\PY{l+m+mi}{5}\PY{p}{,}\PY{l+m+mi}{4}\PY{p}{]}\PY{p}{,}\PY{n}{dpi}\PY{o}{=}\PY{l+m+mi}{300}\PY{p}{)}
\PY{n}{ax} \PY{o}{=} \PY{n}{fig}\PY{o}{.}\PY{n}{add\PYZus{}subplot}\PY{p}{(}\PY{l+m+mi}{1}\PY{p}{,}\PY{l+m+mi}{1}\PY{p}{,}\PY{l+m+mi}{1}\PY{p}{)}
\PY{n}{ax}\PY{o}{.}\PY{n}{scatter}\PY{p}{(}\PY{n}{nucleotides}\PY{p}{[}\PY{l+s+s2}{\PYZdq{}}\PY{l+s+s2}{AbsorptionWavelength}\PY{l+s+s2}{\PYZdq{}}\PY{p}{]}\PY{p}{,}\PY{n}{nucleotides}\PY{p}{[}\PY{l+s+s2}{\PYZdq{}}\PY{l+s+s2}{EmissionWavelength}\PY{l+s+s2}{\PYZdq{}}\PY{p}{]}\PY{p}{,}\PY{n}{s}\PY{o}{=}\PY{n}{ nucleotides}\PY{p}{[}\PY{l+s+s2}{\PYZdq{}}\PY{l+s+s2}{QuantumYield}\PY{l+s+s2}{\PYZdq{}}\PY{p}{]}\PY{o}{*}\PY{l+m+mi}{5}\PY{p}{)}

\PY{c+c1}{\PYZsh{}\PYZsh{}\PYZsh{} I have included the QuantumYield column values as the marker size variable so that each point gives us the absorption/emission wavelengths and the quantum yields.}
\end{Verbatim}
\end{tcolorbox}

            \begin{tcolorbox}[breakable, size=fbox, boxrule=.5pt, pad at break*=1mm, opacityfill=0]
\prompt{Out}{outcolor}{85}{\boxspacing}
\begin{Verbatim}[commandchars=\\\{\}]
<matplotlib.collections.PathCollection at 0x7f45f705cfd0>
\end{Verbatim}
\end{tcolorbox}
        
    \begin{center}
    \adjustimage{max size={0.9\linewidth}{0.9\paperheight}}{output_31_1.png}
    \end{center}
    { \hspace*{\fill} \\}