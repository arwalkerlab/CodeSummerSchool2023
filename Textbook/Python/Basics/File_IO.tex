\section{File I/O}
I/O stands for ``Input/Output''. Loading files into your programs,
parsing the data inside, and being able to use, manipulate, transform,
present, and save the results is a very important skillset.

    \hypertarget{reading-files}{%
\subsubsection{Reading Files}\label{reading-files}}

Python has a few different ways to read files, but for now we'll stick
with plain-text files like you'll get from various analysis programs or
computational chemistry software packages. Python can also read binary
data files, but that's generally more advanced and requires a bit more
experience.

The number one rule to remember for any and all files is that if you
\textbf{open} it, you must \textbf{close} it. A file that is opened and
never closed can either wind up inaccessible to other programs or the
user, or the data that was meant to be stored within can be lost.

See the example below. With the \texttt{open} function, we create a
\textbf{file object} that has several useful internal functions we can
take advantage of. \texttt{open()} takes two arguments - the filename
first, then the type of action to be taken on the file. ``r'' is for
``read mode'', ``w'' is for ``write mode'' which starts at the beginning
of the file and writes (including overwriting!), and ``a'' is for
``append mode'', which acts like ``write mode'' except that it adds to
the end of the file rather than overwriting it.

    \begin{tcolorbox}[breakable, size=fbox, boxrule=1pt, pad at break*=1mm,colback=cellbackground, colframe=cellborder]
\prompt{In}{incolor}{6}{\boxspacing}
\begin{Verbatim}[commandchars=\\\{\}]
\PY{n}{myfile} \PY{o}{=} \PY{n+nb}{open}\PY{p}{(}\PY{l+s+s2}{\PYZdq{}}\PY{l+s+s2}{test\PYZus{}data/file\PYZus{}reading.txt}\PY{l+s+s2}{\PYZdq{}}\PY{p}{,}\PY{l+s+s2}{\PYZdq{}}\PY{l+s+s2}{r}\PY{l+s+s2}{\PYZdq{}}\PY{p}{)}

\PY{n}{contents} \PY{o}{=} \PY{n}{myfile}\PY{o}{.}\PY{n}{read}\PY{p}{(}\PY{p}{)}

\PY{n+nb}{print}\PY{p}{(}\PY{n+nb}{type}\PY{p}{(}\PY{n}{contents}\PY{p}{)}\PY{p}{)}
\PY{n+nb}{print}\PY{p}{(}\PY{n}{contents}\PY{p}{)}

\PY{n}{myfile}\PY{o}{.}\PY{n}{close}\PY{p}{(}\PY{p}{)}
\end{Verbatim}
\end{tcolorbox}

    \begin{Verbatim}[commandchars=\\\{\}]
<class 'str'>
This is just a normal text file.
There are words.  There are sentences.  There are lines.

There was even a blank line.
After this sentence is the end of the file, usually called "EOF".
    \end{Verbatim}

    As you can see, the entire file can be read into a single string, which
includes all the line breaks and everything within. This can be easiest
if you have a small file, or if you know exactly the structure of the
data inside.

There is also a function that can read one line at a time. This
function, \texttt{readline()}, returns the \emph{next} line in the file,
up to the newline character, \texttt{\textbackslash{}n}. We often use
this if the file is large, or if there are multiple line formats in the
file (such as we might see in a TeraChem output file), or if we want to
keep memory usage low.

    \begin{tcolorbox}[breakable, size=fbox, boxrule=1pt, pad at break*=1mm,colback=cellbackground, colframe=cellborder]
\prompt{In}{incolor}{10}{\boxspacing}
\begin{Verbatim}[commandchars=\\\{\}]
\PY{n}{myfile} \PY{o}{=} \PY{n+nb}{open}\PY{p}{(}\PY{l+s+s2}{\PYZdq{}}\PY{l+s+s2}{test\PYZus{}data/file\PYZus{}reading.txt}\PY{l+s+s2}{\PYZdq{}}\PY{p}{,}\PY{l+s+s2}{\PYZdq{}}\PY{l+s+s2}{r}\PY{l+s+s2}{\PYZdq{}}\PY{p}{)}

\PY{n}{contents} \PY{o}{=} \PY{n}{myfile}\PY{o}{.}\PY{n}{readline}\PY{p}{(}\PY{p}{)}
\PY{n+nb}{print}\PY{p}{(}\PY{n}{contents}\PY{p}{)}

\PY{c+c1}{\PYZsh{}\PYZsh{}\PYZsh{} We\PYZsq{}ve already read the first line, but if we do the same set of commands again, we\PYZsq{}ll get the next line}

\PY{n}{contents} \PY{o}{=} \PY{n}{myfile}\PY{o}{.}\PY{n}{readline}\PY{p}{(}\PY{p}{)}
\PY{n+nb}{print}\PY{p}{(}\PY{n}{contents}\PY{p}{)}

\PY{c+c1}{\PYZsh{}\PYZsh{}\PYZsh{} And again, for the third. (see above, the third line is blank!)}

\PY{n}{contents} \PY{o}{=} \PY{n}{myfile}\PY{o}{.}\PY{n}{readline}\PY{p}{(}\PY{p}{)}
\PY{n+nb}{print}\PY{p}{(}\PY{n}{contents}\PY{p}{)}

\PY{c+c1}{\PYZsh{}\PYZsh{}\PYZsh{} Fourth}

\PY{n}{contents} \PY{o}{=} \PY{n}{myfile}\PY{o}{.}\PY{n}{readline}\PY{p}{(}\PY{p}{)}
\PY{n+nb}{print}\PY{p}{(}\PY{n}{contents}\PY{p}{)}

\PY{n}{myfile}\PY{o}{.}\PY{n}{close}\PY{p}{(}\PY{p}{)}
\end{Verbatim}
\end{tcolorbox}

    \begin{Verbatim}[commandchars=\\\{\}]
This is just a normal text file.

There are words.  There are sentences.  There are lines.



There was even a blank line.

    \end{Verbatim}

    We can also get and keep all the lines in memory stored as a
\texttt{list}, with each element of the list holding one line of the
file as a string. This uses a function called \texttt{readlines()},
which has the \texttt{s} at the end to keep it separate from the last
function, \texttt{readline()}.

    \begin{tcolorbox}[breakable, size=fbox, boxrule=1pt, pad at break*=1mm,colback=cellbackground, colframe=cellborder]
\prompt{In}{incolor}{12}{\boxspacing}
\begin{Verbatim}[commandchars=\\\{\}]
\PY{n}{myfile} \PY{o}{=} \PY{n+nb}{open}\PY{p}{(}\PY{l+s+s2}{\PYZdq{}}\PY{l+s+s2}{test\PYZus{}data/file\PYZus{}reading.txt}\PY{l+s+s2}{\PYZdq{}}\PY{p}{,}\PY{l+s+s2}{\PYZdq{}}\PY{l+s+s2}{r}\PY{l+s+s2}{\PYZdq{}}\PY{p}{)}
\PY{n}{all\PYZus{}lines}\PY{o}{=}\PY{n}{myfile}\PY{o}{.}\PY{n}{readlines}\PY{p}{(}\PY{p}{)}
\PY{n+nb}{print}\PY{p}{(}\PY{n}{all\PYZus{}lines}\PY{p}{)}
\PY{n}{myfile}\PY{o}{.}\PY{n}{close}\PY{p}{(}\PY{p}{)}
\end{Verbatim}
\end{tcolorbox}

    \begin{Verbatim}[commandchars=\\\{\}]
['This is just a normal text file.\textbackslash{}n', 'There are words.  There are sentences.
There are lines.  \textbackslash{}n', '\textbackslash{}n', 'There was even a blank line.\textbackslash{}n', 'After this
sentence is the end of the file, usually called "EOF".']
    \end{Verbatim}

    Now you can see all the lines in the file, but presented as a list of
strings instead of the formatted text. With the list structure of the
\texttt{readlines()} function, you can then do more complicated things
to the individual lines by iterating through the list, or choose
specific lines with list-slicing like we discussed previously.

    \begin{tcolorbox}[breakable, size=fbox, boxrule=1pt, pad at break*=1mm,colback=cellbackground, colframe=cellborder]
\prompt{In}{incolor}{14}{\boxspacing}
\begin{Verbatim}[commandchars=\\\{\}]
\PY{k}{for} \PY{n}{line} \PY{o+ow}{in} \PY{n}{all\PYZus{}lines}\PY{p}{:}
    \PY{k}{if} \PY{l+s+s2}{\PYZdq{}}\PY{l+s+s2}{There}\PY{l+s+s2}{\PYZdq{}} \PY{o+ow}{in} \PY{n}{line}\PY{p}{:}
        \PY{n+nb}{print}\PY{p}{(}\PY{n}{line}\PY{p}{)}
\end{Verbatim}
\end{tcolorbox}

    \begin{Verbatim}[commandchars=\\\{\}]
There are words.  There are sentences.  There are lines.

There was even a blank line.

    \end{Verbatim}

    You can also further split up the lines with the \texttt{.split()}
function available to any string object.

    \begin{tcolorbox}[breakable, size=fbox, boxrule=1pt, pad at break*=1mm,colback=cellbackground, colframe=cellborder]
\prompt{In}{incolor}{15}{\boxspacing}
\begin{Verbatim}[commandchars=\\\{\}]
\PY{k}{for} \PY{n}{line} \PY{o+ow}{in} \PY{n}{all\PYZus{}lines}\PY{p}{:}
    \PY{n+nb}{print}\PY{p}{(}\PY{n}{line}\PY{o}{.}\PY{n}{split}\PY{p}{(}\PY{p}{)}\PY{p}{)}
\end{Verbatim}
\end{tcolorbox}

    \begin{Verbatim}[commandchars=\\\{\}]
['This', 'is', 'just', 'a', 'normal', 'text', 'file.']
['There', 'are', 'words.', 'There', 'are', 'sentences.', 'There', 'are',
'lines.']
[]
['There', 'was', 'even', 'a', 'blank', 'line.']
['After', 'this', 'sentence', 'is', 'the', 'end', 'of', 'the', 'file,',
'usually', 'called', '"EOF".']
    \end{Verbatim}

    With this in mind, can you write a function to search the
\texttt{TeraChemOutput.out} file (included in the \texttt{test\_data}
folder) for every instance of ``RMS grad''?

    \begin{tcolorbox}[breakable, size=fbox, boxrule=1pt, pad at break*=1mm,colback=cellbackground, colframe=cellborder]
\prompt{In}{incolor}{ }{\boxspacing}
\begin{Verbatim}[commandchars=\\\{\}]
\PY{c+c1}{\PYZsh{}\PYZsh{}\PYZsh{} Your code goes here!}
\end{Verbatim}
\end{tcolorbox}

    Now that you can find all the lines with ``RMS grad'' in them, can you
extract the current value at each line and add it to a dataset?

\emph{Hint: use the \texttt{append} function, and don't forget to
convert from a string to a float!}

    \begin{tcolorbox}[breakable, size=fbox, boxrule=1pt, pad at break*=1mm,colback=cellbackground, colframe=cellborder]
\prompt{In}{incolor}{5}{\boxspacing}
\begin{Verbatim}[commandchars=\\\{\}]
\PY{n}{dataset} \PY{o}{=} \PY{p}{[}\PY{p}{]}

\PY{c+c1}{\PYZsh{}\PYZsh{}\PYZsh{} Your code goes here!}
\end{Verbatim}
\end{tcolorbox}

    If you were successful in finding, extracting, and storing the ``RMS
grad'' value for each occurrence in the file, run the next cell to see
that data plotted!

    \begin{tcolorbox}[breakable, size=fbox, boxrule=1pt, pad at break*=1mm,colback=cellbackground, colframe=cellborder]
\prompt{In}{incolor}{6}{\boxspacing}
\begin{Verbatim}[commandchars=\\\{\}]
\PY{k+kn}{import} \PY{n+nn}{matplotlib}\PY{n+nn}{.}\PY{n+nn}{pyplot} \PY{k}{as} \PY{n+nn}{plt}
\PY{n}{plt}\PY{o}{.}\PY{n}{plot}\PY{p}{(}\PY{n}{dataset}\PY{p}{)}
\end{Verbatim}
\end{tcolorbox}

            \begin{tcolorbox}[breakable, size=fbox, boxrule=.5pt, pad at break*=1mm, opacityfill=0]
\prompt{Out}{outcolor}{6}{\boxspacing}
\begin{Verbatim}[commandchars=\\\{\}]
[<matplotlib.lines.Line2D at 0x7fb50353f490>]
\end{Verbatim}
\end{tcolorbox}
        
    \begin{center}
    \adjustimage{max size={0.9\linewidth}{0.9\paperheight}}{Images/output_16_1.png}
    \end{center}
    { \hspace*{\fill} \\}
    
    You can see how useful parsing datafiles can be, especially when those
datafiles aren't as neatly formatted as a CPPTRAJ output, or when there
is lots of different kinds of data that you may not need or want at that
moment.

But what about writing your own outputs?

\hypertarget{writing-files}{%
\subsubsection{Writing Files}\label{writing-files}}

To write to a file, you start in much the same way as reading, except
you open the file in ``w'' or ``a'' mode. The file, if it doesn't
already exist, will be created on \texttt{open} and saved on
\texttt{close}, so again be sure to close every file you open.

    \begin{tcolorbox}[breakable, size=fbox, boxrule=1pt, pad at break*=1mm,colback=cellbackground, colframe=cellborder]
\prompt{In}{incolor}{7}{\boxspacing}
\begin{Verbatim}[commandchars=\\\{\}]
\PY{n}{myfile} \PY{o}{=} \PY{n+nb}{open}\PY{p}{(}\PY{l+s+s2}{\PYZdq{}}\PY{l+s+s2}{test\PYZus{}data/writing\PYZus{}outputs.txt}\PY{l+s+s2}{\PYZdq{}}\PY{p}{,}\PY{l+s+s2}{\PYZdq{}}\PY{l+s+s2}{w}\PY{l+s+s2}{\PYZdq{}}\PY{p}{)}

\PY{n}{myfile}\PY{o}{.}\PY{n}{write}\PY{p}{(}\PY{l+s+s2}{\PYZdq{}}\PY{l+s+s2}{This is the text I am writing. }\PY{l+s+s2}{\PYZdq{}}\PY{p}{)}
\PY{n}{myfile}\PY{o}{.}\PY{n}{write}\PY{p}{(}\PY{l+s+s2}{\PYZdq{}}\PY{l+s+s2}{Notice how there is no newline after the last sentence... }\PY{l+s+s2}{\PYZdq{}}\PY{p}{)}
\PY{n}{myfile}\PY{o}{.}\PY{n}{write}\PY{p}{(}\PY{l+s+s2}{\PYZdq{}}\PY{l+s+s2}{You have to include the newline character explicitly when writing to a file.}\PY{l+s+se}{\PYZbs{}n}\PY{l+s+s2}{\PYZdq{}}\PY{p}{)}
\PY{n}{myfile}\PY{o}{.}\PY{n}{write}\PY{p}{(}\PY{l+s+s2}{\PYZdq{}}\PY{l+s+se}{\PYZbs{}n}\PY{l+s+s2}{\PYZdq{}}\PY{p}{)}
\PY{n}{myfile}\PY{o}{.}\PY{n}{write}\PY{p}{(}\PY{l+s+s2}{\PYZdq{}}\PY{l+s+s2}{Like that!}\PY{l+s+se}{\PYZbs{}n}\PY{l+s+s2}{\PYZdq{}}\PY{p}{)}

\PY{n}{myfile}\PY{o}{.}\PY{n}{close}\PY{p}{(}\PY{p}{)}
\end{Verbatim}
\end{tcolorbox}

    You can also write data with specific formatting, producing nice neat
columns of data that can make it easier to read and analyze. You can use
string-formatting to specify the allowed width of sections of a line.

Consider the following set of data:

    \begin{tcolorbox}[breakable, size=fbox, boxrule=1pt, pad at break*=1mm,colback=cellbackground, colframe=cellborder]
\prompt{In}{incolor}{11}{\boxspacing}
\begin{Verbatim}[commandchars=\\\{\}]
\PY{n}{people} \PY{o}{=} \PY{p}{[}  \PY{p}{\PYZob{}}\PY{l+s+s2}{\PYZdq{}}\PY{l+s+s2}{Name}\PY{l+s+s2}{\PYZdq{}}\PY{p}{:}\PY{l+s+s2}{\PYZdq{}}\PY{l+s+s2}{Bobby}\PY{l+s+s2}{\PYZdq{}}\PY{p}{,}\PY{l+s+s2}{\PYZdq{}}\PY{l+s+s2}{Age}\PY{l+s+s2}{\PYZdq{}}\PY{p}{:}\PY{l+m+mi}{25}\PY{p}{,}\PY{l+s+s2}{\PYZdq{}}\PY{l+s+s2}{Class}\PY{l+s+s2}{\PYZdq{}}\PY{p}{:}\PY{l+s+s2}{\PYZdq{}}\PY{l+s+s2}{First\PYZhy{}Year}\PY{l+s+s2}{\PYZdq{}}\PY{p}{,}\PY{l+s+s2}{\PYZdq{}}\PY{l+s+s2}{GPA}\PY{l+s+s2}{\PYZdq{}}\PY{p}{:}\PY{l+m+mf}{3.95}\PY{p}{\PYZcb{}}\PY{p}{,}
            \PY{p}{\PYZob{}}\PY{l+s+s2}{\PYZdq{}}\PY{l+s+s2}{Name}\PY{l+s+s2}{\PYZdq{}}\PY{p}{:}\PY{l+s+s2}{\PYZdq{}}\PY{l+s+s2}{Charlie}\PY{l+s+s2}{\PYZdq{}}\PY{p}{,}\PY{l+s+s2}{\PYZdq{}}\PY{l+s+s2}{Age}\PY{l+s+s2}{\PYZdq{}}\PY{p}{:}\PY{l+m+mi}{27}\PY{p}{,}\PY{l+s+s2}{\PYZdq{}}\PY{l+s+s2}{Class}\PY{l+s+s2}{\PYZdq{}}\PY{p}{:}\PY{l+s+s2}{\PYZdq{}}\PY{l+s+s2}{Second\PYZhy{}Year}\PY{l+s+s2}{\PYZdq{}}\PY{p}{,}\PY{l+s+s2}{\PYZdq{}}\PY{l+s+s2}{GPA}\PY{l+s+s2}{\PYZdq{}}\PY{p}{:}\PY{l+m+mf}{3.87}\PY{p}{\PYZcb{}}\PY{p}{,}
            \PY{p}{\PYZob{}}\PY{l+s+s2}{\PYZdq{}}\PY{l+s+s2}{Name}\PY{l+s+s2}{\PYZdq{}}\PY{p}{:}\PY{l+s+s2}{\PYZdq{}}\PY{l+s+s2}{David}\PY{l+s+s2}{\PYZdq{}}\PY{p}{,}\PY{l+s+s2}{\PYZdq{}}\PY{l+s+s2}{Age}\PY{l+s+s2}{\PYZdq{}}\PY{p}{:}\PY{l+m+mi}{26}\PY{p}{,}\PY{l+s+s2}{\PYZdq{}}\PY{l+s+s2}{Class}\PY{l+s+s2}{\PYZdq{}}\PY{p}{:}\PY{l+s+s2}{\PYZdq{}}\PY{l+s+s2}{First\PYZhy{}Year}\PY{l+s+s2}{\PYZdq{}}\PY{p}{,}\PY{l+s+s2}{\PYZdq{}}\PY{l+s+s2}{GPA}\PY{l+s+s2}{\PYZdq{}}\PY{p}{:}\PY{l+m+mi}{4}\PY{p}{\PYZcb{}}\PY{p}{]}

\PY{n}{myfile} \PY{o}{=} \PY{n+nb}{open}\PY{p}{(}\PY{l+s+s2}{\PYZdq{}}\PY{l+s+s2}{test\PYZus{}data/people\PYZus{}test.txt}\PY{l+s+s2}{\PYZdq{}}\PY{p}{,}\PY{l+s+s2}{\PYZdq{}}\PY{l+s+s2}{w}\PY{l+s+s2}{\PYZdq{}}\PY{p}{)}
\PY{n}{myfile}\PY{o}{.}\PY{n}{write}\PY{p}{(}\PY{l+s+sa}{f}\PY{l+s+s2}{\PYZdq{}}\PY{l+s+si}{\PYZob{}}\PY{l+s+s1}{\PYZsq{}}\PY{l+s+s1}{Name}\PY{l+s+s1}{\PYZsq{}}\PY{l+s+si}{:}\PY{l+s+s2}{\PYZlt{}20}\PY{l+s+si}{\PYZcb{}}\PY{l+s+s2}{ }\PY{l+s+si}{\PYZob{}}\PY{l+s+s1}{\PYZsq{}}\PY{l+s+s1}{Age}\PY{l+s+s1}{\PYZsq{}}\PY{l+s+si}{:}\PY{l+s+s2}{\PYZgt{}10}\PY{l+s+si}{\PYZcb{}}\PY{l+s+s2}{ }\PY{l+s+si}{\PYZob{}}\PY{l+s+s1}{\PYZsq{}}\PY{l+s+s1}{Class}\PY{l+s+s1}{\PYZsq{}}\PY{l+s+si}{:}\PY{l+s+s2}{\PYZlt{}20}\PY{l+s+si}{\PYZcb{}}\PY{l+s+s2}{ }\PY{l+s+si}{\PYZob{}}\PY{l+s+s1}{\PYZsq{}}\PY{l+s+s1}{GPA}\PY{l+s+s1}{\PYZsq{}}\PY{l+s+si}{:}\PY{l+s+s2}{\PYZgt{}10}\PY{l+s+si}{\PYZcb{}}\PY{l+s+se}{\PYZbs{}n}\PY{l+s+s2}{\PYZdq{}}\PY{p}{)}
\PY{n}{myfile}\PY{o}{.}\PY{n}{write}\PY{p}{(}\PY{l+s+s2}{\PYZdq{}}\PY{l+s+s2}{\PYZhy{}\PYZhy{}\PYZhy{}\PYZhy{}\PYZhy{}\PYZhy{}\PYZhy{}\PYZhy{}\PYZhy{}\PYZhy{}\PYZhy{}\PYZhy{}\PYZhy{}\PYZhy{}\PYZhy{}\PYZhy{}\PYZhy{}\PYZhy{}\PYZhy{}\PYZhy{}\PYZhy{}\PYZhy{}\PYZhy{}\PYZhy{}\PYZhy{}\PYZhy{}\PYZhy{}\PYZhy{}\PYZhy{}\PYZhy{}\PYZhy{}\PYZhy{}\PYZhy{}\PYZhy{}\PYZhy{}\PYZhy{}\PYZhy{}\PYZhy{}\PYZhy{}\PYZhy{}\PYZhy{}\PYZhy{}\PYZhy{}\PYZhy{}\PYZhy{}\PYZhy{}\PYZhy{}\PYZhy{}\PYZhy{}\PYZhy{}\PYZhy{}\PYZhy{}\PYZhy{}\PYZhy{}\PYZhy{}\PYZhy{}\PYZhy{}\PYZhy{}\PYZhy{}\PYZhy{}\PYZhy{}\PYZhy{}\PYZhy{}}\PY{l+s+se}{\PYZbs{}n}\PY{l+s+s2}{\PYZdq{}}\PY{p}{)}
\PY{k}{for} \PY{n}{person} \PY{o+ow}{in} \PY{n}{people}\PY{p}{:}
    \PY{n}{myfile}\PY{o}{.}\PY{n}{write}\PY{p}{(}\PY{l+s+sa}{f}\PY{l+s+s2}{\PYZdq{}}\PY{l+s+si}{\PYZob{}}\PY{n}{person}\PY{p}{[}\PY{l+s+s1}{\PYZsq{}}\PY{l+s+s1}{Name}\PY{l+s+s1}{\PYZsq{}}\PY{p}{]}\PY{l+s+si}{:}\PY{l+s+s2}{\PYZlt{}20}\PY{l+s+si}{\PYZcb{}}\PY{l+s+s2}{ }\PY{l+s+si}{\PYZob{}}\PY{n}{person}\PY{p}{[}\PY{l+s+s1}{\PYZsq{}}\PY{l+s+s1}{Age}\PY{l+s+s1}{\PYZsq{}}\PY{p}{]}\PY{l+s+si}{:}\PY{l+s+s2}{\PYZgt{}10}\PY{l+s+si}{\PYZcb{}}\PY{l+s+s2}{ }\PY{l+s+si}{\PYZob{}}\PY{n}{person}\PY{p}{[}\PY{l+s+s1}{\PYZsq{}}\PY{l+s+s1}{Class}\PY{l+s+s1}{\PYZsq{}}\PY{p}{]}\PY{l+s+si}{:}\PY{l+s+s2}{\PYZlt{}20}\PY{l+s+si}{\PYZcb{}}\PY{l+s+s2}{ }\PY{l+s+si}{\PYZob{}}\PY{n}{person}\PY{p}{[}\PY{l+s+s1}{\PYZsq{}}\PY{l+s+s1}{GPA}\PY{l+s+s1}{\PYZsq{}}\PY{p}{]}\PY{l+s+si}{:}\PY{l+s+s2}{\PYZgt{}10.4f}\PY{l+s+si}{\PYZcb{}}\PY{l+s+se}{\PYZbs{}n}\PY{l+s+s2}{\PYZdq{}}\PY{p}{)}
\PY{n}{myfile}\PY{o}{.}\PY{n}{close}\PY{p}{(}\PY{p}{)}
\end{Verbatim}
\end{tcolorbox}

    If you open the file we've just created, it should look like this:

\begin{verbatim}
Name                        Age Class                       GPA
-------------------- ---------- -------------------- ----------
Bobby                        25 First-Year               3.9500
Charlie                      27 Second-Year              3.8700
David                        26 First-Year               4.0000
\end{verbatim}

Notice that the first column is left justified and has empty spaces out
to the specified length of 20 characters. Next, ``Age'' is
right-justified with 10 spaces total. Class is left-justified again with
twenty, and GPA is right-justified \emph{with trailing zeros}, which we
specified with the \texttt{.4f} addition, indicating we wanted four
decimal places and that the value is to be treated like a float. I also
included some blank spaces between each of the variables being printed
1) to show you where each section begins and ends, and 2) to illustrate
how you should be careful when using justifications so you don't wind up
with Age and Class stuck together as a single value.

Also notice that there is a \texttt{\textbackslash{}n} character at the
end of each line in the code. This ensures that the text file is
properly separated by lines - python won't write what you don't tell it
to write.

What if we want to add to a file that already exists? Simple, use
``append mode'', or open with ``a''.

    \begin{tcolorbox}[breakable, size=fbox, boxrule=1pt, pad at break*=1mm,colback=cellbackground, colframe=cellborder]
\prompt{In}{incolor}{12}{\boxspacing}
\begin{Verbatim}[commandchars=\\\{\}]
\PY{n}{people} \PY{o}{=} \PY{p}{[}  \PY{p}{\PYZob{}}\PY{l+s+s2}{\PYZdq{}}\PY{l+s+s2}{Name}\PY{l+s+s2}{\PYZdq{}}\PY{p}{:}\PY{l+s+s2}{\PYZdq{}}\PY{l+s+s2}{Esther}\PY{l+s+s2}{\PYZdq{}}\PY{p}{,}\PY{l+s+s2}{\PYZdq{}}\PY{l+s+s2}{Age}\PY{l+s+s2}{\PYZdq{}}\PY{p}{:}\PY{l+m+mi}{24}\PY{p}{,}\PY{l+s+s2}{\PYZdq{}}\PY{l+s+s2}{Class}\PY{l+s+s2}{\PYZdq{}}\PY{p}{:}\PY{l+s+s2}{\PYZdq{}}\PY{l+s+s2}{First\PYZhy{}Year}\PY{l+s+s2}{\PYZdq{}}\PY{p}{,}\PY{l+s+s2}{\PYZdq{}}\PY{l+s+s2}{GPA}\PY{l+s+s2}{\PYZdq{}}\PY{p}{:}\PY{l+m+mf}{3.96}\PY{p}{\PYZcb{}}\PY{p}{,}
            \PY{p}{\PYZob{}}\PY{l+s+s2}{\PYZdq{}}\PY{l+s+s2}{Name}\PY{l+s+s2}{\PYZdq{}}\PY{p}{:}\PY{l+s+s2}{\PYZdq{}}\PY{l+s+s2}{Frances}\PY{l+s+s2}{\PYZdq{}}\PY{p}{,}\PY{l+s+s2}{\PYZdq{}}\PY{l+s+s2}{Age}\PY{l+s+s2}{\PYZdq{}}\PY{p}{:}\PY{l+m+mi}{28}\PY{p}{,}\PY{l+s+s2}{\PYZdq{}}\PY{l+s+s2}{Class}\PY{l+s+s2}{\PYZdq{}}\PY{p}{:}\PY{l+s+s2}{\PYZdq{}}\PY{l+s+s2}{Second\PYZhy{}Year}\PY{l+s+s2}{\PYZdq{}}\PY{p}{,}\PY{l+s+s2}{\PYZdq{}}\PY{l+s+s2}{GPA}\PY{l+s+s2}{\PYZdq{}}\PY{p}{:}\PY{l+m+mf}{3.99}\PY{p}{\PYZcb{}}\PY{p}{,}
            \PY{p}{\PYZob{}}\PY{l+s+s2}{\PYZdq{}}\PY{l+s+s2}{Name}\PY{l+s+s2}{\PYZdq{}}\PY{p}{:}\PY{l+s+s2}{\PYZdq{}}\PY{l+s+s2}{Gloria}\PY{l+s+s2}{\PYZdq{}}\PY{p}{,}\PY{l+s+s2}{\PYZdq{}}\PY{l+s+s2}{Age}\PY{l+s+s2}{\PYZdq{}}\PY{p}{:}\PY{l+m+mi}{27}\PY{p}{,}\PY{l+s+s2}{\PYZdq{}}\PY{l+s+s2}{Class}\PY{l+s+s2}{\PYZdq{}}\PY{p}{:}\PY{l+s+s2}{\PYZdq{}}\PY{l+s+s2}{First\PYZhy{}Year}\PY{l+s+s2}{\PYZdq{}}\PY{p}{,}\PY{l+s+s2}{\PYZdq{}}\PY{l+s+s2}{GPA}\PY{l+s+s2}{\PYZdq{}}\PY{p}{:}\PY{l+m+mi}{4}\PY{p}{\PYZcb{}}\PY{p}{]}

\PY{n}{myfile} \PY{o}{=} \PY{n+nb}{open}\PY{p}{(}\PY{l+s+s2}{\PYZdq{}}\PY{l+s+s2}{test\PYZus{}data/people\PYZus{}test.txt}\PY{l+s+s2}{\PYZdq{}}\PY{p}{,}\PY{l+s+s2}{\PYZdq{}}\PY{l+s+s2}{a}\PY{l+s+s2}{\PYZdq{}}\PY{p}{)}
\PY{k}{for} \PY{n}{person} \PY{o+ow}{in} \PY{n}{people}\PY{p}{:}
    \PY{n}{myfile}\PY{o}{.}\PY{n}{write}\PY{p}{(}\PY{l+s+sa}{f}\PY{l+s+s2}{\PYZdq{}}\PY{l+s+si}{\PYZob{}}\PY{n}{person}\PY{p}{[}\PY{l+s+s1}{\PYZsq{}}\PY{l+s+s1}{Name}\PY{l+s+s1}{\PYZsq{}}\PY{p}{]}\PY{l+s+si}{:}\PY{l+s+s2}{\PYZlt{}20}\PY{l+s+si}{\PYZcb{}}\PY{l+s+s2}{ }\PY{l+s+si}{\PYZob{}}\PY{n}{person}\PY{p}{[}\PY{l+s+s1}{\PYZsq{}}\PY{l+s+s1}{Age}\PY{l+s+s1}{\PYZsq{}}\PY{p}{]}\PY{l+s+si}{:}\PY{l+s+s2}{\PYZgt{}10}\PY{l+s+si}{\PYZcb{}}\PY{l+s+s2}{ }\PY{l+s+si}{\PYZob{}}\PY{n}{person}\PY{p}{[}\PY{l+s+s1}{\PYZsq{}}\PY{l+s+s1}{Class}\PY{l+s+s1}{\PYZsq{}}\PY{p}{]}\PY{l+s+si}{:}\PY{l+s+s2}{\PYZlt{}20}\PY{l+s+si}{\PYZcb{}}\PY{l+s+s2}{ }\PY{l+s+si}{\PYZob{}}\PY{n}{person}\PY{p}{[}\PY{l+s+s1}{\PYZsq{}}\PY{l+s+s1}{GPA}\PY{l+s+s1}{\PYZsq{}}\PY{p}{]}\PY{l+s+si}{:}\PY{l+s+s2}{\PYZgt{}10.4f}\PY{l+s+si}{\PYZcb{}}\PY{l+s+se}{\PYZbs{}n}\PY{l+s+s2}{\PYZdq{}}\PY{p}{)}
\PY{n}{myfile}\PY{o}{.}\PY{n}{close}\PY{p}{(}\PY{p}{)}
\end{Verbatim}
\end{tcolorbox}

    Now if we reopen the file we created, we can see the addition of the
next three people to the list! This can be useful for things like
logging results over the course of a longer calculation, where you might
not want the file to be kept open and in memory for the entire time, but
still want to add to it on occasion.

File I/O can help you find, organize, and process your data in a way
that makes sense to you.
