\section{System Interaction}
One of the included modules in python is the \texttt{subprocess} module,
which allows you to execute shell commands on a Linux system from within
your python script. You can also use it to read the outputs from these
commands. There are a few different options available to manage these
kinds of system/program interactions.

\hypertarget{subprocess.call}{%
\paragraph{subprocess.call()}\label{subprocess.call}}

The \texttt{subprocess.call()} function is fairly straightforward. It is
useful if you need to execute a command without concern for the output
to the terminal of that command.

    \begin{tcolorbox}[breakable, size=fbox, boxrule=1pt, pad at break*=1mm,colback=cellbackground, colframe=cellborder]
\prompt{In}{incolor}{1}{\boxspacing}
\begin{Verbatim}[commandchars=\\\{\}]
\PY{k+kn}{import} \PY{n+nn}{subprocess}

\PY{n}{subprocess}\PY{o}{.}\PY{n}{call}\PY{p}{(}\PY{l+s+s2}{\PYZdq{}}\PY{l+s+s2}{ls \PYZhy{}lrth}\PY{l+s+s2}{\PYZdq{}}\PY{p}{,}\PY{n}{shell}\PY{o}{=}\PY{k+kc}{True}\PY{p}{)}
\end{Verbatim}
\end{tcolorbox}

    \begin{Verbatim}[commandchars=\\\{\}]
total 48K
-rw-rw-r-- 1 mark mark    0 Jun 13 16:37 19\_Python\_System\_Interaction.ipynb
-rw-rw-r-- 1 mark mark    0 Jun 13 16:38 20\_Python\_Final\_Project.ipynb
drwxrwxr-x 2 mark mark 4.0K Jun 17 16:45 test\_data
-rw-rw-r-- 1 mark mark  33K Jun 17 17:01 17\_Python\_File\_IO.ipynb
-rw-rw-r-- 1 mark mark 4.8K Jun 17 17:35 18\_Python\_Error\_Handling.ipynb
    \end{Verbatim}

            \begin{tcolorbox}[breakable, size=fbox, boxrule=.5pt, pad at break*=1mm, opacityfill=0]
\prompt{Out}{outcolor}{1}{\boxspacing}
\begin{Verbatim}[commandchars=\\\{\}]
0
\end{Verbatim}
\end{tcolorbox}
        
    In a notebook environment, the results of the \texttt{subprocess.call()}
command are simply printed to the output of the cell. In a regular
system environment, this output is printed to the terminal. The
important thing to remember, however, is that neither of those options
allow for the capture and storage of the output by python. In the
example above, we as a user can see the output of our \texttt{ls\ -lrth}
command, but our \textbf{program} cannot.

I often use the \texttt{call()} function when calling external programs
from within my python scripts, specifically when those programs will
produce their own output files that I can use later. In these scenarios,
the outputs to the terminal are not as important.

Also, note that \texttt{subprocess.call} includes a keyword argument
\texttt{shell=True}. This is \emph{required} to have the command
actually run in your shell environment.

\hypertarget{subprocess.popen}{%
\paragraph{subprocess.Popen()}\label{subprocess.popen}}

The other kind of system interaction requires the capture and storage of
the terminal outputs. The \texttt{Popen} command can be used in the same
way as \texttt{call}, but with some additional keywords and with the
functionality included to capture the response from the commands.

    \begin{tcolorbox}[breakable, size=fbox, boxrule=1pt, pad at break*=1mm,colback=cellbackground, colframe=cellborder]
\prompt{In}{incolor}{2}{\boxspacing}
\begin{Verbatim}[commandchars=\\\{\}]
\PY{k+kn}{import} \PY{n+nn}{subprocess}

\PY{n}{proc} \PY{o}{=} \PY{n}{subprocess}\PY{o}{.}\PY{n}{Popen}\PY{p}{(}\PY{l+s+s2}{\PYZdq{}}\PY{l+s+s2}{ls \PYZhy{}lrth}\PY{l+s+s2}{\PYZdq{}}\PY{p}{,}\PY{n}{shell}\PY{o}{=}\PY{k+kc}{True}\PY{p}{,}\PY{n}{stdout}\PY{o}{=}\PY{n}{subprocess}\PY{o}{.}\PY{n}{PIPE}\PY{p}{,}\PY{n}{stderr}\PY{o}{=}\PY{n}{subprocess}\PY{o}{.}\PY{n}{PIPE}\PY{p}{)}
\PY{n}{output}\PY{p}{,}\PY{n}{errors} \PY{o}{=} \PY{n}{proc}\PY{o}{.}\PY{n}{communicate}\PY{p}{(}\PY{p}{)}
\end{Verbatim}
\end{tcolorbox}

    \begin{tcolorbox}[breakable, size=fbox, boxrule=1pt, pad at break*=1mm,colback=cellbackground, colframe=cellborder]
\prompt{In}{incolor}{8}{\boxspacing}
\begin{Verbatim}[commandchars=\\\{\}]
\PY{n+nb}{print}\PY{p}{(}\PY{n}{output}\PY{p}{)}
\end{Verbatim}
\end{tcolorbox}

    \begin{Verbatim}[commandchars=\\\{\}]
b'total 48K\textbackslash{}n-rw-rw-r-- 1 mark mark    0 Jun 13 16:37
19\_Python\_System\_Interaction.ipynb\textbackslash{}n-rw-rw-r-- 1 mark mark    0 Jun 13 16:38
20\_Python\_Final\_Project.ipynb\textbackslash{}ndrwxrwxr-x 2 mark mark 4.0K Jun 17 16:45
test\_data\textbackslash{}n-rw-rw-r-- 1 mark mark  33K Jun 17 17:01
17\_Python\_File\_IO.ipynb\textbackslash{}n-rw-rw-r-- 1 mark mark 4.8K Jun 17 17:35
18\_Python\_Error\_Handling.ipynb\textbackslash{}n'
    \end{Verbatim}

    \begin{tcolorbox}[breakable, size=fbox, boxrule=1pt, pad at break*=1mm,colback=cellbackground, colframe=cellborder]
\prompt{In}{incolor}{9}{\boxspacing}
\begin{Verbatim}[commandchars=\\\{\}]
\PY{n+nb}{print}\PY{p}{(}\PY{n}{errors}\PY{p}{)}
\end{Verbatim}
\end{tcolorbox}

    \begin{Verbatim}[commandchars=\\\{\}]
b''
    \end{Verbatim}

    You will note that when the \texttt{output} and \texttt{errors}
variables are printed, they have a \texttt{b} at the front of the
string. This is an important thing to be aware of - these outputs are
\emph{technically} in binary format, not plain-text. Fortunately, the
python \texttt{print} function can translate from binary to text.
However, working with binary data can be slightly more difficult in some
ways, and it might be easier for us to convert to plain strings before
parsing information. We can use the \texttt{decode} function for this.

    \begin{tcolorbox}[breakable, size=fbox, boxrule=1pt, pad at break*=1mm,colback=cellbackground, colframe=cellborder]
\prompt{In}{incolor}{12}{\boxspacing}
\begin{Verbatim}[commandchars=\\\{\}]
\PY{n}{decoded} \PY{o}{=} \PY{n}{output}\PY{o}{.}\PY{n}{decode}\PY{p}{(}\PY{l+s+s2}{\PYZdq{}}\PY{l+s+s2}{utf\PYZhy{}8}\PY{l+s+s2}{\PYZdq{}}\PY{p}{)}

\PY{n+nb}{print}\PY{p}{(}\PY{n}{decoded}\PY{p}{)}
\end{Verbatim}
\end{tcolorbox}

    \begin{Verbatim}[commandchars=\\\{\}]
total 48K
-rw-rw-r-- 1 mark mark    0 Jun 13 16:37 19\_Python\_System\_Interaction.ipynb
-rw-rw-r-- 1 mark mark    0 Jun 13 16:38 20\_Python\_Final\_Project.ipynb
drwxrwxr-x 2 mark mark 4.0K Jun 17 16:45 test\_data
-rw-rw-r-- 1 mark mark  33K Jun 17 17:01 17\_Python\_File\_IO.ipynb
-rw-rw-r-- 1 mark mark 4.8K Jun 17 17:35 18\_Python\_Error\_Handling.ipynb

    \end{Verbatim}

    The variable \texttt{output} that we got from the \texttt{communicate}
function has its own internal function called \texttt{decode}, which we
call with the argument \texttt{"utf-8"}, indicating we want to use that
particular Unicode formatting. Note that the resulting string does NOT
have the \texttt{b} at the front, and is printed like any other string.
We can also use this data like any other string, with \texttt{split}
commands and other types of parsing to get the information we want from
the output.

Anyone who has looked into the \texttt{TCFreeze} code will recognize the
\texttt{subprocess} module as an integral part of the larger algorithm.
\texttt{TCFreeze} uses \texttt{subprocess.call()} to use TeraChem at
every iteration of the process.

Try out some of your own commands to get a feel for how the different
interactions work.
