\section{Error Handling}
Error handling refers to a set of instructions the program follows if it
encounters an error. If you've gotten any kind of an error in python,
you've seen error handling in action.

What if you wanted to write code that would just take care of the
problem if it encountered an error, rather than crashing and complaining
to the user? If you can anticipate possible errors that may arise, you
can account for them and work around them.

Error handling in python is based first on the \texttt{try/except}
commands. Let's look at the examples below.

    \begin{tcolorbox}[breakable, size=fbox, boxrule=1pt, pad at break*=1mm,colback=cellbackground, colframe=cellborder]
\prompt{In}{incolor}{8}{\boxspacing}
\begin{Verbatim}[commandchars=\\\{\}]
\PY{k+kn}{import} \PY{n+nn}{foobar} \PY{k}{as} \PY{n+nn}{fb}
\PY{n+nb}{print}\PY{p}{(}\PY{l+s+s2}{\PYZdq{}}\PY{l+s+s2}{Hello world!}\PY{l+s+s2}{\PYZdq{}}\PY{p}{)}
\end{Verbatim}
\end{tcolorbox}

    \begin{Verbatim}[commandchars=\\\{\}, frame=single, framerule=2mm, rulecolor=\color{outerrorbackground}]
\textcolor{ansi-red}{---------------------------------------------------------------------------}
\textcolor{ansi-red}{ModuleNotFoundError}                       Traceback (most recent call last)
\textcolor{ansi-green-intense}{\textbf{/home/mark/GH\_Repositories/CodingSummerSchool/Day05\_Python\_WrapUp/18\_Python\_Error\_Handling.ipynb Cell 2'}} in \textcolor{ansi-cyan}{<cell line: 1>}\textcolor{ansi-blue}{()}
\textcolor{ansi-green}{----> <a href='vscode-notebook-cell:/home/mark/GH\_Repositories/CodingSummerSchool/Day05\_Python\_WrapUp/18\_Python\_Error\_Handling.ipynb\#ch0000000?line=0'>1</a>} import foobar as fb
\textcolor{ansi-green-intense}{\textbf{      <a href='vscode-notebook-cell:/home/mark/GH\_Repositories/CodingSummerSchool/Day05\_Python\_WrapUp/18\_Python\_Error\_Handling.ipynb\#ch0000000?line=2'>3</a>}} print("Hello world!")

\textcolor{ansi-red}{ModuleNotFoundError}: No module named 'foobar'
    \end{Verbatim}

    \begin{tcolorbox}[breakable, size=fbox, boxrule=1pt, pad at break*=1mm,colback=cellbackground, colframe=cellborder]
\prompt{In}{incolor}{4}{\boxspacing}
\begin{Verbatim}[commandchars=\\\{\}]
\PY{k}{try}\PY{p}{:}
    \PY{k+kn}{import} \PY{n+nn}{foobar} \PY{k}{as} \PY{n+nn}{fb}
\PY{k}{except}\PY{p}{:}
    \PY{n+nb}{print}\PY{p}{(}\PY{l+s+s2}{\PYZdq{}}\PY{l+s+s2}{foobar module is not available.}\PY{l+s+s2}{\PYZdq{}}\PY{p}{)}
\PY{n+nb}{print}\PY{p}{(}\PY{l+s+s2}{\PYZdq{}}\PY{l+s+s2}{Hello world!}\PY{l+s+s2}{\PYZdq{}}\PY{p}{)}
\end{Verbatim}
\end{tcolorbox}

    \begin{Verbatim}[commandchars=\\\{\}]
pytraj module is not available.
Hello world!
    \end{Verbatim}

    In the first example, we tried to directly import a python module that
doesn't exist (on the computer or even at all), and then print ``Hello
World!'' But because \texttt{foobar} doesn't exist, python encountered
an error and immediately crashed. In the second example, however, we
included the \texttt{try/except} commands, which allows us to follow a
specific set of instructions when the error is encountered, and then
continue on with the program.

I commonly use this in scripts and programs that I provide to the lab
that use non-standard python libraries. For example, a script may use
the \texttt{mdanalysis} library, which is not included with python by
default.

    \begin{tcolorbox}[breakable, size=fbox, boxrule=1pt, pad at break*=1mm,colback=cellbackground, colframe=cellborder]
\prompt{In}{incolor}{ }{\boxspacing}
\begin{Verbatim}[commandchars=\\\{\}]
\PY{k}{try}\PY{p}{:}
    \PY{k+kn}{import} \PY{n+nn}{mdanalysis} \PY{k}{as} \PY{n+nn}{mda}
\PY{k}{except}\PY{p}{:}
    \PY{k+kn}{import} \PY{n+nn}{subprocess}
    \PY{n}{subprocess}\PY{o}{.}\PY{n}{call}\PY{p}{(}\PY{l+s+s2}{\PYZdq{}}\PY{l+s+s2}{conda install \PYZhy{}y mdanalysis}\PY{l+s+s2}{\PYZdq{}}\PY{p}{,}\PY{n}{shell}\PY{o}{=}\PY{k+kc}{True}\PY{p}{)}
    \PY{k+kn}{import} \PY{n+nn}{mdanalysis} \PY{k}{as} \PY{n+nn}{mda}
\end{Verbatim}
\end{tcolorbox}

    The cell above tries to import \texttt{mdanalysis}, and if it fails, it
loads a standard python library called \texttt{subprocess}, then uses
that to install \texttt{mdanalysis}. Once that has completed, the code
continues and imports mdanalysis as though it were always there.

Error handling can get more complex than this, with specific
instructions being assigned to specific \textbf{types} of errors.

If you wanted to be really mean, you could include a handling of the
``KeyboardInterruptException'' to prevent your code from being killed by
the usual \texttt{ctrl-C} method.

But don't do that.
