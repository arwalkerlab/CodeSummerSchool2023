\section{Unit Tests}
It's important when creating programs to ensure each function works
properly on its own and as part of the larger ecosystem. For this
purpose, we'll use \texttt{pytest}, though there are other options out
there.

You can create a test script with your modules to ensure that each
function you make does its specific task appropriately. Each test
function should be preceded by \texttt{test\_} for pytest to detect and
run them.

Any test scripts should include the same prefix as well. These rules
ensure that \texttt{pytest} discovers the tests it is expected to run
without trying to treat other files and functions as tests.

A common approach to testing is to have a specific folder for tests in
the main module directory, with any necessary test data and scripts held
inside.

The next cell will check to see if you have \texttt{pytest} installed,
and if not, will install it for you. It should only need to run once.

    \begin{tcolorbox}[breakable, size=fbox, boxrule=1pt, pad at break*=1mm,colback=cellbackground, colframe=cellborder]
\prompt{In}{incolor}{5}{\boxspacing}
\begin{Verbatim}[commandchars=\\\{\}]
\PY{k}{try}\PY{p}{:}
    \PY{k+kn}{import} \PY{n+nn}{pytest}
\PY{k}{except}\PY{p}{:}
    \PY{o}{!}pip install \PYZhy{}U pytest
    \PY{k+kn}{import} \PY{n+nn}{pytest}
\end{Verbatim}
\end{tcolorbox}

    \hypertarget{tests-and-the-assert-function}{%
\paragraph{\texorpdfstring{Tests and the \texttt{assert}
function}{Tests and the assert function}}\label{tests-and-the-assert-function}}

Pytest makes use of the built-in \texttt{assert} function in python.
Assert makes a comparison between two values and returns a \texttt{True}
or \texttt{False}, which gets interpreted as a pass/fail in the test
environment.

Consider the following functions in a script called
\texttt{test\_mystuff.py} (the cell below is the same as the contents of
that file.)

    \begin{tcolorbox}[breakable, size=fbox, boxrule=1pt, pad at break*=1mm,colback=cellbackground, colframe=cellborder]
\prompt{In}{incolor}{10}{\boxspacing}
\begin{Verbatim}[commandchars=\\\{\}]
\PY{k}{def} \PY{n+nf}{cool\PYZus{}function}\PY{p}{(}\PY{n}{x}\PY{p}{,}\PY{n}{y}\PY{p}{)}\PY{p}{:}
    \PY{l+s+sd}{\PYZdq{}\PYZdq{}\PYZdq{}We expect this function to return the numerical product of two numbers x and y.\PYZdq{}\PYZdq{}\PYZdq{}}
    \PY{k}{return} \PY{n}{x}\PY{o}{*}\PY{n}{y}

\PY{k}{def} \PY{n+nf}{test\PYZus{}cool\PYZus{}function\PYZus{}pass1}\PY{p}{(}\PY{p}{)}\PY{p}{:}
    \PY{k}{assert} \PY{n}{cool\PYZus{}function}\PY{p}{(}\PY{l+m+mi}{3}\PY{p}{,}\PY{l+m+mi}{4}\PY{p}{)} \PY{o}{==} \PY{l+m+mi}{12}  \PY{c+c1}{\PYZsh{}\PYZsh{} SHOULD PASS}
\PY{k}{def} \PY{n+nf}{test\PYZus{}cool\PYZus{}function\PYZus{}pass2}\PY{p}{(}\PY{p}{)}\PY{p}{:}
    \PY{k}{assert} \PY{n}{cool\PYZus{}function}\PY{p}{(}\PY{l+m+mi}{2}\PY{p}{,}\PY{l+m+mi}{4}\PY{p}{)} \PY{o}{==} \PY{l+m+mi}{8}  \PY{c+c1}{\PYZsh{}\PYZsh{} SHOULD PASS}
\PY{k}{def} \PY{n+nf}{test\PYZus{}cool\PYZus{}function\PYZus{}pass3}\PY{p}{(}\PY{p}{)}\PY{p}{:}
    \PY{k}{assert} \PY{n}{cool\PYZus{}function}\PY{p}{(}\PY{l+m+mi}{4}\PY{p}{,}\PY{l+m+mi}{4}\PY{p}{)} \PY{o}{==} \PY{l+m+mi}{16}  \PY{c+c1}{\PYZsh{}\PYZsh{} SHOULD PASS}
\PY{k}{def} \PY{n+nf}{test\PYZus{}cool\PYZus{}function\PYZus{}fail1}\PY{p}{(}\PY{p}{)}\PY{p}{:}
    \PY{k}{assert} \PY{n}{cool\PYZus{}function}\PY{p}{(}\PY{l+m+mi}{3}\PY{p}{,}\PY{l+m+mi}{4}\PY{p}{)} \PY{o}{==} \PY{l+m+mi}{11}  \PY{c+c1}{\PYZsh{}\PYZsh{} SHOULD FAIL}
\PY{k}{def} \PY{n+nf}{test\PYZus{}cool\PYZus{}function\PYZus{}pass4}\PY{p}{(}\PY{p}{)}\PY{p}{:}
    \PY{k}{assert} \PY{n}{cool\PYZus{}function}\PY{p}{(}\PY{l+m+mi}{5}\PY{p}{,}\PY{l+m+mi}{20}\PY{p}{)} \PY{o}{==} \PY{l+m+mi}{100}  \PY{c+c1}{\PYZsh{}\PYZsh{} SHOULD PASS}
\PY{k}{def} \PY{n+nf}{test\PYZus{}cool\PYZus{}function\PYZus{}pass5}\PY{p}{(}\PY{p}{)}\PY{p}{:}
    \PY{k}{assert} \PY{n}{cool\PYZus{}function}\PY{p}{(}\PY{l+m+mi}{3}\PY{p}{,}\PY{l+m+mi}{5}\PY{p}{)} \PY{o}{==} \PY{l+m+mi}{15}  \PY{c+c1}{\PYZsh{}\PYZsh{} SHOULD PASS}
\end{Verbatim}
\end{tcolorbox}

    With the function we wrote and the test functions, we can run
\texttt{pytest} in the directory and it will find any and all test
files, and any test functions, and check them for us. The following cell
will do this for us and print out the results. Keep in mind, we wrote
one of our tests to intentionally fail - this is to show you what it
will look like so you'll know how to recognize it in the future.

    \begin{tcolorbox}[breakable, size=fbox, boxrule=1pt, pad at break*=1mm,colback=cellbackground, colframe=cellborder]
\prompt{In}{incolor}{12}{\boxspacing}
\begin{Verbatim}[commandchars=\\\{\}]
\PY{c+c1}{\PYZsh{} In Jupyter/VSCode notebook environments, the ! before a normal shell command }
\PY{c+c1}{\PYZsh{} will direct the notebook to run that line in the shell environment rather }
\PY{c+c1}{\PYZsh{} than as a python command.}
\PY{o}{!}pytest 
\end{Verbatim}
\end{tcolorbox}

    \begin{Verbatim}[commandchars=\\\{\}]
\textbf{============================= test session starts
==============================}
platform linux -- Python 3.9.7, pytest-7.1.2, pluggy-0.13.1
rootdir: /home/mark/GH\_Repositories/CodingSummerSchool/Day04\_Python\_Modules
plugins: anyio-2.2.0
collected 6 items


test\_mystuff.py
\textcolor{ansi-green}{.}\textcolor{ansi-green}{.}\textcolor{ansi-green}{.}\textcolor{ansi-red}{F}\textcolor{ansi-green}{.}\textcolor{ansi-green}{.}\textcolor{ansi-red}{
[100\%]}

=================================== FAILURES ===================================
\textcolor{ansi-red-intense}{\textbf{\_\_\_\_\_\_\_\_\_\_\_\_\_\_\_\_\_\_\_\_\_\_\_\_\_\_\_ test\_cool\_function\_fail1
\_\_\_\_\_\_\_\_\_\_\_\_\_\_\_\_\_\_\_\_\_\_\_\_\_\_\_}}

    \textcolor{ansi-blue-intense}{def} \textcolor{ansi-green-intense}{test\_cool\_function\_fail1}():
>       \textcolor{ansi-blue-intense}{assert}
cool\_function(\textcolor{ansi-blue-intense}{3},\textcolor{ansi-blue-intense}{4}) == \textcolor{ansi-blue-intense}{11}
\textcolor{ansi-red-intense}{\textbf{E       assert 12 == 11}}
\textcolor{ansi-red-intense}{\textbf{E        +  where 12 = cool\_function(3, 4)}}

\textcolor{ansi-red-intense}{\textbf{test\_mystuff.py}}:12: AssertionError
=========================== short test summary info ============================
FAILED test\_mystuff.py::test\_cool\_function\_fail1 - assert 12 == 11
\textcolor{ansi-red}{========================= }\textcolor{ansi-red-intense}{\textbf{1 failed}}, \textcolor{ansi-green}{5 passed}\textcolor{ansi-red}{ in
0.04s}\textcolor{ansi-red}{ ==========================}
    \end{Verbatim}

    Notice the first section, which lists the file that was tested
(\texttt{test\_mystuff.py}), and lists the passes and failures in order
with the dot/``F''. It also shows that \texttt{pytest} completed 100\%
of the file. Below, you see the more detailed report on the failures
(nothing is given for the passes by default because\ldots{} they
passed).

The failures are shown so that you can see exactly \emph{where} a
function failed. You can use this method to test complex constructs like
classes as well, by simply having multiple \texttt{assert} commands in
your test functions. Here's another example with a basic class. The
following cell is also in \texttt{test\_myclass.py}

    \begin{tcolorbox}[breakable, size=fbox, boxrule=1pt, pad at break*=1mm,colback=cellbackground, colframe=cellborder]
\prompt{In}{incolor}{14}{\boxspacing}
\begin{Verbatim}[commandchars=\\\{\}]
\PY{k}{class} \PY{n+nc}{Person}\PY{p}{:}
    \PY{k}{def} \PY{n+nf+fm}{\PYZus{}\PYZus{}init\PYZus{}\PYZus{}}\PY{p}{(}\PY{n+nb+bp}{self}\PY{p}{,}\PY{n}{name}\PY{p}{,}\PY{n}{age}\PY{p}{,}\PY{n}{university}\PY{p}{)}\PY{p}{:}
        \PY{n+nb+bp}{self}\PY{o}{.}\PY{n}{name} \PY{o}{=} \PY{n}{name}
        \PY{n+nb+bp}{self}\PY{o}{.}\PY{n}{age} \PY{o}{=} \PY{n}{age}
        \PY{n+nb+bp}{self}\PY{o}{.}\PY{n}{uni} \PY{o}{=} \PY{n}{university}

\PY{k}{def} \PY{n+nf}{test\PYZus{}Person\PYZus{}PASS}\PY{p}{(}\PY{p}{)}\PY{p}{:}
    \PY{n}{test} \PY{o}{=} \PY{n}{Person}\PY{p}{(}\PY{l+s+s2}{\PYZdq{}}\PY{l+s+s2}{Mark}\PY{l+s+s2}{\PYZdq{}}\PY{p}{,}\PY{l+m+mi}{37}\PY{p}{,}\PY{l+s+s2}{\PYZdq{}}\PY{l+s+s2}{Wayne State University}\PY{l+s+s2}{\PYZdq{}}\PY{p}{)}
    \PY{k}{assert} \PY{n}{test}\PY{o}{.}\PY{n}{name} \PY{o}{==} \PY{l+s+s2}{\PYZdq{}}\PY{l+s+s2}{Mark}\PY{l+s+s2}{\PYZdq{}}
    \PY{k}{assert} \PY{n}{test}\PY{o}{.}\PY{n}{age} \PY{o}{==} \PY{l+m+mi}{37}
    \PY{k}{assert} \PY{n}{test}\PY{o}{.}\PY{n}{uni} \PY{o}{==} \PY{l+s+s2}{\PYZdq{}}\PY{l+s+s2}{Wayne State University}\PY{l+s+s2}{\PYZdq{}}

\PY{k}{def} \PY{n+nf}{test\PYZus{}Person\PYZus{}FAIL}\PY{p}{(}\PY{p}{)}\PY{p}{:}
    \PY{n}{test} \PY{o}{=} \PY{n}{Person}\PY{p}{(}\PY{l+s+s2}{\PYZdq{}}\PY{l+s+s2}{Mark}\PY{l+s+s2}{\PYZdq{}}\PY{p}{,}\PY{l+m+mi}{37}\PY{p}{,}\PY{l+s+s2}{\PYZdq{}}\PY{l+s+s2}{Wayne State University}\PY{l+s+s2}{\PYZdq{}}\PY{p}{)}
    \PY{k}{assert} \PY{n}{test}\PY{o}{.}\PY{n}{name} \PY{o}{==} \PY{l+s+s2}{\PYZdq{}}\PY{l+s+s2}{Mark}\PY{l+s+s2}{\PYZdq{}}
    \PY{k}{assert} \PY{n}{test}\PY{o}{.}\PY{n}{age} \PY{o}{==} \PY{l+m+mi}{37}
    \PY{k}{assert} \PY{n}{test}\PY{o}{.}\PY{n}{uni} \PY{o}{==} \PY{l+s+s2}{\PYZdq{}}\PY{l+s+s2}{Wayne State Universe}\PY{l+s+s2}{\PYZdq{}} \PY{c+c1}{\PYZsh{}\PYZsh{}\PYZsh{} Notice how this is not correct.}
\end{Verbatim}
\end{tcolorbox}

    \begin{tcolorbox}[breakable, size=fbox, boxrule=1pt, pad at break*=1mm,colback=cellbackground, colframe=cellborder]
\prompt{In}{incolor}{15}{\boxspacing}
\begin{Verbatim}[commandchars=\\\{\}]
\PY{o}{!}pytest 
\end{Verbatim}
\end{tcolorbox}

    \begin{Verbatim}[commandchars=\\\{\}]
\textbf{============================= test session starts
==============================}
platform linux -- Python 3.9.7, pytest-7.1.2, pluggy-0.13.1
rootdir: /home/mark/GH\_Repositories/CodingSummerSchool/Day04\_Python\_Modules
plugins: anyio-2.2.0
collected 8 items


test\_myclass.py \textcolor{ansi-green}{.}\textcolor{ansi-red}{F}\textcolor{ansi-red}{
[ 25\%]}
test\_mystuff.py
\textcolor{ansi-green}{.}\textcolor{ansi-green}{.}\textcolor{ansi-green}{.}\textcolor{ansi-red}{F}\textcolor{ansi-green}{.}\textcolor{ansi-green}{.}\textcolor{ansi-red}{
[100\%]}

=================================== FAILURES ===================================
\textcolor{ansi-red-intense}{\textbf{\_\_\_\_\_\_\_\_\_\_\_\_\_\_\_\_\_\_\_\_\_\_\_\_\_\_\_\_\_\_\_ test\_Person\_FAIL
\_\_\_\_\_\_\_\_\_\_\_\_\_\_\_\_\_\_\_\_\_\_\_\_\_\_\_\_\_\_\_}}

    \textcolor{ansi-blue-intense}{def} \textcolor{ansi-green-intense}{test\_Person\_FAIL}():
        test = Person(\textcolor{ansi-yellow}{"}\textcolor{ansi-yellow}{Mark}\textcolor{ansi-yellow}{"},[9
4m37,\textcolor{ansi-yellow}{"}\textcolor{ansi-yellow}{Wayne State
University}\textcolor{ansi-yellow}{"})
        \textcolor{ansi-blue-intense}{assert} test.name ==
\textcolor{ansi-yellow}{"}\textcolor{ansi-yellow}{Mark}\textcolor{ansi-yellow}{"}
        \textcolor{ansi-blue-intense}{assert} test.age == \textcolor{ansi-blue-intense}{37}
>       \textcolor{ansi-blue-intense}{assert} test.uni == \textcolor{ansi-yellow}{"}\textcolor{ansi-yellow}{Wayne State
Universe}\textcolor{ansi-yellow}{"} \textcolor{ansi-black-intense}{\#\#\# Notice how this is not
correct.}
\textcolor{ansi-red-intense}{\textbf{E       AssertionError: assert 'Wayne State University' == 'Wayne State
Universe'}}
\textcolor{ansi-red-intense}{\textbf{E         - Wayne State Universe}}
\textcolor{ansi-red-intense}{\textbf{E         ?                    \^{}}}
\textcolor{ansi-red-intense}{\textbf{E         + Wayne State University}}
\textcolor{ansi-red-intense}{\textbf{E         ?                    \^{}\^{}\^{}}}

\textcolor{ansi-red-intense}{\textbf{test\_myclass.py}}:17: AssertionError
\textcolor{ansi-red-intense}{\textbf{\_\_\_\_\_\_\_\_\_\_\_\_\_\_\_\_\_\_\_\_\_\_\_\_\_\_\_ test\_cool\_function\_fail1
\_\_\_\_\_\_\_\_\_\_\_\_\_\_\_\_\_\_\_\_\_\_\_\_\_\_\_}}

    \textcolor{ansi-blue-intense}{def} \textcolor{ansi-green-intense}{test\_cool\_function\_fail1}():
>       \textcolor{ansi-blue-intense}{assert}
cool\_function(\textcolor{ansi-blue-intense}{3},\textcolor{ansi-blue-intense}{4}) == \textcolor{ansi-blue-intense}{11}
\textcolor{ansi-red-intense}{\textbf{E       assert 12 == 11}}
\textcolor{ansi-red-intense}{\textbf{E        +  where 12 = cool\_function(3, 4)}}

\textcolor{ansi-red-intense}{\textbf{test\_mystuff.py}}:12: AssertionError
=========================== short test summary info ============================
FAILED test\_myclass.py::test\_Person\_FAIL - AssertionError: assert 'Wayne Stat{\ldots}
FAILED test\_mystuff.py::test\_cool\_function\_fail1 - assert 12 == 11
\textcolor{ansi-red}{========================= }\textcolor{ansi-red-intense}{\textbf{2 failed}}, \textcolor{ansi-green}{6 passed}\textcolor{ansi-red}{ in
0.04s}\textcolor{ansi-red}{ ==========================}
    \end{Verbatim}

    The \texttt{pytest} run now also gives us a summary of the different
files and their failures in addition to the more thorough breakdown of
each failure. Pytest can be extremely useful in finding and fixing bugs
in your code, especially as the code gets bigger and more complex.

There are many more things you can do with pytest as you become more
advanced in other aspects of python programming, such as testing error
handling and making sure that the code handles user errors properly.

\href{https://docs.pytest.org/_/downloads/en/latest/pdf/}{Pytest
Documentation}

You can also build test \emph{classes} in addition to test
\emph{functions}. This can allow you to use more complex data structures
with multiple test functions as well as ensure that the different
functions of a class interact appropriately and correctly.

In most code-heavy jobs, it is generally expected that you will include
tests with code you write. Thus, it's good practice to write tests
alongside your actual programs and functions. It'll also make it easier
to spot problems early before they become larger problems that are
harder to deal with because there are more functions and classes built
upon them.
