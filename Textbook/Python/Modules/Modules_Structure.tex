Modules are self-contained larger objects in Python that can be imported
and used to make your code easier to manage, easier to read, and easier
to modify. If you've used python at all (including the different scripts
Dr.~Walker has provided at various stages of a project), you've seen and
used modules already. The most easily recognized python modules are
\texttt{numpy} and \texttt{matplotlib}, but there are hundreds more
included by default with python, and thousands more available through
various channels.
\section{Structure of Modules}

\hypertarget{importing-moduleslibraries}{%
\subsubsection{Importing
Modules/Libraries}\label{importing-moduleslibraries}}

There are many ways to import modules into your code. The first example
uses the command \texttt{import} to load in the module \texttt{numpy}

\begin{Shaded}
\begin{Highlighting}[]
\ImportTok{import}\NormalTok{ numpy}
\end{Highlighting}
\end{Shaded}

With \texttt{numpy} loaded like this, we can use any of the numpy
functions and submodules by calling them appropriately. For example, to
use \texttt{numpy}'s \texttt{linspace} function to get 1000 numbers
between 0 and 1, we'd call it like this:

\begin{Shaded}
\begin{Highlighting}[]
\NormalTok{data }\OperatorTok{=}\NormalTok{ numpy.linspace(}\DecValTok{0}\NormalTok{,}\DecValTok{1}\NormalTok{,}\DecValTok{1000}\NormalTok{)}
\end{Highlighting}
\end{Shaded}

It's also common to see imports assigned to shorter variable names.

\begin{Shaded}
\begin{Highlighting}[]
\ImportTok{import}\NormalTok{ numpy }\ImportTok{as}\NormalTok{ np}
\end{Highlighting}
\end{Shaded}

This shortened name is the standard convention, and while in this case
it saves only three letters, these assignments become cleaner and easier
when working in larger and more complex modules like
\texttt{matplotlib}, which has dozens of its own submodules. A common
shortening we see is this:

\begin{Shaded}
\begin{Highlighting}[]
\ImportTok{import}\NormalTok{ matplotlib.pyplot }\ImportTok{as}\NormalTok{ plt}
\end{Highlighting}
\end{Shaded}

If you had to type \texttt{matplotlib.pyplot.plot(data)} every time you
wanted to plot your data, it would get tedious \emph{and} introduce the
possibility of typos and errors in the code.

You may encounter some well-established conventions if you find yourself
searching for various python-related code help on the internet. The
following is a (very not comprehensive) list of some of these
conventions.

\begin{Shaded}
\begin{Highlighting}[]
\ImportTok{import}\NormalTok{ numpy }\ImportTok{as}\NormalTok{ np}
\ImportTok{import}\NormalTok{ matplotlib.pyplot }\ImportTok{as}\NormalTok{ plt}
\ImportTok{import}\NormalTok{ pandas }\ImportTok{as}\NormalTok{ pd}
\ImportTok{import}\NormalTok{ mdanalysis }\ImportTok{as}\NormalTok{ mda}
\ImportTok{import}\NormalTok{ pytraj }\ImportTok{as}\NormalTok{ pt}
\ImportTok{import}\NormalTok{ parmed }\ImportTok{as}\NormalTok{ pmd}
\end{Highlighting}
\end{Shaded}

Much of the import convention can be traced to the original
documentation for the various modules.

Another convention you may run across - especially if you're digging
into any of Mark's code repositories on GitHub - is the use of CAPS to
indicate module bases.

\begin{Shaded}
\begin{Highlighting}[]
\ImportTok{import}\NormalTok{ numpy }\ImportTok{as}\NormalTok{ NP}
\ImportTok{import}\NormalTok{ matplotlib.pyplot }\ImportTok{as}\NormalTok{ PLT}
\ImportTok{import}\NormalTok{ glob }\ImportTok{as}\NormalTok{ G}
\ImportTok{import}\NormalTok{ subprocess }\ImportTok{as}\NormalTok{ S}
\end{Highlighting}
\end{Shaded}

\ldots{} and so forth. The specific naming convention you use is up to
you, just as long as you don't mix up what you've imported. Personally,
I use the caps solely because they stand out in the rest of my code,
making it easier for me to keep track of where I'm pulling
module-specific functionality into my own code.

\hypertarget{importing-from-moduleslibraries}{%
\subsubsection{Importing FROM
Modules/Libraries}\label{importing-from-moduleslibraries}}

Let's say there's a specific function in a much larger library that you
want to use, but you don't need everything else that comes with it. This
is usually because of things like module loading times. If a module is
considerably larger than is reasonable, and it forces your computer to
lag when importing it, you can consider a different approach. In the
example below, there's a module called \texttt{glob}, which has a
function called \texttt{glob()} (super helpful/original, I know). I
don't want all of the module, just the function. So I'll import only the
function like this:

\begin{Shaded}
\begin{Highlighting}[]
\ImportTok{from}\NormalTok{ glob }\ImportTok{import}\NormalTok{ glob }\ImportTok{as}\NormalTok{ g}
\end{Highlighting}
\end{Shaded}

Now I can use the \texttt{glob.glob()} function simply by using
\texttt{g()}, and without having to load all the rest of the module with
it. Another example some of you may have seen is the
\texttt{gaussian\_filter} function from \texttt{scipy.ndimage}, which
gets used to smooth out rough data into cleaner curves.

\begin{Shaded}
\begin{Highlighting}[]
\ImportTok{from}\NormalTok{ scipy.ndimage }\ImportTok{import}\NormalTok{ gaussian\_filter}
\end{Highlighting}
\end{Shaded}

\hypertarget{importing-a-module-without-a-name}{%
\subsubsection{Importing a Module Without a
Name}\label{importing-a-module-without-a-name}}

While this is not something you should generally do, there are occasions
where it might be easier to just import everything from a module without
including the actual name assignment. I'm not a fan of this because it
can have unexpected side effects if there are competing functions of the
same name in different modules, but I feel it's better to make you aware
in the first place.

\begin{Shaded}
\begin{Highlighting}[]
\ImportTok{from}\NormalTok{ subprocess }\ImportTok{import} \OperatorTok{*}
\end{Highlighting}
\end{Shaded}

This imports everything from the \texttt{subprocess} module without
requiring the use of \texttt{subprocess.} before any of the
function/submodule calls. Usually this is seen in code examples on the
internet.

\begin{Shaded}
\begin{Highlighting}[]
\ImportTok{import}\NormalTok{ subprocess}
\NormalTok{proc }\OperatorTok{=}\NormalTok{ subprocess.Popen(}\StringTok{"ls {-}lrth"}\NormalTok{,shell}\OperatorTok{=}\VariableTok{True}\NormalTok{,stdout}\OperatorTok{=}\NormalTok{subprocess.PIPE, stderr}\OperatorTok{=}\NormalTok{subprocess.PIPE)}
\end{Highlighting}
\end{Shaded}

becomes

\begin{Shaded}
\begin{Highlighting}[]
\ImportTok{from}\NormalTok{ subprocess }\ImportTok{import} \OperatorTok{*}
\NormalTok{proc }\OperatorTok{=}\NormalTok{ Popen(}\StringTok{"ls {-}lrth"}\NormalTok{,shell}\OperatorTok{=}\VariableTok{True}\NormalTok{,stdout}\OperatorTok{=}\NormalTok{PIPE, stderr}\OperatorTok{=}\NormalTok{PIPE)}
\end{Highlighting}
\end{Shaded}

Now this may seem shorter and cleaner, but in reality it exposes the
user to potential problems. Consider some other libraries that have a
very commonly-named \texttt{load()} function in them. Importing modules
in this manner means that the \texttt{load()} function will
\emph{always} be the last one loaded by your script. Better to be safe
than sorry and not do this.

\hypertarget{importing-files-as-modules}{%
\subsubsection{Importing Files as
Modules}\label{importing-files-as-modules}}

Sometimes, we write functions or classes that are consistently useful
for us, and rather than have to keep writing them into our code (or risk
accidentally changing it in a way that makes it no longer useful), it
can be very helpful to just load them from another file.

Python has a default \texttt{PATH} that it looks in for modules when
importing them. We can add directories to this path inside a python
script and then import files/modules from that directory.

Let's say we have a file full of functions we wrote called
\texttt{utilities.py}:

\begin{Shaded}
\begin{Highlighting}[]
\ImportTok{import}\NormalTok{ sys}
\NormalTok{sys.path.append(}\StringTok{"path/to/folder/with/file"}\NormalTok{)}
\ImportTok{import}\NormalTok{ utilities}
\end{Highlighting}
\end{Shaded}

Now we can use any of the functions we wrote by calling them like
\texttt{utilities.myfunction()}. We can also use the same import tricks
and shortcuts we've described above, like
\texttt{import\ utilities\ as\ U} or even
\texttt{from\ utilities\ import\ *}.

I'm sure you can see where this is going\ldots{}
