\section{Creating a Module}
As you progress in coding, you may find yourself with a whole pile of
various tools and functions that do a wide array of useful things. It
might become helpful to keep them all in a single location, but
organized for easier access.

Python modules can be made in a fairly straightforward way. Each module
is contained inside a folder, and at a minimum there will be one file in
the folder called \texttt{\_\_init\_\_.py}. This file must be present
for the module to be imported, including any submodules. Any submodules
should be their own folders, each with their own
\texttt{\_\_init\_\_.py} inside, and so forth. This can be arbitrarily
deep, but be careful not to go too far down a well of submodules.

\begin{verbatim}
MyModule/
|- __init__.py
|- Submodule1/
|  |- __init__.py
|  |- blueberry.py
|  |- lemon_meringue.py
|- Submodule2/
   |- __init__.py
   |- cherry.py
   |- raspberry.py
\end{verbatim}

This structure allows multiple import options. Each
\texttt{\_\_init\_\_.py} file needs to include its own import commands
for subdirectories and files as well. The complexity of these files can
be very low, simply importing the subfolders and files as they are
named, or complex, including individual function definitions and
specific actions that occur during import.

The first level file, \texttt{MyModule/\_\_init\_\_.py}, would look
something like this:

\begin{Shaded}
\begin{Highlighting}[]
\ImportTok{import}\NormalTok{ Submodule1}
\ImportTok{import}\NormalTok{ Submodule2}
\end{Highlighting}
\end{Shaded}

And the second level file, \texttt{MyModule/Submodule1/\_\_init\_\_.py}
would look like:

\begin{Shaded}
\begin{Highlighting}[]
\ImportTok{import}\NormalTok{ blueberry }\ImportTok{as}\NormalTok{ bb}
\ImportTok{import}\NormalTok{ lemon\_meringue }\ImportTok{as}\NormalTok{ lm}
\end{Highlighting}
\end{Shaded}

which would lead to a user's script looking like this:

\begin{Shaded}
\begin{Highlighting}[]
\ImportTok{import}\NormalTok{ sys}
\NormalTok{sys.path.append(}\StringTok{"path/to/MyModule/location/"}\NormalTok{)}
\ImportTok{import}\NormalTok{ MyModule }\ImportTok{as}\NormalTok{ mm}
\NormalTok{mm.Submodule1.bb.function()}
\NormalTok{mm.Submodule1.lm.other\_function()}
\end{Highlighting}
\end{Shaded}

This is a fairly one-directional import structure. Each thing can really
only import from below themselves in the module's structural heirarchy.
However, what if a function in \texttt{cherry.py} needs to use
\texttt{other\_function()} from \texttt{lemon\_meringue.py}? That's up
one level, then over, and then back down a level. How do we get that
\texttt{other\_function()} into \texttt{cherry.py}? Or what if
\texttt{cherry.py} needed a function in \texttt{raspberry.py}?

In the \texttt{cherry.py} file, we can include this line:

\begin{Shaded}
\begin{Highlighting}[]
\ImportTok{from}\NormalTok{ ..Submodule1.lemon\_meringue }\ImportTok{import}\NormalTok{ other\_function}
\ImportTok{from}\NormalTok{ .raspberry }\ImportTok{import}\NormalTok{ rasp\_function}
\end{Highlighting}
\end{Shaded}

The dots before the modules are the key here. One dot before
\texttt{raspberry} indicates that the file we're importing from is in
the same directory as the current file. Two dots indicates we're up one
level. Three dots would be up two levels, etc. We can go up levels and
then do the normal imports of modules and submodules, including subfiles
as well. This is particularly helpful when writing larger modules with
lots of different tasks but use several of the same functions. It's not
uncommon to have a particular set of functions that are used all over a
module kept in a separate folder to prevent confusion, while also
allowing them to be accessed by other submodules.
