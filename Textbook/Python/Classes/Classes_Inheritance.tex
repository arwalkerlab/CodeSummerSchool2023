\section{Inheritance}
Inheritance refers to building a class that is based on a framework of
another class. This can be done when you have need of classes for
different subcategories of object that all fall under the same larger
category. This can be thought of in terms of different people having
different jobs. You can have a class, \texttt{Person} that has all the
usual information about a person such as name and age, and then have a
class called \texttt{Employee} which inherits from \texttt{Person}. All
\texttt{Employee}s are also a \texttt{Person}, but all \texttt{Person}s
are not \texttt{Employee}s.

Let's look at it from a scientific perspective.

    \begin{tcolorbox}[breakable, size=fbox, boxrule=1pt, pad at break*=1mm,colback=cellbackground, colframe=cellborder]
\prompt{In}{incolor}{2}{\boxspacing}
\begin{Verbatim}[commandchars=\\\{\}]
\PY{k+kn}{import} \PY{n+nn}{numpy} \PY{k}{as} \PY{n+nn}{np}

\PY{k}{class} \PY{n+nc}{Point}\PY{p}{:}
    \PY{k}{def} \PY{n+nf+fm}{\PYZus{}\PYZus{}init\PYZus{}\PYZus{}}\PY{p}{(}\PY{n+nb+bp}{self}\PY{p}{,}\PY{n}{x}\PY{p}{,}\PY{n}{y}\PY{p}{,}\PY{n}{z}\PY{p}{)}\PY{p}{:}
        \PY{n+nb+bp}{self}\PY{o}{.}\PY{n}{x} \PY{o}{=} \PY{n}{x}
        \PY{n+nb+bp}{self}\PY{o}{.}\PY{n}{y} \PY{o}{=} \PY{n}{y}
        \PY{n+nb+bp}{self}\PY{o}{.}\PY{n}{z} \PY{o}{=} \PY{n}{z}
    \PY{k}{def} \PY{n+nf}{distance\PYZus{}from}\PY{p}{(}\PY{n+nb+bp}{self}\PY{p}{,}\PY{n}{point}\PY{p}{)}\PY{p}{:}
        \PY{k}{return} \PY{n}{np}\PY{o}{.}\PY{n}{sqrt}\PY{p}{(} \PY{p}{(}\PY{n+nb+bp}{self}\PY{o}{.}\PY{n}{x}\PY{o}{\PYZhy{}}\PY{n}{point}\PY{o}{.}\PY{n}{x}\PY{p}{)}\PY{o}{*}\PY{o}{*}\PY{l+m+mi}{2} \PY{o}{+} \PY{p}{(}\PY{n+nb+bp}{self}\PY{o}{.}\PY{n}{y}\PY{o}{\PYZhy{}}\PY{n}{point}\PY{o}{.}\PY{n}{y}\PY{p}{)}\PY{o}{*}\PY{o}{*}\PY{l+m+mi}{2} \PY{o}{+} \PY{p}{(}\PY{n+nb+bp}{self}\PY{o}{.}\PY{n}{z}\PY{o}{\PYZhy{}}\PY{n}{point}\PY{o}{.}\PY{n}{z}\PY{p}{)}\PY{o}{*}\PY{o}{*}\PY{l+m+mi}{2} \PY{p}{)}

\PY{c+c1}{\PYZsh{}\PYZsh{} Now we have a class that defines a point in cartesian space (x,y,z) and has a function that can be used to calculate the distance from another point.}
\PY{c+c1}{\PYZsh{}\PYZsh{} What if we wanted a point\PYZhy{}charge class but didn\PYZsq{}t want to have to rewrite the entire class function for it?}
\PY{k}{class} \PY{n+nc}{PointCharge}\PY{p}{(}\PY{n}{Point}\PY{p}{)}\PY{p}{:}
    \PY{k}{def} \PY{n+nf+fm}{\PYZus{}\PYZus{}init\PYZus{}\PYZus{}}\PY{p}{(}\PY{n+nb+bp}{self}\PY{p}{,}\PY{n}{x}\PY{p}{,}\PY{n}{y}\PY{p}{,}\PY{n}{z}\PY{p}{,}\PY{n}{charge}\PY{p}{)}\PY{p}{:}
        \PY{n+nb}{super}\PY{p}{(}\PY{p}{)}\PY{o}{.}\PY{n+nf+fm}{\PYZus{}\PYZus{}init\PYZus{}\PYZus{}}\PY{p}{(}\PY{n}{x}\PY{p}{,}\PY{n}{y}\PY{p}{,}\PY{n}{z}\PY{p}{)}
        \PY{n+nb+bp}{self}\PY{o}{.}\PY{n}{charge} \PY{o}{=} \PY{n}{charge}
    \PY{k}{def} \PY{n+nf}{force\PYZus{}between}\PY{p}{(}\PY{n+nb+bp}{self}\PY{p}{,}\PY{n}{pointcharge}\PY{p}{)}\PY{p}{:}
        \PY{n}{dist} \PY{o}{=} \PY{n+nb+bp}{self}\PY{o}{.}\PY{n}{distance\PYZus{}from}\PY{p}{(}\PY{n}{pointcharge}\PY{p}{)}
        \PY{n}{k} \PY{o}{=} \PY{l+m+mf}{8.988e9}
        \PY{n}{force} \PY{o}{=} \PY{n}{k} \PY{o}{*} \PY{p}{(}\PY{n+nb+bp}{self}\PY{o}{.}\PY{n}{charge} \PY{o}{*} \PY{n}{pointcharge}\PY{o}{.}\PY{n}{charge}\PY{p}{)} \PY{o}{/} \PY{p}{(}\PY{n}{dist}\PY{o}{*}\PY{o}{*}\PY{l+m+mi}{2}\PY{p}{)}
        \PY{k}{return} \PY{n}{force}

\PY{c+c1}{\PYZsh{}\PYZsh{} With the PointCharge class, we included the type of class it\PYZsq{}s inheriting from, Point.  }
\PY{c+c1}{\PYZsh{}\PYZsh{} PointCharge now has all the same variables and functions as Point, without having to redefine them.}

\PY{n}{p1} \PY{o}{=} \PY{n}{Point}\PY{p}{(}\PY{l+m+mi}{0}\PY{p}{,}\PY{l+m+mi}{1}\PY{p}{,}\PY{l+m+mi}{2}\PY{p}{)}
\PY{n}{pc1} \PY{o}{=} \PY{n}{PointCharge}\PY{p}{(}\PY{l+m+mi}{1}\PY{p}{,}\PY{l+m+mi}{2}\PY{p}{,}\PY{l+m+mi}{3}\PY{p}{,}\PY{o}{\PYZhy{}}\PY{l+m+mi}{1}\PY{p}{)}
\PY{n}{pc2} \PY{o}{=} \PY{n}{PointCharge}\PY{p}{(}\PY{l+m+mi}{0}\PY{p}{,}\PY{l+m+mi}{0}\PY{p}{,}\PY{l+m+mi}{0}\PY{p}{,}\PY{o}{+}\PY{l+m+mi}{1}\PY{p}{)}

\PY{n+nb}{print}\PY{p}{(}\PY{n}{pc2}\PY{o}{.}\PY{n}{force\PYZus{}between}\PY{p}{(}\PY{n}{pc1}\PY{p}{)}\PY{p}{)}
\end{Verbatim}
\end{tcolorbox}

    \begin{Verbatim}[commandchars=\\\{\}]
-642000000.0
    \end{Verbatim}

    Take note of how the first function of the \texttt{PointCharge} class,
\texttt{\_\_init\_\_()} is written. The first command is
\texttt{super().\_\_init\_\_()}, which refers to the inherited class'
initialization function. Effectively, it says ``do all the stuff the
previous class normally does first, then do these additional steps''.

We can take inheritance further. Any class can be inherited, and
everything in that class comes with it, including its own inheritances.

Consider an atom. We can (roughly!) describe an atom as a point charge
that also has mass. Of course they're more complicated than that, but
we're taking it a step at a time.

    \begin{tcolorbox}[breakable, size=fbox, boxrule=1pt, pad at break*=1mm,colback=cellbackground, colframe=cellborder]
\prompt{In}{incolor}{3}{\boxspacing}
\begin{Verbatim}[commandchars=\\\{\}]
\PY{k}{class} \PY{n+nc}{Atom}\PY{p}{(}\PY{n}{PointCharge}\PY{p}{)}\PY{p}{:}
    \PY{k}{def} \PY{n+nf+fm}{\PYZus{}\PYZus{}init\PYZus{}\PYZus{}}\PY{p}{(}\PY{n+nb+bp}{self}\PY{p}{,}\PY{n}{x}\PY{p}{,}\PY{n}{y}\PY{p}{,}\PY{n}{z}\PY{p}{,}\PY{n}{charge}\PY{p}{,}\PY{n}{mass}\PY{p}{)}\PY{p}{:}
        \PY{n+nb}{super}\PY{p}{(}\PY{p}{)}\PY{o}{.}\PY{n+nf+fm}{\PYZus{}\PYZus{}init\PYZus{}\PYZus{}}\PY{p}{(}\PY{n}{x}\PY{p}{,}\PY{n}{y}\PY{p}{,}\PY{n}{z}\PY{p}{,}\PY{n}{charge}\PY{p}{)}
        \PY{n+nb+bp}{self}\PY{o}{.}\PY{n}{mass} \PY{o}{=} \PY{n}{mass}

    \PY{k}{def} \PY{n+nf}{get\PYZus{}accel\PYZus{}between}\PY{p}{(}\PY{n+nb+bp}{self}\PY{p}{,}\PY{n}{atom}\PY{p}{)}\PY{p}{:}
        \PY{n}{force} \PY{o}{=} \PY{n+nb+bp}{self}\PY{o}{.}\PY{n}{force\PYZus{}between}\PY{p}{(}\PY{n}{atom}\PY{p}{)}
        \PY{n}{accel} \PY{o}{=} \PY{n}{force}\PY{o}{/}\PY{n+nb+bp}{self}\PY{o}{.}\PY{n}{mass}
        \PY{k}{return} \PY{n}{accel}
\end{Verbatim}
\end{tcolorbox}

    Now we have a class that holds all the previous stages' parameters and
has access to all the previous stages' functions. With this, we can
build larger and larger systems and include more and more data, without
having to completely rebuild each class from the ground up.

You may notice that each of the classes we built end up requiring more
and more variables at their creation. But what if you wanted to have
some classes that have standard values? Let's say we wanted to make an
Iron(II) atom? Every iron(II) atom can be represented with the same mass
(we're skipping isotopic effects for now and going with periodic table
values) and same charge.

    \begin{tcolorbox}[breakable, size=fbox, boxrule=1pt, pad at break*=1mm,colback=cellbackground, colframe=cellborder]
\prompt{In}{incolor}{4}{\boxspacing}
\begin{Verbatim}[commandchars=\\\{\}]
\PY{k}{class} \PY{n+nc}{Iron}\PY{p}{(}\PY{n}{Atom}\PY{p}{)}\PY{p}{:}
    \PY{k}{def} \PY{n+nf+fm}{\PYZus{}\PYZus{}init\PYZus{}\PYZus{}}\PY{p}{(}\PY{n+nb+bp}{self}\PY{p}{,}\PY{n}{x}\PY{p}{,}\PY{n}{y}\PY{p}{,}\PY{n}{z}\PY{p}{)}\PY{p}{:}
        \PY{n+nb}{super}\PY{p}{(}\PY{p}{)}\PY{o}{.}\PY{n+nf+fm}{\PYZus{}\PYZus{}init\PYZus{}\PYZus{}}\PY{p}{(}\PY{n}{x}\PY{p}{,}\PY{n}{y}\PY{p}{,}\PY{n}{z}\PY{p}{,}\PY{l+m+mi}{2}\PY{p}{,}\PY{l+m+mf}{55.845}\PY{p}{)}

\PY{n}{iron\PYZus{}atom} \PY{o}{=} \PY{n}{Iron}\PY{p}{(}\PY{l+m+mi}{0}\PY{p}{,}\PY{l+m+mi}{0}\PY{p}{,}\PY{l+m+mi}{0}\PY{p}{)}
\end{Verbatim}
\end{tcolorbox}

    Now we can create inherited classes that are somewhat more specific and
also faster to initialize. In the above case, if you know an atom is
going to be an Iron(II), you don't have to give the call anything for
\texttt{mass} or \texttt{charge}, because those values are already
known.

Try making some other classes for anything you want. You can also mix in
things from previous lessons like decorators to make your classes unique
and highly customized.
