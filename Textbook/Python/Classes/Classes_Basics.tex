
Classes are one of the most powerful, versatile, and useful tools in
modern programming. A \emph{class} is a construct that can hold internal
data and has its own internal functions that can be called. Classes can
also use external functions, other classes, and can range from the
extremely simple to the incredibly complex.

Classes are defined in specific way, with some required functions built
into all of them.

\section{Basic Classes}

Check out the example class below. Note the two functions that are
already included. Of these two, \texttt{\_\_init\_\_} is
\textbf{required} for any and all classes, otherwise the class won't
actually be initialized when called, and the data you're trying to hold
won't be kept.

    \begin{tcolorbox}[breakable, size=fbox, boxrule=1pt, pad at break*=1mm,colback=cellbackground, colframe=cellborder]
\prompt{In}{incolor}{5}{\boxspacing}
\begin{Verbatim}[commandchars=\\\{\}]
\PY{k}{class} \PY{n+nc}{MyClass}\PY{p}{:}
    \PY{k}{def} \PY{n+nf+fm}{\PYZus{}\PYZus{}init\PYZus{}\PYZus{}}\PY{p}{(}\PY{n+nb+bp}{self}\PY{p}{,}\PY{n}{name}\PY{p}{,}\PY{n}{age}\PY{p}{)}\PY{p}{:}
        \PY{n+nb+bp}{self}\PY{o}{.}\PY{n}{name} \PY{o}{=} \PY{n}{name}
        \PY{n+nb+bp}{self}\PY{o}{.}\PY{n}{age}  \PY{o}{=} \PY{n}{age}
    \PY{k}{def} \PY{n+nf+fm}{\PYZus{}\PYZus{}str\PYZus{}\PYZus{}}\PY{p}{(}\PY{n+nb+bp}{self}\PY{p}{)}\PY{p}{:}
        \PY{k}{return} \PY{l+s+sa}{f}\PY{l+s+s2}{\PYZdq{}}\PY{l+s+si}{\PYZob{}}\PY{n+nb+bp}{self}\PY{o}{.}\PY{n}{name}\PY{l+s+si}{\PYZcb{}}\PY{l+s+s2}{ is }\PY{l+s+si}{\PYZob{}}\PY{n+nb+bp}{self}\PY{o}{.}\PY{n}{age}\PY{l+s+si}{\PYZcb{}}\PY{l+s+s2}{ years old.}\PY{l+s+s2}{\PYZdq{}}
\end{Verbatim}
\end{tcolorbox}

    \texttt{MyClass} is an example of a very basic class structure.

\begin{itemize}
\tightlist
\item
  The references to \texttt{self} are important. It is necessary when
  referencing variables held by the class, especially in internal
  functions. You may notice that the \texttt{\_\_str\_\_} function
  doesn't take any arguments except for \texttt{self}, yet it is able to
  use the \texttt{age} and \texttt{name} variables we stored during the
  \texttt{\_\_init\_\_} function call.
\item
  The first class function defined is \texttt{\_\_init\_\_}, which is
  understood by Python to be the \emph{constructor}. This function is
  called the very first time a new instance of \texttt{MyClass} is
  created.
\item
  The next class function is \texttt{\_\_str\_\_}, which we can use to
  define what is displayed if the user calls the class object as a
  string, such as in a print statement. This function is optional, and
  only changes the default behavior of python when converting a class
  object into a string. See the examples below.
\end{itemize}

    \begin{tcolorbox}[breakable, size=fbox, boxrule=1pt, pad at break*=1mm,colback=cellbackground, colframe=cellborder]
\prompt{In}{incolor}{7}{\boxspacing}
\begin{Verbatim}[commandchars=\\\{\}]
\PY{k}{class} \PY{n+nc}{MyStringlessClass}\PY{p}{:}
    \PY{k}{def} \PY{n+nf+fm}{\PYZus{}\PYZus{}init\PYZus{}\PYZus{}}\PY{p}{(}\PY{n+nb+bp}{self}\PY{p}{,}\PY{n}{name}\PY{p}{,}\PY{n}{age}\PY{p}{)}\PY{p}{:}
        \PY{n+nb+bp}{self}\PY{o}{.}\PY{n}{name} \PY{o}{=} \PY{n}{name}
        \PY{n+nb+bp}{self}\PY{o}{.}\PY{n}{age}  \PY{o}{=} \PY{n}{age}

\PY{n}{stringless} \PY{o}{=} \PY{n}{MyStringlessClass}\PY{p}{(}\PY{l+s+s2}{\PYZdq{}}\PY{l+s+s2}{Mark}\PY{l+s+s2}{\PYZdq{}}\PY{p}{,}\PY{l+m+mi}{37}\PY{p}{)}
\PY{n+nb}{print}\PY{p}{(}\PY{l+s+s2}{\PYZdq{}}\PY{l+s+s2}{MyStringlessClass}\PY{l+s+s2}{\PYZdq{}}\PY{p}{)}
\PY{n+nb}{print}\PY{p}{(}\PY{n}{stringless}\PY{p}{)}
\PY{n+nb}{print}\PY{p}{(}\PY{l+s+s2}{\PYZdq{}}\PY{l+s+s2}{\PYZdq{}}\PY{p}{)}

\PY{n}{not\PYZus{}stringless} \PY{o}{=} \PY{n}{MyClass}\PY{p}{(}\PY{l+s+s2}{\PYZdq{}}\PY{l+s+s2}{Mark}\PY{l+s+s2}{\PYZdq{}}\PY{p}{,}\PY{l+m+mi}{37}\PY{p}{)}
\PY{n+nb}{print}\PY{p}{(}\PY{l+s+s2}{\PYZdq{}}\PY{l+s+s2}{MyClass}\PY{l+s+s2}{\PYZdq{}}\PY{p}{)}
\PY{n+nb}{print}\PY{p}{(}\PY{n}{not\PYZus{}stringless}\PY{p}{)}
\PY{n+nb}{print}\PY{p}{(}\PY{l+s+s2}{\PYZdq{}}\PY{l+s+s2}{\PYZdq{}}\PY{p}{)}
\end{Verbatim}
\end{tcolorbox}

    \begin{Verbatim}[commandchars=\\\{\}]
MyStringlessClass
<\_\_main\_\_.MyStringlessClass object at 0x7f00a8471ee0>

MyClass
Mark is 37 years old.

    \end{Verbatim}

    We can see the differences between the two classes in how they are
displayed through a \texttt{print()} function. We can also create our
own additional functions that can be called after the class is initially
created.

We'll continue with an example class to hold information about a single
atom in a system. I'll be importing \texttt{numpy} as well to make some
of the internal functions a little easier to work with.

    \begin{tcolorbox}[breakable, size=fbox, boxrule=1pt, pad at break*=1mm,colback=cellbackground, colframe=cellborder]
\prompt{In}{incolor}{13}{\boxspacing}
\begin{Verbatim}[commandchars=\\\{\}]
\PY{k+kn}{import} \PY{n+nn}{numpy} \PY{k}{as} \PY{n+nn}{np}

\PY{k}{class} \PY{n+nc}{Atom}\PY{p}{:}
    \PY{k}{def} \PY{n+nf+fm}{\PYZus{}\PYZus{}init\PYZus{}\PYZus{}}\PY{p}{(}\PY{n+nb+bp}{self}\PY{p}{,}\PY{n}{x}\PY{p}{,}\PY{n}{y}\PY{p}{,}\PY{n}{z}\PY{p}{,}\PY{n}{atomic\PYZus{}number}\PY{p}{,}\PY{n}{charge}\PY{p}{)}\PY{p}{:}
        \PY{n+nb+bp}{self}\PY{o}{.}\PY{n}{x} \PY{o}{=} \PY{n}{x}
        \PY{n+nb+bp}{self}\PY{o}{.}\PY{n}{y} \PY{o}{=} \PY{n}{y}
        \PY{n+nb+bp}{self}\PY{o}{.}\PY{n}{z} \PY{o}{=} \PY{n}{z}
        \PY{n+nb+bp}{self}\PY{o}{.}\PY{n}{number} \PY{o}{=} \PY{n}{atomic\PYZus{}number}
        \PY{n+nb+bp}{self}\PY{o}{.}\PY{n}{charge} \PY{o}{=} \PY{n}{charge}
    \PY{k}{def} \PY{n+nf}{distance\PYZus{}from\PYZus{}center}\PY{p}{(}\PY{n+nb+bp}{self}\PY{p}{)}\PY{p}{:}
        \PY{k}{return} \PY{n}{np}\PY{o}{.}\PY{n}{sqrt}\PY{p}{(}\PY{n+nb+bp}{self}\PY{o}{.}\PY{n}{x}\PY{o}{*}\PY{o}{*}\PY{l+m+mi}{2} \PY{o}{+} \PY{n+nb+bp}{self}\PY{o}{.}\PY{n}{y}\PY{o}{*}\PY{o}{*}\PY{l+m+mi}{2} \PY{o}{+} \PY{n+nb+bp}{self}\PY{o}{.}\PY{n}{z}\PY{o}{*}\PY{o}{*}\PY{l+m+mi}{2}\PY{p}{)}
    \PY{k}{def} \PY{n+nf}{update\PYZus{}location}\PY{p}{(}\PY{n+nb+bp}{self}\PY{p}{,}\PY{n}{x}\PY{p}{,}\PY{n}{y}\PY{p}{,}\PY{n}{z}\PY{p}{)}\PY{p}{:}
        \PY{n+nb+bp}{self}\PY{o}{.}\PY{n}{x} \PY{o}{=} \PY{n}{x}
        \PY{n+nb+bp}{self}\PY{o}{.}\PY{n}{y} \PY{o}{=} \PY{n}{y}
        \PY{n+nb+bp}{self}\PY{o}{.}\PY{n}{z} \PY{o}{=} \PY{n}{z}
    \PY{k}{def} \PY{n+nf}{get\PYZus{}location}\PY{p}{(}\PY{n+nb+bp}{self}\PY{p}{)}\PY{p}{:}
        \PY{k}{return} \PY{l+s+sa}{f}\PY{l+s+s2}{\PYZdq{}}\PY{l+s+s2}{(}\PY{l+s+si}{\PYZob{}}\PY{n+nb+bp}{self}\PY{o}{.}\PY{n}{x}\PY{l+s+si}{\PYZcb{}}\PY{l+s+s2}{,}\PY{l+s+si}{\PYZob{}}\PY{n+nb+bp}{self}\PY{o}{.}\PY{n}{y}\PY{l+s+si}{\PYZcb{}}\PY{l+s+s2}{,}\PY{l+s+si}{\PYZob{}}\PY{n+nb+bp}{self}\PY{o}{.}\PY{n}{z}\PY{l+s+si}{\PYZcb{}}\PY{l+s+s2}{)}\PY{l+s+s2}{\PYZdq{}}
    \PY{k}{def} \PY{n+nf}{str\PYZus{}charge}\PY{p}{(}\PY{n+nb+bp}{self}\PY{p}{)}\PY{p}{:}
        \PY{k}{if} \PY{n+nb+bp}{self}\PY{o}{.}\PY{n}{charge} \PY{o}{\PYZgt{}} \PY{l+m+mi}{0}\PY{p}{:}
            \PY{k}{return} \PY{l+s+sa}{f}\PY{l+s+s2}{\PYZdq{}}\PY{l+s+s2}{+}\PY{l+s+si}{\PYZob{}}\PY{n+nb+bp}{self}\PY{o}{.}\PY{n}{charge}\PY{l+s+si}{\PYZcb{}}\PY{l+s+s2}{\PYZdq{}}
        \PY{k}{return} \PY{l+s+sa}{f}\PY{l+s+s2}{\PYZdq{}}\PY{l+s+si}{\PYZob{}}\PY{n+nb+bp}{self}\PY{o}{.}\PY{n}{charge}\PY{l+s+si}{\PYZcb{}}\PY{l+s+s2}{\PYZdq{}}
    \PY{k}{def} \PY{n+nf+fm}{\PYZus{}\PYZus{}str\PYZus{}\PYZus{}}\PY{p}{(}\PY{n+nb+bp}{self}\PY{p}{)}\PY{p}{:}
        \PY{k}{return} \PY{l+s+sa}{f}\PY{l+s+s2}{\PYZdq{}}\PY{l+s+s2}{Atom located at }\PY{l+s+si}{\PYZob{}}\PY{n+nb+bp}{self}\PY{o}{.}\PY{n}{get\PYZus{}location}\PY{p}{(}\PY{p}{)}\PY{l+s+si}{\PYZcb{}}\PY{l+s+s2}{ with charge of }\PY{l+s+si}{\PYZob{}}\PY{n+nb+bp}{self}\PY{o}{.}\PY{n}{str\PYZus{}charge}\PY{p}{(}\PY{p}{)}\PY{l+s+si}{\PYZcb{}}\PY{l+s+s2}{\PYZdq{}}
        
\end{Verbatim}
\end{tcolorbox}

    \begin{tcolorbox}[breakable, size=fbox, boxrule=1pt, pad at break*=1mm,colback=cellbackground, colframe=cellborder]
\prompt{In}{incolor}{21}{\boxspacing}
\begin{Verbatim}[commandchars=\\\{\}]
\PY{n}{chloride} \PY{o}{=} \PY{n}{Atom}\PY{p}{(}\PY{l+m+mf}{3.5}\PY{p}{,}\PY{l+m+mf}{4.2}\PY{p}{,}\PY{l+m+mf}{5.9}\PY{p}{,}\PY{l+m+mi}{17}\PY{p}{,}\PY{o}{\PYZhy{}}\PY{l+m+mi}{1}\PY{p}{)}
\PY{n+nb}{print}\PY{p}{(}\PY{n}{chloride}\PY{p}{)}

\PY{n}{lithium} \PY{o}{=} \PY{n}{Atom}\PY{p}{(}\PY{l+m+mf}{1.0}\PY{p}{,}\PY{l+m+mf}{1.0}\PY{p}{,}\PY{l+m+mf}{1.0}\PY{p}{,}\PY{l+m+mi}{3}\PY{p}{,}\PY{l+m+mi}{1}\PY{p}{)}
\PY{n+nb}{print}\PY{p}{(}\PY{n}{lithium}\PY{p}{)}
\end{Verbatim}
\end{tcolorbox}

    \begin{Verbatim}[commandchars=\\\{\}]
Atom located at (3.5,4.2,5.9) with charge of -1.
Atom located at (1.0,1.0,1.0) with charge of +1.
    \end{Verbatim}

    \begin{tcolorbox}[breakable, size=fbox, boxrule=1pt, pad at break*=1mm,colback=cellbackground, colframe=cellborder]
\prompt{In}{incolor}{22}{\boxspacing}
\begin{Verbatim}[commandchars=\\\{\}]
\PY{n+nb}{print}\PY{p}{(}\PY{n}{chloride}\PY{p}{)}
\PY{n+nb}{print}\PY{p}{(}\PY{n}{chloride}\PY{o}{.}\PY{n}{distance\PYZus{}from\PYZus{}center}\PY{p}{(}\PY{p}{)}\PY{p}{)}
\PY{n}{chloride}\PY{o}{.}\PY{n}{update\PYZus{}location}\PY{p}{(}\PY{l+m+mi}{3}\PY{p}{,}\PY{l+m+mi}{3}\PY{p}{,}\PY{l+m+mi}{3}\PY{p}{)}
\PY{n+nb}{print}\PY{p}{(}\PY{n}{chloride}\PY{p}{)}
\PY{n+nb}{print}\PY{p}{(}\PY{n}{chloride}\PY{o}{.}\PY{n}{distance\PYZus{}from\PYZus{}center}\PY{p}{(}\PY{p}{)}\PY{p}{)}
\end{Verbatim}
\end{tcolorbox}

    \begin{Verbatim}[commandchars=\\\{\}]
Atom located at (3.5,4.2,5.9) with charge of -1.
8.043631020876083
Atom located at (3,3,3) with charge of -1.
5.196152422706632
    \end{Verbatim}

    \begin{tcolorbox}[breakable, size=fbox, boxrule=1pt, pad at break*=1mm,colback=cellbackground, colframe=cellborder]
\prompt{In}{incolor}{18}{\boxspacing}
\begin{Verbatim}[commandchars=\\\{\}]
\PY{n}{lithium}\PY{o}{.}\PY{n}{distance\PYZus{}from\PYZus{}center}\PY{p}{(}\PY{p}{)}
\end{Verbatim}
\end{tcolorbox}

            \begin{tcolorbox}[breakable, size=fbox, boxrule=.5pt, pad at break*=1mm, opacityfill=0]
\prompt{Out}{outcolor}{18}{\boxspacing}
\begin{Verbatim}[commandchars=\\\{\}]
1.7320508075688772
\end{Verbatim}
\end{tcolorbox}
        
    As mentioned above, classes can also use other classes inside
themselves, Let's try making a class for a Molecule that uses Atoms.

First, we want to think about what additional information we need for a
molecule that isn't already included in the Atoms.

The first thing that comes to mind is bonds.

We can create a class that includes a list of atoms and bonds.

    \begin{tcolorbox}[breakable, size=fbox, boxrule=1pt, pad at break*=1mm,colback=cellbackground, colframe=cellborder]
\prompt{In}{incolor}{34}{\boxspacing}
\begin{Verbatim}[commandchars=\\\{\}]
\PY{k}{class} \PY{n+nc}{Molecule}\PY{p}{:}
    \PY{k}{def} \PY{n+nf+fm}{\PYZus{}\PYZus{}init\PYZus{}\PYZus{}}\PY{p}{(}\PY{n+nb+bp}{self}\PY{p}{)}\PY{p}{:}
        \PY{n+nb+bp}{self}\PY{o}{.}\PY{n}{atoms} \PY{o}{=} \PY{p}{[}\PY{p}{]}
        \PY{n+nb+bp}{self}\PY{o}{.}\PY{n}{bonds} \PY{o}{=} \PY{p}{[}\PY{p}{]}
    \PY{k}{def} \PY{n+nf+fm}{\PYZus{}\PYZus{}str\PYZus{}\PYZus{}}\PY{p}{(}\PY{n+nb+bp}{self}\PY{p}{)}\PY{p}{:}
        \PY{n}{string}  \PY{o}{=} \PY{l+s+s2}{\PYZdq{}}\PY{l+s+s2}{Molecule made of }\PY{l+s+s2}{\PYZdq{}}
        \PY{n}{string} \PY{o}{+}\PY{o}{=} \PY{l+s+s2}{\PYZdq{}}\PY{l+s+s2}{,}\PY{l+s+s2}{\PYZdq{}}\PY{o}{.}\PY{n}{join}\PY{p}{(}\PY{p}{[}\PY{n+nb}{str}\PY{p}{(}\PY{n}{atom}\PY{p}{)} \PY{k}{for} \PY{n}{atom} \PY{o+ow}{in} \PY{n+nb+bp}{self}\PY{o}{.}\PY{n}{atoms}\PY{p}{]}\PY{p}{)}
        \PY{k}{return} \PY{n}{string}
    \PY{k}{def} \PY{n+nf}{add\PYZus{}atom}\PY{p}{(}\PY{n+nb+bp}{self}\PY{p}{,}\PY{n}{x}\PY{p}{,}\PY{n}{y}\PY{p}{,}\PY{n}{z}\PY{p}{,}\PY{n}{number}\PY{p}{,}\PY{n}{charge}\PY{p}{)}\PY{p}{:}
        \PY{n+nb+bp}{self}\PY{o}{.}\PY{n}{atoms}\PY{o}{.}\PY{n}{append}\PY{p}{(}\PY{n}{Atom}\PY{p}{(}\PY{n}{x}\PY{p}{,}\PY{n}{y}\PY{p}{,}\PY{n}{z}\PY{p}{,}\PY{n}{number}\PY{p}{,}\PY{n}{charge}\PY{p}{)}\PY{p}{)}
    \PY{k}{def} \PY{n+nf}{add\PYZus{}bond}\PY{p}{(}\PY{n+nb+bp}{self}\PY{p}{,}\PY{n}{index1}\PY{p}{,}\PY{n}{index2}\PY{p}{)}\PY{p}{:}
        \PY{n+nb+bp}{self}\PY{o}{.}\PY{n}{bonds}\PY{o}{.}\PY{n}{append}\PY{p}{(}\PY{p}{[}\PY{n}{index1}\PY{p}{,}\PY{n}{index2}\PY{p}{]}\PY{p}{)}
    \PY{k}{def} \PY{n+nf}{get\PYZus{}bond\PYZus{}distance}\PY{p}{(}\PY{n+nb+bp}{self}\PY{p}{,}\PY{n}{index}\PY{p}{)}\PY{p}{:}
        \PY{n}{idx1}\PY{p}{,}\PY{n}{idx2} \PY{o}{=} \PY{n+nb+bp}{self}\PY{o}{.}\PY{n}{bonds}\PY{p}{[}\PY{n}{index}\PY{p}{]}
        \PY{n}{x1} \PY{o}{=} \PY{n+nb+bp}{self}\PY{o}{.}\PY{n}{atoms}\PY{p}{[}\PY{n}{idx1}\PY{p}{]}\PY{o}{.}\PY{n}{x}
        \PY{n}{y1} \PY{o}{=} \PY{n+nb+bp}{self}\PY{o}{.}\PY{n}{atoms}\PY{p}{[}\PY{n}{idx1}\PY{p}{]}\PY{o}{.}\PY{n}{y}
        \PY{n}{z1} \PY{o}{=} \PY{n+nb+bp}{self}\PY{o}{.}\PY{n}{atoms}\PY{p}{[}\PY{n}{idx1}\PY{p}{]}\PY{o}{.}\PY{n}{z}
        \PY{n}{x2} \PY{o}{=} \PY{n+nb+bp}{self}\PY{o}{.}\PY{n}{atoms}\PY{p}{[}\PY{n}{idx2}\PY{p}{]}\PY{o}{.}\PY{n}{x}
        \PY{n}{y2} \PY{o}{=} \PY{n+nb+bp}{self}\PY{o}{.}\PY{n}{atoms}\PY{p}{[}\PY{n}{idx2}\PY{p}{]}\PY{o}{.}\PY{n}{y}
        \PY{n}{z2} \PY{o}{=} \PY{n+nb+bp}{self}\PY{o}{.}\PY{n}{atoms}\PY{p}{[}\PY{n}{idx2}\PY{p}{]}\PY{o}{.}\PY{n}{z}
        \PY{n}{dist} \PY{o}{=} \PY{n}{np}\PY{o}{.}\PY{n}{sqrt}\PY{p}{(}\PY{p}{(}\PY{n}{x2}\PY{o}{\PYZhy{}}\PY{n}{x1}\PY{p}{)}\PY{o}{*}\PY{o}{*}\PY{l+m+mi}{2} \PY{o}{+} \PY{p}{(}\PY{n}{y2}\PY{o}{\PYZhy{}}\PY{n}{y1}\PY{p}{)}\PY{o}{*}\PY{o}{*}\PY{l+m+mi}{2} \PY{o}{+} \PY{p}{(}\PY{n}{z2}\PY{o}{\PYZhy{}}\PY{n}{z1}\PY{p}{)}\PY{o}{*}\PY{o}{*}\PY{l+m+mi}{2}\PY{p}{)}
        \PY{k}{return} \PY{n}{dist}
    \PY{k}{def} \PY{n+nf}{get\PYZus{}total\PYZus{}charge}\PY{p}{(}\PY{n+nb+bp}{self}\PY{p}{)}\PY{p}{:}
        \PY{n}{charge} \PY{o}{=} \PY{l+m+mi}{0}
        \PY{k}{for} \PY{n}{atom} \PY{o+ow}{in} \PY{n+nb+bp}{self}\PY{o}{.}\PY{n}{atoms}\PY{p}{:}
            \PY{n}{charge} \PY{o}{+}\PY{o}{=} \PY{n}{atom}\PY{o}{.}\PY{n}{charge}
        \PY{k}{return} \PY{n}{charge}
\end{Verbatim}
\end{tcolorbox}

    \begin{tcolorbox}[breakable, size=fbox, boxrule=1pt, pad at break*=1mm,colback=cellbackground, colframe=cellborder]
\prompt{In}{incolor}{35}{\boxspacing}
\begin{Verbatim}[commandchars=\\\{\}]
\PY{n}{sulfurhexafluoride} \PY{o}{=} \PY{n}{Molecule}\PY{p}{(}\PY{p}{)}
\end{Verbatim}
\end{tcolorbox}

    \begin{tcolorbox}[breakable, size=fbox, boxrule=1pt, pad at break*=1mm,colback=cellbackground, colframe=cellborder]
\prompt{In}{incolor}{36}{\boxspacing}
\begin{Verbatim}[commandchars=\\\{\}]
\PY{n}{sulfurhexafluoride}\PY{o}{.}\PY{n}{add\PYZus{}atom}\PY{p}{(}\PY{l+m+mi}{0}\PY{p}{,}\PY{l+m+mi}{0}\PY{p}{,}\PY{l+m+mi}{0}\PY{p}{,}\PY{l+m+mi}{16}\PY{p}{,}\PY{o}{+}\PY{l+m+mi}{6}\PY{p}{)}
\PY{n}{sulfurhexafluoride}\PY{o}{.}\PY{n}{add\PYZus{}atom}\PY{p}{(}\PY{l+m+mi}{1}\PY{p}{,}\PY{l+m+mi}{0}\PY{p}{,}\PY{l+m+mi}{0}\PY{p}{,}\PY{l+m+mi}{9}\PY{p}{,}\PY{o}{\PYZhy{}}\PY{l+m+mi}{1}\PY{p}{)}
\PY{n}{sulfurhexafluoride}\PY{o}{.}\PY{n}{add\PYZus{}atom}\PY{p}{(}\PY{o}{\PYZhy{}}\PY{l+m+mi}{1}\PY{p}{,}\PY{l+m+mi}{0}\PY{p}{,}\PY{l+m+mi}{0}\PY{p}{,}\PY{l+m+mi}{9}\PY{p}{,}\PY{o}{\PYZhy{}}\PY{l+m+mi}{1}\PY{p}{)}
\PY{n}{sulfurhexafluoride}\PY{o}{.}\PY{n}{add\PYZus{}atom}\PY{p}{(}\PY{l+m+mi}{0}\PY{p}{,}\PY{l+m+mi}{1}\PY{p}{,}\PY{l+m+mi}{0}\PY{p}{,}\PY{l+m+mi}{9}\PY{p}{,}\PY{o}{\PYZhy{}}\PY{l+m+mi}{1}\PY{p}{)}
\PY{n}{sulfurhexafluoride}\PY{o}{.}\PY{n}{add\PYZus{}atom}\PY{p}{(}\PY{l+m+mi}{0}\PY{p}{,}\PY{o}{\PYZhy{}}\PY{l+m+mi}{1}\PY{p}{,}\PY{l+m+mi}{0}\PY{p}{,}\PY{l+m+mi}{9}\PY{p}{,}\PY{o}{\PYZhy{}}\PY{l+m+mi}{1}\PY{p}{)}
\PY{n}{sulfurhexafluoride}\PY{o}{.}\PY{n}{add\PYZus{}atom}\PY{p}{(}\PY{l+m+mi}{0}\PY{p}{,}\PY{l+m+mi}{0}\PY{p}{,}\PY{l+m+mi}{1}\PY{p}{,}\PY{l+m+mi}{9}\PY{p}{,}\PY{o}{\PYZhy{}}\PY{l+m+mi}{1}\PY{p}{)}
\PY{n}{sulfurhexafluoride}\PY{o}{.}\PY{n}{add\PYZus{}atom}\PY{p}{(}\PY{l+m+mi}{0}\PY{p}{,}\PY{l+m+mi}{0}\PY{p}{,}\PY{o}{\PYZhy{}}\PY{l+m+mi}{1}\PY{p}{,}\PY{l+m+mi}{9}\PY{p}{,}\PY{o}{\PYZhy{}}\PY{l+m+mi}{1}\PY{p}{)}
\end{Verbatim}
\end{tcolorbox}

    \begin{tcolorbox}[breakable, size=fbox, boxrule=1pt, pad at break*=1mm,colback=cellbackground, colframe=cellborder]
\prompt{In}{incolor}{38}{\boxspacing}
\begin{Verbatim}[commandchars=\\\{\}]
\PY{n+nb}{print}\PY{p}{(}\PY{n}{sulfurhexafluoride}\PY{o}{.}\PY{n}{get\PYZus{}total\PYZus{}charge}\PY{p}{(}\PY{p}{)}\PY{p}{)}
\PY{n+nb}{print}\PY{p}{(}\PY{n}{sulfurhexafluoride}\PY{p}{)}
\end{Verbatim}
\end{tcolorbox}

    \begin{Verbatim}[commandchars=\\\{\}]
0
Molecule made of Atom located at (0,0,0) with charge of +6.,Atom located at
(1,0,0) with charge of -1.,Atom located at (-1,0,0) with charge of -1.,Atom
located at (0,1,0) with charge of -1.,Atom located at (0,-1,0) with charge of
-1.,Atom located at (0,0,1) with charge of -1.,Atom located at (0,0,-1) with
charge of -1.
    \end{Verbatim}

    As you can see, you can build classes of increasing complexity to serve
as more manageable data structures for whatever your specific needs may
be.
