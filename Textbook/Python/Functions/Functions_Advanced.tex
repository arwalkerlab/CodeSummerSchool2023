\section{Advanced Functions - Decorators and Wrappers}
You can nest functions inside other functions. In this way, you can
contain these nested functions to ensure they're only accessible inside
the outer functions. This can be useful if you want to ensure that any
data being used is processed in a specific way beforehand, depending on
the way you call the function.

    \begin{tcolorbox}[breakable, size=fbox, boxrule=1pt, pad at break*=1mm,colback=cellbackground, colframe=cellborder]
\prompt{In}{incolor}{7}{\boxspacing}
\begin{Verbatim}[commandchars=\\\{\}]
\PY{k}{def} \PY{n+nf}{outer\PYZus{}function}\PY{p}{(}\PY{n}{x}\PY{p}{)}\PY{p}{:}
    \PY{k}{def} \PY{n+nf}{inner\PYZus{}function1}\PY{p}{(}\PY{n}{x}\PY{p}{)}\PY{p}{:}
        \PY{n+nb}{print}\PY{p}{(}\PY{n}{x}\PY{p}{,}\PY{l+s+s2}{\PYZdq{}}\PY{l+s+s2}{was given.}\PY{l+s+s2}{\PYZdq{}}\PY{p}{)}
    \PY{k}{def} \PY{n+nf}{inner\PYZus{}function2}\PY{p}{(}\PY{n}{x}\PY{p}{)}\PY{p}{:}
        \PY{n+nb}{print}\PY{p}{(}\PY{n}{x}\PY{o}{*}\PY{o}{*}\PY{l+m+mi}{2}\PY{p}{,}\PY{l+s+s2}{\PYZdq{}}\PY{l+s+s2}{is x squared}\PY{l+s+s2}{\PYZdq{}}\PY{p}{)}
    \PY{n}{inner\PYZus{}function1}\PY{p}{(}\PY{n}{x}\PY{p}{)}
    \PY{n}{inner\PYZus{}function2}\PY{p}{(}\PY{n}{x}\PY{p}{)}

\PY{n}{outer\PYZus{}function}\PY{p}{(}\PY{l+m+mi}{4}\PY{p}{)}
\end{Verbatim}
\end{tcolorbox}

    \begin{Verbatim}[commandchars=\\\{\}]
4 was given.
16 is x squared
    \end{Verbatim}

    As you can see, \texttt{outer\_function} calls its own
\texttt{inner\_function}s without issue. Let's try calling the
\texttt{inner\_function}s directly.

    \begin{tcolorbox}[breakable, size=fbox, boxrule=1pt, pad at break*=1mm,colback=cellbackground, colframe=cellborder]
\prompt{In}{incolor}{8}{\boxspacing}
\begin{Verbatim}[commandchars=\\\{\}]
\PY{n}{inner\PYZus{}function1}\PY{p}{(}\PY{l+m+mi}{4}\PY{p}{)}
\end{Verbatim}
\end{tcolorbox}

    \begin{Verbatim}[commandchars=\\\{\}, frame=single, framerule=2mm, rulecolor=\color{outerrorbackground}]
\textcolor{ansi-red}{---------------------------------------------------------------------------}
\textcolor{ansi-red}{NameError}                                 Traceback (most recent call last)
\textcolor{ansi-green}{/tmp/ipykernel\_269786/3089592963.py} in \textcolor{ansi-cyan}{<module>}
\textcolor{ansi-green}{----> 1}\textcolor{ansi-red}{ }inner\_function1\textcolor{ansi-blue}{(}\textcolor{ansi-cyan}{4}\textcolor{ansi-blue}{)}

\textcolor{ansi-red}{NameError}: name 'inner\_function1' is not defined
    \end{Verbatim}

    Now we get an error. This is because the inner functions only exist in
the scope of the outer function. These small examples illustrate the
concept in a basic way, but let's consider a more complicated function.

    \begin{tcolorbox}[breakable, size=fbox, boxrule=1pt, pad at break*=1mm,colback=cellbackground, colframe=cellborder]
\prompt{In}{incolor}{29}{\boxspacing}
\begin{Verbatim}[commandchars=\\\{\}]
\PY{k+kn}{import} \PY{n+nn}{matplotlib}\PY{n+nn}{.}\PY{n+nn}{pyplot} \PY{k}{as} \PY{n+nn}{plt}
\PY{k+kn}{from} \PY{n+nn}{scipy}\PY{n+nn}{.}\PY{n+nn}{ndimage} \PY{k+kn}{import} \PY{n}{gaussian\PYZus{}filter}
\PY{k+kn}{import} \PY{n+nn}{numpy} \PY{k}{as} \PY{n+nn}{np}
\PY{k+kn}{import} \PY{n+nn}{matplotlib}\PY{n+nn}{.}\PY{n+nn}{cm} \PY{k}{as} \PY{n+nn}{cm}

\PY{k}{def} \PY{n+nf}{plot\PYZus{}with\PYZus{}smoothing}\PY{p}{(}\PY{n}{data}\PY{p}{,}\PY{n}{sigma}\PY{p}{)}\PY{p}{:}
    \PY{l+s+sd}{\PYZdq{}\PYZdq{}\PYZdq{}Takes data and a smoothing factor sigma and plots the raw data plus the smoothed data\PYZdq{}\PYZdq{}\PYZdq{}}
    \PY{n}{cmap} \PY{o}{=} \PY{n}{cm}\PY{o}{.}\PY{n}{get\PYZus{}cmap}\PY{p}{(}\PY{l+s+s2}{\PYZdq{}}\PY{l+s+s2}{Reds\PYZus{}r}\PY{l+s+s2}{\PYZdq{}}\PY{p}{)}
    \PY{n}{fig} \PY{o}{=} \PY{n}{plt}\PY{o}{.}\PY{n}{figure}\PY{p}{(}\PY{n}{figsize}\PY{o}{=}\PY{p}{[}\PY{l+m+mi}{3}\PY{p}{,}\PY{l+m+mi}{2}\PY{p}{]}\PY{p}{,}\PY{n}{dpi}\PY{o}{=}\PY{l+m+mi}{300}\PY{p}{)}
    \PY{n}{ax} \PY{o}{=} \PY{n}{fig}\PY{o}{.}\PY{n}{add\PYZus{}subplot}\PY{p}{(}\PY{l+m+mi}{1}\PY{p}{,}\PY{l+m+mi}{1}\PY{p}{,}\PY{l+m+mi}{1}\PY{p}{)}
    \PY{k}{def} \PY{n+nf}{smoothing}\PY{p}{(}\PY{n}{data}\PY{p}{,}\PY{n}{sigma}\PY{p}{,}\PY{n}{color}\PY{p}{)}\PY{p}{:}
        \PY{l+s+sd}{\PYZdq{}\PYZdq{}\PYZdq{}Plots the smoothed data\PYZdq{}\PYZdq{}\PYZdq{}}
        \PY{n}{ax}\PY{o}{.}\PY{n}{plot}\PY{p}{(}\PY{n}{gaussian\PYZus{}filter}\PY{p}{(}\PY{n}{data}\PY{p}{,}\PY{n}{sigma}\PY{o}{=}\PY{n}{sigma}\PY{p}{)}\PY{p}{,}\PY{n}{color}\PY{o}{=}\PY{n}{color}\PY{p}{,}\PY{n}{lw}\PY{o}{=}\PY{l+m+mf}{0.5}\PY{p}{)}
    \PY{c+c1}{\PYZsh{} ax.plot(data,color=\PYZdq{}grey\PYZdq{},lw=0.5)}
    \PY{k}{for} \PY{n}{i} \PY{o+ow}{in} \PY{n+nb}{range}\PY{p}{(}\PY{n}{sigma}\PY{p}{)}\PY{p}{:}
        \PY{n}{smoothing}\PY{p}{(}\PY{n}{data}\PY{p}{,}\PY{n}{i}\PY{p}{,}\PY{n}{cmap}\PY{p}{(}\PY{n}{i}\PY{o}{/}\PY{n}{sigma}\PY{p}{)}\PY{p}{)}

\PY{n}{data} \PY{o}{=} \PY{n}{np}\PY{o}{.}\PY{n}{random}\PY{o}{.}\PY{n}{rand}\PY{p}{(}\PY{l+m+mi}{10000}\PY{p}{)}
\PY{n}{plot\PYZus{}with\PYZus{}smoothing}\PY{p}{(}\PY{n}{data}\PY{p}{,}\PY{l+m+mi}{100}\PY{p}{)}
\end{Verbatim}
\end{tcolorbox}

    \begin{center}
    \adjustimage{max size={0.9\linewidth}{0.9\paperheight}}{output_5_0.png}
    \end{center}
    { \hspace*{\fill} \\}
    
    Internally defined functions can also allow you to use the same function
name in many different contexts. It can be helpful when you have
functions that are doing some small task a large number of times, but in
a different way depending on the context. In the above example,
\texttt{smoothing} works on 1-dimensional data. There are other
occasions where you may have 2-dimensional data, which requires
different techniques to smooth properly. Therefore, you might want to
ensure that the smoothing functions for 1-d data and 2-d data are kept
isolated to prevent misuse.

Internal functions can also be called as results of if-else statements
or other conditional checks.

    \hypertarget{advanced-functions---decorators}{%
\subsubsection{Advanced Functions -
Decorators}\label{advanced-functions---decorators}}

Decorators are another version of nested functions, but in the opposite
direction. Let's say you want to get timings for individual functions during program execution so you can see where the program spends the most time.  With decorators, you can simple wrap your other functions in another function that grabs the start and end times for that function and prints out how long it took to run.

    \begin{tcolorbox}[breakable, size=fbox, boxrule=1pt, pad at break*=1mm,colback=cellbackground, colframe=cellborder]
\prompt{In}{incolor}{46}{\boxspacing}
\begin{Verbatim}[commandchars=\\\{\}]
\PY{c+c1}{\PYZsh{}\PYZsh{}\PYZsh{} The function \PYZsq{}wrapper\PYZsq{} takes the function as an argument, and the nested function \PYZsq{}wrapped\PYZsq{} uses the incoming arguments in the external function.}
\PY{c+c1}{\PYZsh{}\PYZsh{}\PYZsh{} This allows you to pass arguments through a decorator.}
\PY{c+c1}{\PYZsh{}\PYZsh{}\PYZsh{} The decorator does its own stuff around the called function.}
\PY{k}{def} \PY{n+nf}{decorator\PYZus{}function}\PY{p}{(}\PY{n}{func}\PY{p}{)}\PY{p}{:}
    \PY{k}{def} \PY{n+nf}{wrapper\PYZus{}function}\PY{p}{(}\PY{o}{*}\PY{n}{args}\PY{p}{,}\PY{o}{*}\PY{o}{*}\PY{n}{kwargs}\PY{p}{)}\PY{p}{:}
        \PY{n+nb}{print}\PY{p}{(}\PY{l+s+s2}{\PYZdq{}}\PY{l+s+s2}{Before the function!}\PY{l+s+s2}{\PYZdq{}}\PY{p}{)}
        \PY{n}{result} \PY{o}{=} \PY{n}{func}\PY{p}{(}\PY{o}{*}\PY{n}{args}\PY{p}{,}\PY{o}{*}\PY{o}{*}\PY{n}{kwargs}\PY{p}{)}
        \PY{n+nb}{print}\PY{p}{(}\PY{l+s+s2}{\PYZdq{}}\PY{l+s+s2}{After the function!}\PY{l+s+s2}{\PYZdq{}}\PY{p}{)}
        \PY{k}{return} \PY{n}{result}
    \PY{k}{return} \PY{n}{wrapper\PYZus{}function}


\PY{n+nd}{@decorator\PYZus{}function} \PY{c+c1}{\PYZsh{}\PYZsh{}\PYZsh{} Here\PYZsq{}s where we \PYZdq{}wrap\PYZdq{} the new function rather than having to nest the entire thing inside the decorator.}
\PY{k}{def} \PY{n+nf}{f}\PY{p}{(}\PY{n}{x}\PY{p}{)}\PY{p}{:}
    \PY{n+nb}{print}\PY{p}{(}\PY{l+s+sa}{f}\PY{l+s+s2}{\PYZdq{}}\PY{l+s+s2}{The value of x is }\PY{l+s+si}{\PYZob{}}\PY{n}{x}\PY{l+s+si}{\PYZcb{}}\PY{l+s+s2}{.  Returning x**2}\PY{l+s+s2}{\PYZdq{}}\PY{p}{)}
    \PY{k}{return} \PY{n}{x}\PY{o}{*}\PY{o}{*}\PY{l+m+mi}{2}

\PY{n}{answer} \PY{o}{=} \PY{n}{f}\PY{p}{(}\PY{l+m+mi}{5}\PY{p}{)}
\PY{n+nb}{print}\PY{p}{(}\PY{n}{answer}\PY{p}{)}
\end{Verbatim}
\end{tcolorbox}

    \begin{Verbatim}[commandchars=\\\{\}]
Before the function!
The value of x is 5.  Returning x**2
After the function!
25
    \end{Verbatim}

    Decorators are extremely useful tools, especially with larger projects.
You can also have multiple decorators on a single function, which can
further expand your code!

    \begin{tcolorbox}[breakable, size=fbox, boxrule=1pt, pad at break*=1mm,colback=cellbackground, colframe=cellborder]
\prompt{In}{incolor}{50}{\boxspacing}
\begin{Verbatim}[commandchars=\\\{\}]
\PY{k+kn}{from} \PY{n+nn}{datetime} \PY{k+kn}{import} \PY{n}{datetime} \PY{k}{as} \PY{n}{dt}

\PY{k}{def} \PY{n+nf}{timing\PYZus{}wrapper}\PY{p}{(}\PY{n}{func}\PY{p}{)}\PY{p}{:}
    \PY{k}{def} \PY{n+nf}{wrapped}\PY{p}{(}\PY{o}{*}\PY{n}{args}\PY{p}{,}\PY{o}{*}\PY{o}{*}\PY{n}{kwargs}\PY{p}{)}\PY{p}{:}
        \PY{n}{start} \PY{o}{=} \PY{n}{dt}\PY{o}{.}\PY{n}{now}\PY{p}{(}\PY{p}{)}
        \PY{n+nb}{print}\PY{p}{(}\PY{l+s+s2}{\PYZdq{}}\PY{l+s+s2}{Start time is }\PY{l+s+s2}{\PYZdq{}}\PY{p}{,}\PY{n}{start}\PY{p}{)}
        \PY{n}{result} \PY{o}{=} \PY{n}{func}\PY{p}{(}\PY{o}{*}\PY{n}{args}\PY{p}{,}\PY{o}{*}\PY{o}{*}\PY{n}{kwargs}\PY{p}{)}
        \PY{n}{end} \PY{o}{=} \PY{n}{dt}\PY{o}{.}\PY{n}{now}\PY{p}{(}\PY{p}{)}
        \PY{n+nb}{print}\PY{p}{(}\PY{l+s+s2}{\PYZdq{}}\PY{l+s+s2}{End time is }\PY{l+s+s2}{\PYZdq{}}\PY{p}{,}\PY{n}{end}\PY{p}{)}
        \PY{n+nb}{print}\PY{p}{(}\PY{l+s+s2}{\PYZdq{}}\PY{l+s+s2}{Total Time:}\PY{l+s+s2}{\PYZdq{}}\PY{p}{,}\PY{n}{end} \PY{o}{\PYZhy{}} \PY{n}{start}\PY{p}{)}
        \PY{k}{return} \PY{n}{result}
    \PY{k}{return} \PY{n}{wrapped}

\PY{n+nd}{@timing\PYZus{}wrapper}
\PY{n+nd}{@decorator\PYZus{}function}
\PY{k}{def} \PY{n+nf}{g}\PY{p}{(}\PY{n}{x}\PY{p}{)}\PY{p}{:}
    \PY{n+nb}{print}\PY{p}{(}\PY{l+s+s2}{\PYZdq{}}\PY{l+s+s2}{X is }\PY{l+s+s2}{\PYZdq{}}\PY{p}{,}\PY{n}{x}\PY{p}{,}\PY{l+s+s2}{\PYZdq{}}\PY{l+s+se}{\PYZbs{}n}\PY{l+s+s2}{Returning \PYZhy{}X.}\PY{l+s+s2}{\PYZdq{}}\PY{p}{)}
    \PY{k}{return} \PY{o}{\PYZhy{}}\PY{n}{x}

\PY{n}{g}\PY{p}{(}\PY{l+m+mi}{6}\PY{p}{)}
\end{Verbatim}
\end{tcolorbox}

    \begin{Verbatim}[commandchars=\\\{\}]
Start time is  2022-06-14 16:55:36.759226
Before the function!
X is  6
Returning -X.
After the function!
End time is  2022-06-14 16:55:36.759419
Total Time: 0:00:00.000193
    \end{Verbatim}

            \begin{tcolorbox}[breakable, size=fbox, boxrule=.5pt, pad at break*=1mm, opacityfill=0]
\prompt{Out}{outcolor}{50}{\boxspacing}
\begin{Verbatim}[commandchars=\\\{\}]
-6
\end{Verbatim}
\end{tcolorbox}
        
    This shows us that the order of our decorators matters quite a bit. The
first decorator in the list is the outermost decorator, then the next
one, and so forth.

You can also include additional functionality inside decorators that can
make things easier in your code.

For example, let's say I have a function that opens a file and reads the
first line. I might want to be sure the file exists before I try to open
it. Now, this is something I can do in every function that has a file
being opened and read, which means more code being written and more to
sift through if something goes wrong. However, if I am consistent in my
variable naming conventions (most people develop a style of their own -
unless you work for a company with their own established style
requirements), I can do things like check for a file's existence before
passing the filename on to the reading function. Let's see how that
might look.

I'll use the \texttt{glob} library that comes standard with
\texttt{python}.

    \begin{tcolorbox}[breakable, size=fbox, boxrule=1pt, pad at break*=1mm,colback=cellbackground, colframe=cellborder]
\prompt{In}{incolor}{57}{\boxspacing}
\begin{Verbatim}[commandchars=\\\{\}]
\PY{k+kn}{from} \PY{n+nn}{glob} \PY{k+kn}{import} \PY{n}{glob}

\PY{k}{def} \PY{n+nf}{check\PYZus{}file\PYZus{}exists}\PY{p}{(}\PY{n}{func}\PY{p}{)}\PY{p}{:}
    \PY{k}{def} \PY{n+nf}{wrapper}\PY{p}{(}\PY{o}{*}\PY{n}{args}\PY{p}{,}\PY{o}{*}\PY{o}{*}\PY{n}{kwargs}\PY{p}{)}\PY{p}{:}
        \PY{k}{if} \PY{n}{glob}\PY{p}{(}\PY{n}{kwargs}\PY{p}{[}\PY{l+s+s2}{\PYZdq{}}\PY{l+s+s2}{filename}\PY{l+s+s2}{\PYZdq{}}\PY{p}{]}\PY{p}{)}\PY{p}{:}
            \PY{k}{return} \PY{n}{func}\PY{p}{(}\PY{o}{*}\PY{n}{args}\PY{p}{,}\PY{o}{*}\PY{o}{*}\PY{n}{kwargs}\PY{p}{)}
        \PY{k}{else}\PY{p}{:}
            \PY{n+nb}{print}\PY{p}{(}\PY{n}{args}\PY{p}{)}
            \PY{n+nb}{print}\PY{p}{(}\PY{n}{kwargs}\PY{p}{)}
            \PY{n+nb}{print}\PY{p}{(}\PY{l+s+s2}{\PYZdq{}}\PY{l+s+s2}{File does not exist}\PY{l+s+s2}{\PYZdq{}}\PY{p}{)}
    \PY{k}{return} \PY{n}{wrapper}

\PY{n+nd}{@check\PYZus{}file\PYZus{}exists}
\PY{k}{def} \PY{n+nf}{read\PYZus{}first\PYZus{}line}\PY{p}{(}\PY{n}{filename}\PY{p}{)}\PY{p}{:}
    \PY{k}{with} \PY{n+nb}{open}\PY{p}{(}\PY{n}{filename}\PY{p}{)} \PY{k}{as} \PY{n}{f}\PY{p}{:}
        \PY{n+nb}{print}\PY{p}{(}\PY{n}{f}\PY{o}{.}\PY{n}{readline}\PY{p}{(}\PY{p}{)}\PY{p}{)}
    \PY{n}{f}\PY{o}{.}\PY{n}{close}\PY{p}{(}\PY{p}{)}

\PY{n}{read\PYZus{}first\PYZus{}line}\PY{p}{(}\PY{n}{filename}\PY{o}{=}\PY{l+s+s2}{\PYZdq{}}\PY{l+s+s2}{test.txt}\PY{l+s+s2}{\PYZdq{}}\PY{p}{)}
\PY{n}{read\PYZus{}first\PYZus{}line}\PY{p}{(}\PY{n}{filename}\PY{o}{=}\PY{l+s+s2}{\PYZdq{}}\PY{l+s+s2}{09\PYZus{}Functions\PYZus{}Advanced.ipynb}\PY{l+s+s2}{\PYZdq{}}\PY{p}{)} \PY{c+c1}{\PYZsh{}\PYZsh{} This is this very notebook, so it should return an actual result... which is just a \PYZdq{}\PYZob{}\PYZdq{}.}
\end{Verbatim}
\end{tcolorbox}

    \begin{Verbatim}[commandchars=\\\{\}]
()
\{'filename': 'test.txt'\}
File does not exist
\{

    \end{Verbatim}

    It should be noted that even though the \texttt{read\_first\_line}
function uses the positional argument ``filename'', if the function is
called without the explicit keyword argument
\texttt{filename="test.txt"}, there will be an error in the wrapper. So
take the knowledge that you can mess with parameters inside a wrapper
with a grain of salt - you can also make a big ol' mess.
\newpage
    \begin{tcolorbox}[breakable, size=fbox, boxrule=1pt, pad at break*=1mm,colback=cellbackground, colframe=cellborder]
\prompt{In}{incolor}{ }{\boxspacing}
\begin{Verbatim}[commandchars=\\\{\}]
\PY{c+c1}{\PYZsh{}\PYZsh{}\PYZsh{} For future use, here\PYZsq{}s a basic template!}
\PY{k}{def} \PY{n+nf}{decorator\PYZus{}name}\PY{p}{(}\PY{n}{func}\PY{p}{)}\PY{p}{:}
    \PY{k}{def} \PY{n+nf}{wrapper}\PY{p}{(}\PY{o}{*}\PY{n}{args}\PY{p}{,}\PY{o}{*}\PY{o}{*}\PY{n}{kwargs}\PY{p}{)}\PY{p}{:}
        \PY{k}{return} \PY{n}{func}\PY{p}{(}\PY{o}{*}\PY{n}{args}\PY{p}{,}\PY{o}{*}\PY{o}{*}\PY{n}{kwargs}\PY{p}{)}
    \PY{k}{return} \PY{n}{wrapper}
\end{Verbatim}
\end{tcolorbox}
