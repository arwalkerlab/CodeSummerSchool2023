\section{Intermediate Functions - args and kwargs}
What if you wanted to give a function an unknown number of arguments?
Say, for example, you wanted to give it some random number of
\texttt{float} values and have it multiply the reciprocals of each
together? The mathematical formula would look like this:

\(\prod\limits_{i=1}\frac{1}{x_i}\)

The Python function can use \texttt{*args} as an argument. The
\texttt{*} indicates the value is going to be some arbitrary length list
of values that should be collected by the function.

    \begin{tcolorbox}[breakable, size=fbox, boxrule=1pt, pad at break*=1mm,colback=cellbackground, colframe=cellborder]
\prompt{In}{incolor}{29}{\boxspacing}
\begin{Verbatim}[commandchars=\\\{\}]
\PY{k}{def} \PY{n+nf}{product\PYZus{}of\PYZus{}reciprocals}\PY{p}{(}\PY{o}{*}\PY{n}{args}\PY{p}{)}\PY{p}{:}
    \PY{c+c1}{\PYZsh{} Start with the product set to 1 (since anything multiplied by one is itself).}
    \PY{n}{product}\PY{o}{=}\PY{l+m+mi}{1}
    \PY{k}{for} \PY{n}{arg} \PY{o+ow}{in} \PY{n}{args}\PY{p}{:} \PY{c+c1}{\PYZsh{} iterate through all the values given in the function call}
        \PY{n}{product} \PY{o}{=} \PY{n}{product} \PY{o}{*} \PY{p}{(}\PY{l+m+mi}{1}\PY{o}{/}\PY{n}{arg}\PY{p}{)} \PY{c+c1}{\PYZsh{} multiply the current product value by the reciprocal of the current arg value, and assign it to the product value}
    \PY{k}{return} \PY{n}{product}
\end{Verbatim}
\end{tcolorbox}

    \begin{tcolorbox}[breakable, size=fbox, boxrule=1pt, pad at break*=1mm,colback=cellbackground, colframe=cellborder]
\prompt{In}{incolor}{30}{\boxspacing}
\begin{Verbatim}[commandchars=\\\{\}]
\PY{n}{product\PYZus{}of\PYZus{}reciprocals}\PY{p}{(}\PY{p}{)}
\end{Verbatim}
\end{tcolorbox}

            \begin{tcolorbox}[breakable, size=fbox, boxrule=.5pt, pad at break*=1mm, opacityfill=0]
\prompt{Out}{outcolor}{30}{\boxspacing}
\begin{Verbatim}[commandchars=\\\{\}]
1
\end{Verbatim}
\end{tcolorbox}
        
    \begin{tcolorbox}[breakable, size=fbox, boxrule=1pt, pad at break*=1mm,colback=cellbackground, colframe=cellborder]
\prompt{In}{incolor}{31}{\boxspacing}
\begin{Verbatim}[commandchars=\\\{\}]
\PY{n}{product\PYZus{}of\PYZus{}reciprocals}\PY{p}{(}\PY{l+m+mi}{3}\PY{p}{,}\PY{l+m+mi}{6}\PY{p}{,}\PY{l+m+mi}{95}\PY{p}{)}
\end{Verbatim}
\end{tcolorbox}

            \begin{tcolorbox}[breakable, size=fbox, boxrule=.5pt, pad at break*=1mm, opacityfill=0]
\prompt{Out}{outcolor}{31}{\boxspacing}
\begin{Verbatim}[commandchars=\\\{\}]
0.0005847953216374268
\end{Verbatim}
\end{tcolorbox}
        
    \begin{tcolorbox}[breakable, size=fbox, boxrule=1pt, pad at break*=1mm,colback=cellbackground, colframe=cellborder]
\prompt{In}{incolor}{32}{\boxspacing}
\begin{Verbatim}[commandchars=\\\{\}]
\PY{n}{product\PYZus{}of\PYZus{}reciprocals}\PY{p}{(}\PY{l+m+mi}{1}\PY{p}{,}\PY{l+m+mi}{1}\PY{p}{,}\PY{l+m+mi}{2}\PY{p}{,}\PY{l+m+mi}{3}\PY{p}{,}\PY{l+m+mi}{5}\PY{p}{,}\PY{l+m+mi}{8}\PY{p}{,}\PY{l+m+mi}{11}\PY{p}{)}
\end{Verbatim}
\end{tcolorbox}

            \begin{tcolorbox}[breakable, size=fbox, boxrule=.5pt, pad at break*=1mm, opacityfill=0]
\prompt{Out}{outcolor}{32}{\boxspacing}
\begin{Verbatim}[commandchars=\\\{\}]
0.0003787878787878788
\end{Verbatim}
\end{tcolorbox}
        
    \begin{tcolorbox}[breakable, size=fbox, boxrule=1pt, pad at break*=1mm,colback=cellbackground, colframe=cellborder]
\prompt{In}{incolor}{36}{\boxspacing}
\begin{Verbatim}[commandchars=\\\{\}]
\PY{n}{product\PYZus{}of\PYZus{}reciprocals}\PY{p}{(}\PY{l+m+mf}{0.1}\PY{p}{,} \PY{l+m+mf}{0.2}\PY{p}{,} \PY{l+m+mf}{0.5}\PY{p}{,} \PY{l+m+mf}{1.0}\PY{p}{,} \PY{l+m+mf}{1.5}\PY{p}{,} \PY{l+m+mf}{2.0}\PY{p}{,} \PY{l+m+mf}{5.0}\PY{p}{,} \PY{l+m+mf}{7.0}\PY{p}{,} \PY{l+m+mf}{10.0002}\PY{p}{)}
\end{Verbatim}
\end{tcolorbox}

            \begin{tcolorbox}[breakable, size=fbox, boxrule=.5pt, pad at break*=1mm, opacityfill=0]
\prompt{Out}{outcolor}{36}{\boxspacing}
\begin{Verbatim}[commandchars=\\\{\}]
0.09523619051428493
\end{Verbatim}
\end{tcolorbox}
        
    With \texttt{*args}, any number of values may be passed to the function.

What about unknown and arbitrary number of arguments that must be
assigned to specific keywords?

Here, we can use \texttt{**kwargs}, which is short for ``keyword
arguments''. This specifically requires that each additional argument be
given as a variable and an assignment. These can be useful if there are
specific things you want your function to do, but only if those
arguments are present.

The \texttt{**} indicates that \texttt{kwargs} will be a
\texttt{dictionary} data type, which is a list of mapped keys and
values. Therefore, using the individual keywords requires a little
knowledge of dictionary manipulation.

    \begin{tcolorbox}[breakable, size=fbox, boxrule=1pt, pad at break*=1mm,colback=cellbackground, colframe=cellborder]
\prompt{In}{incolor}{39}{\boxspacing}
\begin{Verbatim}[commandchars=\\\{\}]
\PY{k}{def} \PY{n+nf}{PrintKwargs}\PY{p}{(}\PY{o}{*}\PY{o}{*}\PY{n}{kwargs}\PY{p}{)}\PY{p}{:}
    \PY{k}{for} \PY{n}{key}\PY{p}{,}\PY{n}{val} \PY{o+ow}{in} \PY{n}{kwargs}\PY{o}{.}\PY{n}{items}\PY{p}{(}\PY{p}{)}\PY{p}{:}
        \PY{n+nb}{print}\PY{p}{(}\PY{l+s+s2}{\PYZdq{}}\PY{l+s+s2}{The}\PY{l+s+s2}{\PYZdq{}}\PY{p}{,}\PY{n}{key}\PY{p}{,}\PY{l+s+s2}{\PYZdq{}}\PY{l+s+s2}{says}\PY{l+s+s2}{\PYZdq{}}\PY{p}{,}\PY{n}{val}\PY{p}{)}
    \PY{k}{return}
\end{Verbatim}
\end{tcolorbox}

    \begin{tcolorbox}[breakable, size=fbox, boxrule=1pt, pad at break*=1mm,colback=cellbackground, colframe=cellborder]
\prompt{In}{incolor}{41}{\boxspacing}
\begin{Verbatim}[commandchars=\\\{\}]
\PY{n}{PrintKwargs}\PY{p}{(}\PY{n}{chicken}\PY{o}{=}\PY{l+s+s2}{\PYZdq{}}\PY{l+s+s2}{bawk}\PY{l+s+s2}{\PYZdq{}}\PY{p}{,}\PY{n}{cow}\PY{o}{=}\PY{l+s+s2}{\PYZdq{}}\PY{l+s+s2}{moo}\PY{l+s+s2}{\PYZdq{}}\PY{p}{,}\PY{n}{farmer}\PY{o}{=}\PY{l+s+s2}{\PYZdq{}}\PY{l+s+s2}{it looks like rain}\PY{l+s+s2}{\PYZdq{}}\PY{p}{,}\PY{n}{dog}\PY{o}{=}\PY{l+s+s2}{\PYZdq{}}\PY{l+s+s2}{woof}\PY{l+s+s2}{\PYZdq{}}\PY{p}{)}
\end{Verbatim}
\end{tcolorbox}

    \begin{Verbatim}[commandchars=\\\{\}]
The chicken says bawk
The cow says moo
The farmer says it looks like rain
The dog says woof
    \end{Verbatim}

    What if we include an argument without assigning it to a keyword?

    \begin{tcolorbox}[breakable, size=fbox, boxrule=1pt, pad at break*=1mm,colback=cellbackground, colframe=cellborder]
\prompt{In}{incolor}{43}{\boxspacing}
\begin{Verbatim}[commandchars=\\\{\}]
\PY{n}{PrintKwargs}\PY{p}{(}\PY{n}{chicken}\PY{o}{=}\PY{l+s+s2}{\PYZdq{}}\PY{l+s+s2}{bawk}\PY{l+s+s2}{\PYZdq{}}\PY{p}{,}\PY{n}{cow}\PY{o}{=}\PY{l+s+s2}{\PYZdq{}}\PY{l+s+s2}{moo}\PY{l+s+s2}{\PYZdq{}}\PY{p}{,}\PY{n}{farmer}\PY{o}{=}\PY{l+s+s2}{\PYZdq{}}\PY{l+s+s2}{it looks like rain}\PY{l+s+s2}{\PYZdq{}}\PY{p}{,}\PY{n}{dog}\PY{o}{=}\PY{l+s+s2}{\PYZdq{}}\PY{l+s+s2}{woof}\PY{l+s+s2}{\PYZdq{}}\PY{p}{,}\PY{l+s+s2}{\PYZdq{}}\PY{l+s+s2}{nothing assigned}\PY{l+s+s2}{\PYZdq{}}\PY{p}{)}
\end{Verbatim}
\end{tcolorbox}

    \begin{Verbatim}[commandchars=\\\{\}, frame=single, framerule=2mm, rulecolor=\color{outerrorbackground}]
\textcolor{ansi-cyan}{  File }\textcolor{ansi-green}{"/tmp/ipykernel\_98113/3123196330.py"}\textcolor{ansi-cyan}{, line }\textcolor{ansi-green}{1}
\textcolor{ansi-red}{    PrintKwargs(chicken="bawk",cow="moo",farmer="it looks like rain",dog="woof","nothing assigned")}
                                                                                                  \^{}
\textcolor{ansi-red}{SyntaxError}\textcolor{ansi-red}{:} positional argument follows keyword argument

    \end{Verbatim}

    Not great. In this case, you get an error because there's something
passed to the function that isn't a keyword argument. As an aside, a
``positional argument'' is just the regular arguments we worked with in
the examples before \texttt{*args} and \texttt{**kwargs}.

What if we wanted to account for keyword arguments AND unassigned
arguments?

You can use positional arguments, \texttt{*args}, and \texttt{**kwargs}
in your function calls, so long as they're in that order.

Positional arguments may also have default values assigned to them in
the function definition. Any default variables should be placed at the
end of the \emph{positional arguments}, but before the \texttt{*args}
and \texttt{**kwargs}.

    \begin{tcolorbox}[breakable, size=fbox, boxrule=1pt, pad at break*=1mm,colback=cellbackground, colframe=cellborder]
\prompt{In}{incolor}{1}{\boxspacing}
\begin{Verbatim}[commandchars=\\\{\}]
\PY{k}{def} \PY{n+nf}{BigFunction}\PY{p}{(}\PY{n}{x}\PY{p}{,}\PY{n}{y}\PY{p}{,}\PY{n}{z}\PY{o}{=}\PY{l+m+mi}{10}\PY{p}{,}\PY{o}{*}\PY{n}{args}\PY{p}{,}\PY{o}{*}\PY{o}{*}\PY{n}{kwargs}\PY{p}{)}\PY{p}{:}
    \PY{n+nb}{print}\PY{p}{(}\PY{l+s+s2}{\PYZdq{}}\PY{l+s+s2}{x is}\PY{l+s+s2}{\PYZdq{}}\PY{p}{,}\PY{n}{x}\PY{p}{)}
    \PY{n+nb}{print}\PY{p}{(}\PY{l+s+s2}{\PYZdq{}}\PY{l+s+s2}{y is}\PY{l+s+s2}{\PYZdq{}}\PY{p}{,}\PY{n}{y}\PY{p}{)}
    \PY{n+nb}{print}\PY{p}{(}\PY{l+s+s2}{\PYZdq{}}\PY{l+s+s2}{z is}\PY{l+s+s2}{\PYZdq{}}\PY{p}{,}\PY{n}{z}\PY{p}{)}
    \PY{k}{for} \PY{n}{arg} \PY{o+ow}{in} \PY{n}{args}\PY{p}{:}
        \PY{n+nb}{print}\PY{p}{(}\PY{l+s+s2}{\PYZdq{}}\PY{l+s+s2}{Found arg: }\PY{l+s+s2}{\PYZdq{}}\PY{p}{,}\PY{n}{arg}\PY{p}{)}
    \PY{k}{for} \PY{n}{key}\PY{p}{,}\PY{n}{value} \PY{o+ow}{in} \PY{n}{kwargs}\PY{o}{.}\PY{n}{items}\PY{p}{(}\PY{p}{)}\PY{p}{:}
        \PY{n+nb}{print}\PY{p}{(}\PY{l+s+s2}{\PYZdq{}}\PY{l+s+s2}{keyword}\PY{l+s+s2}{\PYZdq{}}\PY{p}{,}\PY{n}{key}\PY{p}{,}\PY{l+s+s2}{\PYZdq{}}\PY{l+s+s2}{is}\PY{l+s+s2}{\PYZdq{}}\PY{p}{,}\PY{n}{value}\PY{p}{)}
\end{Verbatim}
\end{tcolorbox}

    \begin{tcolorbox}[breakable, size=fbox, boxrule=1pt, pad at break*=1mm,colback=cellbackground, colframe=cellborder]
\prompt{In}{incolor}{2}{\boxspacing}
\begin{Verbatim}[commandchars=\\\{\}]
\PY{n}{BigFunction}\PY{p}{(}\PY{l+m+mi}{3}\PY{p}{,}\PY{l+m+mi}{4}\PY{p}{,}\PY{l+m+mi}{6}\PY{p}{,}\PY{l+s+s2}{\PYZdq{}}\PY{l+s+s2}{bear}\PY{l+s+s2}{\PYZdq{}}\PY{p}{,}\PY{l+s+s2}{\PYZdq{}}\PY{l+s+s2}{goat}\PY{l+s+s2}{\PYZdq{}}\PY{p}{,}\PY{l+s+s2}{\PYZdq{}}\PY{l+s+s2}{llama}\PY{l+s+s2}{\PYZdq{}}\PY{p}{,}\PY{l+s+s2}{\PYZdq{}}\PY{l+s+s2}{emu}\PY{l+s+s2}{\PYZdq{}}\PY{p}{,}\PY{l+s+s2}{\PYZdq{}}\PY{l+s+s2}{shark}\PY{l+s+s2}{\PYZdq{}}\PY{p}{,}\PY{n}{potato}\PY{o}{=}\PY{l+s+s2}{\PYZdq{}}\PY{l+s+s2}{mashed}\PY{l+s+s2}{\PYZdq{}}\PY{p}{,}\PY{n}{dinner}\PY{o}{=}\PY{l+s+s2}{\PYZdq{}}\PY{l+s+s2}{ready}\PY{l+s+s2}{\PYZdq{}}\PY{p}{)}
\end{Verbatim}
\end{tcolorbox}

    \begin{Verbatim}[commandchars=\\\{\}]
x is  3
y is  4
z is 6
Found arg:  bear
Found arg:  goat
Found arg:  llama
Found arg:  emu
Found arg:  shark
keyword potato is mashed
keyword dinner is ready
    \end{Verbatim}

    One thing to keep in mind is that positional arguments get assigned
before anything gets dumped into \texttt{*args}. Above, even though
\texttt{z} has a default value, it got assigned a value of \texttt{6}
from the function call because \texttt{6} was in that position.
Everything afterwards was combined into \texttt{*args}.
