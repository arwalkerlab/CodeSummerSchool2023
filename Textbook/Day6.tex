\chapter{Day 6 - Introduction to C++}
\begin{Goals}
\begin{enumerate}
    \item Compile a basic C++ program
    \item Identify sections of source code
    \item Manage input and output in your code
    \item Handle variables, including initialization, assignment, modification, and recall.
    \item Read from and write to files, including parsing/processing inputs and formatting outputs.
\end{enumerate}
\end{Goals}
\paragraph{}C++ is a compiled language, which means that before a program written with it can be executed, it must be compiled from human-readable text into the binary machine language.

By convention, C++ source code files (the text you can read and edit) end with \texttt{.cpp}, compiled object files end in \texttt{.o}, and completed executables end in \texttt{.x}.

The basic compilation command we will be using is below:

\begin{verbatim}
    g++ main.cpp -o output.x
\end{verbatim}

This calls the compiler \texttt{g++} on the source file \texttt{main.cpp}, with the resulting output (\texttt{-o}) as \texttt{output.x}.  The program can then be executed like any other command-line program by calling it with \texttt{./output.x}.

To start, every C++ program will have the following basic structure:

\begin{minted}{c++}
#include <iostream>
#include <cstdlib>
int main()
// single-line comment
{
    std::cout << "Hello World!" << std::endl;
    return 0;
}
/* This is a multi-
line comment between the start and end asterisk-slashes
*/
\end{minted}

Let's break this down into some major pieces.  The first section is where libraries and other required files are included.  In this example, we're including the \texttt{iostream} library, which manages the input/output streams of C++, and \texttt{cstdlib}, the C Standard Library which contains a wide array of commonly used functions.  As we progress, we will encounter more libraries with their own unique usefulness.
\begin{minted}{c++}
#include <iostream>
#include <cstdlib>
\end{minted}

The next section is where the main program loop is written.  It must be called "main" for C++ compilers to understand where the starting point of the program is when it's run.  The datatype \texttt{int} is used by convention to allow for integer-based exit codes to be returned to the operating system.  This is useful for error reporting - if the program fails in a way the programmer has anticipated, it can return a specific integer value to the operating system to alert it to the failure.

The \textbf{scope} of the \texttt{main()} function is defined by the open- and close- curly braces.  Throughout C++, this will be how scope is defined and maintained.  Inside the scope is where all the program's functions and commands are stored.  At the end of any program or function (with a specific exception we will get into later), there \textbf{must be a \texttt{return} command}.  Also note that at the end of every command line, there is a semicolon.
\begin{minted}{c++}
int main()
{
    std::cout << "Hello World!" << std::endl;
    return 0;
}
\end{minted}
\section{Input/Output}
\paragraph{} The most basic functionality is input from and output to the terminal.  These are accomplished with \texttt{std::cin} and \texttt{std::cout}.  The following example prompts the user to enter an integer value, stores it, and then prints it back out.

\begin{minted}{c++}
#include <iostream>
#include <cstdlib>
int main()
{
    int my_integer;
    std::cout << "Enter an integer: ";
    std::cin >> my_integer;
    std::cout << "You entered " << my_integer << std::endl;
    return 0;
}
\end{minted}
This example also introduces the next section on variables.

\begin{homework}
Try creating some small programs to deal with inputs and outputs from the terminal.  Don't forget to compile before trying to run the programs!
\end{homework}
\section{Variables and Math}
The code block below introduces \texttt{int} and \texttt{double} variable types.  Integers are whole numbers and doubles are floating-point numbers, or decimals, or simply floats.
In C++, variables must be \textbf{declared} before they can be assigned a value, retrieved, or modified.

\begin{minted}{c++}
#include <iostream>
#include <cstdlib>
int main()
{
  int my_integer;
  double my_double;

  my_integer = 10;
  my_double = 10.3;
  // Note how there is no decimal at the end of my_integer.

  std::cout << "my_integer is :" << my_integer << std::endl;
  std::cout << "my_double  is :" << my_double << std::endl;
  
  std::cout << "\n\nLet's do some integer division!\n\n";
  std::cout << "10 / 3 = " << 10/3 << std::endl;
  std::cout << "28 / 5 = " << 28/5 << std::endl;
  
  std::cout << "\n\nNow let's do some double division!" << std::endl;
  std::cout << "10. / 3. = " << 10./3. << std::endl;
  std::cout << "28. / 5. = " << 28./5. << std::endl;
  // Note that even though 10 and 3 are integers, we've included the decimal.  
  // This forces C++ to treat them as doubles instead of integers.
  
  std::cout << "\n\nNow let's do some mixed division!" << std::endl;
  std::cout << "10. / 3 = " << 10./3 << std::endl;
  std::cout << "28 / 5. = " << 28/5. << std::endl;
  // When we run this program, notice the outputs.  Both should be doubles because 
  // C++ gives importance to doubles over ints.
  return 0;
}
\end{minted}
Notice the way each of the different sections are written.  The numerical values are the same in each, with \texttt{10} and \texttt{10.} being equivalent.  However, C++ code treats the former as an integer and the latter as a float.  If you run the program above, you will see the different outputs for what we know to be mathematically identical operations.

It is important to keep these things in mind when working with numerical values in C++.

\subsection{Constants}
It is sometimes useful for us to have values stored as variables that we can adjust once rather than trying to find every instance of them in our code if we need to change them.  For example, we might need to use $\pi$ in a program, but may only need a few decimal places.  Later on, we might find that we need more than a few decimals, and it would be easiest for us if we just had to change one value.  It's also safer because we reduce the risk of introducing an error through a typo in the number if we only have to enter it once.

This is where the use of a \texttt{const} comes in handy.  We add \texttt{const} before a variable type to indicate that the variable is not allowed to be changed during the execution of the program.  In the example with $\pi$, it would look something like this.

\begin{minted}{c++}
const double PI=3.1415;
\end{minted}

In general, constants are declared outside of the scope of functions so that they can be used everywhere.

\section{Strings}
Strings are also known as arrays of characters, or chars.  Both options can be used in C++, however it is important to know the differences in how they can be used.
Strings in C++ have some built-in functions that can be used to find patterns and substrings, modify strings, and slice them.  The code below illustrates the two different variable declarations.  Note that the \texttt{char*} variable is also defined in the same line.

Also, you will notice the line \texttt{using namespace std;} in the code block below.  Previous code blocks have had \texttt{std::} at the start of many of the commands. This line is being added to make it easier to use the functions contained within it. It's similar to python's \texttt{from library import *}, in that it removes the requirement to include \texttt{std::} at the start of many of the library-dependent variable types and functions.  As a result, you will see things like \texttt{cout} instead of \texttt{std::cout}.  This is one of the benefits of using \textbf{namespaces}, and it can be helpful later on down the line as you develop your own code.

\begin{minted}{c++}
#include <iostream>
#include <cstdlib>
#include <cstring>

using namespace std;

int main()
{
  char* my_charstring = "Texas has the best barbecue!";
  string my_string;
  my_string = "Detroit has the best Cheesy Fries!";

  cout << "Now to see how that string is doing.  It was called 'my_string'...\n";
  cout << my_string << endl;

  cout << "And the char* list...  It was called 'my_charstring'...\n";
  cout << my_charstring << endl;

  return 0;
}
\end{minted}

You may also have noticed the use of \texttt{endl} rather than the newline character \texttt{"\\n"}.  These both produce a line break.  The only difference between them is that \texttt{endl} does something called "flushing the buffer", which means that the data stored in the output buffer is forced out into the main output.

For most text-based things, this may not seem important.  But if you're writing things to a file in a loop, you might want to ensure that everything is written to the file as it's generated, rather than all at once at the end of the loop.  Using \texttt{endl} is the preferred option in this scenario, and it's ultimately easier.
\section{Basic Math}
Basic mathematical operators are available in C++.  An example is given below.  Notice that the variables are all declared on one line.  This is done because all of these different variables are doubles, and the compiler can interpret the single line to mean exactly that.

\begin{minted}{c++}
#include <iostream>
#include <cstdlib>
using namespace std;

int main()
{
  double var1, var2, sum, difference, product, dividend;

  cout << "Let's do some math!\nEnter a number: ";
  cin >> var1;
  cout << "Enter another number: ";
  cin >> var2;

  // Assigning results of mathematical functions to new variables.
  sum = var1+var2;
  difference = var1-var2;
  product = var1*var2;
  dividend = var1/var2;

  cout << "Addition: " << var1 << " + " << var2 << " = " << sum << endl;
  cout << "Subtraction: " << var1 << " - " << var2 << " = " << difference << endl;
  cout << "Multiplication: " << var1 << " * " << var2 << " = " << product << endl;
  cout << "Division: " << var1 << " / " << var2 << " = " << dividend << endl;

  return 0;
}
\end{minted}

When run, this program will take any two values given and return the different basic mathematical operations between them - addition, subtraction, multiplication, and division.  Try writing your own mathematical algorithms!
\section{File I/O}

As we discussed before in the Python section, one of the most important things we can do with our code is learn to read and write data in files, or process that data in a way that is useful to us.  The next several code blocks will illustrate some of the different ways file data can be read, processed, formatted, and written.
Keep in mind that this is meant as an introduction to the concepts and not an exhaustive list of methods.

It's important to remember that whenever you open any file in C++, you must close it at some point before the end of the program.  In python, there are instances where the closing of a file is automatic, so it's a little less strict.  In C++, that leniency is not there.  \textbf{Opened files must be closed.}

\subsection{Reading Files}
File I/O is largely controlled by the \texttt{fstream} library.  When opening a file for reading, we use the variable type \texttt{ifstream}, which is short for \textbf{i}nput-\textbf{f}ile-\textbf{stream}.  For files to be written, we use the variable type \texttt{ofstream}.

\begin{minted}{c++}
#include <iostream>
#include <fstream>
#include <cstdlib>
#include <cstring>
using namespace std;

int main()
{
  ifstream inFile;
  string words;
  
  //Whenever you open a file...
  inFile.open("TestData/input_file.txt",ios::in);
  inFile >> words;
  //You have to close it!
  inFile.close();
  cout << words;
  
  cout <<"\n\n Note that we only got one word of the file.\n\n";

  return 0;
}
\end{minted}

In the example above, we have opened the file "TestData/input\_file.txt" for reading and then used it as our input stream.  We then took one the first item in the stream and put it into the variable "words".  Then we closed the file because we were done with it.  And finally, we output the value of "words", which in our case was only the first word of the file.

If we changed the line \texttt{inFile >> words;} to \texttt{inFile >> words >> words;}, we would get the \textbf{second} word in the file, because we assigned the first word to the variable "words", and then overwrote it with the second.  You can also parse lines such that different portions of the line go to different variables.  Take the following lines as an example:

\begin{verbatim}
    State   Frequency       Amplitude
    0       100             3.2
\end{verbatim}

If we assume we've gotten through the file to the point that our input is sitting at the start of the second line, we could use the following code block to assign our different values.

\begin{minted}{c++}
...
int state;
double frequency, amplitude;
inFile >> state >> frequency >> amplitude;
...
\end{minted}

This would assign each of the values to their respective variables.  This is how many programs we use parse input files for molecular geometry, by simply reading in each line with an expected format.

The next code block shows how we can get an entire line of the file at a time.  In many cases, it's easier to work with entire lines rather than trying to account for each individual word in the file.  If we take entire lines at a time, we can also do things like searching for specific patterns within them.

\begin{minted}{c++}
#include <iostream>
#include <fstream>
#include <cstdlib>
#include <cstring>
using namespace std;

int main()
{
  ifstream inFile;
  string words;
 
  cout <<"Let's try getline(): \n";
  inFile.open("TestData/input_file.txt",ios::in);
  getline(inFile,words);
  inFile.close();
  cout <<  words << endl;
  return 0;
}
\end{minted}

The \texttt{getline()} function takes everything in the first argument (the inFile stream) from its current position to the next linebreak character or the EOF (end-of-file) character, and then puts that content into the second argument, in this case our variable "words".

Now let's try going through the entire file.  We can open the file like normal, then put our \texttt{getline()} function inside a \texttt{while} loop.  This is useful because the \texttt{getline()} function will continually iterate through all the lines in the file and return a \texttt{False} value when it gets to the end of the file.  Therefore, inside the loop we can work with each line individually.  In the example below, we're simply printing out each line to the terminal as we go.

\begin{minted}{c++}
#include <iostream>
#include <fstream>
#include <cstdlib>
#include <cstring>
using namespace std;

int main()
{
  ifstream inFile;
  string words;
  inFile.open("TestData/input_file.txt",ios::in);
  while ( getline(inFile,words) )
  {
    cout << words << endl;
  }
  inFile.close();
  return 0;
}
\end{minted}

If you ran this program in the actual Code Summer School folder associated with this document, you'll have gotten the entire lyrics to "I Am the Very Model of a Modern Major-General" from Gilbert and Sullivan's \textit{Pirates of Penzance (1879)}.

What if we wanted to modify lines that had a specific pattern in them?
We can use the \texttt{find()} function available to the string class to seek a pattern inside the string.  We can also use some \texttt{if/else} statements to direct the code's flow.  In the code block below, we simply expanded the previous loop's internal functions to include the check for the word "General" in each line.  If the word is found in the current line, we run the \texttt{transform} function (found in the \texttt{algorithm} library included at the beginning) to convert the whole string to all-caps.

So every time the program encounters "General" in a line, it will print that line in all caps rather than its original format.  And since the file has the lyrics to "I Am The Very Model of a Modern Major-General", we can assume it'll happen a lot.

\begin{minted}{c++}
#include <iostream>
#include <fstream>
#include <cstdlib>
#include <cstring>
#include <algorithm>
using namespace std;

int main()
{
  ifstream inFile;
  string words;
  
  inFile.open("TestData/input_file.txt",ios::in);

  while ( getline(inFile,words) )
  {
    // Let's try some if-statements to see if a word is in the line...
    if (words.find("General") != string::npos)
    {
      transform(words.begin(), words.end(), words.begin(), ::toupper);
      cout << words << endl;
    }
    else
    {
      cout << words << endl;
    }
  }
  inFile.close();
  return 0;
}
\end{minted}

\subsection{Writing Files}
Writing files is similar to reading files in that you have to open and close the files appropriately.  Beyond that, however, writing to the file is much the same as outputting text to the terminal with \texttt{std::cout}.

\begin{minted}{c++}
#include <iostream>
#include <fstream>
#include <cstdlib>
#include <cstring>
#include <algorithm>
using namespace std;

int main()
{
  ofstream outFile;
  string words;
  
  outFile.open("TestData/output_file.txt",ios::out);
  outFile << "This is just a bunch of text." << endl;
  outFile << "This is a second line after the first 'endl'." << endl;
  outFile.close();
  return 0;
}
\end{minted}

The above code block opens a file called "output\_file.txt", outputs two lines of text, then closes the file and exits the program.  This is fairly straightforward.

However, what if you wanted to format the text in a specific and consistent way?  You can use \textit{manipulators} to achieve this.

You can change the justification and fill character with the commands shown below.  Keep in mind that the fill character must be a single character.  You can also set how wide each portion of the output will be by changing the \texttt{setw()} value as necessary.

\begin{minted}{c++}
#include <iostream>
#include <iomanip>
int main()
{
    std::cout << "Left fill:\n" << std::left << std::setfill(' ')
              << std::setw(12) << -1.23  << '\n';

    std::cout << "Internal fill:\n" << std::internal
              << std::setw(12) << -1.23  << '\n';

    std::cout << "Right fill:\n" << std::right
              << std::setw(12) << -1.23  << '\n';
}
\end{minted}

\begin{homework}
Try making your own smaller program to read in information from a file, find some specific pattern, and then output a differently-formatted version of the data to a new file.  You can use any of the outputs you have from your research calculations.
\end{homework}