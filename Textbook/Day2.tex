\chapter{Day 2 - Python Basics}
\section{Error Handling}
Error handling refers to a set of instructions the program follows if it
encounters an error. If you've gotten any kind of an error in python,
you've seen error handling in action.

What if you wanted to write code that would just take care of the
problem if it encountered an error, rather than crashing and complaining
to the user? If you can anticipate possible errors that may arise, you
can account for them and work around them.

Error handling in python is based first on the \texttt{try/except}
commands. Let's look at the examples below.

    \begin{tcolorbox}[breakable, size=fbox, boxrule=1pt, pad at break*=1mm,colback=cellbackground, colframe=cellborder]
\prompt{In}{incolor}{8}{\boxspacing}
\begin{Verbatim}[commandchars=\\\{\}]
\PY{k+kn}{import} \PY{n+nn}{foobar} \PY{k}{as} \PY{n+nn}{fb}
\PY{n+nb}{print}\PY{p}{(}\PY{l+s+s2}{\PYZdq{}}\PY{l+s+s2}{Hello world!}\PY{l+s+s2}{\PYZdq{}}\PY{p}{)}
\end{Verbatim}
\end{tcolorbox}

    \begin{Verbatim}[commandchars=\\\{\}, frame=single, framerule=2mm, rulecolor=\color{outerrorbackground}]
\textcolor{ansi-red}{---------------------------------------------------------------------------}
\textcolor{ansi-red}{ModuleNotFoundError}                       Traceback (most recent call last)
\textcolor{ansi-green-intense}{\textbf{/home/mark/GH\_Repositories/CodingSummerSchool/Day05\_Python\_WrapUp/18\_Python\_Error\_Handling.ipynb Cell 2'}} in \textcolor{ansi-cyan}{<cell line: 1>}\textcolor{ansi-blue}{()}
\textcolor{ansi-green}{----> <a href='vscode-notebook-cell:/home/mark/GH\_Repositories/CodingSummerSchool/Day05\_Python\_WrapUp/18\_Python\_Error\_Handling.ipynb\#ch0000000?line=0'>1</a>} import foobar as fb
\textcolor{ansi-green-intense}{\textbf{      <a href='vscode-notebook-cell:/home/mark/GH\_Repositories/CodingSummerSchool/Day05\_Python\_WrapUp/18\_Python\_Error\_Handling.ipynb\#ch0000000?line=2'>3</a>}} print("Hello world!")

\textcolor{ansi-red}{ModuleNotFoundError}: No module named 'foobar'
    \end{Verbatim}

    \begin{tcolorbox}[breakable, size=fbox, boxrule=1pt, pad at break*=1mm,colback=cellbackground, colframe=cellborder]
\prompt{In}{incolor}{4}{\boxspacing}
\begin{Verbatim}[commandchars=\\\{\}]
\PY{k}{try}\PY{p}{:}
    \PY{k+kn}{import} \PY{n+nn}{foobar} \PY{k}{as} \PY{n+nn}{fb}
\PY{k}{except}\PY{p}{:}
    \PY{n+nb}{print}\PY{p}{(}\PY{l+s+s2}{\PYZdq{}}\PY{l+s+s2}{foobar module is not available.}\PY{l+s+s2}{\PYZdq{}}\PY{p}{)}
\PY{n+nb}{print}\PY{p}{(}\PY{l+s+s2}{\PYZdq{}}\PY{l+s+s2}{Hello world!}\PY{l+s+s2}{\PYZdq{}}\PY{p}{)}
\end{Verbatim}
\end{tcolorbox}

    \begin{Verbatim}[commandchars=\\\{\}]
pytraj module is not available.
Hello world!
    \end{Verbatim}

    In the first example, we tried to directly import a python module that
doesn't exist (on the computer or even at all), and then print ``Hello
World!'' But because \texttt{foobar} doesn't exist, python encountered
an error and immediately crashed. In the second example, however, we
included the \texttt{try/except} commands, which allows us to follow a
specific set of instructions when the error is encountered, and then
continue on with the program.

I commonly use this in scripts and programs that I provide to the lab
that use non-standard python libraries. For example, a script may use
the \texttt{mdanalysis} library, which is not included with python by
default.

    \begin{tcolorbox}[breakable, size=fbox, boxrule=1pt, pad at break*=1mm,colback=cellbackground, colframe=cellborder]
\prompt{In}{incolor}{ }{\boxspacing}
\begin{Verbatim}[commandchars=\\\{\}]
\PY{k}{try}\PY{p}{:}
    \PY{k+kn}{import} \PY{n+nn}{mdanalysis} \PY{k}{as} \PY{n+nn}{mda}
\PY{k}{except}\PY{p}{:}
    \PY{k+kn}{import} \PY{n+nn}{subprocess}
    \PY{n}{subprocess}\PY{o}{.}\PY{n}{call}\PY{p}{(}\PY{l+s+s2}{\PYZdq{}}\PY{l+s+s2}{conda install \PYZhy{}y mdanalysis}\PY{l+s+s2}{\PYZdq{}}\PY{p}{,}\PY{n}{shell}\PY{o}{=}\PY{k+kc}{True}\PY{p}{)}
    \PY{k+kn}{import} \PY{n+nn}{mdanalysis} \PY{k}{as} \PY{n+nn}{mda}
\end{Verbatim}
\end{tcolorbox}

    The cell above tries to import \texttt{mdanalysis}, and if it fails, it
loads a standard python library called \texttt{subprocess}, then uses
that to install \texttt{mdanalysis}. Once that has completed, the code
continues and imports mdanalysis as though it were always there.

Error handling can get more complex than this, with specific
instructions being assigned to specific \textbf{types} of errors.

If you wanted to be really mean, you could include a handling of the
``KeyboardInterruptException'' to prevent your code from being killed by
the usual \texttt{ctrl-C} method.

But don't do that.
\section{File I/O}
I/O stands for ``Input/Output''. Loading files into your programs,
parsing the data inside, and being able to use, manipulate, transform,
present, and save the results is a very important skillset.

    \hypertarget{reading-files}{%
\subsubsection{Reading Files}\label{reading-files}}

Python has a few different ways to read files, but for now we'll stick
with plain-text files like you'll get from various analysis programs or
computational chemistry software packages. Python can also read binary
data files, but that's generally more advanced and requires a bit more
experience.

The number one rule to remember for any and all files is that if you
\textbf{open} it, you must \textbf{close} it. A file that is opened and
never closed can either wind up inaccessible to other programs or the
user, or the data that was meant to be stored within can be lost.

See the example below. With the \texttt{open} function, we create a
\textbf{file object} that has several useful internal functions we can
take advantage of. \texttt{open()} takes two arguments - the filename
first, then the type of action to be taken on the file. ``r'' is for
``read mode'', ``w'' is for ``write mode'' which starts at the beginning
of the file and writes (including overwriting!), and ``a'' is for
``append mode'', which acts like ``write mode'' except that it adds to
the end of the file rather than overwriting it.

    \begin{tcolorbox}[breakable, size=fbox, boxrule=1pt, pad at break*=1mm,colback=cellbackground, colframe=cellborder]
\prompt{In}{incolor}{6}{\boxspacing}
\begin{Verbatim}[commandchars=\\\{\}]
\PY{n}{myfile} \PY{o}{=} \PY{n+nb}{open}\PY{p}{(}\PY{l+s+s2}{\PYZdq{}}\PY{l+s+s2}{test\PYZus{}data/file\PYZus{}reading.txt}\PY{l+s+s2}{\PYZdq{}}\PY{p}{,}\PY{l+s+s2}{\PYZdq{}}\PY{l+s+s2}{r}\PY{l+s+s2}{\PYZdq{}}\PY{p}{)}

\PY{n}{contents} \PY{o}{=} \PY{n}{myfile}\PY{o}{.}\PY{n}{read}\PY{p}{(}\PY{p}{)}

\PY{n+nb}{print}\PY{p}{(}\PY{n+nb}{type}\PY{p}{(}\PY{n}{contents}\PY{p}{)}\PY{p}{)}
\PY{n+nb}{print}\PY{p}{(}\PY{n}{contents}\PY{p}{)}

\PY{n}{myfile}\PY{o}{.}\PY{n}{close}\PY{p}{(}\PY{p}{)}
\end{Verbatim}
\end{tcolorbox}

    \begin{Verbatim}[commandchars=\\\{\}]
<class 'str'>
This is just a normal text file.
There are words.  There are sentences.  There are lines.

There was even a blank line.
After this sentence is the end of the file, usually called "EOF".
    \end{Verbatim}

    As you can see, the entire file can be read into a single string, which
includes all the line breaks and everything within. This can be easiest
if you have a small file, or if you know exactly the structure of the
data inside.

There is also a function that can read one line at a time. This
function, \texttt{readline()}, returns the \emph{next} line in the file,
up to the newline character, \texttt{\textbackslash{}n}. We often use
this if the file is large, or if there are multiple line formats in the
file (such as we might see in a TeraChem output file), or if we want to
keep memory usage low.

    \begin{tcolorbox}[breakable, size=fbox, boxrule=1pt, pad at break*=1mm,colback=cellbackground, colframe=cellborder]
\prompt{In}{incolor}{10}{\boxspacing}
\begin{Verbatim}[commandchars=\\\{\}]
\PY{n}{myfile} \PY{o}{=} \PY{n+nb}{open}\PY{p}{(}\PY{l+s+s2}{\PYZdq{}}\PY{l+s+s2}{test\PYZus{}data/file\PYZus{}reading.txt}\PY{l+s+s2}{\PYZdq{}}\PY{p}{,}\PY{l+s+s2}{\PYZdq{}}\PY{l+s+s2}{r}\PY{l+s+s2}{\PYZdq{}}\PY{p}{)}

\PY{n}{contents} \PY{o}{=} \PY{n}{myfile}\PY{o}{.}\PY{n}{readline}\PY{p}{(}\PY{p}{)}
\PY{n+nb}{print}\PY{p}{(}\PY{n}{contents}\PY{p}{)}

\PY{c+c1}{\PYZsh{}\PYZsh{}\PYZsh{} We\PYZsq{}ve already read the first line, but if we do the same set of commands again, we\PYZsq{}ll get the next line}

\PY{n}{contents} \PY{o}{=} \PY{n}{myfile}\PY{o}{.}\PY{n}{readline}\PY{p}{(}\PY{p}{)}
\PY{n+nb}{print}\PY{p}{(}\PY{n}{contents}\PY{p}{)}

\PY{c+c1}{\PYZsh{}\PYZsh{}\PYZsh{} And again, for the third. (see above, the third line is blank!)}

\PY{n}{contents} \PY{o}{=} \PY{n}{myfile}\PY{o}{.}\PY{n}{readline}\PY{p}{(}\PY{p}{)}
\PY{n+nb}{print}\PY{p}{(}\PY{n}{contents}\PY{p}{)}

\PY{c+c1}{\PYZsh{}\PYZsh{}\PYZsh{} Fourth}

\PY{n}{contents} \PY{o}{=} \PY{n}{myfile}\PY{o}{.}\PY{n}{readline}\PY{p}{(}\PY{p}{)}
\PY{n+nb}{print}\PY{p}{(}\PY{n}{contents}\PY{p}{)}

\PY{n}{myfile}\PY{o}{.}\PY{n}{close}\PY{p}{(}\PY{p}{)}
\end{Verbatim}
\end{tcolorbox}

    \begin{Verbatim}[commandchars=\\\{\}]
This is just a normal text file.

There are words.  There are sentences.  There are lines.



There was even a blank line.

    \end{Verbatim}

    We can also get and keep all the lines in memory stored as a
\texttt{list}, with each element of the list holding one line of the
file as a string. This uses a function called \texttt{readlines()},
which has the \texttt{s} at the end to keep it separate from the last
function, \texttt{readline()}.

    \begin{tcolorbox}[breakable, size=fbox, boxrule=1pt, pad at break*=1mm,colback=cellbackground, colframe=cellborder]
\prompt{In}{incolor}{12}{\boxspacing}
\begin{Verbatim}[commandchars=\\\{\}]
\PY{n}{myfile} \PY{o}{=} \PY{n+nb}{open}\PY{p}{(}\PY{l+s+s2}{\PYZdq{}}\PY{l+s+s2}{test\PYZus{}data/file\PYZus{}reading.txt}\PY{l+s+s2}{\PYZdq{}}\PY{p}{,}\PY{l+s+s2}{\PYZdq{}}\PY{l+s+s2}{r}\PY{l+s+s2}{\PYZdq{}}\PY{p}{)}
\PY{n}{all\PYZus{}lines}\PY{o}{=}\PY{n}{myfile}\PY{o}{.}\PY{n}{readlines}\PY{p}{(}\PY{p}{)}
\PY{n+nb}{print}\PY{p}{(}\PY{n}{all\PYZus{}lines}\PY{p}{)}
\PY{n}{myfile}\PY{o}{.}\PY{n}{close}\PY{p}{(}\PY{p}{)}
\end{Verbatim}
\end{tcolorbox}

    \begin{Verbatim}[commandchars=\\\{\}]
['This is just a normal text file.\textbackslash{}n', 'There are words.  There are sentences.
There are lines.  \textbackslash{}n', '\textbackslash{}n', 'There was even a blank line.\textbackslash{}n', 'After this
sentence is the end of the file, usually called "EOF".']
    \end{Verbatim}

    Now you can see all the lines in the file, but presented as a list of
strings instead of the formatted text. With the list structure of the
\texttt{readlines()} function, you can then do more complicated things
to the individual lines by iterating through the list, or choose
specific lines with list-slicing like we discussed previously.

    \begin{tcolorbox}[breakable, size=fbox, boxrule=1pt, pad at break*=1mm,colback=cellbackground, colframe=cellborder]
\prompt{In}{incolor}{14}{\boxspacing}
\begin{Verbatim}[commandchars=\\\{\}]
\PY{k}{for} \PY{n}{line} \PY{o+ow}{in} \PY{n}{all\PYZus{}lines}\PY{p}{:}
    \PY{k}{if} \PY{l+s+s2}{\PYZdq{}}\PY{l+s+s2}{There}\PY{l+s+s2}{\PYZdq{}} \PY{o+ow}{in} \PY{n}{line}\PY{p}{:}
        \PY{n+nb}{print}\PY{p}{(}\PY{n}{line}\PY{p}{)}
\end{Verbatim}
\end{tcolorbox}

    \begin{Verbatim}[commandchars=\\\{\}]
There are words.  There are sentences.  There are lines.

There was even a blank line.

    \end{Verbatim}

    You can also further split up the lines with the \texttt{.split()}
function available to any string object.

    \begin{tcolorbox}[breakable, size=fbox, boxrule=1pt, pad at break*=1mm,colback=cellbackground, colframe=cellborder]
\prompt{In}{incolor}{15}{\boxspacing}
\begin{Verbatim}[commandchars=\\\{\}]
\PY{k}{for} \PY{n}{line} \PY{o+ow}{in} \PY{n}{all\PYZus{}lines}\PY{p}{:}
    \PY{n+nb}{print}\PY{p}{(}\PY{n}{line}\PY{o}{.}\PY{n}{split}\PY{p}{(}\PY{p}{)}\PY{p}{)}
\end{Verbatim}
\end{tcolorbox}

    \begin{Verbatim}[commandchars=\\\{\}]
['This', 'is', 'just', 'a', 'normal', 'text', 'file.']
['There', 'are', 'words.', 'There', 'are', 'sentences.', 'There', 'are',
'lines.']
[]
['There', 'was', 'even', 'a', 'blank', 'line.']
['After', 'this', 'sentence', 'is', 'the', 'end', 'of', 'the', 'file,',
'usually', 'called', '"EOF".']
    \end{Verbatim}

    With this in mind, can you write a function to search the
\texttt{TeraChemOutput.out} file (included in the \texttt{test\_data}
folder) for every instance of ``RMS grad''?

    \begin{tcolorbox}[breakable, size=fbox, boxrule=1pt, pad at break*=1mm,colback=cellbackground, colframe=cellborder]
\prompt{In}{incolor}{ }{\boxspacing}
\begin{Verbatim}[commandchars=\\\{\}]
\PY{c+c1}{\PYZsh{}\PYZsh{}\PYZsh{} Your code goes here!}
\end{Verbatim}
\end{tcolorbox}

    Now that you can find all the lines with ``RMS grad'' in them, can you
extract the current value at each line and add it to a dataset?

\emph{Hint: use the \texttt{append} function, and don't forget to
convert from a string to a float!}

    \begin{tcolorbox}[breakable, size=fbox, boxrule=1pt, pad at break*=1mm,colback=cellbackground, colframe=cellborder]
\prompt{In}{incolor}{5}{\boxspacing}
\begin{Verbatim}[commandchars=\\\{\}]
\PY{n}{dataset} \PY{o}{=} \PY{p}{[}\PY{p}{]}

\PY{c+c1}{\PYZsh{}\PYZsh{}\PYZsh{} Your code goes here!}
\end{Verbatim}
\end{tcolorbox}

    If you were successful in finding, extracting, and storing the ``RMS
grad'' value for each occurrence in the file, run the next cell to see
that data plotted!

    \begin{tcolorbox}[breakable, size=fbox, boxrule=1pt, pad at break*=1mm,colback=cellbackground, colframe=cellborder]
\prompt{In}{incolor}{6}{\boxspacing}
\begin{Verbatim}[commandchars=\\\{\}]
\PY{k+kn}{import} \PY{n+nn}{matplotlib}\PY{n+nn}{.}\PY{n+nn}{pyplot} \PY{k}{as} \PY{n+nn}{plt}
\PY{n}{plt}\PY{o}{.}\PY{n}{plot}\PY{p}{(}\PY{n}{dataset}\PY{p}{)}
\end{Verbatim}
\end{tcolorbox}

            \begin{tcolorbox}[breakable, size=fbox, boxrule=.5pt, pad at break*=1mm, opacityfill=0]
\prompt{Out}{outcolor}{6}{\boxspacing}
\begin{Verbatim}[commandchars=\\\{\}]
[<matplotlib.lines.Line2D at 0x7fb50353f490>]
\end{Verbatim}
\end{tcolorbox}
        
    \begin{center}
    \adjustimage{max size={0.9\linewidth}{0.9\paperheight}}{Images/output_16_1.png}
    \end{center}
    { \hspace*{\fill} \\}
    
    You can see how useful parsing datafiles can be, especially when those
datafiles aren't as neatly formatted as a CPPTRAJ output, or when there
is lots of different kinds of data that you may not need or want at that
moment.

But what about writing your own outputs?

\hypertarget{writing-files}{%
\subsubsection{Writing Files}\label{writing-files}}

To write to a file, you start in much the same way as reading, except
you open the file in ``w'' or ``a'' mode. The file, if it doesn't
already exist, will be created on \texttt{open} and saved on
\texttt{close}, so again be sure to close every file you open.

    \begin{tcolorbox}[breakable, size=fbox, boxrule=1pt, pad at break*=1mm,colback=cellbackground, colframe=cellborder]
\prompt{In}{incolor}{7}{\boxspacing}
\begin{Verbatim}[commandchars=\\\{\}]
\PY{n}{myfile} \PY{o}{=} \PY{n+nb}{open}\PY{p}{(}\PY{l+s+s2}{\PYZdq{}}\PY{l+s+s2}{test\PYZus{}data/writing\PYZus{}outputs.txt}\PY{l+s+s2}{\PYZdq{}}\PY{p}{,}\PY{l+s+s2}{\PYZdq{}}\PY{l+s+s2}{w}\PY{l+s+s2}{\PYZdq{}}\PY{p}{)}

\PY{n}{myfile}\PY{o}{.}\PY{n}{write}\PY{p}{(}\PY{l+s+s2}{\PYZdq{}}\PY{l+s+s2}{This is the text I am writing. }\PY{l+s+s2}{\PYZdq{}}\PY{p}{)}
\PY{n}{myfile}\PY{o}{.}\PY{n}{write}\PY{p}{(}\PY{l+s+s2}{\PYZdq{}}\PY{l+s+s2}{Notice how there is no newline after the last sentence... }\PY{l+s+s2}{\PYZdq{}}\PY{p}{)}
\PY{n}{myfile}\PY{o}{.}\PY{n}{write}\PY{p}{(}\PY{l+s+s2}{\PYZdq{}}\PY{l+s+s2}{You have to include the newline character explicitly when writing to a file.}\PY{l+s+se}{\PYZbs{}n}\PY{l+s+s2}{\PYZdq{}}\PY{p}{)}
\PY{n}{myfile}\PY{o}{.}\PY{n}{write}\PY{p}{(}\PY{l+s+s2}{\PYZdq{}}\PY{l+s+se}{\PYZbs{}n}\PY{l+s+s2}{\PYZdq{}}\PY{p}{)}
\PY{n}{myfile}\PY{o}{.}\PY{n}{write}\PY{p}{(}\PY{l+s+s2}{\PYZdq{}}\PY{l+s+s2}{Like that!}\PY{l+s+se}{\PYZbs{}n}\PY{l+s+s2}{\PYZdq{}}\PY{p}{)}

\PY{n}{myfile}\PY{o}{.}\PY{n}{close}\PY{p}{(}\PY{p}{)}
\end{Verbatim}
\end{tcolorbox}

    You can also write data with specific formatting, producing nice neat
columns of data that can make it easier to read and analyze. You can use
string-formatting to specify the allowed width of sections of a line.

Consider the following set of data:

    \begin{tcolorbox}[breakable, size=fbox, boxrule=1pt, pad at break*=1mm,colback=cellbackground, colframe=cellborder]
\prompt{In}{incolor}{11}{\boxspacing}
\begin{Verbatim}[commandchars=\\\{\}]
\PY{n}{people} \PY{o}{=} \PY{p}{[}  \PY{p}{\PYZob{}}\PY{l+s+s2}{\PYZdq{}}\PY{l+s+s2}{Name}\PY{l+s+s2}{\PYZdq{}}\PY{p}{:}\PY{l+s+s2}{\PYZdq{}}\PY{l+s+s2}{Bobby}\PY{l+s+s2}{\PYZdq{}}\PY{p}{,}\PY{l+s+s2}{\PYZdq{}}\PY{l+s+s2}{Age}\PY{l+s+s2}{\PYZdq{}}\PY{p}{:}\PY{l+m+mi}{25}\PY{p}{,}\PY{l+s+s2}{\PYZdq{}}\PY{l+s+s2}{Class}\PY{l+s+s2}{\PYZdq{}}\PY{p}{:}\PY{l+s+s2}{\PYZdq{}}\PY{l+s+s2}{First\PYZhy{}Year}\PY{l+s+s2}{\PYZdq{}}\PY{p}{,}\PY{l+s+s2}{\PYZdq{}}\PY{l+s+s2}{GPA}\PY{l+s+s2}{\PYZdq{}}\PY{p}{:}\PY{l+m+mf}{3.95}\PY{p}{\PYZcb{}}\PY{p}{,}
            \PY{p}{\PYZob{}}\PY{l+s+s2}{\PYZdq{}}\PY{l+s+s2}{Name}\PY{l+s+s2}{\PYZdq{}}\PY{p}{:}\PY{l+s+s2}{\PYZdq{}}\PY{l+s+s2}{Charlie}\PY{l+s+s2}{\PYZdq{}}\PY{p}{,}\PY{l+s+s2}{\PYZdq{}}\PY{l+s+s2}{Age}\PY{l+s+s2}{\PYZdq{}}\PY{p}{:}\PY{l+m+mi}{27}\PY{p}{,}\PY{l+s+s2}{\PYZdq{}}\PY{l+s+s2}{Class}\PY{l+s+s2}{\PYZdq{}}\PY{p}{:}\PY{l+s+s2}{\PYZdq{}}\PY{l+s+s2}{Second\PYZhy{}Year}\PY{l+s+s2}{\PYZdq{}}\PY{p}{,}\PY{l+s+s2}{\PYZdq{}}\PY{l+s+s2}{GPA}\PY{l+s+s2}{\PYZdq{}}\PY{p}{:}\PY{l+m+mf}{3.87}\PY{p}{\PYZcb{}}\PY{p}{,}
            \PY{p}{\PYZob{}}\PY{l+s+s2}{\PYZdq{}}\PY{l+s+s2}{Name}\PY{l+s+s2}{\PYZdq{}}\PY{p}{:}\PY{l+s+s2}{\PYZdq{}}\PY{l+s+s2}{David}\PY{l+s+s2}{\PYZdq{}}\PY{p}{,}\PY{l+s+s2}{\PYZdq{}}\PY{l+s+s2}{Age}\PY{l+s+s2}{\PYZdq{}}\PY{p}{:}\PY{l+m+mi}{26}\PY{p}{,}\PY{l+s+s2}{\PYZdq{}}\PY{l+s+s2}{Class}\PY{l+s+s2}{\PYZdq{}}\PY{p}{:}\PY{l+s+s2}{\PYZdq{}}\PY{l+s+s2}{First\PYZhy{}Year}\PY{l+s+s2}{\PYZdq{}}\PY{p}{,}\PY{l+s+s2}{\PYZdq{}}\PY{l+s+s2}{GPA}\PY{l+s+s2}{\PYZdq{}}\PY{p}{:}\PY{l+m+mi}{4}\PY{p}{\PYZcb{}}\PY{p}{]}

\PY{n}{myfile} \PY{o}{=} \PY{n+nb}{open}\PY{p}{(}\PY{l+s+s2}{\PYZdq{}}\PY{l+s+s2}{test\PYZus{}data/people\PYZus{}test.txt}\PY{l+s+s2}{\PYZdq{}}\PY{p}{,}\PY{l+s+s2}{\PYZdq{}}\PY{l+s+s2}{w}\PY{l+s+s2}{\PYZdq{}}\PY{p}{)}
\PY{n}{myfile}\PY{o}{.}\PY{n}{write}\PY{p}{(}\PY{l+s+sa}{f}\PY{l+s+s2}{\PYZdq{}}\PY{l+s+si}{\PYZob{}}\PY{l+s+s1}{\PYZsq{}}\PY{l+s+s1}{Name}\PY{l+s+s1}{\PYZsq{}}\PY{l+s+si}{:}\PY{l+s+s2}{\PYZlt{}20}\PY{l+s+si}{\PYZcb{}}\PY{l+s+s2}{ }\PY{l+s+si}{\PYZob{}}\PY{l+s+s1}{\PYZsq{}}\PY{l+s+s1}{Age}\PY{l+s+s1}{\PYZsq{}}\PY{l+s+si}{:}\PY{l+s+s2}{\PYZgt{}10}\PY{l+s+si}{\PYZcb{}}\PY{l+s+s2}{ }\PY{l+s+si}{\PYZob{}}\PY{l+s+s1}{\PYZsq{}}\PY{l+s+s1}{Class}\PY{l+s+s1}{\PYZsq{}}\PY{l+s+si}{:}\PY{l+s+s2}{\PYZlt{}20}\PY{l+s+si}{\PYZcb{}}\PY{l+s+s2}{ }\PY{l+s+si}{\PYZob{}}\PY{l+s+s1}{\PYZsq{}}\PY{l+s+s1}{GPA}\PY{l+s+s1}{\PYZsq{}}\PY{l+s+si}{:}\PY{l+s+s2}{\PYZgt{}10}\PY{l+s+si}{\PYZcb{}}\PY{l+s+se}{\PYZbs{}n}\PY{l+s+s2}{\PYZdq{}}\PY{p}{)}
\PY{n}{myfile}\PY{o}{.}\PY{n}{write}\PY{p}{(}\PY{l+s+s2}{\PYZdq{}}\PY{l+s+s2}{\PYZhy{}\PYZhy{}\PYZhy{}\PYZhy{}\PYZhy{}\PYZhy{}\PYZhy{}\PYZhy{}\PYZhy{}\PYZhy{}\PYZhy{}\PYZhy{}\PYZhy{}\PYZhy{}\PYZhy{}\PYZhy{}\PYZhy{}\PYZhy{}\PYZhy{}\PYZhy{}\PYZhy{}\PYZhy{}\PYZhy{}\PYZhy{}\PYZhy{}\PYZhy{}\PYZhy{}\PYZhy{}\PYZhy{}\PYZhy{}\PYZhy{}\PYZhy{}\PYZhy{}\PYZhy{}\PYZhy{}\PYZhy{}\PYZhy{}\PYZhy{}\PYZhy{}\PYZhy{}\PYZhy{}\PYZhy{}\PYZhy{}\PYZhy{}\PYZhy{}\PYZhy{}\PYZhy{}\PYZhy{}\PYZhy{}\PYZhy{}\PYZhy{}\PYZhy{}\PYZhy{}\PYZhy{}\PYZhy{}\PYZhy{}\PYZhy{}\PYZhy{}\PYZhy{}\PYZhy{}\PYZhy{}\PYZhy{}\PYZhy{}}\PY{l+s+se}{\PYZbs{}n}\PY{l+s+s2}{\PYZdq{}}\PY{p}{)}
\PY{k}{for} \PY{n}{person} \PY{o+ow}{in} \PY{n}{people}\PY{p}{:}
    \PY{n}{myfile}\PY{o}{.}\PY{n}{write}\PY{p}{(}\PY{l+s+sa}{f}\PY{l+s+s2}{\PYZdq{}}\PY{l+s+si}{\PYZob{}}\PY{n}{person}\PY{p}{[}\PY{l+s+s1}{\PYZsq{}}\PY{l+s+s1}{Name}\PY{l+s+s1}{\PYZsq{}}\PY{p}{]}\PY{l+s+si}{:}\PY{l+s+s2}{\PYZlt{}20}\PY{l+s+si}{\PYZcb{}}\PY{l+s+s2}{ }\PY{l+s+si}{\PYZob{}}\PY{n}{person}\PY{p}{[}\PY{l+s+s1}{\PYZsq{}}\PY{l+s+s1}{Age}\PY{l+s+s1}{\PYZsq{}}\PY{p}{]}\PY{l+s+si}{:}\PY{l+s+s2}{\PYZgt{}10}\PY{l+s+si}{\PYZcb{}}\PY{l+s+s2}{ }\PY{l+s+si}{\PYZob{}}\PY{n}{person}\PY{p}{[}\PY{l+s+s1}{\PYZsq{}}\PY{l+s+s1}{Class}\PY{l+s+s1}{\PYZsq{}}\PY{p}{]}\PY{l+s+si}{:}\PY{l+s+s2}{\PYZlt{}20}\PY{l+s+si}{\PYZcb{}}\PY{l+s+s2}{ }\PY{l+s+si}{\PYZob{}}\PY{n}{person}\PY{p}{[}\PY{l+s+s1}{\PYZsq{}}\PY{l+s+s1}{GPA}\PY{l+s+s1}{\PYZsq{}}\PY{p}{]}\PY{l+s+si}{:}\PY{l+s+s2}{\PYZgt{}10.4f}\PY{l+s+si}{\PYZcb{}}\PY{l+s+se}{\PYZbs{}n}\PY{l+s+s2}{\PYZdq{}}\PY{p}{)}
\PY{n}{myfile}\PY{o}{.}\PY{n}{close}\PY{p}{(}\PY{p}{)}
\end{Verbatim}
\end{tcolorbox}

    If you open the file we've just created, it should look like this:

\begin{verbatim}
Name                        Age Class                       GPA
-------------------- ---------- -------------------- ----------
Bobby                        25 First-Year               3.9500
Charlie                      27 Second-Year              3.8700
David                        26 First-Year               4.0000
\end{verbatim}

Notice that the first column is left justified and has empty spaces out
to the specified length of 20 characters. Next, ``Age'' is
right-justified with 10 spaces total. Class is left-justified again with
twenty, and GPA is right-justified \emph{with trailing zeros}, which we
specified with the \texttt{.4f} addition, indicating we wanted four
decimal places and that the value is to be treated like a float. I also
included some blank spaces between each of the variables being printed
1) to show you where each section begins and ends, and 2) to illustrate
how you should be careful when using justifications so you don't wind up
with Age and Class stuck together as a single value.

Also notice that there is a \texttt{\textbackslash{}n} character at the
end of each line in the code. This ensures that the text file is
properly separated by lines - python won't write what you don't tell it
to write.

What if we want to add to a file that already exists? Simple, use
``append mode'', or open with ``a''.

    \begin{tcolorbox}[breakable, size=fbox, boxrule=1pt, pad at break*=1mm,colback=cellbackground, colframe=cellborder]
\prompt{In}{incolor}{12}{\boxspacing}
\begin{Verbatim}[commandchars=\\\{\}]
\PY{n}{people} \PY{o}{=} \PY{p}{[}  \PY{p}{\PYZob{}}\PY{l+s+s2}{\PYZdq{}}\PY{l+s+s2}{Name}\PY{l+s+s2}{\PYZdq{}}\PY{p}{:}\PY{l+s+s2}{\PYZdq{}}\PY{l+s+s2}{Esther}\PY{l+s+s2}{\PYZdq{}}\PY{p}{,}\PY{l+s+s2}{\PYZdq{}}\PY{l+s+s2}{Age}\PY{l+s+s2}{\PYZdq{}}\PY{p}{:}\PY{l+m+mi}{24}\PY{p}{,}\PY{l+s+s2}{\PYZdq{}}\PY{l+s+s2}{Class}\PY{l+s+s2}{\PYZdq{}}\PY{p}{:}\PY{l+s+s2}{\PYZdq{}}\PY{l+s+s2}{First\PYZhy{}Year}\PY{l+s+s2}{\PYZdq{}}\PY{p}{,}\PY{l+s+s2}{\PYZdq{}}\PY{l+s+s2}{GPA}\PY{l+s+s2}{\PYZdq{}}\PY{p}{:}\PY{l+m+mf}{3.96}\PY{p}{\PYZcb{}}\PY{p}{,}
            \PY{p}{\PYZob{}}\PY{l+s+s2}{\PYZdq{}}\PY{l+s+s2}{Name}\PY{l+s+s2}{\PYZdq{}}\PY{p}{:}\PY{l+s+s2}{\PYZdq{}}\PY{l+s+s2}{Frances}\PY{l+s+s2}{\PYZdq{}}\PY{p}{,}\PY{l+s+s2}{\PYZdq{}}\PY{l+s+s2}{Age}\PY{l+s+s2}{\PYZdq{}}\PY{p}{:}\PY{l+m+mi}{28}\PY{p}{,}\PY{l+s+s2}{\PYZdq{}}\PY{l+s+s2}{Class}\PY{l+s+s2}{\PYZdq{}}\PY{p}{:}\PY{l+s+s2}{\PYZdq{}}\PY{l+s+s2}{Second\PYZhy{}Year}\PY{l+s+s2}{\PYZdq{}}\PY{p}{,}\PY{l+s+s2}{\PYZdq{}}\PY{l+s+s2}{GPA}\PY{l+s+s2}{\PYZdq{}}\PY{p}{:}\PY{l+m+mf}{3.99}\PY{p}{\PYZcb{}}\PY{p}{,}
            \PY{p}{\PYZob{}}\PY{l+s+s2}{\PYZdq{}}\PY{l+s+s2}{Name}\PY{l+s+s2}{\PYZdq{}}\PY{p}{:}\PY{l+s+s2}{\PYZdq{}}\PY{l+s+s2}{Gloria}\PY{l+s+s2}{\PYZdq{}}\PY{p}{,}\PY{l+s+s2}{\PYZdq{}}\PY{l+s+s2}{Age}\PY{l+s+s2}{\PYZdq{}}\PY{p}{:}\PY{l+m+mi}{27}\PY{p}{,}\PY{l+s+s2}{\PYZdq{}}\PY{l+s+s2}{Class}\PY{l+s+s2}{\PYZdq{}}\PY{p}{:}\PY{l+s+s2}{\PYZdq{}}\PY{l+s+s2}{First\PYZhy{}Year}\PY{l+s+s2}{\PYZdq{}}\PY{p}{,}\PY{l+s+s2}{\PYZdq{}}\PY{l+s+s2}{GPA}\PY{l+s+s2}{\PYZdq{}}\PY{p}{:}\PY{l+m+mi}{4}\PY{p}{\PYZcb{}}\PY{p}{]}

\PY{n}{myfile} \PY{o}{=} \PY{n+nb}{open}\PY{p}{(}\PY{l+s+s2}{\PYZdq{}}\PY{l+s+s2}{test\PYZus{}data/people\PYZus{}test.txt}\PY{l+s+s2}{\PYZdq{}}\PY{p}{,}\PY{l+s+s2}{\PYZdq{}}\PY{l+s+s2}{a}\PY{l+s+s2}{\PYZdq{}}\PY{p}{)}
\PY{k}{for} \PY{n}{person} \PY{o+ow}{in} \PY{n}{people}\PY{p}{:}
    \PY{n}{myfile}\PY{o}{.}\PY{n}{write}\PY{p}{(}\PY{l+s+sa}{f}\PY{l+s+s2}{\PYZdq{}}\PY{l+s+si}{\PYZob{}}\PY{n}{person}\PY{p}{[}\PY{l+s+s1}{\PYZsq{}}\PY{l+s+s1}{Name}\PY{l+s+s1}{\PYZsq{}}\PY{p}{]}\PY{l+s+si}{:}\PY{l+s+s2}{\PYZlt{}20}\PY{l+s+si}{\PYZcb{}}\PY{l+s+s2}{ }\PY{l+s+si}{\PYZob{}}\PY{n}{person}\PY{p}{[}\PY{l+s+s1}{\PYZsq{}}\PY{l+s+s1}{Age}\PY{l+s+s1}{\PYZsq{}}\PY{p}{]}\PY{l+s+si}{:}\PY{l+s+s2}{\PYZgt{}10}\PY{l+s+si}{\PYZcb{}}\PY{l+s+s2}{ }\PY{l+s+si}{\PYZob{}}\PY{n}{person}\PY{p}{[}\PY{l+s+s1}{\PYZsq{}}\PY{l+s+s1}{Class}\PY{l+s+s1}{\PYZsq{}}\PY{p}{]}\PY{l+s+si}{:}\PY{l+s+s2}{\PYZlt{}20}\PY{l+s+si}{\PYZcb{}}\PY{l+s+s2}{ }\PY{l+s+si}{\PYZob{}}\PY{n}{person}\PY{p}{[}\PY{l+s+s1}{\PYZsq{}}\PY{l+s+s1}{GPA}\PY{l+s+s1}{\PYZsq{}}\PY{p}{]}\PY{l+s+si}{:}\PY{l+s+s2}{\PYZgt{}10.4f}\PY{l+s+si}{\PYZcb{}}\PY{l+s+se}{\PYZbs{}n}\PY{l+s+s2}{\PYZdq{}}\PY{p}{)}
\PY{n}{myfile}\PY{o}{.}\PY{n}{close}\PY{p}{(}\PY{p}{)}
\end{Verbatim}
\end{tcolorbox}

    Now if we reopen the file we created, we can see the addition of the
next three people to the list! This can be useful for things like
logging results over the course of a longer calculation, where you might
not want the file to be kept open and in memory for the entire time, but
still want to add to it on occasion.

File I/O can help you find, organize, and process your data in a way
that makes sense to you.
\section{System Interaction}
One of the included modules in python is the \texttt{subprocess} module,
which allows you to execute shell commands on a Linux system from within
your python script. You can also use it to read the outputs from these
commands. There are a few different options available to manage these
kinds of system/program interactions.

\hypertarget{subprocess.call}{%
\paragraph{subprocess.call()}\label{subprocess.call}}

The \texttt{subprocess.call()} function is fairly straightforward. It is
useful if you need to execute a command without concern for the output
to the terminal of that command.

    \begin{tcolorbox}[breakable, size=fbox, boxrule=1pt, pad at break*=1mm,colback=cellbackground, colframe=cellborder]
\prompt{In}{incolor}{1}{\boxspacing}
\begin{Verbatim}[commandchars=\\\{\}]
\PY{k+kn}{import} \PY{n+nn}{subprocess}

\PY{n}{subprocess}\PY{o}{.}\PY{n}{call}\PY{p}{(}\PY{l+s+s2}{\PYZdq{}}\PY{l+s+s2}{ls \PYZhy{}lrth}\PY{l+s+s2}{\PYZdq{}}\PY{p}{,}\PY{n}{shell}\PY{o}{=}\PY{k+kc}{True}\PY{p}{)}
\end{Verbatim}
\end{tcolorbox}

    \begin{Verbatim}[commandchars=\\\{\}]
total 48K
-rw-rw-r-- 1 mark mark    0 Jun 13 16:37 19\_Python\_System\_Interaction.ipynb
-rw-rw-r-- 1 mark mark    0 Jun 13 16:38 20\_Python\_Final\_Project.ipynb
drwxrwxr-x 2 mark mark 4.0K Jun 17 16:45 test\_data
-rw-rw-r-- 1 mark mark  33K Jun 17 17:01 17\_Python\_File\_IO.ipynb
-rw-rw-r-- 1 mark mark 4.8K Jun 17 17:35 18\_Python\_Error\_Handling.ipynb
    \end{Verbatim}

            \begin{tcolorbox}[breakable, size=fbox, boxrule=.5pt, pad at break*=1mm, opacityfill=0]
\prompt{Out}{outcolor}{1}{\boxspacing}
\begin{Verbatim}[commandchars=\\\{\}]
0
\end{Verbatim}
\end{tcolorbox}
        
    In a notebook environment, the results of the \texttt{subprocess.call()}
command are simply printed to the output of the cell. In a regular
system environment, this output is printed to the terminal. The
important thing to remember, however, is that neither of those options
allow for the capture and storage of the output by python. In the
example above, we as a user can see the output of our \texttt{ls\ -lrth}
command, but our \textbf{program} cannot.

I often use the \texttt{call()} function when calling external programs
from within my python scripts, specifically when those programs will
produce their own output files that I can use later. In these scenarios,
the outputs to the terminal are not as important.

Also, note that \texttt{subprocess.call} includes a keyword argument
\texttt{shell=True}. This is \emph{required} to have the command
actually run in your shell environment.

\hypertarget{subprocess.popen}{%
\paragraph{subprocess.Popen()}\label{subprocess.popen}}

The other kind of system interaction requires the capture and storage of
the terminal outputs. The \texttt{Popen} command can be used in the same
way as \texttt{call}, but with some additional keywords and with the
functionality included to capture the response from the commands.

    \begin{tcolorbox}[breakable, size=fbox, boxrule=1pt, pad at break*=1mm,colback=cellbackground, colframe=cellborder]
\prompt{In}{incolor}{2}{\boxspacing}
\begin{Verbatim}[commandchars=\\\{\}]
\PY{k+kn}{import} \PY{n+nn}{subprocess}

\PY{n}{proc} \PY{o}{=} \PY{n}{subprocess}\PY{o}{.}\PY{n}{Popen}\PY{p}{(}\PY{l+s+s2}{\PYZdq{}}\PY{l+s+s2}{ls \PYZhy{}lrth}\PY{l+s+s2}{\PYZdq{}}\PY{p}{,}\PY{n}{shell}\PY{o}{=}\PY{k+kc}{True}\PY{p}{,}\PY{n}{stdout}\PY{o}{=}\PY{n}{subprocess}\PY{o}{.}\PY{n}{PIPE}\PY{p}{,}\PY{n}{stderr}\PY{o}{=}\PY{n}{subprocess}\PY{o}{.}\PY{n}{PIPE}\PY{p}{)}
\PY{n}{output}\PY{p}{,}\PY{n}{errors} \PY{o}{=} \PY{n}{proc}\PY{o}{.}\PY{n}{communicate}\PY{p}{(}\PY{p}{)}
\end{Verbatim}
\end{tcolorbox}

    \begin{tcolorbox}[breakable, size=fbox, boxrule=1pt, pad at break*=1mm,colback=cellbackground, colframe=cellborder]
\prompt{In}{incolor}{8}{\boxspacing}
\begin{Verbatim}[commandchars=\\\{\}]
\PY{n+nb}{print}\PY{p}{(}\PY{n}{output}\PY{p}{)}
\end{Verbatim}
\end{tcolorbox}

    \begin{Verbatim}[commandchars=\\\{\}]
b'total 48K\textbackslash{}n-rw-rw-r-- 1 mark mark    0 Jun 13 16:37
19\_Python\_System\_Interaction.ipynb\textbackslash{}n-rw-rw-r-- 1 mark mark    0 Jun 13 16:38
20\_Python\_Final\_Project.ipynb\textbackslash{}ndrwxrwxr-x 2 mark mark 4.0K Jun 17 16:45
test\_data\textbackslash{}n-rw-rw-r-- 1 mark mark  33K Jun 17 17:01
17\_Python\_File\_IO.ipynb\textbackslash{}n-rw-rw-r-- 1 mark mark 4.8K Jun 17 17:35
18\_Python\_Error\_Handling.ipynb\textbackslash{}n'
    \end{Verbatim}

    \begin{tcolorbox}[breakable, size=fbox, boxrule=1pt, pad at break*=1mm,colback=cellbackground, colframe=cellborder]
\prompt{In}{incolor}{9}{\boxspacing}
\begin{Verbatim}[commandchars=\\\{\}]
\PY{n+nb}{print}\PY{p}{(}\PY{n}{errors}\PY{p}{)}
\end{Verbatim}
\end{tcolorbox}

    \begin{Verbatim}[commandchars=\\\{\}]
b''
    \end{Verbatim}

    You will note that when the \texttt{output} and \texttt{errors}
variables are printed, they have a \texttt{b} at the front of the
string. This is an important thing to be aware of - these outputs are
\emph{technically} in binary format, not plain-text. Fortunately, the
python \texttt{print} function can translate from binary to text.
However, working with binary data can be slightly more difficult in some
ways, and it might be easier for us to convert to plain strings before
parsing information. We can use the \texttt{decode} function for this.

    \begin{tcolorbox}[breakable, size=fbox, boxrule=1pt, pad at break*=1mm,colback=cellbackground, colframe=cellborder]
\prompt{In}{incolor}{12}{\boxspacing}
\begin{Verbatim}[commandchars=\\\{\}]
\PY{n}{decoded} \PY{o}{=} \PY{n}{output}\PY{o}{.}\PY{n}{decode}\PY{p}{(}\PY{l+s+s2}{\PYZdq{}}\PY{l+s+s2}{utf\PYZhy{}8}\PY{l+s+s2}{\PYZdq{}}\PY{p}{)}

\PY{n+nb}{print}\PY{p}{(}\PY{n}{decoded}\PY{p}{)}
\end{Verbatim}
\end{tcolorbox}

    \begin{Verbatim}[commandchars=\\\{\}]
total 48K
-rw-rw-r-- 1 mark mark    0 Jun 13 16:37 19\_Python\_System\_Interaction.ipynb
-rw-rw-r-- 1 mark mark    0 Jun 13 16:38 20\_Python\_Final\_Project.ipynb
drwxrwxr-x 2 mark mark 4.0K Jun 17 16:45 test\_data
-rw-rw-r-- 1 mark mark  33K Jun 17 17:01 17\_Python\_File\_IO.ipynb
-rw-rw-r-- 1 mark mark 4.8K Jun 17 17:35 18\_Python\_Error\_Handling.ipynb

    \end{Verbatim}

    The variable \texttt{output} that we got from the \texttt{communicate}
function has its own internal function called \texttt{decode}, which we
call with the argument \texttt{"utf-8"}, indicating we want to use that
particular Unicode formatting. Note that the resulting string does NOT
have the \texttt{b} at the front, and is printed like any other string.
We can also use this data like any other string, with \texttt{split}
commands and other types of parsing to get the information we want from
the output.

Anyone who has looked into the \texttt{TCFreeze} code will recognize the
\texttt{subprocess} module as an integral part of the larger algorithm.
\texttt{TCFreeze} uses \texttt{subprocess.call()} to use TeraChem at
every iteration of the process.

Try out some of your own commands to get a feel for how the different
interactions work.
\section{Library - Matplotlib}
Let's assume you've got some data. Good data that your advisor will be
pleased with. Mythical Data. How should you present this data?

\begin{enumerate}
\def\labelenumi{\arabic{enumi}.}
\tightlist
\item
  Prepare the figure environment with \texttt{matplotlib.pyplot}
\end{enumerate}

    \begin{tcolorbox}[breakable, size=fbox, boxrule=1pt, pad at break*=1mm,colback=cellbackground, colframe=cellborder]
\prompt{In}{incolor}{1}{\boxspacing}
\begin{Verbatim}[commandchars=\\\{\}]
\PY{k+kn}{import} \PY{n+nn}{matplotlib}\PY{n+nn}{.}\PY{n+nn}{pyplot} \PY{k}{as} \PY{n+nn}{plt}
\PY{k+kn}{import} \PY{n+nn}{numpy} \PY{k}{as} \PY{n+nn}{np}

\PY{c+c1}{\PYZsh{}\PYZsh{}\PYZsh{}\PYZsh{} GENERATING RANDOM DATA!}
\PY{n}{my\PYZus{}data}\PY{o}{=}\PY{n}{np}\PY{o}{.}\PY{n}{random}\PY{o}{.}\PY{n}{rand}\PY{p}{(}\PY{l+m+mi}{10000}\PY{p}{)} \PY{c+c1}{\PYZsh{} this is just a dataset of 10,000 random values between 0 and 1}

\PY{n}{fig} \PY{o}{=} \PY{n}{plt}\PY{o}{.}\PY{n}{figure}\PY{p}{(}\PY{n}{figsize}\PY{o}{=}\PY{p}{[}\PY{l+m+mi}{3}\PY{p}{,}\PY{l+m+mi}{2}\PY{p}{]}\PY{p}{,}\PY{n}{dpi}\PY{o}{=}\PY{l+m+mi}{150}\PY{p}{)} \PY{c+c1}{\PYZsh{} this initializes a Figure with the size 3 in. wide by 2 in. tall at 300dpi}
\PY{n}{ax} \PY{o}{=} \PY{n}{fig}\PY{o}{.}\PY{n}{add\PYZus{}subplot}\PY{p}{(}\PY{l+m+mi}{1}\PY{p}{,}\PY{l+m+mi}{1}\PY{p}{,}\PY{l+m+mi}{1}\PY{p}{)} \PY{c+c1}{\PYZsh{} this adds a set of axes as a subplot to the figure.  }

\PY{c+c1}{\PYZsh{} The numbers 1,1,1 indicate the arrangement of subplots should be 1 row, 1 column, and the current axes are in cell 1.}
\PY{c+c1}{\PYZsh{} if we used (3,2,4), this would be 3 rows, 2 columns, and in cell 4.  Cells are numbered left\PYZhy{}to\PYZhy{}right, then top\PYZhy{}to\PYZhy{}bottom.}
\PY{c+c1}{\PYZsh{} [ ] [ ] }
\PY{c+c1}{\PYZsh{} [ ] [X]}
\PY{c+c1}{\PYZsh{} [ ] [ ]}
\PY{c+c1}{\PYZsh{} This is where cell 4 would be in the (3,2,4) example.}

\PY{n}{ax}\PY{o}{.}\PY{n}{plot}\PY{p}{(}\PY{n}{my\PYZus{}data}\PY{p}{)}
\end{Verbatim}
\end{tcolorbox}

            \begin{tcolorbox}[breakable, size=fbox, boxrule=.5pt, pad at break*=1mm, opacityfill=0]
\prompt{Out}{outcolor}{1}{\boxspacing}
\begin{Verbatim}[commandchars=\\\{\}]
[<matplotlib.lines.Line2D at 0x7f775e5faa90>]
\end{Verbatim}
\end{tcolorbox}
        
    \begin{center}
    \adjustimage{max size={0.9\linewidth}{0.9\paperheight}}{output_1_1.png}
    \end{center}
    { \hspace*{\fill} \\}
    
    Lovely.

Let's mess with the x- and y-axis limits

    \begin{tcolorbox}[breakable, size=fbox, boxrule=1pt, pad at break*=1mm,colback=cellbackground, colframe=cellborder]
\prompt{In}{incolor}{2}{\boxspacing}
\begin{Verbatim}[commandchars=\\\{\}]
\PY{n}{fig} \PY{o}{=} \PY{n}{plt}\PY{o}{.}\PY{n}{figure}\PY{p}{(}\PY{n}{figsize}\PY{o}{=}\PY{p}{[}\PY{l+m+mi}{3}\PY{p}{,}\PY{l+m+mi}{2}\PY{p}{]}\PY{p}{,}\PY{n}{dpi}\PY{o}{=}\PY{l+m+mi}{150}\PY{p}{)} 
\PY{n}{ax} \PY{o}{=} \PY{n}{fig}\PY{o}{.}\PY{n}{add\PYZus{}subplot}\PY{p}{(}\PY{l+m+mi}{1}\PY{p}{,}\PY{l+m+mi}{1}\PY{p}{,}\PY{l+m+mi}{1}\PY{p}{)}
\PY{n}{ax}\PY{o}{.}\PY{n}{plot}\PY{p}{(}\PY{n}{my\PYZus{}data}\PY{p}{)}
\PY{c+c1}{\PYZsh{}\PYZsh{}\PYZsh{}\PYZsh{} NEW FUNCTIONALITY!}
\PY{n}{ax}\PY{o}{.}\PY{n}{set\PYZus{}xlim}\PY{p}{(}\PY{l+m+mi}{0}\PY{p}{,}\PY{l+m+mi}{10000}\PY{p}{)}
\PY{n}{ax}\PY{o}{.}\PY{n}{set\PYZus{}ylim}\PY{p}{(}\PY{l+m+mi}{0}\PY{p}{,}\PY{l+m+mi}{1}\PY{p}{)}
\end{Verbatim}
\end{tcolorbox}

            \begin{tcolorbox}[breakable, size=fbox, boxrule=.5pt, pad at break*=1mm, opacityfill=0]
\prompt{Out}{outcolor}{2}{\boxspacing}
\begin{Verbatim}[commandchars=\\\{\}]
(0.0, 1.0)
\end{Verbatim}
\end{tcolorbox}
        
    \begin{center}
    \adjustimage{max size={0.9\linewidth}{0.9\paperheight}}{output_3_1.png}
    \end{center}
    { \hspace*{\fill} \\}
    
    Cool. Now we can adjust the axes to ensure a) our data is presented how
we want it, and b) we don't have empty space around it, which can be
confusing sometimes. Now let's adjust the color, thickness, and style of
the line by adding keywords to the \texttt{ax.plot} function call. I'll
also adjust the x-axis down to \texttt{{[}0,100{]}} to let us see the
line a little more easily - we'll address other methods of dealing with
this later. We can also mess with the figure size.

    \begin{tcolorbox}[breakable, size=fbox, boxrule=1pt, pad at break*=1mm,colback=cellbackground, colframe=cellborder]
\prompt{In}{incolor}{3}{\boxspacing}
\begin{Verbatim}[commandchars=\\\{\}]
\PY{n}{fig} \PY{o}{=} \PY{n}{plt}\PY{o}{.}\PY{n}{figure}\PY{p}{(}\PY{n}{figsize}\PY{o}{=}\PY{p}{[}\PY{l+m+mi}{8}\PY{p}{,}\PY{l+m+mi}{4}\PY{p}{]}\PY{p}{,}\PY{n}{dpi}\PY{o}{=}\PY{l+m+mi}{150}\PY{p}{)} 
\PY{n}{ax} \PY{o}{=} \PY{n}{fig}\PY{o}{.}\PY{n}{add\PYZus{}subplot}\PY{p}{(}\PY{l+m+mi}{1}\PY{p}{,}\PY{l+m+mi}{1}\PY{p}{,}\PY{l+m+mi}{1}\PY{p}{)}
\PY{n}{ax}\PY{o}{.}\PY{n}{set\PYZus{}xlim}\PY{p}{(}\PY{l+m+mi}{0}\PY{p}{,}\PY{l+m+mi}{100}\PY{p}{)}
\PY{n}{ax}\PY{o}{.}\PY{n}{set\PYZus{}ylim}\PY{p}{(}\PY{l+m+mi}{0}\PY{p}{,}\PY{l+m+mi}{1}\PY{p}{)}
\PY{c+c1}{\PYZsh{}\PYZsh{}\PYZsh{}\PYZsh{} NEW FUNCTIONALITY!}
\PY{n}{ax}\PY{o}{.}\PY{n}{plot}\PY{p}{(}\PY{n}{my\PYZus{}data}\PY{p}{,}\PY{n}{color}\PY{o}{=}\PY{l+s+s2}{\PYZdq{}}\PY{l+s+s2}{red}\PY{l+s+s2}{\PYZdq{}}\PY{p}{,}\PY{n}{lw}\PY{o}{=}\PY{l+m+mf}{0.5}\PY{p}{,}\PY{n}{linestyle}\PY{o}{=}\PY{l+s+s2}{\PYZdq{}}\PY{l+s+s2}{\PYZhy{}\PYZhy{}}\PY{l+s+s2}{\PYZdq{}}\PY{p}{)}
\PY{n}{ax}\PY{o}{.}\PY{n}{plot}\PY{p}{(}\PY{l+m+mi}{1}\PY{o}{\PYZhy{}}\PY{n}{my\PYZus{}data}\PY{p}{,}\PY{n}{color}\PY{o}{=}\PY{l+s+s2}{\PYZdq{}}\PY{l+s+s2}{blue}\PY{l+s+s2}{\PYZdq{}}\PY{p}{,}\PY{n}{lw}\PY{o}{=}\PY{l+m+mf}{0.5}\PY{p}{,}\PY{n}{linestyle}\PY{o}{=}\PY{l+s+s2}{\PYZdq{}}\PY{l+s+s2}{:}\PY{l+s+s2}{\PYZdq{}}\PY{p}{)}
\end{Verbatim}
\end{tcolorbox}

            \begin{tcolorbox}[breakable, size=fbox, boxrule=.5pt, pad at break*=1mm, opacityfill=0]
\prompt{Out}{outcolor}{3}{\boxspacing}
\begin{Verbatim}[commandchars=\\\{\}]
[<matplotlib.lines.Line2D at 0x7f775dc7a8e0>]
\end{Verbatim}
\end{tcolorbox}
        
    \begin{center}
    \adjustimage{max size={0.9\linewidth}{0.9\paperheight}}{output_5_1.png}
    \end{center}
    { \hspace*{\fill} \\}
    
    We can also add titles to each axis, to the subplot, or to the entire
figure.

    \begin{tcolorbox}[breakable, size=fbox, boxrule=1pt, pad at break*=1mm,colback=cellbackground, colframe=cellborder]
\prompt{In}{incolor}{7}{\boxspacing}
\begin{Verbatim}[commandchars=\\\{\}]
\PY{n}{fig} \PY{o}{=} \PY{n}{plt}\PY{o}{.}\PY{n}{figure}\PY{p}{(}\PY{n}{figsize}\PY{o}{=}\PY{p}{[}\PY{l+m+mi}{5}\PY{p}{,}\PY{l+m+mi}{4}\PY{p}{]}\PY{p}{,}\PY{n}{dpi}\PY{o}{=}\PY{l+m+mi}{150}\PY{p}{)} 

\PY{n}{ax} \PY{o}{=} \PY{n}{fig}\PY{o}{.}\PY{n}{add\PYZus{}subplot}\PY{p}{(}\PY{l+m+mi}{2}\PY{p}{,}\PY{l+m+mi}{2}\PY{p}{,}\PY{l+m+mi}{1}\PY{p}{)}
\PY{n}{ax}\PY{o}{.}\PY{n}{plot}\PY{p}{(}\PY{n}{my\PYZus{}data}\PY{p}{,}\PY{n}{color}\PY{o}{=}\PY{l+s+s2}{\PYZdq{}}\PY{l+s+s2}{red}\PY{l+s+s2}{\PYZdq{}}\PY{p}{)}
\PY{n}{ax}\PY{o}{.}\PY{n}{set\PYZus{}xlim}\PY{p}{(}\PY{l+m+mi}{0}\PY{p}{,}\PY{l+m+mi}{10000}\PY{p}{)}
\PY{n}{ax}\PY{o}{.}\PY{n}{set\PYZus{}ylim}\PY{p}{(}\PY{l+m+mi}{0}\PY{p}{,}\PY{l+m+mi}{1}\PY{p}{)}
\PY{c+c1}{\PYZsh{}\PYZsh{}\PYZsh{}\PYZsh{} NEW FUNCTIONALITY!}
\PY{n}{ax}\PY{o}{.}\PY{n}{set\PYZus{}xlabel}\PY{p}{(}\PY{l+s+s2}{\PYZdq{}}\PY{l+s+s2}{x\PYZhy{}axis}\PY{l+s+s2}{\PYZdq{}}\PY{p}{)}
\PY{n}{ax}\PY{o}{.}\PY{n}{set\PYZus{}ylabel}\PY{p}{(}\PY{l+s+s2}{\PYZdq{}}\PY{l+s+s2}{y\PYZhy{}axis}\PY{l+s+s2}{\PYZdq{}}\PY{p}{)}
\PY{n}{ax}\PY{o}{.}\PY{n}{set\PYZus{}title}\PY{p}{(}\PY{l+s+s2}{\PYZdq{}}\PY{l+s+s2}{subplot 1}\PY{l+s+s2}{\PYZdq{}}\PY{p}{)}

\PY{n}{ax} \PY{o}{=} \PY{n}{fig}\PY{o}{.}\PY{n}{add\PYZus{}subplot}\PY{p}{(}\PY{l+m+mi}{2}\PY{p}{,}\PY{l+m+mi}{2}\PY{p}{,}\PY{l+m+mi}{2}\PY{p}{)}
\PY{n}{ax}\PY{o}{.}\PY{n}{plot}\PY{p}{(}\PY{n}{my\PYZus{}data}\PY{p}{,}\PY{n}{color}\PY{o}{=}\PY{l+s+s2}{\PYZdq{}}\PY{l+s+s2}{green}\PY{l+s+s2}{\PYZdq{}}\PY{p}{)}
\PY{n}{ax}\PY{o}{.}\PY{n}{set\PYZus{}xlim}\PY{p}{(}\PY{l+m+mi}{0}\PY{p}{,}\PY{l+m+mi}{10000}\PY{p}{)}
\PY{n}{ax}\PY{o}{.}\PY{n}{set\PYZus{}ylim}\PY{p}{(}\PY{l+m+mi}{0}\PY{p}{,}\PY{l+m+mi}{1}\PY{p}{)}
\PY{c+c1}{\PYZsh{}\PYZsh{}\PYZsh{}\PYZsh{} NEW FUNCTIONALITY!}
\PY{n}{ax}\PY{o}{.}\PY{n}{set\PYZus{}xlabel}\PY{p}{(}\PY{l+s+s2}{\PYZdq{}}\PY{l+s+s2}{x\PYZhy{}axis}\PY{l+s+s2}{\PYZdq{}}\PY{p}{)}
\PY{n}{ax}\PY{o}{.}\PY{n}{set\PYZus{}ylabel}\PY{p}{(}\PY{l+s+s2}{\PYZdq{}}\PY{l+s+s2}{y\PYZhy{}axis}\PY{l+s+s2}{\PYZdq{}}\PY{p}{)}
\PY{n}{ax}\PY{o}{.}\PY{n}{set\PYZus{}title}\PY{p}{(}\PY{l+s+s2}{\PYZdq{}}\PY{l+s+s2}{subplot 2}\PY{l+s+s2}{\PYZdq{}}\PY{p}{)}

\PY{n}{ax} \PY{o}{=} \PY{n}{fig}\PY{o}{.}\PY{n}{add\PYZus{}subplot}\PY{p}{(}\PY{l+m+mi}{2}\PY{p}{,}\PY{l+m+mi}{2}\PY{p}{,}\PY{l+m+mi}{3}\PY{p}{)}
\PY{n}{ax}\PY{o}{.}\PY{n}{plot}\PY{p}{(}\PY{n}{my\PYZus{}data}\PY{p}{,}\PY{n}{color}\PY{o}{=}\PY{l+s+s2}{\PYZdq{}}\PY{l+s+s2}{orange}\PY{l+s+s2}{\PYZdq{}}\PY{p}{)}
\PY{n}{ax}\PY{o}{.}\PY{n}{set\PYZus{}xlim}\PY{p}{(}\PY{l+m+mi}{0}\PY{p}{,}\PY{l+m+mi}{10000}\PY{p}{)}
\PY{n}{ax}\PY{o}{.}\PY{n}{set\PYZus{}ylim}\PY{p}{(}\PY{l+m+mi}{0}\PY{p}{,}\PY{l+m+mi}{1}\PY{p}{)}
\PY{c+c1}{\PYZsh{}\PYZsh{}\PYZsh{}\PYZsh{} NEW FUNCTIONALITY!}
\PY{n}{ax}\PY{o}{.}\PY{n}{set\PYZus{}xlabel}\PY{p}{(}\PY{l+s+s2}{\PYZdq{}}\PY{l+s+s2}{x\PYZhy{}axis}\PY{l+s+s2}{\PYZdq{}}\PY{p}{)}
\PY{n}{ax}\PY{o}{.}\PY{n}{set\PYZus{}ylabel}\PY{p}{(}\PY{l+s+s2}{\PYZdq{}}\PY{l+s+s2}{y\PYZhy{}axis}\PY{l+s+s2}{\PYZdq{}}\PY{p}{)}
\PY{n}{ax}\PY{o}{.}\PY{n}{set\PYZus{}title}\PY{p}{(}\PY{l+s+s2}{\PYZdq{}}\PY{l+s+s2}{subplot 3}\PY{l+s+s2}{\PYZdq{}}\PY{p}{)}

\PY{n}{ax} \PY{o}{=} \PY{n}{fig}\PY{o}{.}\PY{n}{add\PYZus{}subplot}\PY{p}{(}\PY{l+m+mi}{2}\PY{p}{,}\PY{l+m+mi}{2}\PY{p}{,}\PY{l+m+mi}{4}\PY{p}{)}
\PY{n}{ax}\PY{o}{.}\PY{n}{plot}\PY{p}{(}\PY{n}{my\PYZus{}data}\PY{p}{,}\PY{n}{color}\PY{o}{=}\PY{l+s+s2}{\PYZdq{}}\PY{l+s+s2}{blue}\PY{l+s+s2}{\PYZdq{}}\PY{p}{)}
\PY{n}{ax}\PY{o}{.}\PY{n}{set\PYZus{}xlim}\PY{p}{(}\PY{l+m+mi}{0}\PY{p}{,}\PY{l+m+mi}{10000}\PY{p}{)}
\PY{n}{ax}\PY{o}{.}\PY{n}{set\PYZus{}ylim}\PY{p}{(}\PY{l+m+mi}{0}\PY{p}{,}\PY{l+m+mi}{1}\PY{p}{)}
\PY{c+c1}{\PYZsh{}\PYZsh{}\PYZsh{}\PYZsh{} NEW FUNCTIONALITY!}
\PY{n}{ax}\PY{o}{.}\PY{n}{set\PYZus{}xlabel}\PY{p}{(}\PY{l+s+s2}{\PYZdq{}}\PY{l+s+s2}{x\PYZhy{}axis}\PY{l+s+s2}{\PYZdq{}}\PY{p}{)}
\PY{n}{ax}\PY{o}{.}\PY{n}{set\PYZus{}ylabel}\PY{p}{(}\PY{l+s+s2}{\PYZdq{}}\PY{l+s+s2}{y\PYZhy{}axis}\PY{l+s+s2}{\PYZdq{}}\PY{p}{)}
\PY{n}{ax}\PY{o}{.}\PY{n}{set\PYZus{}title}\PY{p}{(}\PY{l+s+s2}{\PYZdq{}}\PY{l+s+s2}{subplot 4}\PY{l+s+s2}{\PYZdq{}}\PY{p}{)}

\PY{n}{fig}\PY{o}{.}\PY{n}{suptitle}\PY{p}{(}\PY{l+s+s2}{\PYZdq{}}\PY{l+s+s2}{figure title}\PY{l+s+s2}{\PYZdq{}}\PY{p}{)}
\end{Verbatim}
\end{tcolorbox}

            \begin{tcolorbox}[breakable, size=fbox, boxrule=.5pt, pad at break*=1mm, opacityfill=0]
\prompt{Out}{outcolor}{7}{\boxspacing}
\begin{Verbatim}[commandchars=\\\{\}]
Text(0.5, 0.98, 'figure title')
\end{Verbatim}
\end{tcolorbox}
        
    \begin{center}
    \adjustimage{max size={0.9\linewidth}{0.9\paperheight}}{output_7_1.png}
    \end{center}
    { \hspace*{\fill} \\}
    
    Now we can see a problem. The individual subplots are overlapping each
other and making the figure an unreadable mess.

let's add one line at the end of the cell

    \begin{tcolorbox}[breakable, size=fbox, boxrule=1pt, pad at break*=1mm,colback=cellbackground, colframe=cellborder]
\prompt{In}{incolor}{6}{\boxspacing}
\begin{Verbatim}[commandchars=\\\{\}]
\PY{n}{fig} \PY{o}{=} \PY{n}{plt}\PY{o}{.}\PY{n}{figure}\PY{p}{(}\PY{n}{figsize}\PY{o}{=}\PY{p}{[}\PY{l+m+mi}{5}\PY{p}{,}\PY{l+m+mi}{4}\PY{p}{]}\PY{p}{,}\PY{n}{dpi}\PY{o}{=}\PY{l+m+mi}{150}\PY{p}{)} 

\PY{n}{ax} \PY{o}{=} \PY{n}{fig}\PY{o}{.}\PY{n}{add\PYZus{}subplot}\PY{p}{(}\PY{l+m+mi}{2}\PY{p}{,}\PY{l+m+mi}{2}\PY{p}{,}\PY{l+m+mi}{1}\PY{p}{)}
\PY{n}{ax}\PY{o}{.}\PY{n}{plot}\PY{p}{(}\PY{n}{my\PYZus{}data}\PY{p}{,}\PY{n}{color}\PY{o}{=}\PY{l+s+s2}{\PYZdq{}}\PY{l+s+s2}{red}\PY{l+s+s2}{\PYZdq{}}\PY{p}{)}
\PY{n}{ax}\PY{o}{.}\PY{n}{set\PYZus{}xlim}\PY{p}{(}\PY{l+m+mi}{0}\PY{p}{,}\PY{l+m+mi}{10000}\PY{p}{)}
\PY{n}{ax}\PY{o}{.}\PY{n}{set\PYZus{}ylim}\PY{p}{(}\PY{l+m+mi}{0}\PY{p}{,}\PY{l+m+mi}{1}\PY{p}{)}
\PY{n}{ax}\PY{o}{.}\PY{n}{set\PYZus{}xlabel}\PY{p}{(}\PY{l+s+s2}{\PYZdq{}}\PY{l+s+s2}{x\PYZhy{}axis}\PY{l+s+s2}{\PYZdq{}}\PY{p}{)}
\PY{n}{ax}\PY{o}{.}\PY{n}{set\PYZus{}ylabel}\PY{p}{(}\PY{l+s+s2}{\PYZdq{}}\PY{l+s+s2}{y\PYZhy{}axis}\PY{l+s+s2}{\PYZdq{}}\PY{p}{)}
\PY{n}{ax}\PY{o}{.}\PY{n}{set\PYZus{}title}\PY{p}{(}\PY{l+s+s2}{\PYZdq{}}\PY{l+s+s2}{subplot 1}\PY{l+s+s2}{\PYZdq{}}\PY{p}{)}

\PY{n}{ax} \PY{o}{=} \PY{n}{fig}\PY{o}{.}\PY{n}{add\PYZus{}subplot}\PY{p}{(}\PY{l+m+mi}{2}\PY{p}{,}\PY{l+m+mi}{2}\PY{p}{,}\PY{l+m+mi}{2}\PY{p}{)}
\PY{n}{ax}\PY{o}{.}\PY{n}{plot}\PY{p}{(}\PY{n}{my\PYZus{}data}\PY{p}{,}\PY{n}{color}\PY{o}{=}\PY{l+s+s2}{\PYZdq{}}\PY{l+s+s2}{green}\PY{l+s+s2}{\PYZdq{}}\PY{p}{)}
\PY{n}{ax}\PY{o}{.}\PY{n}{set\PYZus{}xlim}\PY{p}{(}\PY{l+m+mi}{0}\PY{p}{,}\PY{l+m+mi}{10000}\PY{p}{)}
\PY{n}{ax}\PY{o}{.}\PY{n}{set\PYZus{}ylim}\PY{p}{(}\PY{l+m+mi}{0}\PY{p}{,}\PY{l+m+mi}{1}\PY{p}{)}
\PY{n}{ax}\PY{o}{.}\PY{n}{set\PYZus{}xlabel}\PY{p}{(}\PY{l+s+s2}{\PYZdq{}}\PY{l+s+s2}{x\PYZhy{}axis}\PY{l+s+s2}{\PYZdq{}}\PY{p}{)}
\PY{n}{ax}\PY{o}{.}\PY{n}{set\PYZus{}ylabel}\PY{p}{(}\PY{l+s+s2}{\PYZdq{}}\PY{l+s+s2}{y\PYZhy{}axis}\PY{l+s+s2}{\PYZdq{}}\PY{p}{)}
\PY{n}{ax}\PY{o}{.}\PY{n}{set\PYZus{}title}\PY{p}{(}\PY{l+s+s2}{\PYZdq{}}\PY{l+s+s2}{subplot 2}\PY{l+s+s2}{\PYZdq{}}\PY{p}{)}

\PY{n}{ax} \PY{o}{=} \PY{n}{fig}\PY{o}{.}\PY{n}{add\PYZus{}subplot}\PY{p}{(}\PY{l+m+mi}{2}\PY{p}{,}\PY{l+m+mi}{2}\PY{p}{,}\PY{l+m+mi}{3}\PY{p}{)}
\PY{n}{ax}\PY{o}{.}\PY{n}{plot}\PY{p}{(}\PY{n}{my\PYZus{}data}\PY{p}{,}\PY{n}{color}\PY{o}{=}\PY{l+s+s2}{\PYZdq{}}\PY{l+s+s2}{orange}\PY{l+s+s2}{\PYZdq{}}\PY{p}{)}
\PY{n}{ax}\PY{o}{.}\PY{n}{set\PYZus{}xlim}\PY{p}{(}\PY{l+m+mi}{0}\PY{p}{,}\PY{l+m+mi}{10000}\PY{p}{)}
\PY{n}{ax}\PY{o}{.}\PY{n}{set\PYZus{}ylim}\PY{p}{(}\PY{l+m+mi}{0}\PY{p}{,}\PY{l+m+mi}{1}\PY{p}{)}
\PY{n}{ax}\PY{o}{.}\PY{n}{set\PYZus{}xlabel}\PY{p}{(}\PY{l+s+s2}{\PYZdq{}}\PY{l+s+s2}{x\PYZhy{}axis}\PY{l+s+s2}{\PYZdq{}}\PY{p}{)}
\PY{n}{ax}\PY{o}{.}\PY{n}{set\PYZus{}ylabel}\PY{p}{(}\PY{l+s+s2}{\PYZdq{}}\PY{l+s+s2}{y\PYZhy{}axis}\PY{l+s+s2}{\PYZdq{}}\PY{p}{)}
\PY{n}{ax}\PY{o}{.}\PY{n}{set\PYZus{}title}\PY{p}{(}\PY{l+s+s2}{\PYZdq{}}\PY{l+s+s2}{subplot 3}\PY{l+s+s2}{\PYZdq{}}\PY{p}{)}

\PY{n}{ax} \PY{o}{=} \PY{n}{fig}\PY{o}{.}\PY{n}{add\PYZus{}subplot}\PY{p}{(}\PY{l+m+mi}{2}\PY{p}{,}\PY{l+m+mi}{2}\PY{p}{,}\PY{l+m+mi}{4}\PY{p}{)}
\PY{n}{ax}\PY{o}{.}\PY{n}{plot}\PY{p}{(}\PY{n}{my\PYZus{}data}\PY{p}{,}\PY{n}{color}\PY{o}{=}\PY{l+s+s2}{\PYZdq{}}\PY{l+s+s2}{blue}\PY{l+s+s2}{\PYZdq{}}\PY{p}{)}
\PY{n}{ax}\PY{o}{.}\PY{n}{set\PYZus{}xlim}\PY{p}{(}\PY{l+m+mi}{0}\PY{p}{,}\PY{l+m+mi}{10000}\PY{p}{)}
\PY{n}{ax}\PY{o}{.}\PY{n}{set\PYZus{}ylim}\PY{p}{(}\PY{l+m+mi}{0}\PY{p}{,}\PY{l+m+mi}{1}\PY{p}{)}
\PY{n}{ax}\PY{o}{.}\PY{n}{set\PYZus{}xlabel}\PY{p}{(}\PY{l+s+s2}{\PYZdq{}}\PY{l+s+s2}{x\PYZhy{}axis}\PY{l+s+s2}{\PYZdq{}}\PY{p}{)}
\PY{n}{ax}\PY{o}{.}\PY{n}{set\PYZus{}ylabel}\PY{p}{(}\PY{l+s+s2}{\PYZdq{}}\PY{l+s+s2}{y\PYZhy{}axis}\PY{l+s+s2}{\PYZdq{}}\PY{p}{)}
\PY{n}{ax}\PY{o}{.}\PY{n}{set\PYZus{}title}\PY{p}{(}\PY{l+s+s2}{\PYZdq{}}\PY{l+s+s2}{subplot 4}\PY{l+s+s2}{\PYZdq{}}\PY{p}{)}

\PY{n}{fig}\PY{o}{.}\PY{n}{suptitle}\PY{p}{(}\PY{l+s+s2}{\PYZdq{}}\PY{l+s+s2}{figure title}\PY{l+s+s2}{\PYZdq{}}\PY{p}{)}
\PY{c+c1}{\PYZsh{}\PYZsh{}\PYZsh{}\PYZsh{} NEW FUNCTIONALITY!}
\PY{n}{fig}\PY{o}{.}\PY{n}{tight\PYZus{}layout}\PY{p}{(}\PY{p}{)}
\end{Verbatim}
\end{tcolorbox}

    \begin{center}
    \adjustimage{max size={0.9\linewidth}{0.9\paperheight}}{output_9_0.png}
    \end{center}
    { \hspace*{\fill} \\}
    
    Well, they're no longer overlapping each other. But now it looks
ridiculous and not very useful. A larger figure size might help. We can
also mess with font sizes of various elements.

    \begin{tcolorbox}[breakable, size=fbox, boxrule=1pt, pad at break*=1mm,colback=cellbackground, colframe=cellborder]
\prompt{In}{incolor}{8}{\boxspacing}
\begin{Verbatim}[commandchars=\\\{\}]
\PY{c+c1}{\PYZsh{}\PYZsh{}\PYZsh{}\PYZsh{} NEW FUNCTIONALITY!}
\PY{n}{fig} \PY{o}{=} \PY{n}{plt}\PY{o}{.}\PY{n}{figure}\PY{p}{(}\PY{n}{figsize}\PY{o}{=}\PY{p}{[}\PY{l+m+mi}{8}\PY{p}{,}\PY{l+m+mi}{5}\PY{p}{]}\PY{p}{,}\PY{n}{dpi}\PY{o}{=}\PY{l+m+mi}{150}\PY{p}{)} 

\PY{n}{ax} \PY{o}{=} \PY{n}{fig}\PY{o}{.}\PY{n}{add\PYZus{}subplot}\PY{p}{(}\PY{l+m+mi}{2}\PY{p}{,}\PY{l+m+mi}{2}\PY{p}{,}\PY{l+m+mi}{1}\PY{p}{)}
\PY{n}{ax}\PY{o}{.}\PY{n}{plot}\PY{p}{(}\PY{n}{my\PYZus{}data}\PY{p}{,}\PY{n}{color}\PY{o}{=}\PY{l+s+s2}{\PYZdq{}}\PY{l+s+s2}{red}\PY{l+s+s2}{\PYZdq{}}\PY{p}{)}
\PY{n}{ax}\PY{o}{.}\PY{n}{set\PYZus{}xlim}\PY{p}{(}\PY{l+m+mi}{0}\PY{p}{,}\PY{l+m+mi}{10000}\PY{p}{)}
\PY{n}{ax}\PY{o}{.}\PY{n}{set\PYZus{}ylim}\PY{p}{(}\PY{l+m+mi}{0}\PY{p}{,}\PY{l+m+mi}{1}\PY{p}{)}
\PY{c+c1}{\PYZsh{}\PYZsh{}\PYZsh{}\PYZsh{} NEW FUNCTIONALITY!}
\PY{n}{ax}\PY{o}{.}\PY{n}{set\PYZus{}xlabel}\PY{p}{(}\PY{l+s+s2}{\PYZdq{}}\PY{l+s+s2}{x\PYZhy{}axis \PYZhy{} size 10}\PY{l+s+s2}{\PYZdq{}}\PY{p}{,}\PY{n}{fontsize}\PY{o}{=}\PY{l+m+mi}{10}\PY{p}{)}
\PY{c+c1}{\PYZsh{}\PYZsh{}\PYZsh{}\PYZsh{} NEW FUNCTIONALITY!}
\PY{n}{ax}\PY{o}{.}\PY{n}{set\PYZus{}ylabel}\PY{p}{(}\PY{l+s+s2}{\PYZdq{}}\PY{l+s+s2}{y\PYZhy{}axis \PYZhy{} size 12}\PY{l+s+s2}{\PYZdq{}}\PY{p}{,}\PY{n}{fontsize}\PY{o}{=}\PY{l+m+mi}{12}\PY{p}{)}
\PY{c+c1}{\PYZsh{}\PYZsh{}\PYZsh{}\PYZsh{} NEW FUNCTIONALITY!}
\PY{n}{ax}\PY{o}{.}\PY{n}{set\PYZus{}title}\PY{p}{(}\PY{l+s+s2}{\PYZdq{}}\PY{l+s+s2}{subplot 1 \PYZhy{} size 8}\PY{l+s+s2}{\PYZdq{}}\PY{p}{,}\PY{n}{fontsize}\PY{o}{=}\PY{l+m+mi}{8}\PY{p}{)}

\PY{n}{ax} \PY{o}{=} \PY{n}{fig}\PY{o}{.}\PY{n}{add\PYZus{}subplot}\PY{p}{(}\PY{l+m+mi}{2}\PY{p}{,}\PY{l+m+mi}{2}\PY{p}{,}\PY{l+m+mi}{2}\PY{p}{)}
\PY{n}{ax}\PY{o}{.}\PY{n}{plot}\PY{p}{(}\PY{n}{my\PYZus{}data}\PY{p}{,}\PY{n}{color}\PY{o}{=}\PY{l+s+s2}{\PYZdq{}}\PY{l+s+s2}{green}\PY{l+s+s2}{\PYZdq{}}\PY{p}{)}
\PY{n}{ax}\PY{o}{.}\PY{n}{set\PYZus{}xlim}\PY{p}{(}\PY{l+m+mi}{0}\PY{p}{,}\PY{l+m+mi}{10000}\PY{p}{)}
\PY{n}{ax}\PY{o}{.}\PY{n}{set\PYZus{}ylim}\PY{p}{(}\PY{l+m+mi}{0}\PY{p}{,}\PY{l+m+mi}{1}\PY{p}{)}
\PY{c+c1}{\PYZsh{}\PYZsh{}\PYZsh{}\PYZsh{} NEW FUNCTIONALITY!}
\PY{n}{ax}\PY{o}{.}\PY{n}{set\PYZus{}xlabel}\PY{p}{(}\PY{l+s+s2}{\PYZdq{}}\PY{l+s+s2}{x\PYZhy{}axis \PYZhy{} size 9}\PY{l+s+s2}{\PYZdq{}}\PY{p}{,}\PY{n}{fontsize}\PY{o}{=}\PY{l+m+mi}{9}\PY{p}{)}
\PY{c+c1}{\PYZsh{}\PYZsh{}\PYZsh{}\PYZsh{} NEW FUNCTIONALITY!}
\PY{n}{ax}\PY{o}{.}\PY{n}{set\PYZus{}ylabel}\PY{p}{(}\PY{l+s+s2}{\PYZdq{}}\PY{l+s+s2}{y\PYZhy{}axis \PYZhy{} size 8}\PY{l+s+s2}{\PYZdq{}}\PY{p}{,}\PY{n}{fontsize}\PY{o}{=}\PY{l+m+mi}{8}\PY{p}{)}
\PY{c+c1}{\PYZsh{}\PYZsh{}\PYZsh{}\PYZsh{} NEW FUNCTIONALITY!}
\PY{n}{ax}\PY{o}{.}\PY{n}{set\PYZus{}title}\PY{p}{(}\PY{l+s+s2}{\PYZdq{}}\PY{l+s+s2}{subplot 2 \PYZhy{} size 10}\PY{l+s+s2}{\PYZdq{}}\PY{p}{,}\PY{n}{fontsize}\PY{o}{=}\PY{l+m+mi}{10}\PY{p}{)}

\PY{n}{ax} \PY{o}{=} \PY{n}{fig}\PY{o}{.}\PY{n}{add\PYZus{}subplot}\PY{p}{(}\PY{l+m+mi}{2}\PY{p}{,}\PY{l+m+mi}{2}\PY{p}{,}\PY{l+m+mi}{3}\PY{p}{)}
\PY{n}{ax}\PY{o}{.}\PY{n}{plot}\PY{p}{(}\PY{n}{my\PYZus{}data}\PY{p}{,}\PY{n}{color}\PY{o}{=}\PY{l+s+s2}{\PYZdq{}}\PY{l+s+s2}{orange}\PY{l+s+s2}{\PYZdq{}}\PY{p}{)}
\PY{n}{ax}\PY{o}{.}\PY{n}{set\PYZus{}xlim}\PY{p}{(}\PY{l+m+mi}{0}\PY{p}{,}\PY{l+m+mi}{10000}\PY{p}{)}
\PY{n}{ax}\PY{o}{.}\PY{n}{set\PYZus{}ylim}\PY{p}{(}\PY{l+m+mi}{0}\PY{p}{,}\PY{l+m+mi}{1}\PY{p}{)}
\PY{c+c1}{\PYZsh{}\PYZsh{}\PYZsh{}\PYZsh{} NEW FUNCTIONALITY!}
\PY{n}{ax}\PY{o}{.}\PY{n}{set\PYZus{}xlabel}\PY{p}{(}\PY{l+s+s2}{\PYZdq{}}\PY{l+s+s2}{x\PYZhy{}axis \PYZhy{} size 7}\PY{l+s+s2}{\PYZdq{}}\PY{p}{,}\PY{n}{fontsize}\PY{o}{=}\PY{l+m+mi}{7}\PY{p}{)}
\PY{c+c1}{\PYZsh{}\PYZsh{}\PYZsh{}\PYZsh{} NEW FUNCTIONALITY!}
\PY{n}{ax}\PY{o}{.}\PY{n}{set\PYZus{}ylabel}\PY{p}{(}\PY{l+s+s2}{\PYZdq{}}\PY{l+s+s2}{y\PYZhy{}axis \PYZhy{} size 6}\PY{l+s+s2}{\PYZdq{}}\PY{p}{,}\PY{n}{fontsize}\PY{o}{=}\PY{l+m+mi}{6}\PY{p}{)}
\PY{c+c1}{\PYZsh{}\PYZsh{}\PYZsh{}\PYZsh{} NEW FUNCTIONALITY!}
\PY{n}{ax}\PY{o}{.}\PY{n}{set\PYZus{}title}\PY{p}{(}\PY{l+s+s2}{\PYZdq{}}\PY{l+s+s2}{subplot 3 \PYZhy{} size 5}\PY{l+s+s2}{\PYZdq{}}\PY{p}{,}\PY{n}{fontsize}\PY{o}{=}\PY{l+m+mi}{5}\PY{p}{)}

\PY{n}{ax} \PY{o}{=} \PY{n}{fig}\PY{o}{.}\PY{n}{add\PYZus{}subplot}\PY{p}{(}\PY{l+m+mi}{2}\PY{p}{,}\PY{l+m+mi}{2}\PY{p}{,}\PY{l+m+mi}{4}\PY{p}{)}
\PY{n}{ax}\PY{o}{.}\PY{n}{plot}\PY{p}{(}\PY{n}{my\PYZus{}data}\PY{p}{,}\PY{n}{color}\PY{o}{=}\PY{l+s+s2}{\PYZdq{}}\PY{l+s+s2}{blue}\PY{l+s+s2}{\PYZdq{}}\PY{p}{)}
\PY{n}{ax}\PY{o}{.}\PY{n}{set\PYZus{}xlim}\PY{p}{(}\PY{l+m+mi}{0}\PY{p}{,}\PY{l+m+mi}{10000}\PY{p}{)}
\PY{n}{ax}\PY{o}{.}\PY{n}{set\PYZus{}ylim}\PY{p}{(}\PY{l+m+mi}{0}\PY{p}{,}\PY{l+m+mi}{1}\PY{p}{)}
\PY{c+c1}{\PYZsh{}\PYZsh{}\PYZsh{}\PYZsh{} NEW FUNCTIONALITY!}
\PY{n}{ax}\PY{o}{.}\PY{n}{set\PYZus{}xlabel}\PY{p}{(}\PY{l+s+s2}{\PYZdq{}}\PY{l+s+s2}{x\PYZhy{}axis \PYZhy{} size 11}\PY{l+s+s2}{\PYZdq{}}\PY{p}{,}\PY{n}{fontsize}\PY{o}{=}\PY{l+m+mi}{11}\PY{p}{)}
\PY{c+c1}{\PYZsh{}\PYZsh{}\PYZsh{}\PYZsh{} NEW FUNCTIONALITY!}
\PY{n}{ax}\PY{o}{.}\PY{n}{set\PYZus{}ylabel}\PY{p}{(}\PY{l+s+s2}{\PYZdq{}}\PY{l+s+s2}{y\PYZhy{}axis \PYZhy{} size 16}\PY{l+s+s2}{\PYZdq{}}\PY{p}{,}\PY{n}{fontsize}\PY{o}{=}\PY{l+m+mi}{16}\PY{p}{)}
\PY{c+c1}{\PYZsh{}\PYZsh{}\PYZsh{}\PYZsh{} NEW FUNCTIONALITY!}
\PY{n}{ax}\PY{o}{.}\PY{n}{set\PYZus{}title}\PY{p}{(}\PY{l+s+s2}{\PYZdq{}}\PY{l+s+s2}{subplot 4 \PYZhy{} size 18}\PY{l+s+s2}{\PYZdq{}}\PY{p}{,}\PY{n}{fontsize}\PY{o}{=}\PY{l+m+mi}{18}\PY{p}{)}

\PY{c+c1}{\PYZsh{}\PYZsh{}\PYZsh{}\PYZsh{} NEW FUNCTIONALITY!}
\PY{n}{fig}\PY{o}{.}\PY{n}{suptitle}\PY{p}{(}\PY{l+s+s2}{\PYZdq{}}\PY{l+s+s2}{figure title \PYZhy{} size 24}\PY{l+s+s2}{\PYZdq{}}\PY{p}{,}\PY{n}{fontsize}\PY{o}{=}\PY{l+m+mi}{24}\PY{p}{)}
\PY{n}{fig}\PY{o}{.}\PY{n}{tight\PYZus{}layout}\PY{p}{(}\PY{p}{)}
\end{Verbatim}
\end{tcolorbox}

    \begin{center}
    \adjustimage{max size={0.9\linewidth}{0.9\paperheight}}{output_11_0.png}
    \end{center}
    { \hspace*{\fill} \\}
    
    You get the idea. It's considered good practice to have the figure title
as the largest fontsize, then subplot titles, then axis labels, then
axis ticks. This helps keep things clean. In the next cell, I'll just
remove the figure title. For most multi-plot figures, you won't need it
anyway. And let's adjust the fontsize of the actual ticks, since we
didn't do that above.

    \begin{tcolorbox}[breakable, size=fbox, boxrule=1pt, pad at break*=1mm,colback=cellbackground, colframe=cellborder]
\prompt{In}{incolor}{9}{\boxspacing}
\begin{Verbatim}[commandchars=\\\{\}]
\PY{n}{fig} \PY{o}{=} \PY{n}{plt}\PY{o}{.}\PY{n}{figure}\PY{p}{(}\PY{n}{figsize}\PY{o}{=}\PY{p}{[}\PY{l+m+mi}{8}\PY{p}{,}\PY{l+m+mi}{5}\PY{p}{]}\PY{p}{,}\PY{n}{dpi}\PY{o}{=}\PY{l+m+mi}{150}\PY{p}{)} 

\PY{n}{ax} \PY{o}{=} \PY{n}{fig}\PY{o}{.}\PY{n}{add\PYZus{}subplot}\PY{p}{(}\PY{l+m+mi}{2}\PY{p}{,}\PY{l+m+mi}{2}\PY{p}{,}\PY{l+m+mi}{1}\PY{p}{)}
\PY{n}{ax}\PY{o}{.}\PY{n}{plot}\PY{p}{(}\PY{n}{my\PYZus{}data}\PY{p}{,}\PY{n}{color}\PY{o}{=}\PY{l+s+s2}{\PYZdq{}}\PY{l+s+s2}{red}\PY{l+s+s2}{\PYZdq{}}\PY{p}{)}
\PY{n}{ax}\PY{o}{.}\PY{n}{set\PYZus{}xlim}\PY{p}{(}\PY{l+m+mi}{0}\PY{p}{,}\PY{l+m+mi}{10000}\PY{p}{)}
\PY{n}{ax}\PY{o}{.}\PY{n}{set\PYZus{}ylim}\PY{p}{(}\PY{l+m+mi}{0}\PY{p}{,}\PY{l+m+mi}{1}\PY{p}{)}
\PY{n}{ax}\PY{o}{.}\PY{n}{set\PYZus{}xlabel}\PY{p}{(}\PY{l+s+s2}{\PYZdq{}}\PY{l+s+s2}{x\PYZhy{}axis \PYZhy{} size 8}\PY{l+s+s2}{\PYZdq{}}\PY{p}{,}\PY{n}{fontsize}\PY{o}{=}\PY{l+m+mi}{8}\PY{p}{)}
\PY{n}{ax}\PY{o}{.}\PY{n}{set\PYZus{}ylabel}\PY{p}{(}\PY{l+s+s2}{\PYZdq{}}\PY{l+s+s2}{y\PYZhy{}axis \PYZhy{} size 8}\PY{l+s+s2}{\PYZdq{}}\PY{p}{,}\PY{n}{fontsize}\PY{o}{=}\PY{l+m+mi}{8}\PY{p}{)}
\PY{n}{ax}\PY{o}{.}\PY{n}{set\PYZus{}title}\PY{p}{(}\PY{l+s+s2}{\PYZdq{}}\PY{l+s+s2}{subplot 1 \PYZhy{} size 12}\PY{l+s+s2}{\PYZdq{}}\PY{p}{,}\PY{n}{fontsize}\PY{o}{=}\PY{l+m+mi}{12}\PY{p}{)}
\PY{c+c1}{\PYZsh{}\PYZsh{}\PYZsh{}\PYZsh{} NEW FUNCTIONALITY!}
\PY{n}{plt}\PY{o}{.}\PY{n}{xticks}\PY{p}{(}\PY{n}{fontsize}\PY{o}{=}\PY{l+m+mi}{7}\PY{p}{)}
\PY{n}{plt}\PY{o}{.}\PY{n}{yticks}\PY{p}{(}\PY{n}{fontsize}\PY{o}{=}\PY{l+m+mi}{7}\PY{p}{)}

\PY{n}{ax} \PY{o}{=} \PY{n}{fig}\PY{o}{.}\PY{n}{add\PYZus{}subplot}\PY{p}{(}\PY{l+m+mi}{2}\PY{p}{,}\PY{l+m+mi}{2}\PY{p}{,}\PY{l+m+mi}{2}\PY{p}{)}
\PY{n}{ax}\PY{o}{.}\PY{n}{plot}\PY{p}{(}\PY{n}{my\PYZus{}data}\PY{p}{,}\PY{n}{color}\PY{o}{=}\PY{l+s+s2}{\PYZdq{}}\PY{l+s+s2}{green}\PY{l+s+s2}{\PYZdq{}}\PY{p}{)}
\PY{n}{ax}\PY{o}{.}\PY{n}{set\PYZus{}xlim}\PY{p}{(}\PY{l+m+mi}{0}\PY{p}{,}\PY{l+m+mi}{10000}\PY{p}{)}
\PY{n}{ax}\PY{o}{.}\PY{n}{set\PYZus{}ylim}\PY{p}{(}\PY{l+m+mi}{0}\PY{p}{,}\PY{l+m+mi}{1}\PY{p}{)}
\PY{n}{ax}\PY{o}{.}\PY{n}{set\PYZus{}xlabel}\PY{p}{(}\PY{l+s+s2}{\PYZdq{}}\PY{l+s+s2}{x\PYZhy{}axis \PYZhy{} size 8}\PY{l+s+s2}{\PYZdq{}}\PY{p}{,}\PY{n}{fontsize}\PY{o}{=}\PY{l+m+mi}{8}\PY{p}{)}
\PY{n}{ax}\PY{o}{.}\PY{n}{set\PYZus{}ylabel}\PY{p}{(}\PY{l+s+s2}{\PYZdq{}}\PY{l+s+s2}{y\PYZhy{}axis \PYZhy{} size 8}\PY{l+s+s2}{\PYZdq{}}\PY{p}{,}\PY{n}{fontsize}\PY{o}{=}\PY{l+m+mi}{8}\PY{p}{)}
\PY{n}{ax}\PY{o}{.}\PY{n}{set\PYZus{}title}\PY{p}{(}\PY{l+s+s2}{\PYZdq{}}\PY{l+s+s2}{subplot 2 \PYZhy{} size 12}\PY{l+s+s2}{\PYZdq{}}\PY{p}{,}\PY{n}{fontsize}\PY{o}{=}\PY{l+m+mi}{12}\PY{p}{)}
\PY{c+c1}{\PYZsh{}\PYZsh{}\PYZsh{}\PYZsh{} NEW FUNCTIONALITY!}
\PY{n}{plt}\PY{o}{.}\PY{n}{xticks}\PY{p}{(}\PY{n}{fontsize}\PY{o}{=}\PY{l+m+mi}{7}\PY{p}{)}
\PY{n}{plt}\PY{o}{.}\PY{n}{yticks}\PY{p}{(}\PY{n}{fontsize}\PY{o}{=}\PY{l+m+mi}{7}\PY{p}{)}

\PY{n}{ax} \PY{o}{=} \PY{n}{fig}\PY{o}{.}\PY{n}{add\PYZus{}subplot}\PY{p}{(}\PY{l+m+mi}{2}\PY{p}{,}\PY{l+m+mi}{2}\PY{p}{,}\PY{l+m+mi}{3}\PY{p}{)}
\PY{n}{ax}\PY{o}{.}\PY{n}{plot}\PY{p}{(}\PY{n}{my\PYZus{}data}\PY{p}{,}\PY{n}{color}\PY{o}{=}\PY{l+s+s2}{\PYZdq{}}\PY{l+s+s2}{orange}\PY{l+s+s2}{\PYZdq{}}\PY{p}{)}
\PY{n}{ax}\PY{o}{.}\PY{n}{set\PYZus{}xlim}\PY{p}{(}\PY{l+m+mi}{0}\PY{p}{,}\PY{l+m+mi}{10000}\PY{p}{)}
\PY{n}{ax}\PY{o}{.}\PY{n}{set\PYZus{}ylim}\PY{p}{(}\PY{l+m+mi}{0}\PY{p}{,}\PY{l+m+mi}{1}\PY{p}{)}
\PY{n}{ax}\PY{o}{.}\PY{n}{set\PYZus{}xlabel}\PY{p}{(}\PY{l+s+s2}{\PYZdq{}}\PY{l+s+s2}{x\PYZhy{}axis \PYZhy{} size 8}\PY{l+s+s2}{\PYZdq{}}\PY{p}{,}\PY{n}{fontsize}\PY{o}{=}\PY{l+m+mi}{8}\PY{p}{)}
\PY{n}{ax}\PY{o}{.}\PY{n}{set\PYZus{}ylabel}\PY{p}{(}\PY{l+s+s2}{\PYZdq{}}\PY{l+s+s2}{y\PYZhy{}axis \PYZhy{} size 8}\PY{l+s+s2}{\PYZdq{}}\PY{p}{,}\PY{n}{fontsize}\PY{o}{=}\PY{l+m+mi}{8}\PY{p}{)}
\PY{n}{ax}\PY{o}{.}\PY{n}{set\PYZus{}title}\PY{p}{(}\PY{l+s+s2}{\PYZdq{}}\PY{l+s+s2}{subplot 3 \PYZhy{} size 12}\PY{l+s+s2}{\PYZdq{}}\PY{p}{,}\PY{n}{fontsize}\PY{o}{=}\PY{l+m+mi}{12}\PY{p}{)}
\PY{c+c1}{\PYZsh{}\PYZsh{}\PYZsh{}\PYZsh{} NEW FUNCTIONALITY!}
\PY{n}{plt}\PY{o}{.}\PY{n}{xticks}\PY{p}{(}\PY{n}{fontsize}\PY{o}{=}\PY{l+m+mi}{7}\PY{p}{)}
\PY{n}{plt}\PY{o}{.}\PY{n}{yticks}\PY{p}{(}\PY{n}{fontsize}\PY{o}{=}\PY{l+m+mi}{7}\PY{p}{)}

\PY{n}{ax} \PY{o}{=} \PY{n}{fig}\PY{o}{.}\PY{n}{add\PYZus{}subplot}\PY{p}{(}\PY{l+m+mi}{2}\PY{p}{,}\PY{l+m+mi}{2}\PY{p}{,}\PY{l+m+mi}{4}\PY{p}{)}
\PY{n}{ax}\PY{o}{.}\PY{n}{plot}\PY{p}{(}\PY{n}{my\PYZus{}data}\PY{p}{,}\PY{n}{color}\PY{o}{=}\PY{l+s+s2}{\PYZdq{}}\PY{l+s+s2}{blue}\PY{l+s+s2}{\PYZdq{}}\PY{p}{)}
\PY{n}{ax}\PY{o}{.}\PY{n}{set\PYZus{}xlim}\PY{p}{(}\PY{l+m+mi}{0}\PY{p}{,}\PY{l+m+mi}{10000}\PY{p}{)}
\PY{n}{ax}\PY{o}{.}\PY{n}{set\PYZus{}ylim}\PY{p}{(}\PY{l+m+mi}{0}\PY{p}{,}\PY{l+m+mi}{1}\PY{p}{)}
\PY{n}{ax}\PY{o}{.}\PY{n}{set\PYZus{}xlabel}\PY{p}{(}\PY{l+s+s2}{\PYZdq{}}\PY{l+s+s2}{x\PYZhy{}axis \PYZhy{} size 8}\PY{l+s+s2}{\PYZdq{}}\PY{p}{,}\PY{n}{fontsize}\PY{o}{=}\PY{l+m+mi}{8}\PY{p}{)}
\PY{n}{ax}\PY{o}{.}\PY{n}{set\PYZus{}ylabel}\PY{p}{(}\PY{l+s+s2}{\PYZdq{}}\PY{l+s+s2}{y\PYZhy{}axis \PYZhy{} size 8}\PY{l+s+s2}{\PYZdq{}}\PY{p}{,}\PY{n}{fontsize}\PY{o}{=}\PY{l+m+mi}{8}\PY{p}{)}
\PY{n}{ax}\PY{o}{.}\PY{n}{set\PYZus{}title}\PY{p}{(}\PY{l+s+s2}{\PYZdq{}}\PY{l+s+s2}{subplot 4 \PYZhy{} size 12}\PY{l+s+s2}{\PYZdq{}}\PY{p}{,}\PY{n}{fontsize}\PY{o}{=}\PY{l+m+mi}{12}\PY{p}{)}
\PY{c+c1}{\PYZsh{}\PYZsh{}\PYZsh{}\PYZsh{} NEW FUNCTIONALITY!}
\PY{n}{plt}\PY{o}{.}\PY{n}{xticks}\PY{p}{(}\PY{n}{fontsize}\PY{o}{=}\PY{l+m+mi}{7}\PY{p}{)}
\PY{n}{plt}\PY{o}{.}\PY{n}{yticks}\PY{p}{(}\PY{n}{fontsize}\PY{o}{=}\PY{l+m+mi}{7}\PY{p}{)}

\PY{n}{fig}\PY{o}{.}\PY{n}{tight\PYZus{}layout}\PY{p}{(}\PY{p}{)}
\end{Verbatim}
\end{tcolorbox}

    \begin{center}
    \adjustimage{max size={0.9\linewidth}{0.9\paperheight}}{output_13_0.png}
    \end{center}
    { \hspace*{\fill} \\}
    
    We've figured out how to plot data as a line plot here, but the data
happens to be very difficult to read. It's being overwhelmed by the
sheer quantity of data here, and ultimately we can't really see much.
Fortunately, there are lots of functions we can use to try to smooth the
data out, or extract trends. One example is the
\texttt{gaussian\_filter} function included with the SciPy module.

We'll smooth our data using the gaussian filter, then plot it over the
original data, which we will color grey and make 50\% transparent using
the ``alpha'' keyword. I'll also add labels for each set of data, which
will be shown in the legend.

    \begin{tcolorbox}[breakable, size=fbox, boxrule=1pt, pad at break*=1mm,colback=cellbackground, colframe=cellborder]
\prompt{In}{incolor}{10}{\boxspacing}
\begin{Verbatim}[commandchars=\\\{\}]
\PY{c+c1}{\PYZsh{}\PYZsh{}\PYZsh{}\PYZsh{} NEW IMPORT!}
\PY{k+kn}{from} \PY{n+nn}{scipy}\PY{n+nn}{.}\PY{n+nn}{ndimage} \PY{k+kn}{import} \PY{n}{gaussian\PYZus{}filter}
\PY{c+c1}{\PYZsh{}\PYZsh{}\PYZsh{}\PYZsh{} NEW FUNCTIONALITY!}
\PY{n}{smoothed\PYZus{}data} \PY{o}{=} \PY{n}{gaussian\PYZus{}filter}\PY{p}{(}\PY{n}{my\PYZus{}data}\PY{p}{,}\PY{n}{sigma}\PY{o}{=}\PY{l+m+mi}{100}\PY{p}{)} \PY{c+c1}{\PYZsh{} here we smooth our original over 100 values.}

\PY{n}{fig} \PY{o}{=} \PY{n}{plt}\PY{o}{.}\PY{n}{figure}\PY{p}{(}\PY{n}{figsize}\PY{o}{=}\PY{p}{[}\PY{l+m+mi}{4}\PY{p}{,}\PY{l+m+mi}{3}\PY{p}{]}\PY{p}{,}\PY{n}{dpi}\PY{o}{=}\PY{l+m+mi}{150}\PY{p}{)} 
\PY{n}{ax} \PY{o}{=} \PY{n}{fig}\PY{o}{.}\PY{n}{add\PYZus{}subplot}\PY{p}{(}\PY{l+m+mi}{1}\PY{p}{,}\PY{l+m+mi}{1}\PY{p}{,}\PY{l+m+mi}{1}\PY{p}{)}
\PY{n}{ax}\PY{o}{.}\PY{n}{plot}\PY{p}{(}\PY{n}{my\PYZus{}data}\PY{p}{,}\PY{n}{color}\PY{o}{=}\PY{l+s+s2}{\PYZdq{}}\PY{l+s+s2}{grey}\PY{l+s+s2}{\PYZdq{}}\PY{p}{,}\PY{n}{alpha}\PY{o}{=}\PY{l+m+mf}{0.5}\PY{p}{,}\PY{n}{label}\PY{o}{=}\PY{l+s+s2}{\PYZdq{}}\PY{l+s+s2}{Original}\PY{l+s+s2}{\PYZdq{}}\PY{p}{)}
\PY{c+c1}{\PYZsh{}\PYZsh{}\PYZsh{}\PYZsh{} NEW DATA!}
\PY{n}{ax}\PY{o}{.}\PY{n}{plot}\PY{p}{(}\PY{n}{smoothed\PYZus{}data}\PY{p}{,}\PY{n}{color}\PY{o}{=}\PY{l+s+s2}{\PYZdq{}}\PY{l+s+s2}{red}\PY{l+s+s2}{\PYZdq{}}\PY{p}{,}\PY{n}{label}\PY{o}{=}\PY{l+s+s2}{\PYZdq{}}\PY{l+s+s2}{Smoothed}\PY{l+s+s2}{\PYZdq{}}\PY{p}{)}
\PY{n}{ax}\PY{o}{.}\PY{n}{set\PYZus{}xlim}\PY{p}{(}\PY{l+m+mi}{0}\PY{p}{,}\PY{l+m+mi}{10000}\PY{p}{)}
\PY{n}{ax}\PY{o}{.}\PY{n}{set\PYZus{}ylim}\PY{p}{(}\PY{l+m+mi}{0}\PY{p}{,}\PY{l+m+mi}{1}\PY{p}{)}
\PY{n}{ax}\PY{o}{.}\PY{n}{set\PYZus{}xlabel}\PY{p}{(}\PY{l+s+s2}{\PYZdq{}}\PY{l+s+s2}{x\PYZhy{}axis \PYZhy{} size 8}\PY{l+s+s2}{\PYZdq{}}\PY{p}{,}\PY{n}{fontsize}\PY{o}{=}\PY{l+m+mi}{8}\PY{p}{)}
\PY{n}{ax}\PY{o}{.}\PY{n}{set\PYZus{}ylabel}\PY{p}{(}\PY{l+s+s2}{\PYZdq{}}\PY{l+s+s2}{y\PYZhy{}axis \PYZhy{} size 8}\PY{l+s+s2}{\PYZdq{}}\PY{p}{,}\PY{n}{fontsize}\PY{o}{=}\PY{l+m+mi}{8}\PY{p}{)}
\PY{n}{ax}\PY{o}{.}\PY{n}{set\PYZus{}title}\PY{p}{(}\PY{l+s+s2}{\PYZdq{}}\PY{l+s+s2}{subplot 1 \PYZhy{} size 12}\PY{l+s+s2}{\PYZdq{}}\PY{p}{,}\PY{n}{fontsize}\PY{o}{=}\PY{l+m+mi}{12}\PY{p}{)}
\PY{n}{plt}\PY{o}{.}\PY{n}{xticks}\PY{p}{(}\PY{n}{fontsize}\PY{o}{=}\PY{l+m+mi}{7}\PY{p}{)}
\PY{n}{plt}\PY{o}{.}\PY{n}{yticks}\PY{p}{(}\PY{n}{fontsize}\PY{o}{=}\PY{l+m+mi}{7}\PY{p}{)}
\PY{c+c1}{\PYZsh{}\PYZsh{}\PYZsh{}\PYZsh{} NEW FUNCTIONALITY!}
\PY{n}{ax}\PY{o}{.}\PY{n}{legend}\PY{p}{(}\PY{n}{loc}\PY{o}{=}\PY{l+m+mi}{2}\PY{p}{,}\PY{n}{fontsize}\PY{o}{=}\PY{l+m+mi}{6}\PY{p}{)}
\PY{n}{fig}\PY{o}{.}\PY{n}{tight\PYZus{}layout}\PY{p}{(}\PY{p}{)}
\end{Verbatim}
\end{tcolorbox}

    \begin{center}
    \adjustimage{max size={0.9\linewidth}{0.9\paperheight}}{output_15_0.png}
    \end{center}
    { \hspace*{\fill} \\}
    
    What if we wanted to plot something that wasn't a dependent variable
like above? What if we wanted to compare two different measurements over
time to see how they related? A scatter plot might come in handy. First,
let's generate some new data. Then we'll use what we've learned above,
but use the scatter plot function instead of \texttt{plot}, which
defaults to a line.

    \begin{tcolorbox}[breakable, size=fbox, boxrule=1pt, pad at break*=1mm,colback=cellbackground, colframe=cellborder]
\prompt{In}{incolor}{11}{\boxspacing}
\begin{Verbatim}[commandchars=\\\{\}]
\PY{n}{new\PYZus{}data1}\PY{o}{=}\PY{n}{np}\PY{o}{.}\PY{n}{random}\PY{o}{.}\PY{n}{rand}\PY{p}{(}\PY{l+m+mi}{500}\PY{p}{)}
\PY{n}{new\PYZus{}data2}\PY{o}{=}\PY{n}{np}\PY{o}{.}\PY{n}{random}\PY{o}{.}\PY{n}{rand}\PY{p}{(}\PY{l+m+mi}{500}\PY{p}{)}

\PY{n}{fig} \PY{o}{=} \PY{n}{plt}\PY{o}{.}\PY{n}{figure}\PY{p}{(}\PY{n}{figsize}\PY{o}{=}\PY{p}{[}\PY{l+m+mi}{4}\PY{p}{,}\PY{l+m+mi}{3}\PY{p}{]}\PY{p}{,}\PY{n}{dpi}\PY{o}{=}\PY{l+m+mi}{150}\PY{p}{)} 
\PY{n}{ax} \PY{o}{=} \PY{n}{fig}\PY{o}{.}\PY{n}{add\PYZus{}subplot}\PY{p}{(}\PY{l+m+mi}{1}\PY{p}{,}\PY{l+m+mi}{1}\PY{p}{,}\PY{l+m+mi}{1}\PY{p}{)}
\PY{c+c1}{\PYZsh{}\PYZsh{}\PYZsh{}\PYZsh{} NEW FUNCTIONALITY!}
\PY{n}{ax}\PY{o}{.}\PY{n}{scatter}\PY{p}{(}\PY{n}{new\PYZus{}data1}\PY{p}{,}\PY{n}{new\PYZus{}data2}\PY{p}{,}\PY{n}{marker}\PY{o}{=}\PY{l+s+s2}{\PYZdq{}}\PY{l+s+s2}{+}\PY{l+s+s2}{\PYZdq{}}\PY{p}{,}\PY{n}{lw}\PY{o}{=}\PY{l+m+mf}{0.5}\PY{p}{)}
\PY{n}{ax}\PY{o}{.}\PY{n}{set\PYZus{}xlim}\PY{p}{(}\PY{l+m+mi}{0}\PY{p}{,}\PY{l+m+mi}{1}\PY{p}{)}
\PY{n}{ax}\PY{o}{.}\PY{n}{set\PYZus{}ylim}\PY{p}{(}\PY{l+m+mi}{0}\PY{p}{,}\PY{l+m+mi}{1}\PY{p}{)}
\PY{n}{ax}\PY{o}{.}\PY{n}{set\PYZus{}xlabel}\PY{p}{(}\PY{l+s+s2}{\PYZdq{}}\PY{l+s+s2}{new\PYZus{}data1}\PY{l+s+s2}{\PYZdq{}}\PY{p}{,}\PY{n}{fontsize}\PY{o}{=}\PY{l+m+mi}{8}\PY{p}{)}
\PY{n}{ax}\PY{o}{.}\PY{n}{set\PYZus{}ylabel}\PY{p}{(}\PY{l+s+s2}{\PYZdq{}}\PY{l+s+s2}{new\PYZus{}data2}\PY{l+s+s2}{\PYZdq{}}\PY{p}{,}\PY{n}{fontsize}\PY{o}{=}\PY{l+m+mi}{8}\PY{p}{)}
\PY{n}{ax}\PY{o}{.}\PY{n}{set\PYZus{}title}\PY{p}{(}\PY{l+s+s2}{\PYZdq{}}\PY{l+s+s2}{scatterplotting}\PY{l+s+s2}{\PYZdq{}}\PY{p}{,}\PY{n}{fontsize}\PY{o}{=}\PY{l+m+mi}{12}\PY{p}{)}
\PY{n}{plt}\PY{o}{.}\PY{n}{xticks}\PY{p}{(}\PY{n}{fontsize}\PY{o}{=}\PY{l+m+mi}{7}\PY{p}{)}
\PY{n}{plt}\PY{o}{.}\PY{n}{yticks}\PY{p}{(}\PY{n}{fontsize}\PY{o}{=}\PY{l+m+mi}{7}\PY{p}{)}
\PY{n}{fig}\PY{o}{.}\PY{n}{tight\PYZus{}layout}\PY{p}{(}\PY{p}{)}
\end{Verbatim}
\end{tcolorbox}

    \begin{center}
    \adjustimage{max size={0.9\linewidth}{0.9\paperheight}}{output_17_0.png}
    \end{center}
    { \hspace*{\fill} \\}
    
    This shows us that our two sets of randomly generated data look \ldots{}
random. What if we wanted to show the order in which these datapoints
occur in the lists, with earlier values shown in one color and later
values in a different color?

MatPlotLib has colormaps for this sort of thing! We can color each point
based on a value. Since we already have the x and y values in the plot,
it seems a waste to color based on those. However, we can create a list,
\texttt{z}, made up of 500 values between 0 and 1, using
\texttt{np.linspace}, and then pass that to the scatter plot function
for the color value, along with the \texttt{cmap} keyword to choose the
specific colormap we want (check out MatPlotLib documentation for a huge
selection of colormaps - or for how to build your own!)

    \begin{tcolorbox}[breakable, size=fbox, boxrule=1pt, pad at break*=1mm,colback=cellbackground, colframe=cellborder]
\prompt{In}{incolor}{12}{\boxspacing}
\begin{Verbatim}[commandchars=\\\{\}]
\PY{n}{new\PYZus{}data1}\PY{o}{=}\PY{n}{np}\PY{o}{.}\PY{n}{random}\PY{o}{.}\PY{n}{rand}\PY{p}{(}\PY{l+m+mi}{500}\PY{p}{)}
\PY{n}{new\PYZus{}data2}\PY{o}{=}\PY{n}{np}\PY{o}{.}\PY{n}{random}\PY{o}{.}\PY{n}{rand}\PY{p}{(}\PY{l+m+mi}{500}\PY{p}{)}
\PY{c+c1}{\PYZsh{}\PYZsh{}\PYZsh{}\PYZsh{} NEW FUNCTIONALITY!}
\PY{n}{z}\PY{o}{=}\PY{n}{np}\PY{o}{.}\PY{n}{linspace}\PY{p}{(}\PY{l+m+mi}{0}\PY{p}{,}\PY{l+m+mi}{1}\PY{p}{,}\PY{l+m+mi}{500}\PY{p}{)}

\PY{n}{fig} \PY{o}{=} \PY{n}{plt}\PY{o}{.}\PY{n}{figure}\PY{p}{(}\PY{n}{figsize}\PY{o}{=}\PY{p}{[}\PY{l+m+mi}{8}\PY{p}{,}\PY{l+m+mi}{6}\PY{p}{]}\PY{p}{,}\PY{n}{dpi}\PY{o}{=}\PY{l+m+mi}{150}\PY{p}{)} 
\PY{n}{ax} \PY{o}{=} \PY{n}{fig}\PY{o}{.}\PY{n}{add\PYZus{}subplot}\PY{p}{(}\PY{l+m+mi}{2}\PY{p}{,}\PY{l+m+mi}{2}\PY{p}{,}\PY{l+m+mi}{1}\PY{p}{)}
\PY{c+c1}{\PYZsh{}\PYZsh{}\PYZsh{}\PYZsh{} NEW FUNCTIONALITY!}
\PY{n}{ax}\PY{o}{.}\PY{n}{scatter}\PY{p}{(}\PY{n}{new\PYZus{}data1}\PY{p}{,}\PY{n}{new\PYZus{}data2}\PY{p}{,}\PY{n}{marker}\PY{o}{=}\PY{l+s+s2}{\PYZdq{}}\PY{l+s+s2}{.}\PY{l+s+s2}{\PYZdq{}}\PY{p}{,}\PY{n}{lw}\PY{o}{=}\PY{l+m+mf}{0.5}\PY{p}{,}\PY{n}{c}\PY{o}{=}\PY{n}{z}\PY{p}{,}\PY{n}{cmap}\PY{o}{=}\PY{l+s+s2}{\PYZdq{}}\PY{l+s+s2}{rainbow}\PY{l+s+s2}{\PYZdq{}}\PY{p}{)}
\PY{n}{ax}\PY{o}{.}\PY{n}{set\PYZus{}xlim}\PY{p}{(}\PY{l+m+mi}{0}\PY{p}{,}\PY{l+m+mi}{1}\PY{p}{)}
\PY{n}{ax}\PY{o}{.}\PY{n}{set\PYZus{}ylim}\PY{p}{(}\PY{l+m+mi}{0}\PY{p}{,}\PY{l+m+mi}{1}\PY{p}{)}
\PY{n}{ax}\PY{o}{.}\PY{n}{set\PYZus{}xlabel}\PY{p}{(}\PY{l+s+s2}{\PYZdq{}}\PY{l+s+s2}{new\PYZus{}data1}\PY{l+s+s2}{\PYZdq{}}\PY{p}{,}\PY{n}{fontsize}\PY{o}{=}\PY{l+m+mi}{8}\PY{p}{)}
\PY{n}{ax}\PY{o}{.}\PY{n}{set\PYZus{}ylabel}\PY{p}{(}\PY{l+s+s2}{\PYZdq{}}\PY{l+s+s2}{new\PYZus{}data2}\PY{l+s+s2}{\PYZdq{}}\PY{p}{,}\PY{n}{fontsize}\PY{o}{=}\PY{l+m+mi}{8}\PY{p}{)}
\PY{n}{ax}\PY{o}{.}\PY{n}{set\PYZus{}title}\PY{p}{(}\PY{l+s+s2}{\PYZdq{}}\PY{l+s+s2}{rainbow, marker .}\PY{l+s+s2}{\PYZdq{}}\PY{p}{,}\PY{n}{fontsize}\PY{o}{=}\PY{l+m+mi}{12}\PY{p}{)}
\PY{n}{plt}\PY{o}{.}\PY{n}{xticks}\PY{p}{(}\PY{n}{fontsize}\PY{o}{=}\PY{l+m+mi}{7}\PY{p}{)}
\PY{n}{plt}\PY{o}{.}\PY{n}{yticks}\PY{p}{(}\PY{n}{fontsize}\PY{o}{=}\PY{l+m+mi}{7}\PY{p}{)}

\PY{n}{ax} \PY{o}{=} \PY{n}{fig}\PY{o}{.}\PY{n}{add\PYZus{}subplot}\PY{p}{(}\PY{l+m+mi}{2}\PY{p}{,}\PY{l+m+mi}{2}\PY{p}{,}\PY{l+m+mi}{2}\PY{p}{)}
\PY{c+c1}{\PYZsh{}\PYZsh{}\PYZsh{}\PYZsh{} NEW FUNCTIONALITY!}
\PY{n}{ax}\PY{o}{.}\PY{n}{scatter}\PY{p}{(}\PY{n}{new\PYZus{}data1}\PY{p}{,}\PY{n}{new\PYZus{}data2}\PY{p}{,}\PY{n}{marker}\PY{o}{=}\PY{l+s+s2}{\PYZdq{}}\PY{l+s+s2}{o}\PY{l+s+s2}{\PYZdq{}}\PY{p}{,}\PY{n}{lw}\PY{o}{=}\PY{l+m+mf}{0.5}\PY{p}{,}\PY{n}{c}\PY{o}{=}\PY{n}{z}\PY{p}{,}\PY{n}{cmap}\PY{o}{=}\PY{l+s+s2}{\PYZdq{}}\PY{l+s+s2}{viridis}\PY{l+s+s2}{\PYZdq{}}\PY{p}{)}
\PY{n}{ax}\PY{o}{.}\PY{n}{set\PYZus{}xlim}\PY{p}{(}\PY{l+m+mi}{0}\PY{p}{,}\PY{l+m+mi}{1}\PY{p}{)}
\PY{n}{ax}\PY{o}{.}\PY{n}{set\PYZus{}ylim}\PY{p}{(}\PY{l+m+mi}{0}\PY{p}{,}\PY{l+m+mi}{1}\PY{p}{)}
\PY{n}{ax}\PY{o}{.}\PY{n}{set\PYZus{}xlabel}\PY{p}{(}\PY{l+s+s2}{\PYZdq{}}\PY{l+s+s2}{new\PYZus{}data1}\PY{l+s+s2}{\PYZdq{}}\PY{p}{,}\PY{n}{fontsize}\PY{o}{=}\PY{l+m+mi}{8}\PY{p}{)}
\PY{n}{ax}\PY{o}{.}\PY{n}{set\PYZus{}ylabel}\PY{p}{(}\PY{l+s+s2}{\PYZdq{}}\PY{l+s+s2}{new\PYZus{}data2}\PY{l+s+s2}{\PYZdq{}}\PY{p}{,}\PY{n}{fontsize}\PY{o}{=}\PY{l+m+mi}{8}\PY{p}{)}
\PY{n}{ax}\PY{o}{.}\PY{n}{set\PYZus{}title}\PY{p}{(}\PY{l+s+s2}{\PYZdq{}}\PY{l+s+s2}{viridis, marker o}\PY{l+s+s2}{\PYZdq{}}\PY{p}{,}\PY{n}{fontsize}\PY{o}{=}\PY{l+m+mi}{12}\PY{p}{)}
\PY{n}{plt}\PY{o}{.}\PY{n}{xticks}\PY{p}{(}\PY{n}{fontsize}\PY{o}{=}\PY{l+m+mi}{7}\PY{p}{)}
\PY{n}{plt}\PY{o}{.}\PY{n}{yticks}\PY{p}{(}\PY{n}{fontsize}\PY{o}{=}\PY{l+m+mi}{7}\PY{p}{)}

\PY{n}{ax} \PY{o}{=} \PY{n}{fig}\PY{o}{.}\PY{n}{add\PYZus{}subplot}\PY{p}{(}\PY{l+m+mi}{2}\PY{p}{,}\PY{l+m+mi}{2}\PY{p}{,}\PY{l+m+mi}{3}\PY{p}{)}
\PY{c+c1}{\PYZsh{}\PYZsh{}\PYZsh{}\PYZsh{} NEW FUNCTIONALITY!}
\PY{n}{ax}\PY{o}{.}\PY{n}{scatter}\PY{p}{(}\PY{n}{new\PYZus{}data1}\PY{p}{,}\PY{n}{new\PYZus{}data2}\PY{p}{,}\PY{n}{marker}\PY{o}{=}\PY{l+s+s2}{\PYZdq{}}\PY{l+s+s2}{+}\PY{l+s+s2}{\PYZdq{}}\PY{p}{,}\PY{n}{lw}\PY{o}{=}\PY{l+m+mf}{0.5}\PY{p}{,}\PY{n}{c}\PY{o}{=}\PY{n}{z}\PY{p}{,}\PY{n}{cmap}\PY{o}{=}\PY{l+s+s2}{\PYZdq{}}\PY{l+s+s2}{magma}\PY{l+s+s2}{\PYZdq{}}\PY{p}{)}
\PY{n}{ax}\PY{o}{.}\PY{n}{set\PYZus{}xlim}\PY{p}{(}\PY{l+m+mi}{0}\PY{p}{,}\PY{l+m+mi}{1}\PY{p}{)}
\PY{n}{ax}\PY{o}{.}\PY{n}{set\PYZus{}ylim}\PY{p}{(}\PY{l+m+mi}{0}\PY{p}{,}\PY{l+m+mi}{1}\PY{p}{)}
\PY{n}{ax}\PY{o}{.}\PY{n}{set\PYZus{}xlabel}\PY{p}{(}\PY{l+s+s2}{\PYZdq{}}\PY{l+s+s2}{new\PYZus{}data1}\PY{l+s+s2}{\PYZdq{}}\PY{p}{,}\PY{n}{fontsize}\PY{o}{=}\PY{l+m+mi}{8}\PY{p}{)}
\PY{n}{ax}\PY{o}{.}\PY{n}{set\PYZus{}ylabel}\PY{p}{(}\PY{l+s+s2}{\PYZdq{}}\PY{l+s+s2}{new\PYZus{}data2}\PY{l+s+s2}{\PYZdq{}}\PY{p}{,}\PY{n}{fontsize}\PY{o}{=}\PY{l+m+mi}{8}\PY{p}{)}
\PY{n}{ax}\PY{o}{.}\PY{n}{set\PYZus{}title}\PY{p}{(}\PY{l+s+s2}{\PYZdq{}}\PY{l+s+s2}{magma, marker +}\PY{l+s+s2}{\PYZdq{}}\PY{p}{,}\PY{n}{fontsize}\PY{o}{=}\PY{l+m+mi}{12}\PY{p}{)}
\PY{n}{plt}\PY{o}{.}\PY{n}{xticks}\PY{p}{(}\PY{n}{fontsize}\PY{o}{=}\PY{l+m+mi}{7}\PY{p}{)}
\PY{n}{plt}\PY{o}{.}\PY{n}{yticks}\PY{p}{(}\PY{n}{fontsize}\PY{o}{=}\PY{l+m+mi}{7}\PY{p}{)}

\PY{n}{ax} \PY{o}{=} \PY{n}{fig}\PY{o}{.}\PY{n}{add\PYZus{}subplot}\PY{p}{(}\PY{l+m+mi}{2}\PY{p}{,}\PY{l+m+mi}{2}\PY{p}{,}\PY{l+m+mi}{4}\PY{p}{)}
\PY{c+c1}{\PYZsh{}\PYZsh{}\PYZsh{}\PYZsh{} NEW FUNCTIONALITY!}
\PY{n}{ax}\PY{o}{.}\PY{n}{scatter}\PY{p}{(}\PY{n}{new\PYZus{}data1}\PY{p}{,}\PY{n}{new\PYZus{}data2}\PY{p}{,}\PY{n}{marker}\PY{o}{=}\PY{l+s+s2}{\PYZdq{}}\PY{l+s+s2}{x}\PY{l+s+s2}{\PYZdq{}}\PY{p}{,}\PY{n}{lw}\PY{o}{=}\PY{l+m+mf}{0.5}\PY{p}{,}\PY{n}{c}\PY{o}{=}\PY{n}{z}\PY{p}{,}\PY{n}{cmap}\PY{o}{=}\PY{l+s+s2}{\PYZdq{}}\PY{l+s+s2}{ocean\PYZus{}r}\PY{l+s+s2}{\PYZdq{}}\PY{p}{)}
\PY{n}{ax}\PY{o}{.}\PY{n}{set\PYZus{}xlim}\PY{p}{(}\PY{l+m+mi}{0}\PY{p}{,}\PY{l+m+mi}{1}\PY{p}{)}
\PY{n}{ax}\PY{o}{.}\PY{n}{set\PYZus{}ylim}\PY{p}{(}\PY{l+m+mi}{0}\PY{p}{,}\PY{l+m+mi}{1}\PY{p}{)}
\PY{n}{ax}\PY{o}{.}\PY{n}{set\PYZus{}xlabel}\PY{p}{(}\PY{l+s+s2}{\PYZdq{}}\PY{l+s+s2}{new\PYZus{}data1}\PY{l+s+s2}{\PYZdq{}}\PY{p}{,}\PY{n}{fontsize}\PY{o}{=}\PY{l+m+mi}{8}\PY{p}{)}
\PY{n}{ax}\PY{o}{.}\PY{n}{set\PYZus{}ylabel}\PY{p}{(}\PY{l+s+s2}{\PYZdq{}}\PY{l+s+s2}{new\PYZus{}data2}\PY{l+s+s2}{\PYZdq{}}\PY{p}{,}\PY{n}{fontsize}\PY{o}{=}\PY{l+m+mi}{8}\PY{p}{)}
\PY{n}{ax}\PY{o}{.}\PY{n}{set\PYZus{}title}\PY{p}{(}\PY{l+s+s2}{\PYZdq{}}\PY{l+s+s2}{ocean\PYZus{}r, marker x}\PY{l+s+s2}{\PYZdq{}}\PY{p}{,}\PY{n}{fontsize}\PY{o}{=}\PY{l+m+mi}{12}\PY{p}{)}
\PY{n}{plt}\PY{o}{.}\PY{n}{xticks}\PY{p}{(}\PY{n}{fontsize}\PY{o}{=}\PY{l+m+mi}{7}\PY{p}{)}
\PY{n}{plt}\PY{o}{.}\PY{n}{yticks}\PY{p}{(}\PY{n}{fontsize}\PY{o}{=}\PY{l+m+mi}{7}\PY{p}{)}


\PY{n}{fig}\PY{o}{.}\PY{n}{tight\PYZus{}layout}\PY{p}{(}\PY{p}{)}
\end{Verbatim}
\end{tcolorbox}

    \begin{center}
    \adjustimage{max size={0.9\linewidth}{0.9\paperheight}}{output_19_0.png}
    \end{center}
    { \hspace*{\fill} \\}
    
    In the example above, you might have noticed one of the colormaps was
\texttt{ocean\_r}. MatPlotLib has the functionality built in to reverse
any given colormap by adding \texttt{\_r} to the name. This offers a
quick way to reverse the colors being used for values of interest rather
than trying to rearrange your data.

Now, scatter plots are great if your data isn't overly dense for the
area in which it lies. In the examples above, we've used 500 data points
scattered about. What if that were 10,000 data points?

    \begin{tcolorbox}[breakable, size=fbox, boxrule=1pt, pad at break*=1mm,colback=cellbackground, colframe=cellborder]
\prompt{In}{incolor}{13}{\boxspacing}
\begin{Verbatim}[commandchars=\\\{\}]
\PY{n}{new\PYZus{}data1}\PY{o}{=}\PY{n}{np}\PY{o}{.}\PY{n}{random}\PY{o}{.}\PY{n}{rand}\PY{p}{(}\PY{l+m+mi}{10000}\PY{p}{)}
\PY{n}{new\PYZus{}data2}\PY{o}{=}\PY{n}{np}\PY{o}{.}\PY{n}{random}\PY{o}{.}\PY{n}{rand}\PY{p}{(}\PY{l+m+mi}{10000}\PY{p}{)}
\PY{c+c1}{\PYZsh{}\PYZsh{}\PYZsh{}\PYZsh{} NEW FUNCTIONALITY!}
\PY{n}{z}\PY{o}{=}\PY{n}{np}\PY{o}{.}\PY{n}{linspace}\PY{p}{(}\PY{l+m+mi}{0}\PY{p}{,}\PY{l+m+mi}{1}\PY{p}{,}\PY{l+m+mi}{10000}\PY{p}{)}

\PY{n}{fig} \PY{o}{=} \PY{n}{plt}\PY{o}{.}\PY{n}{figure}\PY{p}{(}\PY{n}{figsize}\PY{o}{=}\PY{p}{[}\PY{l+m+mi}{8}\PY{p}{,}\PY{l+m+mi}{3}\PY{p}{]}\PY{p}{,}\PY{n}{dpi}\PY{o}{=}\PY{l+m+mi}{150}\PY{p}{)} 
\PY{n}{ax} \PY{o}{=} \PY{n}{fig}\PY{o}{.}\PY{n}{add\PYZus{}subplot}\PY{p}{(}\PY{l+m+mi}{1}\PY{p}{,}\PY{l+m+mi}{2}\PY{p}{,}\PY{l+m+mi}{1}\PY{p}{)}
\PY{c+c1}{\PYZsh{}\PYZsh{}\PYZsh{}\PYZsh{} NEW FUNCTIONALITY!}
\PY{n}{ax}\PY{o}{.}\PY{n}{scatter}\PY{p}{(}\PY{n}{new\PYZus{}data1}\PY{p}{,}\PY{n}{new\PYZus{}data2}\PY{p}{,}\PY{n}{marker}\PY{o}{=}\PY{l+s+s2}{\PYZdq{}}\PY{l+s+s2}{.}\PY{l+s+s2}{\PYZdq{}}\PY{p}{,}\PY{n}{lw}\PY{o}{=}\PY{l+m+mf}{0.5}\PY{p}{,}\PY{n}{c}\PY{o}{=}\PY{l+s+s2}{\PYZdq{}}\PY{l+s+s2}{blue}\PY{l+s+s2}{\PYZdq{}}\PY{p}{)}
\PY{n}{ax}\PY{o}{.}\PY{n}{set\PYZus{}xlim}\PY{p}{(}\PY{l+m+mi}{0}\PY{p}{,}\PY{l+m+mi}{1}\PY{p}{)}
\PY{n}{ax}\PY{o}{.}\PY{n}{set\PYZus{}ylim}\PY{p}{(}\PY{l+m+mi}{0}\PY{p}{,}\PY{l+m+mi}{1}\PY{p}{)}
\PY{n}{ax}\PY{o}{.}\PY{n}{set\PYZus{}xlabel}\PY{p}{(}\PY{l+s+s2}{\PYZdq{}}\PY{l+s+s2}{new\PYZus{}data1}\PY{l+s+s2}{\PYZdq{}}\PY{p}{,}\PY{n}{fontsize}\PY{o}{=}\PY{l+m+mi}{8}\PY{p}{)}
\PY{n}{ax}\PY{o}{.}\PY{n}{set\PYZus{}ylabel}\PY{p}{(}\PY{l+s+s2}{\PYZdq{}}\PY{l+s+s2}{new\PYZus{}data2}\PY{l+s+s2}{\PYZdq{}}\PY{p}{,}\PY{n}{fontsize}\PY{o}{=}\PY{l+m+mi}{8}\PY{p}{)}
\PY{n}{ax}\PY{o}{.}\PY{n}{set\PYZus{}title}\PY{p}{(}\PY{l+s+s2}{\PYZdq{}}\PY{l+s+s2}{blue, marker .}\PY{l+s+s2}{\PYZdq{}}\PY{p}{,}\PY{n}{fontsize}\PY{o}{=}\PY{l+m+mi}{12}\PY{p}{)}
\PY{n}{plt}\PY{o}{.}\PY{n}{xticks}\PY{p}{(}\PY{n}{fontsize}\PY{o}{=}\PY{l+m+mi}{7}\PY{p}{)}
\PY{n}{plt}\PY{o}{.}\PY{n}{yticks}\PY{p}{(}\PY{n}{fontsize}\PY{o}{=}\PY{l+m+mi}{7}\PY{p}{)}

\PY{n}{ax} \PY{o}{=} \PY{n}{fig}\PY{o}{.}\PY{n}{add\PYZus{}subplot}\PY{p}{(}\PY{l+m+mi}{1}\PY{p}{,}\PY{l+m+mi}{2}\PY{p}{,}\PY{l+m+mi}{2}\PY{p}{)}
\PY{c+c1}{\PYZsh{}\PYZsh{}\PYZsh{}\PYZsh{} NEW FUNCTIONALITY!}
\PY{n}{ax}\PY{o}{.}\PY{n}{scatter}\PY{p}{(}\PY{n}{new\PYZus{}data1}\PY{p}{,}\PY{n}{new\PYZus{}data2}\PY{p}{,}\PY{n}{marker}\PY{o}{=}\PY{l+s+s2}{\PYZdq{}}\PY{l+s+s2}{.}\PY{l+s+s2}{\PYZdq{}}\PY{p}{,}\PY{n}{lw}\PY{o}{=}\PY{l+m+mf}{0.5}\PY{p}{,}\PY{n}{c}\PY{o}{=}\PY{n}{z}\PY{p}{,}\PY{n}{cmap}\PY{o}{=}\PY{l+s+s2}{\PYZdq{}}\PY{l+s+s2}{rainbow}\PY{l+s+s2}{\PYZdq{}}\PY{p}{)}
\PY{n}{ax}\PY{o}{.}\PY{n}{set\PYZus{}xlim}\PY{p}{(}\PY{l+m+mi}{0}\PY{p}{,}\PY{l+m+mi}{1}\PY{p}{)}
\PY{n}{ax}\PY{o}{.}\PY{n}{set\PYZus{}ylim}\PY{p}{(}\PY{l+m+mi}{0}\PY{p}{,}\PY{l+m+mi}{1}\PY{p}{)}
\PY{n}{ax}\PY{o}{.}\PY{n}{set\PYZus{}xlabel}\PY{p}{(}\PY{l+s+s2}{\PYZdq{}}\PY{l+s+s2}{new\PYZus{}data1}\PY{l+s+s2}{\PYZdq{}}\PY{p}{,}\PY{n}{fontsize}\PY{o}{=}\PY{l+m+mi}{8}\PY{p}{)}
\PY{n}{ax}\PY{o}{.}\PY{n}{set\PYZus{}ylabel}\PY{p}{(}\PY{l+s+s2}{\PYZdq{}}\PY{l+s+s2}{new\PYZus{}data2}\PY{l+s+s2}{\PYZdq{}}\PY{p}{,}\PY{n}{fontsize}\PY{o}{=}\PY{l+m+mi}{8}\PY{p}{)}
\PY{n}{ax}\PY{o}{.}\PY{n}{set\PYZus{}title}\PY{p}{(}\PY{l+s+s2}{\PYZdq{}}\PY{l+s+s2}{rainbow, marker .}\PY{l+s+s2}{\PYZdq{}}\PY{p}{,}\PY{n}{fontsize}\PY{o}{=}\PY{l+m+mi}{12}\PY{p}{)}
\PY{n}{plt}\PY{o}{.}\PY{n}{xticks}\PY{p}{(}\PY{n}{fontsize}\PY{o}{=}\PY{l+m+mi}{7}\PY{p}{)}
\PY{n}{plt}\PY{o}{.}\PY{n}{yticks}\PY{p}{(}\PY{n}{fontsize}\PY{o}{=}\PY{l+m+mi}{7}\PY{p}{)}

\PY{n}{fig}\PY{o}{.}\PY{n}{tight\PYZus{}layout}\PY{p}{(}\PY{p}{)}
\end{Verbatim}
\end{tcolorbox}

    \begin{center}
    \adjustimage{max size={0.9\linewidth}{0.9\paperheight}}{output_21_0.png}
    \end{center}
    { \hspace*{\fill} \\}
    
    Even with the rainbow mapping, we simply have too much data to really
see what's going on in the plot. Situations such as these might suggest
we use a different approach - the Density Plot.

    \begin{tcolorbox}[breakable, size=fbox, boxrule=1pt, pad at break*=1mm,colback=cellbackground, colframe=cellborder]
\prompt{In}{incolor}{14}{\boxspacing}
\begin{Verbatim}[commandchars=\\\{\}]
\PY{c+c1}{\PYZsh{}\PYZsh{}\PYZsh{}\PYZsh{} NEW IMPORT!}
\PY{k+kn}{from} \PY{n+nn}{scipy}\PY{n+nn}{.}\PY{n+nn}{stats} \PY{k+kn}{import} \PY{n}{kde}
\PY{n}{nbins}\PY{o}{=}\PY{l+m+mi}{300}
\PY{c+c1}{\PYZsh{}\PYZsh{}\PYZsh{}\PYZsh{} NEW FUNCTIONALITY!}
\PY{n}{k} \PY{o}{=} \PY{n}{kde}\PY{o}{.}\PY{n}{gaussian\PYZus{}kde}\PY{p}{(}\PY{p}{[}\PY{n}{new\PYZus{}data1}\PY{p}{,}\PY{n}{new\PYZus{}data2}\PY{p}{]}\PY{p}{)}
\PY{n}{X}\PY{p}{,} \PY{n}{Y} \PY{o}{=} \PY{n}{np}\PY{o}{.}\PY{n}{mgrid}\PY{p}{[}\PY{n}{new\PYZus{}data1}\PY{o}{.}\PY{n}{min}\PY{p}{(}\PY{p}{)}\PY{p}{:}\PY{n}{new\PYZus{}data1}\PY{o}{.}\PY{n}{max}\PY{p}{(}\PY{p}{)}\PY{p}{:}\PY{n}{nbins}\PY{o}{*}\PY{l+m+mi}{1}\PY{n}{j}\PY{p}{,} \PY{n}{new\PYZus{}data2}\PY{o}{.}\PY{n}{min}\PY{p}{(}\PY{p}{)}\PY{p}{:}\PY{n}{new\PYZus{}data2}\PY{o}{.}\PY{n}{max}\PY{p}{(}\PY{p}{)}\PY{p}{:}\PY{n}{nbins}\PY{o}{*}\PY{l+m+mi}{1}\PY{n}{j}\PY{p}{]}
\PY{n}{Z} \PY{o}{=} \PY{n}{k}\PY{p}{(}\PY{n}{np}\PY{o}{.}\PY{n}{vstack}\PY{p}{(}\PY{p}{[}\PY{n}{X}\PY{o}{.}\PY{n}{flatten}\PY{p}{(}\PY{p}{)}\PY{p}{,} \PY{n}{Y}\PY{o}{.}\PY{n}{flatten}\PY{p}{(}\PY{p}{)}\PY{p}{]}\PY{p}{)}\PY{p}{)}


\PY{n}{fig} \PY{o}{=} \PY{n}{plt}\PY{o}{.}\PY{n}{figure}\PY{p}{(}\PY{n}{figsize}\PY{o}{=}\PY{p}{[}\PY{l+m+mi}{4}\PY{p}{,}\PY{l+m+mi}{3}\PY{p}{]}\PY{p}{,}\PY{n}{dpi}\PY{o}{=}\PY{l+m+mi}{150}\PY{p}{)} 
\PY{n}{ax} \PY{o}{=} \PY{n}{fig}\PY{o}{.}\PY{n}{add\PYZus{}subplot}\PY{p}{(}\PY{l+m+mi}{1}\PY{p}{,}\PY{l+m+mi}{1}\PY{p}{,}\PY{l+m+mi}{1}\PY{p}{)}
\PY{c+c1}{\PYZsh{}\PYZsh{}\PYZsh{}\PYZsh{} NEW FUNCTIONALITY!}
\PY{n}{ax}\PY{o}{.}\PY{n}{pcolormesh}\PY{p}{(}\PY{n}{X}\PY{p}{,} \PY{n}{Y}\PY{p}{,} \PY{n}{Z}\PY{o}{.}\PY{n}{reshape}\PY{p}{(}\PY{n}{X}\PY{o}{.}\PY{n}{shape}\PY{p}{)}\PY{p}{,} \PY{n}{shading}\PY{o}{=}\PY{l+s+s1}{\PYZsq{}}\PY{l+s+s1}{auto}\PY{l+s+s1}{\PYZsq{}}\PY{p}{,}\PY{n}{cmap}\PY{o}{=}\PY{l+s+s2}{\PYZdq{}}\PY{l+s+s2}{inferno}\PY{l+s+s2}{\PYZdq{}}\PY{p}{)}
\PY{n}{ax}\PY{o}{.}\PY{n}{set\PYZus{}xlim}\PY{p}{(}\PY{l+m+mi}{0}\PY{p}{,}\PY{l+m+mi}{1}\PY{p}{)}
\PY{n}{ax}\PY{o}{.}\PY{n}{set\PYZus{}ylim}\PY{p}{(}\PY{l+m+mi}{0}\PY{p}{,}\PY{l+m+mi}{1}\PY{p}{)}
\PY{n}{ax}\PY{o}{.}\PY{n}{set\PYZus{}xlabel}\PY{p}{(}\PY{l+s+s2}{\PYZdq{}}\PY{l+s+s2}{new\PYZus{}data1}\PY{l+s+s2}{\PYZdq{}}\PY{p}{,}\PY{n}{fontsize}\PY{o}{=}\PY{l+m+mi}{8}\PY{p}{)}
\PY{n}{ax}\PY{o}{.}\PY{n}{set\PYZus{}ylabel}\PY{p}{(}\PY{l+s+s2}{\PYZdq{}}\PY{l+s+s2}{new\PYZus{}data2}\PY{l+s+s2}{\PYZdq{}}\PY{p}{,}\PY{n}{fontsize}\PY{o}{=}\PY{l+m+mi}{8}\PY{p}{)}
\PY{n}{ax}\PY{o}{.}\PY{n}{set\PYZus{}title}\PY{p}{(}\PY{l+s+s2}{\PYZdq{}}\PY{l+s+s2}{inferno, density plot}\PY{l+s+s2}{\PYZdq{}}\PY{p}{,}\PY{n}{fontsize}\PY{o}{=}\PY{l+m+mi}{12}\PY{p}{)}
\PY{n}{plt}\PY{o}{.}\PY{n}{xticks}\PY{p}{(}\PY{n}{fontsize}\PY{o}{=}\PY{l+m+mi}{7}\PY{p}{)}
\PY{n}{plt}\PY{o}{.}\PY{n}{yticks}\PY{p}{(}\PY{n}{fontsize}\PY{o}{=}\PY{l+m+mi}{7}\PY{p}{)}

\PY{n}{fig}\PY{o}{.}\PY{n}{tight\PYZus{}layout}\PY{p}{(}\PY{p}{)}
\end{Verbatim}
\end{tcolorbox}

    \begin{center}
    \adjustimage{max size={0.9\linewidth}{0.9\paperheight}}{output_23_0.png}
    \end{center}
    { \hspace*{\fill} \\}
    
    You might have noticed,when running that last cell, that it took a
little longer than normal. This is because the data is being processed
from individual data points into a mesh where each point is connected to
adjacent points. Notice how there are regions of different color shown
around the plot. While you cannot distinguish individual data points
anymore, you can see trends and regions of higher or lower density of
data.

The number of bins affects the smoothness of the density plot, but at
the cost of differentiation. The example below shows how different bins
produce different plots.

    \begin{tcolorbox}[breakable, size=fbox, boxrule=1pt, pad at break*=1mm,colback=cellbackground, colframe=cellborder]
\prompt{In}{incolor}{15}{\boxspacing}
\begin{Verbatim}[commandchars=\\\{\}]
\PY{c+c1}{\PYZsh{}\PYZsh{}\PYZsh{}\PYZsh{} NEW IMPORT!}
\PY{k+kn}{from} \PY{n+nn}{scipy}\PY{n+nn}{.}\PY{n+nn}{stats} \PY{k+kn}{import} \PY{n}{kde}
\PY{n}{fig} \PY{o}{=} \PY{n}{plt}\PY{o}{.}\PY{n}{figure}\PY{p}{(}\PY{n}{figsize}\PY{o}{=}\PY{p}{[}\PY{l+m+mi}{8}\PY{p}{,}\PY{l+m+mi}{6}\PY{p}{]}\PY{p}{,}\PY{n}{dpi}\PY{o}{=}\PY{l+m+mi}{150}\PY{p}{)} 

\PY{c+c1}{\PYZsh{}\PYZsh{}\PYZsh{}\PYZsh{} NEW FUNCTIONALITY!}
\PY{n}{k} \PY{o}{=} \PY{n}{kde}\PY{o}{.}\PY{n}{gaussian\PYZus{}kde}\PY{p}{(}\PY{p}{[}\PY{n}{new\PYZus{}data1}\PY{p}{,}\PY{n}{new\PYZus{}data2}\PY{p}{]}\PY{p}{)}
\PY{n}{nbins}\PY{o}{=}\PY{l+m+mi}{300}
\PY{n}{X}\PY{p}{,} \PY{n}{Y} \PY{o}{=} \PY{n}{np}\PY{o}{.}\PY{n}{mgrid}\PY{p}{[}\PY{n}{new\PYZus{}data1}\PY{o}{.}\PY{n}{min}\PY{p}{(}\PY{p}{)}\PY{p}{:}\PY{n}{new\PYZus{}data1}\PY{o}{.}\PY{n}{max}\PY{p}{(}\PY{p}{)}\PY{p}{:}\PY{n}{nbins}\PY{o}{*}\PY{l+m+mi}{1}\PY{n}{j}\PY{p}{,} \PY{n}{new\PYZus{}data2}\PY{o}{.}\PY{n}{min}\PY{p}{(}\PY{p}{)}\PY{p}{:}\PY{n}{new\PYZus{}data2}\PY{o}{.}\PY{n}{max}\PY{p}{(}\PY{p}{)}\PY{p}{:}\PY{n}{nbins}\PY{o}{*}\PY{l+m+mi}{1}\PY{n}{j}\PY{p}{]}
\PY{n}{Z} \PY{o}{=} \PY{n}{k}\PY{p}{(}\PY{n}{np}\PY{o}{.}\PY{n}{vstack}\PY{p}{(}\PY{p}{[}\PY{n}{X}\PY{o}{.}\PY{n}{flatten}\PY{p}{(}\PY{p}{)}\PY{p}{,} \PY{n}{Y}\PY{o}{.}\PY{n}{flatten}\PY{p}{(}\PY{p}{)}\PY{p}{]}\PY{p}{)}\PY{p}{)}
\PY{n}{ax} \PY{o}{=} \PY{n}{fig}\PY{o}{.}\PY{n}{add\PYZus{}subplot}\PY{p}{(}\PY{l+m+mi}{2}\PY{p}{,}\PY{l+m+mi}{2}\PY{p}{,}\PY{l+m+mi}{1}\PY{p}{)}
\PY{n}{ax}\PY{o}{.}\PY{n}{pcolormesh}\PY{p}{(}\PY{n}{X}\PY{p}{,} \PY{n}{Y}\PY{p}{,} \PY{n}{Z}\PY{o}{.}\PY{n}{reshape}\PY{p}{(}\PY{n}{X}\PY{o}{.}\PY{n}{shape}\PY{p}{)}\PY{p}{,} \PY{n}{shading}\PY{o}{=}\PY{l+s+s1}{\PYZsq{}}\PY{l+s+s1}{auto}\PY{l+s+s1}{\PYZsq{}}\PY{p}{,}\PY{n}{cmap}\PY{o}{=}\PY{l+s+s2}{\PYZdq{}}\PY{l+s+s2}{inferno}\PY{l+s+s2}{\PYZdq{}}\PY{p}{)}
\PY{n}{ax}\PY{o}{.}\PY{n}{set\PYZus{}xlim}\PY{p}{(}\PY{l+m+mi}{0}\PY{p}{,}\PY{l+m+mi}{1}\PY{p}{)}
\PY{n}{ax}\PY{o}{.}\PY{n}{set\PYZus{}ylim}\PY{p}{(}\PY{l+m+mi}{0}\PY{p}{,}\PY{l+m+mi}{1}\PY{p}{)}
\PY{n}{ax}\PY{o}{.}\PY{n}{set\PYZus{}xlabel}\PY{p}{(}\PY{l+s+s2}{\PYZdq{}}\PY{l+s+s2}{new\PYZus{}data1}\PY{l+s+s2}{\PYZdq{}}\PY{p}{,}\PY{n}{fontsize}\PY{o}{=}\PY{l+m+mi}{8}\PY{p}{)}
\PY{n}{ax}\PY{o}{.}\PY{n}{set\PYZus{}ylabel}\PY{p}{(}\PY{l+s+s2}{\PYZdq{}}\PY{l+s+s2}{new\PYZus{}data2}\PY{l+s+s2}{\PYZdq{}}\PY{p}{,}\PY{n}{fontsize}\PY{o}{=}\PY{l+m+mi}{8}\PY{p}{)}
\PY{n}{ax}\PY{o}{.}\PY{n}{set\PYZus{}title}\PY{p}{(}\PY{l+s+s2}{\PYZdq{}}\PY{l+s+s2}{inferno, nbins = 300}\PY{l+s+s2}{\PYZdq{}}\PY{p}{,}\PY{n}{fontsize}\PY{o}{=}\PY{l+m+mi}{12}\PY{p}{)}
\PY{n}{plt}\PY{o}{.}\PY{n}{xticks}\PY{p}{(}\PY{n}{fontsize}\PY{o}{=}\PY{l+m+mi}{7}\PY{p}{)}
\PY{n}{plt}\PY{o}{.}\PY{n}{yticks}\PY{p}{(}\PY{n}{fontsize}\PY{o}{=}\PY{l+m+mi}{7}\PY{p}{)}


\PY{n}{k} \PY{o}{=} \PY{n}{kde}\PY{o}{.}\PY{n}{gaussian\PYZus{}kde}\PY{p}{(}\PY{p}{[}\PY{n}{new\PYZus{}data1}\PY{p}{,}\PY{n}{new\PYZus{}data2}\PY{p}{]}\PY{p}{)}
\PY{n}{nbins}\PY{o}{=}\PY{l+m+mi}{200}
\PY{n}{X}\PY{p}{,} \PY{n}{Y} \PY{o}{=} \PY{n}{np}\PY{o}{.}\PY{n}{mgrid}\PY{p}{[}\PY{n}{new\PYZus{}data1}\PY{o}{.}\PY{n}{min}\PY{p}{(}\PY{p}{)}\PY{p}{:}\PY{n}{new\PYZus{}data1}\PY{o}{.}\PY{n}{max}\PY{p}{(}\PY{p}{)}\PY{p}{:}\PY{n}{nbins}\PY{o}{*}\PY{l+m+mi}{1}\PY{n}{j}\PY{p}{,} \PY{n}{new\PYZus{}data2}\PY{o}{.}\PY{n}{min}\PY{p}{(}\PY{p}{)}\PY{p}{:}\PY{n}{new\PYZus{}data2}\PY{o}{.}\PY{n}{max}\PY{p}{(}\PY{p}{)}\PY{p}{:}\PY{n}{nbins}\PY{o}{*}\PY{l+m+mi}{1}\PY{n}{j}\PY{p}{]}
\PY{n}{Z} \PY{o}{=} \PY{n}{k}\PY{p}{(}\PY{n}{np}\PY{o}{.}\PY{n}{vstack}\PY{p}{(}\PY{p}{[}\PY{n}{X}\PY{o}{.}\PY{n}{flatten}\PY{p}{(}\PY{p}{)}\PY{p}{,} \PY{n}{Y}\PY{o}{.}\PY{n}{flatten}\PY{p}{(}\PY{p}{)}\PY{p}{]}\PY{p}{)}\PY{p}{)}
\PY{n}{ax} \PY{o}{=} \PY{n}{fig}\PY{o}{.}\PY{n}{add\PYZus{}subplot}\PY{p}{(}\PY{l+m+mi}{2}\PY{p}{,}\PY{l+m+mi}{2}\PY{p}{,}\PY{l+m+mi}{2}\PY{p}{)}
\PY{n}{ax}\PY{o}{.}\PY{n}{pcolormesh}\PY{p}{(}\PY{n}{X}\PY{p}{,} \PY{n}{Y}\PY{p}{,} \PY{n}{Z}\PY{o}{.}\PY{n}{reshape}\PY{p}{(}\PY{n}{X}\PY{o}{.}\PY{n}{shape}\PY{p}{)}\PY{p}{,} \PY{n}{shading}\PY{o}{=}\PY{l+s+s1}{\PYZsq{}}\PY{l+s+s1}{auto}\PY{l+s+s1}{\PYZsq{}}\PY{p}{,}\PY{n}{cmap}\PY{o}{=}\PY{l+s+s2}{\PYZdq{}}\PY{l+s+s2}{inferno}\PY{l+s+s2}{\PYZdq{}}\PY{p}{)}
\PY{n}{ax}\PY{o}{.}\PY{n}{set\PYZus{}xlim}\PY{p}{(}\PY{l+m+mi}{0}\PY{p}{,}\PY{l+m+mi}{1}\PY{p}{)}
\PY{n}{ax}\PY{o}{.}\PY{n}{set\PYZus{}ylim}\PY{p}{(}\PY{l+m+mi}{0}\PY{p}{,}\PY{l+m+mi}{1}\PY{p}{)}
\PY{n}{ax}\PY{o}{.}\PY{n}{set\PYZus{}xlabel}\PY{p}{(}\PY{l+s+s2}{\PYZdq{}}\PY{l+s+s2}{new\PYZus{}data1}\PY{l+s+s2}{\PYZdq{}}\PY{p}{,}\PY{n}{fontsize}\PY{o}{=}\PY{l+m+mi}{8}\PY{p}{)}
\PY{n}{ax}\PY{o}{.}\PY{n}{set\PYZus{}ylabel}\PY{p}{(}\PY{l+s+s2}{\PYZdq{}}\PY{l+s+s2}{new\PYZus{}data2}\PY{l+s+s2}{\PYZdq{}}\PY{p}{,}\PY{n}{fontsize}\PY{o}{=}\PY{l+m+mi}{8}\PY{p}{)}
\PY{n}{ax}\PY{o}{.}\PY{n}{set\PYZus{}title}\PY{p}{(}\PY{l+s+s2}{\PYZdq{}}\PY{l+s+s2}{inferno, nbins = 200}\PY{l+s+s2}{\PYZdq{}}\PY{p}{,}\PY{n}{fontsize}\PY{o}{=}\PY{l+m+mi}{12}\PY{p}{)}
\PY{n}{plt}\PY{o}{.}\PY{n}{xticks}\PY{p}{(}\PY{n}{fontsize}\PY{o}{=}\PY{l+m+mi}{7}\PY{p}{)}
\PY{n}{plt}\PY{o}{.}\PY{n}{yticks}\PY{p}{(}\PY{n}{fontsize}\PY{o}{=}\PY{l+m+mi}{7}\PY{p}{)}

\PY{n}{k} \PY{o}{=} \PY{n}{kde}\PY{o}{.}\PY{n}{gaussian\PYZus{}kde}\PY{p}{(}\PY{p}{[}\PY{n}{new\PYZus{}data1}\PY{p}{,}\PY{n}{new\PYZus{}data2}\PY{p}{]}\PY{p}{)}
\PY{n}{nbins}\PY{o}{=}\PY{l+m+mi}{100}
\PY{n}{X}\PY{p}{,} \PY{n}{Y} \PY{o}{=} \PY{n}{np}\PY{o}{.}\PY{n}{mgrid}\PY{p}{[}\PY{n}{new\PYZus{}data1}\PY{o}{.}\PY{n}{min}\PY{p}{(}\PY{p}{)}\PY{p}{:}\PY{n}{new\PYZus{}data1}\PY{o}{.}\PY{n}{max}\PY{p}{(}\PY{p}{)}\PY{p}{:}\PY{n}{nbins}\PY{o}{*}\PY{l+m+mi}{1}\PY{n}{j}\PY{p}{,} \PY{n}{new\PYZus{}data2}\PY{o}{.}\PY{n}{min}\PY{p}{(}\PY{p}{)}\PY{p}{:}\PY{n}{new\PYZus{}data2}\PY{o}{.}\PY{n}{max}\PY{p}{(}\PY{p}{)}\PY{p}{:}\PY{n}{nbins}\PY{o}{*}\PY{l+m+mi}{1}\PY{n}{j}\PY{p}{]}
\PY{n}{Z} \PY{o}{=} \PY{n}{k}\PY{p}{(}\PY{n}{np}\PY{o}{.}\PY{n}{vstack}\PY{p}{(}\PY{p}{[}\PY{n}{X}\PY{o}{.}\PY{n}{flatten}\PY{p}{(}\PY{p}{)}\PY{p}{,} \PY{n}{Y}\PY{o}{.}\PY{n}{flatten}\PY{p}{(}\PY{p}{)}\PY{p}{]}\PY{p}{)}\PY{p}{)}
\PY{n}{ax} \PY{o}{=} \PY{n}{fig}\PY{o}{.}\PY{n}{add\PYZus{}subplot}\PY{p}{(}\PY{l+m+mi}{2}\PY{p}{,}\PY{l+m+mi}{2}\PY{p}{,}\PY{l+m+mi}{3}\PY{p}{)}
\PY{n}{ax}\PY{o}{.}\PY{n}{pcolormesh}\PY{p}{(}\PY{n}{X}\PY{p}{,} \PY{n}{Y}\PY{p}{,} \PY{n}{Z}\PY{o}{.}\PY{n}{reshape}\PY{p}{(}\PY{n}{X}\PY{o}{.}\PY{n}{shape}\PY{p}{)}\PY{p}{,} \PY{n}{shading}\PY{o}{=}\PY{l+s+s1}{\PYZsq{}}\PY{l+s+s1}{auto}\PY{l+s+s1}{\PYZsq{}}\PY{p}{,}\PY{n}{cmap}\PY{o}{=}\PY{l+s+s2}{\PYZdq{}}\PY{l+s+s2}{inferno}\PY{l+s+s2}{\PYZdq{}}\PY{p}{)}
\PY{n}{ax}\PY{o}{.}\PY{n}{set\PYZus{}xlim}\PY{p}{(}\PY{l+m+mi}{0}\PY{p}{,}\PY{l+m+mi}{1}\PY{p}{)}
\PY{n}{ax}\PY{o}{.}\PY{n}{set\PYZus{}ylim}\PY{p}{(}\PY{l+m+mi}{0}\PY{p}{,}\PY{l+m+mi}{1}\PY{p}{)}
\PY{n}{ax}\PY{o}{.}\PY{n}{set\PYZus{}xlabel}\PY{p}{(}\PY{l+s+s2}{\PYZdq{}}\PY{l+s+s2}{new\PYZus{}data1}\PY{l+s+s2}{\PYZdq{}}\PY{p}{,}\PY{n}{fontsize}\PY{o}{=}\PY{l+m+mi}{8}\PY{p}{)}
\PY{n}{ax}\PY{o}{.}\PY{n}{set\PYZus{}ylabel}\PY{p}{(}\PY{l+s+s2}{\PYZdq{}}\PY{l+s+s2}{new\PYZus{}data2}\PY{l+s+s2}{\PYZdq{}}\PY{p}{,}\PY{n}{fontsize}\PY{o}{=}\PY{l+m+mi}{8}\PY{p}{)}
\PY{n}{ax}\PY{o}{.}\PY{n}{set\PYZus{}title}\PY{p}{(}\PY{l+s+s2}{\PYZdq{}}\PY{l+s+s2}{inferno, nbins = 100}\PY{l+s+s2}{\PYZdq{}}\PY{p}{,}\PY{n}{fontsize}\PY{o}{=}\PY{l+m+mi}{12}\PY{p}{)}
\PY{n}{plt}\PY{o}{.}\PY{n}{xticks}\PY{p}{(}\PY{n}{fontsize}\PY{o}{=}\PY{l+m+mi}{7}\PY{p}{)}
\PY{n}{plt}\PY{o}{.}\PY{n}{yticks}\PY{p}{(}\PY{n}{fontsize}\PY{o}{=}\PY{l+m+mi}{7}\PY{p}{)}

\PY{n}{k} \PY{o}{=} \PY{n}{kde}\PY{o}{.}\PY{n}{gaussian\PYZus{}kde}\PY{p}{(}\PY{p}{[}\PY{n}{new\PYZus{}data1}\PY{p}{,}\PY{n}{new\PYZus{}data2}\PY{p}{]}\PY{p}{)}
\PY{n}{nbins}\PY{o}{=}\PY{l+m+mi}{25}
\PY{n}{X}\PY{p}{,} \PY{n}{Y} \PY{o}{=} \PY{n}{np}\PY{o}{.}\PY{n}{mgrid}\PY{p}{[}\PY{n}{new\PYZus{}data1}\PY{o}{.}\PY{n}{min}\PY{p}{(}\PY{p}{)}\PY{p}{:}\PY{n}{new\PYZus{}data1}\PY{o}{.}\PY{n}{max}\PY{p}{(}\PY{p}{)}\PY{p}{:}\PY{n}{nbins}\PY{o}{*}\PY{l+m+mi}{1}\PY{n}{j}\PY{p}{,} \PY{n}{new\PYZus{}data2}\PY{o}{.}\PY{n}{min}\PY{p}{(}\PY{p}{)}\PY{p}{:}\PY{n}{new\PYZus{}data2}\PY{o}{.}\PY{n}{max}\PY{p}{(}\PY{p}{)}\PY{p}{:}\PY{n}{nbins}\PY{o}{*}\PY{l+m+mi}{1}\PY{n}{j}\PY{p}{]}
\PY{n}{Z} \PY{o}{=} \PY{n}{k}\PY{p}{(}\PY{n}{np}\PY{o}{.}\PY{n}{vstack}\PY{p}{(}\PY{p}{[}\PY{n}{X}\PY{o}{.}\PY{n}{flatten}\PY{p}{(}\PY{p}{)}\PY{p}{,} \PY{n}{Y}\PY{o}{.}\PY{n}{flatten}\PY{p}{(}\PY{p}{)}\PY{p}{]}\PY{p}{)}\PY{p}{)}
\PY{n}{ax} \PY{o}{=} \PY{n}{fig}\PY{o}{.}\PY{n}{add\PYZus{}subplot}\PY{p}{(}\PY{l+m+mi}{2}\PY{p}{,}\PY{l+m+mi}{2}\PY{p}{,}\PY{l+m+mi}{4}\PY{p}{)}
\PY{n}{ax}\PY{o}{.}\PY{n}{pcolormesh}\PY{p}{(}\PY{n}{X}\PY{p}{,} \PY{n}{Y}\PY{p}{,} \PY{n}{Z}\PY{o}{.}\PY{n}{reshape}\PY{p}{(}\PY{n}{X}\PY{o}{.}\PY{n}{shape}\PY{p}{)}\PY{p}{,} \PY{n}{shading}\PY{o}{=}\PY{l+s+s1}{\PYZsq{}}\PY{l+s+s1}{auto}\PY{l+s+s1}{\PYZsq{}}\PY{p}{,}\PY{n}{cmap}\PY{o}{=}\PY{l+s+s2}{\PYZdq{}}\PY{l+s+s2}{inferno}\PY{l+s+s2}{\PYZdq{}}\PY{p}{)}
\PY{n}{ax}\PY{o}{.}\PY{n}{set\PYZus{}xlim}\PY{p}{(}\PY{l+m+mi}{0}\PY{p}{,}\PY{l+m+mi}{1}\PY{p}{)}
\PY{n}{ax}\PY{o}{.}\PY{n}{set\PYZus{}ylim}\PY{p}{(}\PY{l+m+mi}{0}\PY{p}{,}\PY{l+m+mi}{1}\PY{p}{)}
\PY{n}{ax}\PY{o}{.}\PY{n}{set\PYZus{}xlabel}\PY{p}{(}\PY{l+s+s2}{\PYZdq{}}\PY{l+s+s2}{new\PYZus{}data1}\PY{l+s+s2}{\PYZdq{}}\PY{p}{,}\PY{n}{fontsize}\PY{o}{=}\PY{l+m+mi}{8}\PY{p}{)}
\PY{n}{ax}\PY{o}{.}\PY{n}{set\PYZus{}ylabel}\PY{p}{(}\PY{l+s+s2}{\PYZdq{}}\PY{l+s+s2}{new\PYZus{}data2}\PY{l+s+s2}{\PYZdq{}}\PY{p}{,}\PY{n}{fontsize}\PY{o}{=}\PY{l+m+mi}{8}\PY{p}{)}
\PY{n}{ax}\PY{o}{.}\PY{n}{set\PYZus{}title}\PY{p}{(}\PY{l+s+s2}{\PYZdq{}}\PY{l+s+s2}{inferno, nbins = 25}\PY{l+s+s2}{\PYZdq{}}\PY{p}{,}\PY{n}{fontsize}\PY{o}{=}\PY{l+m+mi}{12}\PY{p}{)}
\PY{n}{plt}\PY{o}{.}\PY{n}{xticks}\PY{p}{(}\PY{n}{fontsize}\PY{o}{=}\PY{l+m+mi}{7}\PY{p}{)}
\PY{n}{plt}\PY{o}{.}\PY{n}{yticks}\PY{p}{(}\PY{n}{fontsize}\PY{o}{=}\PY{l+m+mi}{7}\PY{p}{)}

\PY{n}{fig}\PY{o}{.}\PY{n}{tight\PYZus{}layout}\PY{p}{(}\PY{p}{)}
\end{Verbatim}
\end{tcolorbox}

    \begin{center}
    \adjustimage{max size={0.9\linewidth}{0.9\paperheight}}{output_25_0.png}
    \end{center}
    { \hspace*{\fill} \\}
    
    You can also try with different colormaps until you find the one that's
right for the data you're presenting. For example, the \texttt{terrain}
colormap runs blue-brown-green-white similar to a topological map
showing altitudes!

    \begin{tcolorbox}[breakable, size=fbox, boxrule=1pt, pad at break*=1mm,colback=cellbackground, colframe=cellborder]
\prompt{In}{incolor}{16}{\boxspacing}
\begin{Verbatim}[commandchars=\\\{\}]
\PY{c+c1}{\PYZsh{}\PYZsh{}\PYZsh{}\PYZsh{} NEW IMPORT!}
\PY{k+kn}{from} \PY{n+nn}{scipy}\PY{n+nn}{.}\PY{n+nn}{stats} \PY{k+kn}{import} \PY{n}{kde}
\PY{n}{fig} \PY{o}{=} \PY{n}{plt}\PY{o}{.}\PY{n}{figure}\PY{p}{(}\PY{n}{figsize}\PY{o}{=}\PY{p}{[}\PY{l+m+mi}{8}\PY{p}{,}\PY{l+m+mi}{6}\PY{p}{]}\PY{p}{,}\PY{n}{dpi}\PY{o}{=}\PY{l+m+mi}{150}\PY{p}{)} 

\PY{c+c1}{\PYZsh{}\PYZsh{}\PYZsh{}\PYZsh{} NEW FUNCTIONALITY!}
\PY{n}{k} \PY{o}{=} \PY{n}{kde}\PY{o}{.}\PY{n}{gaussian\PYZus{}kde}\PY{p}{(}\PY{p}{[}\PY{n}{new\PYZus{}data1}\PY{p}{,}\PY{n}{new\PYZus{}data2}\PY{p}{]}\PY{p}{)}
\PY{n}{nbins}\PY{o}{=}\PY{l+m+mi}{300}
\PY{n}{X}\PY{p}{,} \PY{n}{Y} \PY{o}{=} \PY{n}{np}\PY{o}{.}\PY{n}{mgrid}\PY{p}{[}\PY{n}{new\PYZus{}data1}\PY{o}{.}\PY{n}{min}\PY{p}{(}\PY{p}{)}\PY{p}{:}\PY{n}{new\PYZus{}data1}\PY{o}{.}\PY{n}{max}\PY{p}{(}\PY{p}{)}\PY{p}{:}\PY{n}{nbins}\PY{o}{*}\PY{l+m+mi}{1}\PY{n}{j}\PY{p}{,} \PY{n}{new\PYZus{}data2}\PY{o}{.}\PY{n}{min}\PY{p}{(}\PY{p}{)}\PY{p}{:}\PY{n}{new\PYZus{}data2}\PY{o}{.}\PY{n}{max}\PY{p}{(}\PY{p}{)}\PY{p}{:}\PY{n}{nbins}\PY{o}{*}\PY{l+m+mi}{1}\PY{n}{j}\PY{p}{]}
\PY{n}{Z} \PY{o}{=} \PY{n}{k}\PY{p}{(}\PY{n}{np}\PY{o}{.}\PY{n}{vstack}\PY{p}{(}\PY{p}{[}\PY{n}{X}\PY{o}{.}\PY{n}{flatten}\PY{p}{(}\PY{p}{)}\PY{p}{,} \PY{n}{Y}\PY{o}{.}\PY{n}{flatten}\PY{p}{(}\PY{p}{)}\PY{p}{]}\PY{p}{)}\PY{p}{)}
\PY{n}{ax} \PY{o}{=} \PY{n}{fig}\PY{o}{.}\PY{n}{add\PYZus{}subplot}\PY{p}{(}\PY{l+m+mi}{2}\PY{p}{,}\PY{l+m+mi}{2}\PY{p}{,}\PY{l+m+mi}{1}\PY{p}{)}
\PY{n}{ax}\PY{o}{.}\PY{n}{pcolormesh}\PY{p}{(}\PY{n}{X}\PY{p}{,} \PY{n}{Y}\PY{p}{,} \PY{n}{Z}\PY{o}{.}\PY{n}{reshape}\PY{p}{(}\PY{n}{X}\PY{o}{.}\PY{n}{shape}\PY{p}{)}\PY{p}{,} \PY{n}{shading}\PY{o}{=}\PY{l+s+s1}{\PYZsq{}}\PY{l+s+s1}{auto}\PY{l+s+s1}{\PYZsq{}}\PY{p}{,}\PY{n}{cmap}\PY{o}{=}\PY{l+s+s2}{\PYZdq{}}\PY{l+s+s2}{inferno}\PY{l+s+s2}{\PYZdq{}}\PY{p}{)}
\PY{n}{ax}\PY{o}{.}\PY{n}{set\PYZus{}xlim}\PY{p}{(}\PY{l+m+mi}{0}\PY{p}{,}\PY{l+m+mi}{1}\PY{p}{)}
\PY{n}{ax}\PY{o}{.}\PY{n}{set\PYZus{}ylim}\PY{p}{(}\PY{l+m+mi}{0}\PY{p}{,}\PY{l+m+mi}{1}\PY{p}{)}
\PY{n}{ax}\PY{o}{.}\PY{n}{set\PYZus{}xlabel}\PY{p}{(}\PY{l+s+s2}{\PYZdq{}}\PY{l+s+s2}{new\PYZus{}data1}\PY{l+s+s2}{\PYZdq{}}\PY{p}{,}\PY{n}{fontsize}\PY{o}{=}\PY{l+m+mi}{8}\PY{p}{)}
\PY{n}{ax}\PY{o}{.}\PY{n}{set\PYZus{}ylabel}\PY{p}{(}\PY{l+s+s2}{\PYZdq{}}\PY{l+s+s2}{new\PYZus{}data2}\PY{l+s+s2}{\PYZdq{}}\PY{p}{,}\PY{n}{fontsize}\PY{o}{=}\PY{l+m+mi}{8}\PY{p}{)}
\PY{n}{ax}\PY{o}{.}\PY{n}{set\PYZus{}title}\PY{p}{(}\PY{l+s+s2}{\PYZdq{}}\PY{l+s+s2}{inferno}\PY{l+s+s2}{\PYZdq{}}\PY{p}{,}\PY{n}{fontsize}\PY{o}{=}\PY{l+m+mi}{12}\PY{p}{)}
\PY{n}{plt}\PY{o}{.}\PY{n}{xticks}\PY{p}{(}\PY{n}{fontsize}\PY{o}{=}\PY{l+m+mi}{7}\PY{p}{)}
\PY{n}{plt}\PY{o}{.}\PY{n}{yticks}\PY{p}{(}\PY{n}{fontsize}\PY{o}{=}\PY{l+m+mi}{7}\PY{p}{)}

\PY{n}{ax} \PY{o}{=} \PY{n}{fig}\PY{o}{.}\PY{n}{add\PYZus{}subplot}\PY{p}{(}\PY{l+m+mi}{2}\PY{p}{,}\PY{l+m+mi}{2}\PY{p}{,}\PY{l+m+mi}{2}\PY{p}{)}
\PY{n}{ax}\PY{o}{.}\PY{n}{pcolormesh}\PY{p}{(}\PY{n}{X}\PY{p}{,} \PY{n}{Y}\PY{p}{,} \PY{n}{Z}\PY{o}{.}\PY{n}{reshape}\PY{p}{(}\PY{n}{X}\PY{o}{.}\PY{n}{shape}\PY{p}{)}\PY{p}{,} \PY{n}{shading}\PY{o}{=}\PY{l+s+s1}{\PYZsq{}}\PY{l+s+s1}{auto}\PY{l+s+s1}{\PYZsq{}}\PY{p}{,}\PY{n}{cmap}\PY{o}{=}\PY{l+s+s2}{\PYZdq{}}\PY{l+s+s2}{jet}\PY{l+s+s2}{\PYZdq{}}\PY{p}{)}
\PY{n}{ax}\PY{o}{.}\PY{n}{set\PYZus{}xlim}\PY{p}{(}\PY{l+m+mi}{0}\PY{p}{,}\PY{l+m+mi}{1}\PY{p}{)}
\PY{n}{ax}\PY{o}{.}\PY{n}{set\PYZus{}ylim}\PY{p}{(}\PY{l+m+mi}{0}\PY{p}{,}\PY{l+m+mi}{1}\PY{p}{)}
\PY{n}{ax}\PY{o}{.}\PY{n}{set\PYZus{}xlabel}\PY{p}{(}\PY{l+s+s2}{\PYZdq{}}\PY{l+s+s2}{new\PYZus{}data1}\PY{l+s+s2}{\PYZdq{}}\PY{p}{,}\PY{n}{fontsize}\PY{o}{=}\PY{l+m+mi}{8}\PY{p}{)}
\PY{n}{ax}\PY{o}{.}\PY{n}{set\PYZus{}ylabel}\PY{p}{(}\PY{l+s+s2}{\PYZdq{}}\PY{l+s+s2}{new\PYZus{}data2}\PY{l+s+s2}{\PYZdq{}}\PY{p}{,}\PY{n}{fontsize}\PY{o}{=}\PY{l+m+mi}{8}\PY{p}{)}
\PY{n}{ax}\PY{o}{.}\PY{n}{set\PYZus{}title}\PY{p}{(}\PY{l+s+s2}{\PYZdq{}}\PY{l+s+s2}{jet}\PY{l+s+s2}{\PYZdq{}}\PY{p}{,}\PY{n}{fontsize}\PY{o}{=}\PY{l+m+mi}{12}\PY{p}{)}
\PY{n}{plt}\PY{o}{.}\PY{n}{xticks}\PY{p}{(}\PY{n}{fontsize}\PY{o}{=}\PY{l+m+mi}{7}\PY{p}{)}
\PY{n}{plt}\PY{o}{.}\PY{n}{yticks}\PY{p}{(}\PY{n}{fontsize}\PY{o}{=}\PY{l+m+mi}{7}\PY{p}{)}

\PY{n}{ax} \PY{o}{=} \PY{n}{fig}\PY{o}{.}\PY{n}{add\PYZus{}subplot}\PY{p}{(}\PY{l+m+mi}{2}\PY{p}{,}\PY{l+m+mi}{2}\PY{p}{,}\PY{l+m+mi}{3}\PY{p}{)}
\PY{n}{ax}\PY{o}{.}\PY{n}{pcolormesh}\PY{p}{(}\PY{n}{X}\PY{p}{,} \PY{n}{Y}\PY{p}{,} \PY{n}{Z}\PY{o}{.}\PY{n}{reshape}\PY{p}{(}\PY{n}{X}\PY{o}{.}\PY{n}{shape}\PY{p}{)}\PY{p}{,} \PY{n}{shading}\PY{o}{=}\PY{l+s+s1}{\PYZsq{}}\PY{l+s+s1}{auto}\PY{l+s+s1}{\PYZsq{}}\PY{p}{,}\PY{n}{cmap}\PY{o}{=}\PY{l+s+s2}{\PYZdq{}}\PY{l+s+s2}{terrain}\PY{l+s+s2}{\PYZdq{}}\PY{p}{)}
\PY{n}{ax}\PY{o}{.}\PY{n}{set\PYZus{}xlim}\PY{p}{(}\PY{l+m+mi}{0}\PY{p}{,}\PY{l+m+mi}{1}\PY{p}{)}
\PY{n}{ax}\PY{o}{.}\PY{n}{set\PYZus{}ylim}\PY{p}{(}\PY{l+m+mi}{0}\PY{p}{,}\PY{l+m+mi}{1}\PY{p}{)}
\PY{n}{ax}\PY{o}{.}\PY{n}{set\PYZus{}xlabel}\PY{p}{(}\PY{l+s+s2}{\PYZdq{}}\PY{l+s+s2}{new\PYZus{}data1}\PY{l+s+s2}{\PYZdq{}}\PY{p}{,}\PY{n}{fontsize}\PY{o}{=}\PY{l+m+mi}{8}\PY{p}{)}
\PY{n}{ax}\PY{o}{.}\PY{n}{set\PYZus{}ylabel}\PY{p}{(}\PY{l+s+s2}{\PYZdq{}}\PY{l+s+s2}{new\PYZus{}data2}\PY{l+s+s2}{\PYZdq{}}\PY{p}{,}\PY{n}{fontsize}\PY{o}{=}\PY{l+m+mi}{8}\PY{p}{)}
\PY{n}{ax}\PY{o}{.}\PY{n}{set\PYZus{}title}\PY{p}{(}\PY{l+s+s2}{\PYZdq{}}\PY{l+s+s2}{terrain}\PY{l+s+s2}{\PYZdq{}}\PY{p}{,}\PY{n}{fontsize}\PY{o}{=}\PY{l+m+mi}{12}\PY{p}{)}
\PY{n}{plt}\PY{o}{.}\PY{n}{xticks}\PY{p}{(}\PY{n}{fontsize}\PY{o}{=}\PY{l+m+mi}{7}\PY{p}{)}
\PY{n}{plt}\PY{o}{.}\PY{n}{yticks}\PY{p}{(}\PY{n}{fontsize}\PY{o}{=}\PY{l+m+mi}{7}\PY{p}{)}

\PY{n}{ax} \PY{o}{=} \PY{n}{fig}\PY{o}{.}\PY{n}{add\PYZus{}subplot}\PY{p}{(}\PY{l+m+mi}{2}\PY{p}{,}\PY{l+m+mi}{2}\PY{p}{,}\PY{l+m+mi}{4}\PY{p}{)}
\PY{n}{ax}\PY{o}{.}\PY{n}{pcolormesh}\PY{p}{(}\PY{n}{X}\PY{p}{,} \PY{n}{Y}\PY{p}{,} \PY{n}{Z}\PY{o}{.}\PY{n}{reshape}\PY{p}{(}\PY{n}{X}\PY{o}{.}\PY{n}{shape}\PY{p}{)}\PY{p}{,} \PY{n}{shading}\PY{o}{=}\PY{l+s+s1}{\PYZsq{}}\PY{l+s+s1}{auto}\PY{l+s+s1}{\PYZsq{}}\PY{p}{,}\PY{n}{cmap}\PY{o}{=}\PY{l+s+s2}{\PYZdq{}}\PY{l+s+s2}{gist\PYZus{}ncar}\PY{l+s+s2}{\PYZdq{}}\PY{p}{)}
\PY{n}{ax}\PY{o}{.}\PY{n}{set\PYZus{}xlim}\PY{p}{(}\PY{l+m+mi}{0}\PY{p}{,}\PY{l+m+mi}{1}\PY{p}{)}
\PY{n}{ax}\PY{o}{.}\PY{n}{set\PYZus{}ylim}\PY{p}{(}\PY{l+m+mi}{0}\PY{p}{,}\PY{l+m+mi}{1}\PY{p}{)}
\PY{n}{ax}\PY{o}{.}\PY{n}{set\PYZus{}xlabel}\PY{p}{(}\PY{l+s+s2}{\PYZdq{}}\PY{l+s+s2}{new\PYZus{}data1}\PY{l+s+s2}{\PYZdq{}}\PY{p}{,}\PY{n}{fontsize}\PY{o}{=}\PY{l+m+mi}{8}\PY{p}{)}
\PY{n}{ax}\PY{o}{.}\PY{n}{set\PYZus{}ylabel}\PY{p}{(}\PY{l+s+s2}{\PYZdq{}}\PY{l+s+s2}{new\PYZus{}data2}\PY{l+s+s2}{\PYZdq{}}\PY{p}{,}\PY{n}{fontsize}\PY{o}{=}\PY{l+m+mi}{8}\PY{p}{)}
\PY{n}{ax}\PY{o}{.}\PY{n}{set\PYZus{}title}\PY{p}{(}\PY{l+s+s2}{\PYZdq{}}\PY{l+s+s2}{gist\PYZus{}ncar}\PY{l+s+s2}{\PYZdq{}}\PY{p}{,}\PY{n}{fontsize}\PY{o}{=}\PY{l+m+mi}{12}\PY{p}{)}
\PY{n}{plt}\PY{o}{.}\PY{n}{xticks}\PY{p}{(}\PY{n}{fontsize}\PY{o}{=}\PY{l+m+mi}{7}\PY{p}{)}
\PY{n}{plt}\PY{o}{.}\PY{n}{yticks}\PY{p}{(}\PY{n}{fontsize}\PY{o}{=}\PY{l+m+mi}{7}\PY{p}{)}

\PY{n}{fig}\PY{o}{.}\PY{n}{tight\PYZus{}layout}\PY{p}{(}\PY{p}{)}
\end{Verbatim}
\end{tcolorbox}

    \begin{center}
    \adjustimage{max size={0.9\linewidth}{0.9\paperheight}}{output_27_0.png}
    \end{center}
    { \hspace*{\fill} \\}
    
    \hypertarget{histograms}{%
\subsubsection{Histograms}\label{histograms}}

2D Density plots are great for cases where you're comparing two
different data sets, but what if you want something similar for just one
dataset? You can use a histogram!

In the example below, I've generated random data that follows a normal
(gaussian) distribution. See how different numbers of bins can change
the shape and structure of the histogram.

Note the y-axis values as well. As we increase the number of bins, the
number of points in a given bin goes down on average.

    \begin{tcolorbox}[breakable, size=fbox, boxrule=1pt, pad at break*=1mm,colback=cellbackground, colframe=cellborder]
\prompt{In}{incolor}{17}{\boxspacing}
\begin{Verbatim}[commandchars=\\\{\}]
\PY{n}{hist\PYZus{}data}\PY{o}{=}\PY{n}{np}\PY{o}{.}\PY{n}{random}\PY{o}{.}\PY{n}{normal}\PY{p}{(}\PY{l+m+mf}{0.5}\PY{p}{,}\PY{l+m+mf}{0.1}\PY{p}{,}\PY{n}{size}\PY{o}{=}\PY{l+m+mi}{10000}\PY{p}{)}
\PY{n}{fig} \PY{o}{=} \PY{n}{plt}\PY{o}{.}\PY{n}{figure}\PY{p}{(}\PY{n}{figsize}\PY{o}{=}\PY{p}{[}\PY{l+m+mi}{8}\PY{p}{,}\PY{l+m+mi}{6}\PY{p}{]}\PY{p}{,}\PY{n}{dpi}\PY{o}{=}\PY{l+m+mi}{150}\PY{p}{)} 

\PY{n}{ax} \PY{o}{=} \PY{n}{fig}\PY{o}{.}\PY{n}{add\PYZus{}subplot}\PY{p}{(}\PY{l+m+mi}{2}\PY{p}{,}\PY{l+m+mi}{2}\PY{p}{,}\PY{l+m+mi}{1}\PY{p}{)}
\PY{n}{ax}\PY{o}{.}\PY{n}{hist}\PY{p}{(}\PY{n}{hist\PYZus{}data}\PY{p}{,}\PY{n}{bins}\PY{o}{=}\PY{l+m+mi}{10}\PY{p}{)}
\PY{n}{ax}\PY{o}{.}\PY{n}{set\PYZus{}xlim}\PY{p}{(}\PY{l+m+mi}{0}\PY{p}{,}\PY{l+m+mi}{1}\PY{p}{)}
\PY{n}{ax}\PY{o}{.}\PY{n}{set\PYZus{}xlabel}\PY{p}{(}\PY{l+s+s2}{\PYZdq{}}\PY{l+s+s2}{new\PYZus{}data1}\PY{l+s+s2}{\PYZdq{}}\PY{p}{,}\PY{n}{fontsize}\PY{o}{=}\PY{l+m+mi}{8}\PY{p}{)}
\PY{n}{ax}\PY{o}{.}\PY{n}{set\PYZus{}ylabel}\PY{p}{(}\PY{l+s+s2}{\PYZdq{}}\PY{l+s+s2}{new\PYZus{}data2}\PY{l+s+s2}{\PYZdq{}}\PY{p}{,}\PY{n}{fontsize}\PY{o}{=}\PY{l+m+mi}{8}\PY{p}{)}
\PY{n}{ax}\PY{o}{.}\PY{n}{set\PYZus{}title}\PY{p}{(}\PY{l+s+s2}{\PYZdq{}}\PY{l+s+s2}{bins: 10}\PY{l+s+s2}{\PYZdq{}}\PY{p}{,}\PY{n}{fontsize}\PY{o}{=}\PY{l+m+mi}{12}\PY{p}{)}
\PY{n}{plt}\PY{o}{.}\PY{n}{xticks}\PY{p}{(}\PY{n}{fontsize}\PY{o}{=}\PY{l+m+mi}{7}\PY{p}{)}
\PY{n}{plt}\PY{o}{.}\PY{n}{yticks}\PY{p}{(}\PY{n}{fontsize}\PY{o}{=}\PY{l+m+mi}{7}\PY{p}{)}

\PY{n}{ax} \PY{o}{=} \PY{n}{fig}\PY{o}{.}\PY{n}{add\PYZus{}subplot}\PY{p}{(}\PY{l+m+mi}{2}\PY{p}{,}\PY{l+m+mi}{2}\PY{p}{,}\PY{l+m+mi}{2}\PY{p}{)}
\PY{n}{ax}\PY{o}{.}\PY{n}{hist}\PY{p}{(}\PY{n}{hist\PYZus{}data}\PY{p}{,}\PY{n}{bins}\PY{o}{=}\PY{l+m+mi}{25}\PY{p}{)}
\PY{n}{ax}\PY{o}{.}\PY{n}{set\PYZus{}xlim}\PY{p}{(}\PY{l+m+mi}{0}\PY{p}{,}\PY{l+m+mi}{1}\PY{p}{)}
\PY{n}{ax}\PY{o}{.}\PY{n}{set\PYZus{}xlabel}\PY{p}{(}\PY{l+s+s2}{\PYZdq{}}\PY{l+s+s2}{new\PYZus{}data1}\PY{l+s+s2}{\PYZdq{}}\PY{p}{,}\PY{n}{fontsize}\PY{o}{=}\PY{l+m+mi}{8}\PY{p}{)}
\PY{n}{ax}\PY{o}{.}\PY{n}{set\PYZus{}ylabel}\PY{p}{(}\PY{l+s+s2}{\PYZdq{}}\PY{l+s+s2}{new\PYZus{}data2}\PY{l+s+s2}{\PYZdq{}}\PY{p}{,}\PY{n}{fontsize}\PY{o}{=}\PY{l+m+mi}{8}\PY{p}{)}
\PY{n}{ax}\PY{o}{.}\PY{n}{set\PYZus{}title}\PY{p}{(}\PY{l+s+s2}{\PYZdq{}}\PY{l+s+s2}{bins: 25}\PY{l+s+s2}{\PYZdq{}}\PY{p}{,}\PY{n}{fontsize}\PY{o}{=}\PY{l+m+mi}{12}\PY{p}{)}
\PY{n}{plt}\PY{o}{.}\PY{n}{xticks}\PY{p}{(}\PY{n}{fontsize}\PY{o}{=}\PY{l+m+mi}{7}\PY{p}{)}
\PY{n}{plt}\PY{o}{.}\PY{n}{yticks}\PY{p}{(}\PY{n}{fontsize}\PY{o}{=}\PY{l+m+mi}{7}\PY{p}{)}

\PY{n}{ax} \PY{o}{=} \PY{n}{fig}\PY{o}{.}\PY{n}{add\PYZus{}subplot}\PY{p}{(}\PY{l+m+mi}{2}\PY{p}{,}\PY{l+m+mi}{2}\PY{p}{,}\PY{l+m+mi}{3}\PY{p}{)}
\PY{n}{ax}\PY{o}{.}\PY{n}{hist}\PY{p}{(}\PY{n}{hist\PYZus{}data}\PY{p}{,}\PY{n}{bins}\PY{o}{=}\PY{l+m+mi}{100}\PY{p}{)}
\PY{n}{ax}\PY{o}{.}\PY{n}{set\PYZus{}xlim}\PY{p}{(}\PY{l+m+mi}{0}\PY{p}{,}\PY{l+m+mi}{1}\PY{p}{)}
\PY{n}{ax}\PY{o}{.}\PY{n}{set\PYZus{}xlabel}\PY{p}{(}\PY{l+s+s2}{\PYZdq{}}\PY{l+s+s2}{new\PYZus{}data1}\PY{l+s+s2}{\PYZdq{}}\PY{p}{,}\PY{n}{fontsize}\PY{o}{=}\PY{l+m+mi}{8}\PY{p}{)}
\PY{n}{ax}\PY{o}{.}\PY{n}{set\PYZus{}ylabel}\PY{p}{(}\PY{l+s+s2}{\PYZdq{}}\PY{l+s+s2}{new\PYZus{}data2}\PY{l+s+s2}{\PYZdq{}}\PY{p}{,}\PY{n}{fontsize}\PY{o}{=}\PY{l+m+mi}{8}\PY{p}{)}
\PY{n}{ax}\PY{o}{.}\PY{n}{set\PYZus{}title}\PY{p}{(}\PY{l+s+s2}{\PYZdq{}}\PY{l+s+s2}{bins: 100}\PY{l+s+s2}{\PYZdq{}}\PY{p}{,}\PY{n}{fontsize}\PY{o}{=}\PY{l+m+mi}{12}\PY{p}{)}
\PY{n}{plt}\PY{o}{.}\PY{n}{xticks}\PY{p}{(}\PY{n}{fontsize}\PY{o}{=}\PY{l+m+mi}{7}\PY{p}{)}
\PY{n}{plt}\PY{o}{.}\PY{n}{yticks}\PY{p}{(}\PY{n}{fontsize}\PY{o}{=}\PY{l+m+mi}{7}\PY{p}{)}

\PY{n}{ax} \PY{o}{=} \PY{n}{fig}\PY{o}{.}\PY{n}{add\PYZus{}subplot}\PY{p}{(}\PY{l+m+mi}{2}\PY{p}{,}\PY{l+m+mi}{2}\PY{p}{,}\PY{l+m+mi}{4}\PY{p}{)}
\PY{n}{ax}\PY{o}{.}\PY{n}{hist}\PY{p}{(}\PY{n}{hist\PYZus{}data}\PY{p}{,}\PY{n}{bins}\PY{o}{=}\PY{l+m+mi}{500}\PY{p}{)}
\PY{n}{ax}\PY{o}{.}\PY{n}{set\PYZus{}xlim}\PY{p}{(}\PY{l+m+mi}{0}\PY{p}{,}\PY{l+m+mi}{1}\PY{p}{)}
\PY{n}{ax}\PY{o}{.}\PY{n}{set\PYZus{}xlabel}\PY{p}{(}\PY{l+s+s2}{\PYZdq{}}\PY{l+s+s2}{new\PYZus{}data1}\PY{l+s+s2}{\PYZdq{}}\PY{p}{,}\PY{n}{fontsize}\PY{o}{=}\PY{l+m+mi}{8}\PY{p}{)}
\PY{n}{ax}\PY{o}{.}\PY{n}{set\PYZus{}ylabel}\PY{p}{(}\PY{l+s+s2}{\PYZdq{}}\PY{l+s+s2}{new\PYZus{}data2}\PY{l+s+s2}{\PYZdq{}}\PY{p}{,}\PY{n}{fontsize}\PY{o}{=}\PY{l+m+mi}{8}\PY{p}{)}
\PY{n}{ax}\PY{o}{.}\PY{n}{set\PYZus{}title}\PY{p}{(}\PY{l+s+s2}{\PYZdq{}}\PY{l+s+s2}{bins: 500}\PY{l+s+s2}{\PYZdq{}}\PY{p}{,}\PY{n}{fontsize}\PY{o}{=}\PY{l+m+mi}{12}\PY{p}{)}
\PY{n}{plt}\PY{o}{.}\PY{n}{xticks}\PY{p}{(}\PY{n}{fontsize}\PY{o}{=}\PY{l+m+mi}{7}\PY{p}{)}
\PY{n}{plt}\PY{o}{.}\PY{n}{yticks}\PY{p}{(}\PY{n}{fontsize}\PY{o}{=}\PY{l+m+mi}{7}\PY{p}{)}

\PY{n}{fig}\PY{o}{.}\PY{n}{tight\PYZus{}layout}\PY{p}{(}\PY{p}{)}
\end{Verbatim}
\end{tcolorbox}

    \begin{center}
    \adjustimage{max size={0.9\linewidth}{0.9\paperheight}}{output_29_0.png}
    \end{center}
    { \hspace*{\fill} \\}
    
    Histograms can also be used to plot probability functions on the same
graph. I'm also adding the \texttt{density=1} keyword argument to the
histogram plot command to convert the y-axis values from the number of
points in each bin to the percentage of the total. This will result in
more consistent ranges and let the probability function line match with
the histogram's distribution.

    \begin{tcolorbox}[breakable, size=fbox, boxrule=1pt, pad at break*=1mm,colback=cellbackground, colframe=cellborder]
\prompt{In}{incolor}{18}{\boxspacing}
\begin{Verbatim}[commandchars=\\\{\}]
\PY{n}{mu} \PY{o}{=} \PY{l+m+mf}{0.5}
\PY{n}{sigma} \PY{o}{=} \PY{l+m+mf}{0.1}
\PY{n}{hist\PYZus{}data} \PY{o}{=} \PY{n}{np}\PY{o}{.}\PY{n}{random}\PY{o}{.}\PY{n}{normal}\PY{p}{(}\PY{n}{mu}\PY{p}{,}\PY{n}{sigma}\PY{p}{,}\PY{n}{size}\PY{o}{=}\PY{l+m+mi}{10000}\PY{p}{)}

\PY{n}{fig} \PY{o}{=} \PY{n}{plt}\PY{o}{.}\PY{n}{figure}\PY{p}{(}\PY{n}{figsize}\PY{o}{=}\PY{p}{[}\PY{l+m+mi}{8}\PY{p}{,}\PY{l+m+mi}{6}\PY{p}{]}\PY{p}{,}\PY{n}{dpi}\PY{o}{=}\PY{l+m+mi}{150}\PY{p}{)} 

\PY{n}{ax} \PY{o}{=} \PY{n}{fig}\PY{o}{.}\PY{n}{add\PYZus{}subplot}\PY{p}{(}\PY{l+m+mi}{2}\PY{p}{,}\PY{l+m+mi}{2}\PY{p}{,}\PY{l+m+mi}{1}\PY{p}{)}
\PY{n}{nbins}\PY{o}{=}\PY{l+m+mi}{10}
\PY{n}{count}\PY{p}{,} \PY{n}{bins}\PY{p}{,} \PY{n}{ignored} \PY{o}{=} \PY{n}{ax}\PY{o}{.}\PY{n}{hist}\PY{p}{(}\PY{n}{hist\PYZus{}data}\PY{p}{,}\PY{n}{bins}\PY{o}{=}\PY{n}{nbins}\PY{p}{,} \PY{n}{density}\PY{o}{=}\PY{l+m+mi}{1}\PY{p}{)}
\PY{n}{plt}\PY{o}{.}\PY{n}{plot}\PY{p}{(}\PY{n}{bins}\PY{p}{,} \PY{l+m+mi}{1}\PY{o}{/}\PY{p}{(}\PY{n}{sigma}\PY{o}{*}\PY{n}{np}\PY{o}{.}\PY{n}{sqrt}\PY{p}{(}\PY{l+m+mi}{2}\PY{o}{*}\PY{n}{np}\PY{o}{.}\PY{n}{pi}\PY{p}{)}\PY{p}{)}\PY{o}{*}\PY{n}{np}\PY{o}{.}\PY{n}{exp}\PY{p}{(}\PY{o}{\PYZhy{}}\PY{p}{(}\PY{n}{bins}\PY{o}{\PYZhy{}}\PY{n}{mu}\PY{p}{)}\PY{o}{*}\PY{o}{*}\PY{l+m+mi}{2} \PY{o}{/} \PY{p}{(}\PY{l+m+mi}{2}\PY{o}{*}\PY{n}{sigma}\PY{o}{*}\PY{o}{*}\PY{l+m+mi}{2}\PY{p}{)}\PY{p}{)}\PY{p}{,}\PY{n}{linewidth}\PY{o}{=}\PY{l+m+mi}{2}\PY{p}{,} \PY{n}{color}\PY{o}{=}\PY{l+s+s1}{\PYZsq{}}\PY{l+s+s1}{r}\PY{l+s+s1}{\PYZsq{}}\PY{p}{)}
\PY{n}{ax}\PY{o}{.}\PY{n}{set\PYZus{}xlim}\PY{p}{(}\PY{l+m+mi}{0}\PY{p}{,}\PY{l+m+mi}{1}\PY{p}{)}
\PY{n}{ax}\PY{o}{.}\PY{n}{set\PYZus{}xlabel}\PY{p}{(}\PY{l+s+s2}{\PYZdq{}}\PY{l+s+s2}{hist\PYZus{}data}\PY{l+s+s2}{\PYZdq{}}\PY{p}{,}\PY{n}{fontsize}\PY{o}{=}\PY{l+m+mi}{8}\PY{p}{)}
\PY{n}{ax}\PY{o}{.}\PY{n}{set\PYZus{}title}\PY{p}{(}\PY{l+s+s2}{\PYZdq{}}\PY{l+s+s2}{bins: 10}\PY{l+s+s2}{\PYZdq{}}\PY{p}{,}\PY{n}{fontsize}\PY{o}{=}\PY{l+m+mi}{12}\PY{p}{)}
\PY{n}{plt}\PY{o}{.}\PY{n}{xticks}\PY{p}{(}\PY{n}{fontsize}\PY{o}{=}\PY{l+m+mi}{7}\PY{p}{)}
\PY{n}{plt}\PY{o}{.}\PY{n}{yticks}\PY{p}{(}\PY{n}{fontsize}\PY{o}{=}\PY{l+m+mi}{7}\PY{p}{)}

\PY{n}{ax} \PY{o}{=} \PY{n}{fig}\PY{o}{.}\PY{n}{add\PYZus{}subplot}\PY{p}{(}\PY{l+m+mi}{2}\PY{p}{,}\PY{l+m+mi}{2}\PY{p}{,}\PY{l+m+mi}{2}\PY{p}{)}
\PY{n}{nbins}\PY{o}{=}\PY{l+m+mi}{25}
\PY{n}{count}\PY{p}{,} \PY{n}{bins}\PY{p}{,} \PY{n}{ignored} \PY{o}{=} \PY{n}{ax}\PY{o}{.}\PY{n}{hist}\PY{p}{(}\PY{n}{hist\PYZus{}data}\PY{p}{,}\PY{n}{bins}\PY{o}{=}\PY{n}{nbins}\PY{p}{,} \PY{n}{density}\PY{o}{=}\PY{l+m+mi}{1}\PY{p}{)}
\PY{n}{plt}\PY{o}{.}\PY{n}{plot}\PY{p}{(}\PY{n}{bins}\PY{p}{,} \PY{l+m+mi}{1}\PY{o}{/}\PY{p}{(}\PY{n}{sigma}\PY{o}{*}\PY{n}{np}\PY{o}{.}\PY{n}{sqrt}\PY{p}{(}\PY{l+m+mi}{2}\PY{o}{*}\PY{n}{np}\PY{o}{.}\PY{n}{pi}\PY{p}{)}\PY{p}{)}\PY{o}{*}\PY{n}{np}\PY{o}{.}\PY{n}{exp}\PY{p}{(}\PY{o}{\PYZhy{}}\PY{p}{(}\PY{n}{bins}\PY{o}{\PYZhy{}}\PY{n}{mu}\PY{p}{)}\PY{o}{*}\PY{o}{*}\PY{l+m+mi}{2} \PY{o}{/} \PY{p}{(}\PY{l+m+mi}{2}\PY{o}{*}\PY{n}{sigma}\PY{o}{*}\PY{o}{*}\PY{l+m+mi}{2}\PY{p}{)}\PY{p}{)}\PY{p}{,}\PY{n}{linewidth}\PY{o}{=}\PY{l+m+mi}{2}\PY{p}{,} \PY{n}{color}\PY{o}{=}\PY{l+s+s1}{\PYZsq{}}\PY{l+s+s1}{r}\PY{l+s+s1}{\PYZsq{}}\PY{p}{)}
\PY{n}{ax}\PY{o}{.}\PY{n}{set\PYZus{}xlim}\PY{p}{(}\PY{l+m+mi}{0}\PY{p}{,}\PY{l+m+mi}{1}\PY{p}{)}
\PY{n}{ax}\PY{o}{.}\PY{n}{set\PYZus{}xlabel}\PY{p}{(}\PY{l+s+s2}{\PYZdq{}}\PY{l+s+s2}{hist\PYZus{}data}\PY{l+s+s2}{\PYZdq{}}\PY{p}{,}\PY{n}{fontsize}\PY{o}{=}\PY{l+m+mi}{8}\PY{p}{)}
\PY{n}{ax}\PY{o}{.}\PY{n}{set\PYZus{}title}\PY{p}{(}\PY{l+s+s2}{\PYZdq{}}\PY{l+s+s2}{bins: 25}\PY{l+s+s2}{\PYZdq{}}\PY{p}{,}\PY{n}{fontsize}\PY{o}{=}\PY{l+m+mi}{12}\PY{p}{)}
\PY{n}{plt}\PY{o}{.}\PY{n}{xticks}\PY{p}{(}\PY{n}{fontsize}\PY{o}{=}\PY{l+m+mi}{7}\PY{p}{)}
\PY{n}{plt}\PY{o}{.}\PY{n}{yticks}\PY{p}{(}\PY{n}{fontsize}\PY{o}{=}\PY{l+m+mi}{7}\PY{p}{)}

\PY{n}{ax} \PY{o}{=} \PY{n}{fig}\PY{o}{.}\PY{n}{add\PYZus{}subplot}\PY{p}{(}\PY{l+m+mi}{2}\PY{p}{,}\PY{l+m+mi}{2}\PY{p}{,}\PY{l+m+mi}{3}\PY{p}{)}
\PY{n}{nbins}\PY{o}{=}\PY{l+m+mi}{100}
\PY{n}{count}\PY{p}{,} \PY{n}{bins}\PY{p}{,} \PY{n}{ignored} \PY{o}{=} \PY{n}{ax}\PY{o}{.}\PY{n}{hist}\PY{p}{(}\PY{n}{hist\PYZus{}data}\PY{p}{,}\PY{n}{bins}\PY{o}{=}\PY{n}{nbins}\PY{p}{,} \PY{n}{density}\PY{o}{=}\PY{l+m+mi}{1}\PY{p}{)}
\PY{n}{plt}\PY{o}{.}\PY{n}{plot}\PY{p}{(}\PY{n}{bins}\PY{p}{,} \PY{l+m+mi}{1}\PY{o}{/}\PY{p}{(}\PY{n}{sigma}\PY{o}{*}\PY{n}{np}\PY{o}{.}\PY{n}{sqrt}\PY{p}{(}\PY{l+m+mi}{2}\PY{o}{*}\PY{n}{np}\PY{o}{.}\PY{n}{pi}\PY{p}{)}\PY{p}{)}\PY{o}{*}\PY{n}{np}\PY{o}{.}\PY{n}{exp}\PY{p}{(}\PY{o}{\PYZhy{}}\PY{p}{(}\PY{n}{bins}\PY{o}{\PYZhy{}}\PY{n}{mu}\PY{p}{)}\PY{o}{*}\PY{o}{*}\PY{l+m+mi}{2} \PY{o}{/} \PY{p}{(}\PY{l+m+mi}{2}\PY{o}{*}\PY{n}{sigma}\PY{o}{*}\PY{o}{*}\PY{l+m+mi}{2}\PY{p}{)}\PY{p}{)}\PY{p}{,}\PY{n}{linewidth}\PY{o}{=}\PY{l+m+mi}{2}\PY{p}{,} \PY{n}{color}\PY{o}{=}\PY{l+s+s1}{\PYZsq{}}\PY{l+s+s1}{r}\PY{l+s+s1}{\PYZsq{}}\PY{p}{)}
\PY{n}{ax}\PY{o}{.}\PY{n}{set\PYZus{}xlim}\PY{p}{(}\PY{l+m+mi}{0}\PY{p}{,}\PY{l+m+mi}{1}\PY{p}{)}
\PY{n}{ax}\PY{o}{.}\PY{n}{set\PYZus{}xlabel}\PY{p}{(}\PY{l+s+s2}{\PYZdq{}}\PY{l+s+s2}{hist\PYZus{}data}\PY{l+s+s2}{\PYZdq{}}\PY{p}{,}\PY{n}{fontsize}\PY{o}{=}\PY{l+m+mi}{8}\PY{p}{)}
\PY{n}{ax}\PY{o}{.}\PY{n}{set\PYZus{}title}\PY{p}{(}\PY{l+s+s2}{\PYZdq{}}\PY{l+s+s2}{bins: 100}\PY{l+s+s2}{\PYZdq{}}\PY{p}{,}\PY{n}{fontsize}\PY{o}{=}\PY{l+m+mi}{12}\PY{p}{)}
\PY{n}{plt}\PY{o}{.}\PY{n}{xticks}\PY{p}{(}\PY{n}{fontsize}\PY{o}{=}\PY{l+m+mi}{7}\PY{p}{)}
\PY{n}{plt}\PY{o}{.}\PY{n}{yticks}\PY{p}{(}\PY{n}{fontsize}\PY{o}{=}\PY{l+m+mi}{7}\PY{p}{)}

\PY{n}{ax} \PY{o}{=} \PY{n}{fig}\PY{o}{.}\PY{n}{add\PYZus{}subplot}\PY{p}{(}\PY{l+m+mi}{2}\PY{p}{,}\PY{l+m+mi}{2}\PY{p}{,}\PY{l+m+mi}{4}\PY{p}{)}
\PY{n}{nbins}\PY{o}{=}\PY{l+m+mi}{1000}
\PY{n}{count}\PY{p}{,} \PY{n}{bins}\PY{p}{,} \PY{n}{ignored} \PY{o}{=} \PY{n}{ax}\PY{o}{.}\PY{n}{hist}\PY{p}{(}\PY{n}{hist\PYZus{}data}\PY{p}{,}\PY{n}{bins}\PY{o}{=}\PY{n}{nbins}\PY{p}{,} \PY{n}{density}\PY{o}{=}\PY{l+m+mi}{1}\PY{p}{)}
\PY{n}{plt}\PY{o}{.}\PY{n}{plot}\PY{p}{(}\PY{n}{bins}\PY{p}{,} \PY{l+m+mi}{1}\PY{o}{/}\PY{p}{(}\PY{n}{sigma}\PY{o}{*}\PY{n}{np}\PY{o}{.}\PY{n}{sqrt}\PY{p}{(}\PY{l+m+mi}{2}\PY{o}{*}\PY{n}{np}\PY{o}{.}\PY{n}{pi}\PY{p}{)}\PY{p}{)}\PY{o}{*}\PY{n}{np}\PY{o}{.}\PY{n}{exp}\PY{p}{(}\PY{o}{\PYZhy{}}\PY{p}{(}\PY{n}{bins}\PY{o}{\PYZhy{}}\PY{n}{mu}\PY{p}{)}\PY{o}{*}\PY{o}{*}\PY{l+m+mi}{2} \PY{o}{/} \PY{p}{(}\PY{l+m+mi}{2}\PY{o}{*}\PY{n}{sigma}\PY{o}{*}\PY{o}{*}\PY{l+m+mi}{2}\PY{p}{)}\PY{p}{)}\PY{p}{,}\PY{n}{linewidth}\PY{o}{=}\PY{l+m+mi}{2}\PY{p}{,} \PY{n}{color}\PY{o}{=}\PY{l+s+s1}{\PYZsq{}}\PY{l+s+s1}{r}\PY{l+s+s1}{\PYZsq{}}\PY{p}{)}
\PY{n}{ax}\PY{o}{.}\PY{n}{set\PYZus{}xlim}\PY{p}{(}\PY{l+m+mi}{0}\PY{p}{,}\PY{l+m+mi}{1}\PY{p}{)}
\PY{n}{ax}\PY{o}{.}\PY{n}{set\PYZus{}xlabel}\PY{p}{(}\PY{l+s+s2}{\PYZdq{}}\PY{l+s+s2}{hist\PYZus{}data}\PY{l+s+s2}{\PYZdq{}}\PY{p}{,}\PY{n}{fontsize}\PY{o}{=}\PY{l+m+mi}{8}\PY{p}{)}
\PY{n}{ax}\PY{o}{.}\PY{n}{set\PYZus{}title}\PY{p}{(}\PY{l+s+s2}{\PYZdq{}}\PY{l+s+s2}{bins: 1000}\PY{l+s+s2}{\PYZdq{}}\PY{p}{,}\PY{n}{fontsize}\PY{o}{=}\PY{l+m+mi}{12}\PY{p}{)}
\PY{n}{plt}\PY{o}{.}\PY{n}{xticks}\PY{p}{(}\PY{n}{fontsize}\PY{o}{=}\PY{l+m+mi}{7}\PY{p}{)}
\PY{n}{plt}\PY{o}{.}\PY{n}{yticks}\PY{p}{(}\PY{n}{fontsize}\PY{o}{=}\PY{l+m+mi}{7}\PY{p}{)}

\PY{n}{fig}\PY{o}{.}\PY{n}{tight\PYZus{}layout}\PY{p}{(}\PY{p}{)}
\end{Verbatim}
\end{tcolorbox}

    \begin{center}
    \adjustimage{max size={0.9\linewidth}{0.9\paperheight}}{output_31_0.png}
    \end{center}
    { \hspace*{\fill} \\}
    
    Histograms are a great lead in to bar plots, even if this text is very
much not.

Consider some set of values for a bunch of individual things. Let's say
you took a poll of a thousand people to figure out what pizza topping is
the most popular.

You can simply provide a list of toppings and a list of values and a jab
at the people who put pineapple on their pizza and produce a nice
barplot!

    \begin{tcolorbox}[breakable, size=fbox, boxrule=1pt, pad at break*=1mm,colback=cellbackground, colframe=cellborder]
\prompt{In}{incolor}{31}{\boxspacing}
\begin{Verbatim}[commandchars=\\\{\}]
\PY{n}{toppings} \PY{o}{=} \PY{p}{[}\PY{l+s+s2}{\PYZdq{}}\PY{l+s+s2}{Pepperoni}\PY{l+s+s2}{\PYZdq{}}\PY{p}{,}\PY{l+s+s2}{\PYZdq{}}\PY{l+s+s2}{Mushrooms}\PY{l+s+s2}{\PYZdq{}}\PY{p}{,}\PY{l+s+s2}{\PYZdq{}}\PY{l+s+s2}{Olives}\PY{l+s+s2}{\PYZdq{}}\PY{p}{,}\PY{l+s+s2}{\PYZdq{}}\PY{l+s+s2}{Extra Cheese}\PY{l+s+s2}{\PYZdq{}}\PY{p}{,}\PY{l+s+s2}{\PYZdq{}}\PY{l+s+s2}{Pineapple}\PY{l+s+s2}{\PYZdq{}}\PY{p}{]}
\PY{n}{results}  \PY{o}{=} \PY{p}{[}\PY{l+m+mi}{120}\PY{p}{,}\PY{l+m+mi}{295}\PY{p}{,}\PY{l+m+mi}{180}\PY{p}{,}\PY{l+m+mi}{400}\PY{p}{,}\PY{l+m+mi}{1}\PY{p}{]}
\PY{n}{fig} \PY{o}{=} \PY{n}{plt}\PY{o}{.}\PY{n}{figure}\PY{p}{(}\PY{n}{figsize}\PY{o}{=}\PY{p}{[}\PY{l+m+mi}{4}\PY{p}{,}\PY{l+m+mi}{3}\PY{p}{]}\PY{p}{,}\PY{n}{dpi}\PY{o}{=}\PY{l+m+mi}{150}\PY{p}{)} 
\PY{n}{ax} \PY{o}{=} \PY{n}{fig}\PY{o}{.}\PY{n}{add\PYZus{}subplot}\PY{p}{(}\PY{l+m+mi}{1}\PY{p}{,}\PY{l+m+mi}{1}\PY{p}{,}\PY{l+m+mi}{1}\PY{p}{)}
\PY{n}{ax}\PY{o}{.}\PY{n}{bar}\PY{p}{(}\PY{n}{toppings}\PY{p}{,}\PY{n}{results}\PY{p}{,}\PY{n}{width}\PY{o}{=}\PY{l+m+mf}{1.0}\PY{p}{,}\PY{n}{color}\PY{o}{=}\PY{l+s+s2}{\PYZdq{}}\PY{l+s+s2}{red}\PY{l+s+s2}{\PYZdq{}}\PY{p}{,}\PY{n}{edgecolor}\PY{o}{=}\PY{l+s+s2}{\PYZdq{}}\PY{l+s+s2}{black}\PY{l+s+s2}{\PYZdq{}}\PY{p}{)}
\PY{n}{ax}\PY{o}{.}\PY{n}{set\PYZus{}xlabel}\PY{p}{(}\PY{l+s+s2}{\PYZdq{}}\PY{l+s+s2}{Toppings}\PY{l+s+s2}{\PYZdq{}}\PY{p}{,}\PY{n}{fontsize}\PY{o}{=}\PY{l+m+mi}{8}\PY{p}{)}
\PY{n}{ax}\PY{o}{.}\PY{n}{set\PYZus{}title}\PY{p}{(}\PY{l+s+s2}{\PYZdq{}}\PY{l+s+s2}{Popularity of Toppings}\PY{l+s+s2}{\PYZdq{}}\PY{p}{,}\PY{n}{fontsize}\PY{o}{=}\PY{l+m+mi}{12}\PY{p}{)}
\PY{n}{plt}\PY{o}{.}\PY{n}{xticks}\PY{p}{(}\PY{n}{fontsize}\PY{o}{=}\PY{l+m+mi}{6}\PY{p}{)}
\PY{n}{plt}\PY{o}{.}\PY{n}{yticks}\PY{p}{(}\PY{n}{fontsize}\PY{o}{=}\PY{l+m+mi}{6}\PY{p}{)}

\PY{n}{fig}\PY{o}{.}\PY{n}{tight\PYZus{}layout}\PY{p}{(}\PY{p}{)}
\end{Verbatim}
\end{tcolorbox}

    \begin{center}
    \adjustimage{max size={0.9\linewidth}{0.9\paperheight}}{output_33_0.png}
    \end{center}
    { \hspace*{\fill} \\}
    
    These results are valid and not up for debate. Moving on!

\hypertarget{dimensional-surfaces}{%
\subsubsection{3-Dimensional Surfaces}\label{dimensional-surfaces}}

MatPlotLib has the functionality to produce 3D surfaces. This means you
can present values that are dependent on two variables, or even data
dependent on 3 variables, using color to signify the value at a given
set of x,y,z coordinates in 3D space.

    \begin{tcolorbox}[breakable, size=fbox, boxrule=1pt, pad at break*=1mm,colback=cellbackground, colframe=cellborder]
\prompt{In}{incolor}{20}{\boxspacing}
\begin{Verbatim}[commandchars=\\\{\}]
\PY{n}{X} \PY{o}{=} \PY{n}{np}\PY{o}{.}\PY{n}{arange}\PY{p}{(}\PY{o}{\PYZhy{}}\PY{l+m+mi}{5}\PY{p}{,} \PY{l+m+mi}{5}\PY{p}{,} \PY{l+m+mf}{0.005}\PY{p}{)}
\PY{n}{Y} \PY{o}{=} \PY{n}{np}\PY{o}{.}\PY{n}{arange}\PY{p}{(}\PY{o}{\PYZhy{}}\PY{l+m+mi}{5}\PY{p}{,} \PY{l+m+mi}{5}\PY{p}{,} \PY{l+m+mf}{0.005}\PY{p}{)}
\PY{n}{X}\PY{p}{,} \PY{n}{Y} \PY{o}{=} \PY{n}{np}\PY{o}{.}\PY{n}{meshgrid}\PY{p}{(}\PY{n}{X}\PY{p}{,} \PY{n}{Y}\PY{p}{)}
\PY{n}{R} \PY{o}{=} \PY{n}{np}\PY{o}{.}\PY{n}{sqrt}\PY{p}{(}\PY{n}{X}\PY{o}{*}\PY{o}{*}\PY{l+m+mi}{2} \PY{o}{+} \PY{n}{Y}\PY{o}{*}\PY{o}{*}\PY{l+m+mi}{2}\PY{p}{)}
\PY{n}{Z} \PY{o}{=} \PY{n}{np}\PY{o}{.}\PY{n}{sin}\PY{p}{(}\PY{n}{R}\PY{p}{)}\PY{o}{*}\PY{n}{np}\PY{o}{.}\PY{n}{cos}\PY{p}{(}\PY{n}{R}\PY{p}{)}

\PY{n}{fig} \PY{o}{=} \PY{n}{plt}\PY{o}{.}\PY{n}{figure}\PY{p}{(}\PY{n}{figsize}\PY{o}{=}\PY{p}{[}\PY{l+m+mi}{10}\PY{p}{,}\PY{l+m+mi}{4}\PY{p}{]}\PY{p}{,}\PY{n}{dpi}\PY{o}{=}\PY{l+m+mi}{150}\PY{p}{)}
\PY{n}{ax} \PY{o}{=} \PY{n}{fig}\PY{o}{.}\PY{n}{add\PYZus{}subplot}\PY{p}{(}\PY{l+m+mi}{2}\PY{p}{,}\PY{l+m+mi}{4}\PY{p}{,}\PY{l+m+mi}{1}\PY{p}{,}\PY{n}{projection}\PY{o}{=}\PY{l+s+s2}{\PYZdq{}}\PY{l+s+s2}{3d}\PY{l+s+s2}{\PYZdq{}}\PY{p}{)}
\PY{n}{ax}\PY{o}{.}\PY{n}{plot\PYZus{}surface}\PY{p}{(}\PY{n}{X}\PY{p}{,} \PY{n}{Y}\PY{p}{,} \PY{n}{Z}\PY{p}{,} \PY{n}{cmap}\PY{o}{=}\PY{l+s+s2}{\PYZdq{}}\PY{l+s+s2}{inferno}\PY{l+s+s2}{\PYZdq{}}\PY{p}{,}\PY{n}{linewidth}\PY{o}{=}\PY{l+m+mi}{0}\PY{p}{,} \PY{n}{antialiased}\PY{o}{=}\PY{k+kc}{True}\PY{p}{,}\PY{n}{alpha}\PY{o}{=}\PY{l+m+mf}{1.0}\PY{p}{)}
\PY{n}{ax}\PY{o}{.}\PY{n}{set\PYZus{}zlim}\PY{p}{(}\PY{o}{\PYZhy{}}\PY{l+m+mf}{1.01}\PY{p}{,} \PY{l+m+mf}{1.01}\PY{p}{)}
\PY{n}{ax}\PY{o}{.}\PY{n}{set\PYZus{}title}\PY{p}{(}\PY{l+s+s2}{\PYZdq{}}\PY{l+s+s2}{Inferno,Antialiased}\PY{l+s+s2}{\PYZdq{}}\PY{p}{,}\PY{n}{fontsize}\PY{o}{=}\PY{l+m+mi}{10}\PY{p}{)}
\PY{n}{ax} \PY{o}{=} \PY{n}{fig}\PY{o}{.}\PY{n}{add\PYZus{}subplot}\PY{p}{(}\PY{l+m+mi}{2}\PY{p}{,}\PY{l+m+mi}{4}\PY{p}{,}\PY{l+m+mi}{2}\PY{p}{,}\PY{n}{projection}\PY{o}{=}\PY{l+s+s2}{\PYZdq{}}\PY{l+s+s2}{3d}\PY{l+s+s2}{\PYZdq{}}\PY{p}{)}
\PY{n}{ax}\PY{o}{.}\PY{n}{plot\PYZus{}surface}\PY{p}{(}\PY{n}{X}\PY{p}{,} \PY{n}{Y}\PY{p}{,} \PY{n}{Z}\PY{p}{,} \PY{n}{cmap}\PY{o}{=}\PY{l+s+s2}{\PYZdq{}}\PY{l+s+s2}{inferno}\PY{l+s+s2}{\PYZdq{}}\PY{p}{,}\PY{n}{linewidth}\PY{o}{=}\PY{l+m+mi}{0}\PY{p}{,} \PY{n}{antialiased}\PY{o}{=}\PY{k+kc}{False}\PY{p}{,}\PY{n}{alpha}\PY{o}{=}\PY{l+m+mf}{1.0}\PY{p}{)}
\PY{n}{ax}\PY{o}{.}\PY{n}{set\PYZus{}zlim}\PY{p}{(}\PY{o}{\PYZhy{}}\PY{l+m+mf}{1.01}\PY{p}{,} \PY{l+m+mf}{1.01}\PY{p}{)}
\PY{n}{ax}\PY{o}{.}\PY{n}{set\PYZus{}title}\PY{p}{(}\PY{l+s+s2}{\PYZdq{}}\PY{l+s+s2}{Inferno}\PY{l+s+s2}{\PYZdq{}}\PY{p}{,}\PY{n}{fontsize}\PY{o}{=}\PY{l+m+mi}{10}\PY{p}{)}
\PY{n}{ax} \PY{o}{=} \PY{n}{fig}\PY{o}{.}\PY{n}{add\PYZus{}subplot}\PY{p}{(}\PY{l+m+mi}{2}\PY{p}{,}\PY{l+m+mi}{4}\PY{p}{,}\PY{l+m+mi}{3}\PY{p}{,}\PY{n}{projection}\PY{o}{=}\PY{l+s+s2}{\PYZdq{}}\PY{l+s+s2}{3d}\PY{l+s+s2}{\PYZdq{}}\PY{p}{)}
\PY{n}{ax}\PY{o}{.}\PY{n}{plot\PYZus{}surface}\PY{p}{(}\PY{n}{X}\PY{p}{,} \PY{n}{Y}\PY{p}{,} \PY{n}{Z}\PY{p}{,} \PY{n}{cmap}\PY{o}{=}\PY{l+s+s2}{\PYZdq{}}\PY{l+s+s2}{inferno}\PY{l+s+s2}{\PYZdq{}}\PY{p}{,}\PY{n}{linewidth}\PY{o}{=}\PY{l+m+mi}{0}\PY{p}{,} \PY{n}{antialiased}\PY{o}{=}\PY{k+kc}{True}\PY{p}{,}\PY{n}{alpha}\PY{o}{=}\PY{l+m+mf}{0.5}\PY{p}{)}
\PY{n}{ax}\PY{o}{.}\PY{n}{set\PYZus{}zlim}\PY{p}{(}\PY{o}{\PYZhy{}}\PY{l+m+mf}{1.01}\PY{p}{,} \PY{l+m+mf}{1.01}\PY{p}{)}
\PY{n}{ax}\PY{o}{.}\PY{n}{set\PYZus{}title}\PY{p}{(}\PY{l+s+s2}{\PYZdq{}}\PY{l+s+s2}{Inferno,Antialiased,Alpha=0.5}\PY{l+s+s2}{\PYZdq{}}\PY{p}{,}\PY{n}{fontsize}\PY{o}{=}\PY{l+m+mi}{10}\PY{p}{)}
\PY{n}{ax} \PY{o}{=} \PY{n}{fig}\PY{o}{.}\PY{n}{add\PYZus{}subplot}\PY{p}{(}\PY{l+m+mi}{2}\PY{p}{,}\PY{l+m+mi}{4}\PY{p}{,}\PY{l+m+mi}{4}\PY{p}{,}\PY{n}{projection}\PY{o}{=}\PY{l+s+s2}{\PYZdq{}}\PY{l+s+s2}{3d}\PY{l+s+s2}{\PYZdq{}}\PY{p}{)}
\PY{n}{ax}\PY{o}{.}\PY{n}{plot\PYZus{}surface}\PY{p}{(}\PY{n}{X}\PY{p}{,} \PY{n}{Y}\PY{p}{,} \PY{n}{Z}\PY{p}{,} \PY{n}{cmap}\PY{o}{=}\PY{l+s+s2}{\PYZdq{}}\PY{l+s+s2}{inferno}\PY{l+s+s2}{\PYZdq{}}\PY{p}{,}\PY{n}{linewidth}\PY{o}{=}\PY{l+m+mi}{0}\PY{p}{,} \PY{n}{antialiased}\PY{o}{=}\PY{k+kc}{False}\PY{p}{,}\PY{n}{alpha}\PY{o}{=}\PY{l+m+mf}{0.5}\PY{p}{)}
\PY{n}{ax}\PY{o}{.}\PY{n}{set\PYZus{}zlim}\PY{p}{(}\PY{o}{\PYZhy{}}\PY{l+m+mf}{1.01}\PY{p}{,} \PY{l+m+mf}{1.01}\PY{p}{)}
\PY{n}{ax}\PY{o}{.}\PY{n}{set\PYZus{}title}\PY{p}{(}\PY{l+s+s2}{\PYZdq{}}\PY{l+s+s2}{Inferno,Alpha=0.5}\PY{l+s+s2}{\PYZdq{}}\PY{p}{,}\PY{n}{fontsize}\PY{o}{=}\PY{l+m+mi}{10}\PY{p}{)}


\PY{n}{ax} \PY{o}{=} \PY{n}{fig}\PY{o}{.}\PY{n}{add\PYZus{}subplot}\PY{p}{(}\PY{l+m+mi}{2}\PY{p}{,}\PY{l+m+mi}{4}\PY{p}{,}\PY{l+m+mi}{5}\PY{p}{,}\PY{n}{projection}\PY{o}{=}\PY{l+s+s2}{\PYZdq{}}\PY{l+s+s2}{3d}\PY{l+s+s2}{\PYZdq{}}\PY{p}{)}
\PY{n}{ax}\PY{o}{.}\PY{n}{plot\PYZus{}surface}\PY{p}{(}\PY{n}{X}\PY{p}{,} \PY{n}{Y}\PY{p}{,} \PY{n}{Z}\PY{p}{,} \PY{n}{cmap}\PY{o}{=}\PY{l+s+s2}{\PYZdq{}}\PY{l+s+s2}{jet}\PY{l+s+s2}{\PYZdq{}}\PY{p}{,}\PY{n}{linewidth}\PY{o}{=}\PY{l+m+mi}{0}\PY{p}{,} \PY{n}{antialiased}\PY{o}{=}\PY{k+kc}{True}\PY{p}{,}\PY{n}{alpha}\PY{o}{=}\PY{l+m+mf}{1.0}\PY{p}{)}
\PY{n}{ax}\PY{o}{.}\PY{n}{set\PYZus{}zlim}\PY{p}{(}\PY{o}{\PYZhy{}}\PY{l+m+mf}{1.01}\PY{p}{,} \PY{l+m+mf}{1.01}\PY{p}{)}
\PY{n}{ax}\PY{o}{.}\PY{n}{set\PYZus{}title}\PY{p}{(}\PY{l+s+s2}{\PYZdq{}}\PY{l+s+s2}{Jet,Antialiased}\PY{l+s+s2}{\PYZdq{}}\PY{p}{,}\PY{n}{fontsize}\PY{o}{=}\PY{l+m+mi}{10}\PY{p}{)}
\PY{n}{ax} \PY{o}{=} \PY{n}{fig}\PY{o}{.}\PY{n}{add\PYZus{}subplot}\PY{p}{(}\PY{l+m+mi}{2}\PY{p}{,}\PY{l+m+mi}{4}\PY{p}{,}\PY{l+m+mi}{6}\PY{p}{,}\PY{n}{projection}\PY{o}{=}\PY{l+s+s2}{\PYZdq{}}\PY{l+s+s2}{3d}\PY{l+s+s2}{\PYZdq{}}\PY{p}{)}
\PY{n}{ax}\PY{o}{.}\PY{n}{plot\PYZus{}surface}\PY{p}{(}\PY{n}{X}\PY{p}{,} \PY{n}{Y}\PY{p}{,} \PY{n}{Z}\PY{p}{,} \PY{n}{cmap}\PY{o}{=}\PY{l+s+s2}{\PYZdq{}}\PY{l+s+s2}{jet}\PY{l+s+s2}{\PYZdq{}}\PY{p}{,}\PY{n}{linewidth}\PY{o}{=}\PY{l+m+mi}{0}\PY{p}{,} \PY{n}{antialiased}\PY{o}{=}\PY{k+kc}{False}\PY{p}{,}\PY{n}{alpha}\PY{o}{=}\PY{l+m+mf}{1.0}\PY{p}{)}
\PY{n}{ax}\PY{o}{.}\PY{n}{set\PYZus{}zlim}\PY{p}{(}\PY{o}{\PYZhy{}}\PY{l+m+mf}{1.01}\PY{p}{,} \PY{l+m+mf}{1.01}\PY{p}{)}
\PY{n}{ax}\PY{o}{.}\PY{n}{set\PYZus{}title}\PY{p}{(}\PY{l+s+s2}{\PYZdq{}}\PY{l+s+s2}{Jet}\PY{l+s+s2}{\PYZdq{}}\PY{p}{,}\PY{n}{fontsize}\PY{o}{=}\PY{l+m+mi}{10}\PY{p}{)}
\PY{n}{ax} \PY{o}{=} \PY{n}{fig}\PY{o}{.}\PY{n}{add\PYZus{}subplot}\PY{p}{(}\PY{l+m+mi}{2}\PY{p}{,}\PY{l+m+mi}{4}\PY{p}{,}\PY{l+m+mi}{7}\PY{p}{,}\PY{n}{projection}\PY{o}{=}\PY{l+s+s2}{\PYZdq{}}\PY{l+s+s2}{3d}\PY{l+s+s2}{\PYZdq{}}\PY{p}{)}
\PY{n}{ax}\PY{o}{.}\PY{n}{plot\PYZus{}surface}\PY{p}{(}\PY{n}{X}\PY{p}{,} \PY{n}{Y}\PY{p}{,} \PY{n}{Z}\PY{p}{,} \PY{n}{cmap}\PY{o}{=}\PY{l+s+s2}{\PYZdq{}}\PY{l+s+s2}{jet}\PY{l+s+s2}{\PYZdq{}}\PY{p}{,}\PY{n}{linewidth}\PY{o}{=}\PY{l+m+mi}{0}\PY{p}{,} \PY{n}{antialiased}\PY{o}{=}\PY{k+kc}{True}\PY{p}{,}\PY{n}{alpha}\PY{o}{=}\PY{l+m+mf}{0.5}\PY{p}{)}
\PY{n}{ax}\PY{o}{.}\PY{n}{set\PYZus{}zlim}\PY{p}{(}\PY{o}{\PYZhy{}}\PY{l+m+mf}{1.01}\PY{p}{,} \PY{l+m+mf}{1.01}\PY{p}{)}
\PY{n}{ax}\PY{o}{.}\PY{n}{set\PYZus{}title}\PY{p}{(}\PY{l+s+s2}{\PYZdq{}}\PY{l+s+s2}{Jet,Antialiased,Alpha=0.5}\PY{l+s+s2}{\PYZdq{}}\PY{p}{,}\PY{n}{fontsize}\PY{o}{=}\PY{l+m+mi}{10}\PY{p}{)}
\PY{n}{ax} \PY{o}{=} \PY{n}{fig}\PY{o}{.}\PY{n}{add\PYZus{}subplot}\PY{p}{(}\PY{l+m+mi}{2}\PY{p}{,}\PY{l+m+mi}{4}\PY{p}{,}\PY{l+m+mi}{8}\PY{p}{,}\PY{n}{projection}\PY{o}{=}\PY{l+s+s2}{\PYZdq{}}\PY{l+s+s2}{3d}\PY{l+s+s2}{\PYZdq{}}\PY{p}{)}
\PY{n}{ax}\PY{o}{.}\PY{n}{plot\PYZus{}surface}\PY{p}{(}\PY{n}{X}\PY{p}{,} \PY{n}{Y}\PY{p}{,} \PY{n}{Z}\PY{p}{,} \PY{n}{cmap}\PY{o}{=}\PY{l+s+s2}{\PYZdq{}}\PY{l+s+s2}{jet}\PY{l+s+s2}{\PYZdq{}}\PY{p}{,}\PY{n}{linewidth}\PY{o}{=}\PY{l+m+mi}{0}\PY{p}{,} \PY{n}{antialiased}\PY{o}{=}\PY{k+kc}{False}\PY{p}{,}\PY{n}{alpha}\PY{o}{=}\PY{l+m+mf}{0.5}\PY{p}{)}
\PY{n}{ax}\PY{o}{.}\PY{n}{set\PYZus{}zlim}\PY{p}{(}\PY{o}{\PYZhy{}}\PY{l+m+mf}{1.01}\PY{p}{,} \PY{l+m+mf}{1.01}\PY{p}{)}
\PY{n}{ax}\PY{o}{.}\PY{n}{set\PYZus{}title}\PY{p}{(}\PY{l+s+s2}{\PYZdq{}}\PY{l+s+s2}{Jet,Alpha=0.5}\PY{l+s+s2}{\PYZdq{}}\PY{p}{,}\PY{n}{fontsize}\PY{o}{=}\PY{l+m+mi}{10}\PY{p}{)}


\PY{n}{fig}\PY{o}{.}\PY{n}{tight\PYZus{}layout}\PY{p}{(}\PY{p}{)}
\end{Verbatim}
\end{tcolorbox}

    \begin{center}
    \adjustimage{max size={0.9\linewidth}{0.9\paperheight}}{output_35_0.png}
    \end{center}
    { \hspace*{\fill} \\}
    
    That's all well and good as an example, but how does it look with data
we've already worked with?

Let's try the same process, but we'll use the same dataset from the
Density Plots section above.

    \begin{tcolorbox}[breakable, size=fbox, boxrule=1pt, pad at break*=1mm,colback=cellbackground, colframe=cellborder]
\prompt{In}{incolor}{21}{\boxspacing}
\begin{Verbatim}[commandchars=\\\{\}]
\PY{c+c1}{\PYZsh{}\PYZsh{}\PYZsh{}\PYZsh{} NEW IMPORT!}
\PY{k+kn}{from} \PY{n+nn}{scipy}\PY{n+nn}{.}\PY{n+nn}{stats} \PY{k+kn}{import} \PY{n}{kde}
\PY{n}{fig} \PY{o}{=} \PY{n}{plt}\PY{o}{.}\PY{n}{figure}\PY{p}{(}\PY{n}{figsize}\PY{o}{=}\PY{p}{[}\PY{l+m+mi}{8}\PY{p}{,}\PY{l+m+mi}{6}\PY{p}{]}\PY{p}{,}\PY{n}{dpi}\PY{o}{=}\PY{l+m+mi}{150}\PY{p}{)} 

\PY{c+c1}{\PYZsh{}\PYZsh{}\PYZsh{}\PYZsh{} NEW FUNCTIONALITY!}
\PY{n}{k} \PY{o}{=} \PY{n}{kde}\PY{o}{.}\PY{n}{gaussian\PYZus{}kde}\PY{p}{(}\PY{p}{[}\PY{n}{new\PYZus{}data1}\PY{p}{,}\PY{n}{new\PYZus{}data2}\PY{p}{]}\PY{p}{)}
\PY{n}{nbins}\PY{o}{=}\PY{l+m+mi}{300}
\PY{n}{X}\PY{p}{,} \PY{n}{Y} \PY{o}{=} \PY{n}{np}\PY{o}{.}\PY{n}{mgrid}\PY{p}{[}\PY{n}{new\PYZus{}data1}\PY{o}{.}\PY{n}{min}\PY{p}{(}\PY{p}{)}\PY{p}{:}\PY{n}{new\PYZus{}data1}\PY{o}{.}\PY{n}{max}\PY{p}{(}\PY{p}{)}\PY{p}{:}\PY{n}{nbins}\PY{o}{*}\PY{l+m+mi}{1}\PY{n}{j}\PY{p}{,} \PY{n}{new\PYZus{}data2}\PY{o}{.}\PY{n}{min}\PY{p}{(}\PY{p}{)}\PY{p}{:}\PY{n}{new\PYZus{}data2}\PY{o}{.}\PY{n}{max}\PY{p}{(}\PY{p}{)}\PY{p}{:}\PY{n}{nbins}\PY{o}{*}\PY{l+m+mi}{1}\PY{n}{j}\PY{p}{]}
\PY{n}{Z} \PY{o}{=} \PY{n}{k}\PY{p}{(}\PY{n}{np}\PY{o}{.}\PY{n}{vstack}\PY{p}{(}\PY{p}{[}\PY{n}{X}\PY{o}{.}\PY{n}{flatten}\PY{p}{(}\PY{p}{)}\PY{p}{,} \PY{n}{Y}\PY{o}{.}\PY{n}{flatten}\PY{p}{(}\PY{p}{)}\PY{p}{]}\PY{p}{)}\PY{p}{)}
\PY{n}{ax} \PY{o}{=} \PY{n}{fig}\PY{o}{.}\PY{n}{add\PYZus{}subplot}\PY{p}{(}\PY{l+m+mi}{2}\PY{p}{,}\PY{l+m+mi}{2}\PY{p}{,}\PY{l+m+mi}{1}\PY{p}{,}\PY{n}{projection}\PY{o}{=}\PY{l+s+s2}{\PYZdq{}}\PY{l+s+s2}{3d}\PY{l+s+s2}{\PYZdq{}}\PY{p}{)}
\PY{n}{ax}\PY{o}{.}\PY{n}{plot\PYZus{}surface}\PY{p}{(}\PY{n}{X}\PY{p}{,} \PY{n}{Y}\PY{p}{,} \PY{n}{Z}\PY{o}{.}\PY{n}{reshape}\PY{p}{(}\PY{n}{X}\PY{o}{.}\PY{n}{shape}\PY{p}{)}\PY{p}{,}\PY{n}{cmap}\PY{o}{=}\PY{l+s+s2}{\PYZdq{}}\PY{l+s+s2}{inferno}\PY{l+s+s2}{\PYZdq{}}\PY{p}{,}\PY{n}{linewidth}\PY{o}{=}\PY{l+m+mi}{0}\PY{p}{,} \PY{n}{antialiased}\PY{o}{=}\PY{k+kc}{True}\PY{p}{,}\PY{n}{alpha}\PY{o}{=}\PY{l+m+mf}{0.75}\PY{p}{)}
\PY{n}{ax}\PY{o}{.}\PY{n}{set\PYZus{}xlim}\PY{p}{(}\PY{l+m+mi}{0}\PY{p}{,}\PY{l+m+mi}{1}\PY{p}{)}
\PY{n}{ax}\PY{o}{.}\PY{n}{set\PYZus{}ylim}\PY{p}{(}\PY{l+m+mi}{0}\PY{p}{,}\PY{l+m+mi}{1}\PY{p}{)}
\PY{n}{ax}\PY{o}{.}\PY{n}{set\PYZus{}xlabel}\PY{p}{(}\PY{l+s+s2}{\PYZdq{}}\PY{l+s+s2}{new\PYZus{}data1}\PY{l+s+s2}{\PYZdq{}}\PY{p}{,}\PY{n}{fontsize}\PY{o}{=}\PY{l+m+mi}{8}\PY{p}{)}
\PY{n}{ax}\PY{o}{.}\PY{n}{set\PYZus{}ylabel}\PY{p}{(}\PY{l+s+s2}{\PYZdq{}}\PY{l+s+s2}{new\PYZus{}data2}\PY{l+s+s2}{\PYZdq{}}\PY{p}{,}\PY{n}{fontsize}\PY{o}{=}\PY{l+m+mi}{8}\PY{p}{)}
\PY{n}{ax}\PY{o}{.}\PY{n}{set\PYZus{}title}\PY{p}{(}\PY{l+s+s2}{\PYZdq{}}\PY{l+s+s2}{inferno}\PY{l+s+s2}{\PYZdq{}}\PY{p}{,}\PY{n}{fontsize}\PY{o}{=}\PY{l+m+mi}{12}\PY{p}{)}
\PY{n}{plt}\PY{o}{.}\PY{n}{xticks}\PY{p}{(}\PY{n}{fontsize}\PY{o}{=}\PY{l+m+mi}{7}\PY{p}{)}
\PY{n}{plt}\PY{o}{.}\PY{n}{yticks}\PY{p}{(}\PY{n}{fontsize}\PY{o}{=}\PY{l+m+mi}{7}\PY{p}{)}

\PY{n}{ax} \PY{o}{=} \PY{n}{fig}\PY{o}{.}\PY{n}{add\PYZus{}subplot}\PY{p}{(}\PY{l+m+mi}{2}\PY{p}{,}\PY{l+m+mi}{2}\PY{p}{,}\PY{l+m+mi}{2}\PY{p}{,}\PY{n}{projection}\PY{o}{=}\PY{l+s+s2}{\PYZdq{}}\PY{l+s+s2}{3d}\PY{l+s+s2}{\PYZdq{}}\PY{p}{)}
\PY{n}{ax}\PY{o}{.}\PY{n}{plot\PYZus{}surface}\PY{p}{(}\PY{n}{X}\PY{p}{,} \PY{n}{Y}\PY{p}{,} \PY{n}{Z}\PY{o}{.}\PY{n}{reshape}\PY{p}{(}\PY{n}{X}\PY{o}{.}\PY{n}{shape}\PY{p}{)}\PY{p}{,}\PY{n}{cmap}\PY{o}{=}\PY{l+s+s2}{\PYZdq{}}\PY{l+s+s2}{jet}\PY{l+s+s2}{\PYZdq{}}\PY{p}{,}\PY{n}{linewidth}\PY{o}{=}\PY{l+m+mi}{0}\PY{p}{,} \PY{n}{antialiased}\PY{o}{=}\PY{k+kc}{True}\PY{p}{,}\PY{n}{alpha}\PY{o}{=}\PY{l+m+mf}{0.75}\PY{p}{)}
\PY{n}{ax}\PY{o}{.}\PY{n}{set\PYZus{}xlim}\PY{p}{(}\PY{l+m+mi}{0}\PY{p}{,}\PY{l+m+mi}{1}\PY{p}{)}
\PY{n}{ax}\PY{o}{.}\PY{n}{set\PYZus{}ylim}\PY{p}{(}\PY{l+m+mi}{0}\PY{p}{,}\PY{l+m+mi}{1}\PY{p}{)}
\PY{n}{ax}\PY{o}{.}\PY{n}{set\PYZus{}xlabel}\PY{p}{(}\PY{l+s+s2}{\PYZdq{}}\PY{l+s+s2}{new\PYZus{}data1}\PY{l+s+s2}{\PYZdq{}}\PY{p}{,}\PY{n}{fontsize}\PY{o}{=}\PY{l+m+mi}{8}\PY{p}{)}
\PY{n}{ax}\PY{o}{.}\PY{n}{set\PYZus{}ylabel}\PY{p}{(}\PY{l+s+s2}{\PYZdq{}}\PY{l+s+s2}{new\PYZus{}data2}\PY{l+s+s2}{\PYZdq{}}\PY{p}{,}\PY{n}{fontsize}\PY{o}{=}\PY{l+m+mi}{8}\PY{p}{)}
\PY{n}{ax}\PY{o}{.}\PY{n}{set\PYZus{}title}\PY{p}{(}\PY{l+s+s2}{\PYZdq{}}\PY{l+s+s2}{jet}\PY{l+s+s2}{\PYZdq{}}\PY{p}{,}\PY{n}{fontsize}\PY{o}{=}\PY{l+m+mi}{12}\PY{p}{)}
\PY{n}{plt}\PY{o}{.}\PY{n}{xticks}\PY{p}{(}\PY{n}{fontsize}\PY{o}{=}\PY{l+m+mi}{7}\PY{p}{)}
\PY{n}{plt}\PY{o}{.}\PY{n}{yticks}\PY{p}{(}\PY{n}{fontsize}\PY{o}{=}\PY{l+m+mi}{7}\PY{p}{)}

\PY{n}{ax} \PY{o}{=} \PY{n}{fig}\PY{o}{.}\PY{n}{add\PYZus{}subplot}\PY{p}{(}\PY{l+m+mi}{2}\PY{p}{,}\PY{l+m+mi}{2}\PY{p}{,}\PY{l+m+mi}{3}\PY{p}{,}\PY{n}{projection}\PY{o}{=}\PY{l+s+s2}{\PYZdq{}}\PY{l+s+s2}{3d}\PY{l+s+s2}{\PYZdq{}}\PY{p}{)}
\PY{n}{ax}\PY{o}{.}\PY{n}{plot\PYZus{}surface}\PY{p}{(}\PY{n}{X}\PY{p}{,} \PY{n}{Y}\PY{p}{,} \PY{n}{Z}\PY{o}{.}\PY{n}{reshape}\PY{p}{(}\PY{n}{X}\PY{o}{.}\PY{n}{shape}\PY{p}{)}\PY{p}{,}\PY{n}{cmap}\PY{o}{=}\PY{l+s+s2}{\PYZdq{}}\PY{l+s+s2}{terrain}\PY{l+s+s2}{\PYZdq{}}\PY{p}{,}\PY{n}{linewidth}\PY{o}{=}\PY{l+m+mi}{0}\PY{p}{,} \PY{n}{antialiased}\PY{o}{=}\PY{k+kc}{True}\PY{p}{,}\PY{n}{alpha}\PY{o}{=}\PY{l+m+mf}{0.75}\PY{p}{)}
\PY{n}{ax}\PY{o}{.}\PY{n}{set\PYZus{}xlim}\PY{p}{(}\PY{l+m+mi}{0}\PY{p}{,}\PY{l+m+mi}{1}\PY{p}{)}
\PY{n}{ax}\PY{o}{.}\PY{n}{set\PYZus{}ylim}\PY{p}{(}\PY{l+m+mi}{0}\PY{p}{,}\PY{l+m+mi}{1}\PY{p}{)}
\PY{n}{ax}\PY{o}{.}\PY{n}{set\PYZus{}xlabel}\PY{p}{(}\PY{l+s+s2}{\PYZdq{}}\PY{l+s+s2}{new\PYZus{}data1}\PY{l+s+s2}{\PYZdq{}}\PY{p}{,}\PY{n}{fontsize}\PY{o}{=}\PY{l+m+mi}{8}\PY{p}{)}
\PY{n}{ax}\PY{o}{.}\PY{n}{set\PYZus{}ylabel}\PY{p}{(}\PY{l+s+s2}{\PYZdq{}}\PY{l+s+s2}{new\PYZus{}data2}\PY{l+s+s2}{\PYZdq{}}\PY{p}{,}\PY{n}{fontsize}\PY{o}{=}\PY{l+m+mi}{8}\PY{p}{)}
\PY{n}{ax}\PY{o}{.}\PY{n}{set\PYZus{}title}\PY{p}{(}\PY{l+s+s2}{\PYZdq{}}\PY{l+s+s2}{terrain}\PY{l+s+s2}{\PYZdq{}}\PY{p}{,}\PY{n}{fontsize}\PY{o}{=}\PY{l+m+mi}{12}\PY{p}{)}
\PY{n}{plt}\PY{o}{.}\PY{n}{xticks}\PY{p}{(}\PY{n}{fontsize}\PY{o}{=}\PY{l+m+mi}{7}\PY{p}{)}
\PY{n}{plt}\PY{o}{.}\PY{n}{yticks}\PY{p}{(}\PY{n}{fontsize}\PY{o}{=}\PY{l+m+mi}{7}\PY{p}{)}

\PY{n}{ax} \PY{o}{=} \PY{n}{fig}\PY{o}{.}\PY{n}{add\PYZus{}subplot}\PY{p}{(}\PY{l+m+mi}{2}\PY{p}{,}\PY{l+m+mi}{2}\PY{p}{,}\PY{l+m+mi}{4}\PY{p}{,}\PY{n}{projection}\PY{o}{=}\PY{l+s+s2}{\PYZdq{}}\PY{l+s+s2}{3d}\PY{l+s+s2}{\PYZdq{}}\PY{p}{)}
\PY{n}{ax}\PY{o}{.}\PY{n}{plot\PYZus{}surface}\PY{p}{(}\PY{n}{X}\PY{p}{,} \PY{n}{Y}\PY{p}{,} \PY{n}{Z}\PY{o}{.}\PY{n}{reshape}\PY{p}{(}\PY{n}{X}\PY{o}{.}\PY{n}{shape}\PY{p}{)}\PY{p}{,}\PY{n}{cmap}\PY{o}{=}\PY{l+s+s2}{\PYZdq{}}\PY{l+s+s2}{gist\PYZus{}ncar}\PY{l+s+s2}{\PYZdq{}}\PY{p}{,}\PY{n}{linewidth}\PY{o}{=}\PY{l+m+mi}{0}\PY{p}{,} \PY{n}{antialiased}\PY{o}{=}\PY{k+kc}{True}\PY{p}{,}\PY{n}{alpha}\PY{o}{=}\PY{l+m+mf}{0.75}\PY{p}{)}
\PY{n}{ax}\PY{o}{.}\PY{n}{set\PYZus{}xlim}\PY{p}{(}\PY{l+m+mi}{0}\PY{p}{,}\PY{l+m+mi}{1}\PY{p}{)}
\PY{n}{ax}\PY{o}{.}\PY{n}{set\PYZus{}ylim}\PY{p}{(}\PY{l+m+mi}{0}\PY{p}{,}\PY{l+m+mi}{1}\PY{p}{)}
\PY{n}{ax}\PY{o}{.}\PY{n}{set\PYZus{}xlabel}\PY{p}{(}\PY{l+s+s2}{\PYZdq{}}\PY{l+s+s2}{new\PYZus{}data1}\PY{l+s+s2}{\PYZdq{}}\PY{p}{,}\PY{n}{fontsize}\PY{o}{=}\PY{l+m+mi}{8}\PY{p}{)}
\PY{n}{ax}\PY{o}{.}\PY{n}{set\PYZus{}ylabel}\PY{p}{(}\PY{l+s+s2}{\PYZdq{}}\PY{l+s+s2}{new\PYZus{}data2}\PY{l+s+s2}{\PYZdq{}}\PY{p}{,}\PY{n}{fontsize}\PY{o}{=}\PY{l+m+mi}{8}\PY{p}{)}
\PY{n}{ax}\PY{o}{.}\PY{n}{set\PYZus{}title}\PY{p}{(}\PY{l+s+s2}{\PYZdq{}}\PY{l+s+s2}{gist\PYZus{}ncar}\PY{l+s+s2}{\PYZdq{}}\PY{p}{,}\PY{n}{fontsize}\PY{o}{=}\PY{l+m+mi}{12}\PY{p}{)}
\PY{n}{plt}\PY{o}{.}\PY{n}{xticks}\PY{p}{(}\PY{n}{fontsize}\PY{o}{=}\PY{l+m+mi}{7}\PY{p}{)}
\PY{n}{plt}\PY{o}{.}\PY{n}{yticks}\PY{p}{(}\PY{n}{fontsize}\PY{o}{=}\PY{l+m+mi}{7}\PY{p}{)}

\PY{n}{fig}\PY{o}{.}\PY{n}{tight\PYZus{}layout}\PY{p}{(}\PY{p}{)}
\end{Verbatim}
\end{tcolorbox}

    \begin{center}
    \adjustimage{max size={0.9\linewidth}{0.9\paperheight}}{output_37_0.png}
    \end{center}
    { \hspace*{\fill} \\}
    
    \hypertarget{polar-coordinates}{%
\subsubsection{Polar Coordinates}\label{polar-coordinates}}

Polar coordinates can come in very handy when you need to plot something
with angle values, such as the rotation of a substituent group about a
chemical bond. Because the angles are ultimately periodic, it can be
important to present the data in such a way.

    \begin{tcolorbox}[breakable, size=fbox, boxrule=1pt, pad at break*=1mm,colback=cellbackground, colframe=cellborder]
\prompt{In}{incolor}{30}{\boxspacing}
\begin{Verbatim}[commandchars=\\\{\}]
\PY{c+c1}{\PYZsh{}\PYZsh{}\PYZsh{} Random Dihedral Angles (each value is randomly adjusted from the previous value)}
\PY{n}{dihedral\PYZus{}data} \PY{o}{=} \PY{p}{[}\PY{l+m+mi}{0}\PY{p}{]}
\PY{k}{for} \PY{n}{i} \PY{o+ow}{in} \PY{n+nb}{range}\PY{p}{(}\PY{l+m+mi}{10000}\PY{p}{)}\PY{p}{:}
    \PY{n}{move} \PY{o}{=} \PY{n}{np}\PY{o}{.}\PY{n}{random}\PY{o}{.}\PY{n}{uniform}\PY{p}{(}\PY{o}{\PYZhy{}}\PY{l+m+mi}{1}\PY{p}{,}\PY{l+m+mi}{1}\PY{p}{)}
    \PY{n}{dihedral\PYZus{}data}\PY{o}{.}\PY{n}{append}\PY{p}{(}\PY{n}{dihedral\PYZus{}data}\PY{p}{[}\PY{o}{\PYZhy{}}\PY{l+m+mi}{1}\PY{p}{]}\PY{o}{+}\PY{n}{move}\PY{p}{)}
\end{Verbatim}
\end{tcolorbox}

    \begin{tcolorbox}[breakable, size=fbox, boxrule=1pt, pad at break*=1mm,colback=cellbackground, colframe=cellborder]
\prompt{In}{incolor}{29}{\boxspacing}
\begin{Verbatim}[commandchars=\\\{\}]
\PY{c+c1}{\PYZsh{} dihedral\PYZus{}data is just an array of dihedrals I got from cpptraj, reported in degrees, so I have to change to radians}
\PY{n}{plt}\PY{o}{.}\PY{n}{rcParams}\PY{p}{[}\PY{l+s+s2}{\PYZdq{}}\PY{l+s+s2}{font.size}\PY{l+s+s2}{\PYZdq{}}\PY{p}{]} \PY{o}{=} \PY{l+m+mi}{8}

\PY{n}{x} \PY{o}{=} \PY{n}{np}\PY{o}{.}\PY{n}{linspace}\PY{p}{(}\PY{l+m+mi}{0}\PY{p}{,}\PY{l+m+mi}{1}\PY{p}{,}\PY{n+nb}{len}\PY{p}{(}\PY{n}{dihedral\PYZus{}data}\PY{p}{)}\PY{p}{)}
\PY{n}{fig} \PY{o}{=} \PY{n}{plt}\PY{o}{.}\PY{n}{figure}\PY{p}{(}\PY{n}{figsize}\PY{o}{=}\PY{p}{[}\PY{l+m+mi}{4}\PY{p}{,}\PY{l+m+mi}{3}\PY{p}{]}\PY{p}{,}\PY{n}{dpi}\PY{o}{=}\PY{l+m+mi}{150}\PY{p}{)}
\PY{n}{ax} \PY{o}{=} \PY{n}{fig}\PY{o}{.}\PY{n}{add\PYZus{}subplot}\PY{p}{(}\PY{l+m+mi}{1}\PY{p}{,}\PY{l+m+mi}{1}\PY{p}{,}\PY{l+m+mi}{1}\PY{p}{,}\PY{n}{projection}\PY{o}{=}\PY{l+s+s1}{\PYZsq{}}\PY{l+s+s1}{polar}\PY{l+s+s1}{\PYZsq{}}\PY{p}{)}
\PY{n}{data\PYZus{}as\PYZus{}radians} \PY{o}{=} \PY{p}{[}\PY{n}{angle}\PY{o}{/}\PY{l+m+mi}{180}\PY{o}{*}\PY{n}{np}\PY{o}{.}\PY{n}{pi} \PY{k}{for} \PY{n}{angle} \PY{o+ow}{in} \PY{n}{dihedral\PYZus{}data}\PY{p}{]} \PY{c+c1}{\PYZsh{}\PYZsh{} converting angles from degrees to radians}
\PY{n}{ax}\PY{o}{.}\PY{n}{plot}\PY{p}{(}\PY{n}{data\PYZus{}as\PYZus{}radians}\PY{p}{,}\PY{n}{x}\PY{p}{,}\PY{n}{lw}\PY{o}{=}\PY{l+m+mf}{0.5}\PY{p}{,}\PY{n}{color}\PY{o}{=}\PY{l+s+s2}{\PYZdq{}}\PY{l+s+s2}{red}\PY{l+s+s2}{\PYZdq{}}\PY{p}{,}\PY{n}{alpha}\PY{o}{=}\PY{l+m+mf}{0.5}\PY{p}{)}
\PY{n}{ax}\PY{o}{.}\PY{n}{set\PYZus{}rmax}\PY{p}{(}\PY{l+m+mi}{1}\PY{p}{)}
\PY{n}{ax}\PY{o}{.}\PY{n}{set\PYZus{}rlabel\PYZus{}position}\PY{p}{(}\PY{o}{\PYZhy{}}\PY{l+m+mi}{45}\PY{p}{)}  \PY{c+c1}{\PYZsh{} Move radial labels away from plotted line}
\PY{n}{ax}\PY{o}{.}\PY{n}{grid}\PY{p}{(}\PY{k+kc}{True}\PY{p}{)}

\PY{n}{ax}\PY{o}{.}\PY{n}{set\PYZus{}title}\PY{p}{(}\PY{l+s+s2}{\PYZdq{}}\PY{l+s+s2}{Dihedral Angles As Polar Plots}\PY{l+s+s2}{\PYZdq{}}\PY{p}{,} \PY{n}{fontsize}\PY{o}{=}\PY{l+m+mi}{10}\PY{p}{,}\PY{n}{y}\PY{o}{=}\PY{l+m+mf}{1.10}\PY{p}{)}

\PY{n}{plt}\PY{o}{.}\PY{n}{show}\PY{p}{(}\PY{p}{)}

\PY{n}{fig}\PY{o}{.}\PY{n}{tight\PYZus{}layout}\PY{p}{(}\PY{p}{)}
\end{Verbatim}
\end{tcolorbox}

    \begin{center}
    \adjustimage{max size={0.9\linewidth}{0.9\paperheight}}{output_40_0.png}
    \end{center}
    { \hspace*{\fill} \\}
\section{Library - Numpy}
One of the most popular python libraries is \texttt{numpy}, which can be
found in hundreds of other libraries. NumPy takes advantage of the speed
of C++ and FORTRAN code while being usable inside python, making it
extremely powerful. As their own website states, ``A solution in NumPy
is often clear and elegant''.

We'll cover some of the basic or common functions below to get started,
but keep in mind that there is quite a bit more that numpy can do to
make your python-coding life easier. NumPy has no dependencies outside
of python itself. It also comes standard with most python installations
(like Anaconda). Loading numpy is therefore simple. The convention is to
use the code below.

    \begin{tcolorbox}[breakable, size=fbox, boxrule=1pt, pad at break*=1mm,colback=cellbackground, colframe=cellborder]
\prompt{In}{incolor}{3}{\boxspacing}
\begin{Verbatim}[commandchars=\\\{\}]
\PY{k+kn}{import} \PY{n+nn}{numpy} \PY{k}{as} \PY{n+nn}{np}
\end{Verbatim}
\end{tcolorbox}

    NumPy is most often associated with the ``numpy array'', which is
effectively a list of elements. However, the \texttt{numpy} library
takes the array to a higher level and has a lot more included
functionality. Let's look at how a basic array can be made with
\texttt{numpy}.

    \begin{tcolorbox}[breakable, size=fbox, boxrule=1pt, pad at break*=1mm,colback=cellbackground, colframe=cellborder]
\prompt{In}{incolor}{8}{\boxspacing}
\begin{Verbatim}[commandchars=\\\{\}]
\PY{n}{my\PYZus{}array} \PY{o}{=} \PY{n}{np}\PY{o}{.}\PY{n}{arange}\PY{p}{(}\PY{l+m+mi}{0}\PY{p}{,}\PY{l+m+mi}{100}\PY{p}{)}
\PY{n+nb}{print}\PY{p}{(}\PY{n}{my\PYZus{}array}\PY{p}{)}
\end{Verbatim}
\end{tcolorbox}

    \begin{Verbatim}[commandchars=\\\{\}]
[ 0  1  2  3  4  5  6  7  8  9 10 11 12 13 14 15 16 17 18 19 20 21 22 23
 24 25 26 27 28 29 30 31 32 33 34 35 36 37 38 39 40 41 42 43 44 45 46 47
 48 49 50 51 52 53 54 55 56 57 58 59 60 61 62 63 64 65 66 67 68 69 70 71
 72 73 74 75 76 77 78 79 80 81 82 83 84 85 86 87 88 89 90 91 92 93 94 95
 96 97 98 99]
    \end{Verbatim}

    We can see the use of \texttt{np.arange} above. It has two required
arguments for the start and ending values. Note that in the code above,
the end value is 100, but the actual list stops at 99. This means if you
actually want 0 to 100, you should set an end value of 101.

We can also add a third value for stride, or how much to increment the
value.

    \begin{tcolorbox}[breakable, size=fbox, boxrule=1pt, pad at break*=1mm,colback=cellbackground, colframe=cellborder]
\prompt{In}{incolor}{9}{\boxspacing}
\begin{Verbatim}[commandchars=\\\{\}]
\PY{n}{my\PYZus{}array} \PY{o}{=} \PY{n}{np}\PY{o}{.}\PY{n}{arange}\PY{p}{(}\PY{l+m+mi}{0}\PY{p}{,}\PY{l+m+mi}{100}\PY{p}{,}\PY{l+m+mi}{5}\PY{p}{)}
\PY{n+nb}{print}\PY{p}{(}\PY{n}{my\PYZus{}array}\PY{p}{)}
\end{Verbatim}
\end{tcolorbox}

    \begin{Verbatim}[commandchars=\\\{\}]
[ 0  5 10 15 20 25 30 35 40 45 50 55 60 65 70 75 80 85 90 95]
    \end{Verbatim}

    In the example above, we get values that are 5 apart. If we adjust the
end point to 101, we'll get the last value (100) as well.

If we want to get a range of values that are evenly spaced between two
endpoints \emph{and includes both of those endpoints}, we can use the
\texttt{linspace} function. The \texttt{linspace} function requires
three values - start, end, and number of values to return. It is
important to keep in mind how the array is created in this function. If
a set of N values is requested between X and Y, it will calculate the
increment as follows:

\[ I = \frac{Y-X}{N-1} \]

This will produce values of \([X, X+I, X+2I, ..., X+(N-1)I, Y]\), which
are equally spaced.

The reason this is being explained is shown below. If we wanted 20
values between 0 and 10, we might intuitively think it would give us
increments of 0.5, so that the array would look like
\([0,0.5,1.0,1.5,...]\). However, the \texttt{linspace} array includes
both endpoints in the list.

    \begin{tcolorbox}[breakable, size=fbox, boxrule=1pt, pad at break*=1mm,colback=cellbackground, colframe=cellborder]
\prompt{In}{incolor}{13}{\boxspacing}
\begin{Verbatim}[commandchars=\\\{\}]
\PY{n}{my\PYZus{}array} \PY{o}{=} \PY{n}{np}\PY{o}{.}\PY{n}{linspace}\PY{p}{(}\PY{l+m+mi}{0}\PY{p}{,}\PY{l+m+mi}{10}\PY{p}{,}\PY{l+m+mi}{20}\PY{p}{)}
\PY{n+nb}{print}\PY{p}{(}\PY{n}{my\PYZus{}array}\PY{p}{)}
\end{Verbatim}
\end{tcolorbox}

    \begin{Verbatim}[commandchars=\\\{\}]
[ 0.          0.52631579  1.05263158  1.57894737  2.10526316  2.63157895
  3.15789474  3.68421053  4.21052632  4.73684211  5.26315789  5.78947368
  6.31578947  6.84210526  7.36842105  7.89473684  8.42105263  8.94736842
  9.47368421 10.        ]
    \end{Verbatim}

    \begin{tcolorbox}[breakable, size=fbox, boxrule=1pt, pad at break*=1mm,colback=cellbackground, colframe=cellborder]
\prompt{In}{incolor}{14}{\boxspacing}
\begin{Verbatim}[commandchars=\\\{\}]
\PY{n}{my\PYZus{}array} \PY{o}{=} \PY{n}{np}\PY{o}{.}\PY{n}{linspace}\PY{p}{(}\PY{l+m+mi}{0}\PY{p}{,}\PY{l+m+mi}{10}\PY{p}{,}\PY{l+m+mi}{21}\PY{p}{)}
\PY{n+nb}{print}\PY{p}{(}\PY{n}{my\PYZus{}array}\PY{p}{)}
\end{Verbatim}
\end{tcolorbox}

    \begin{Verbatim}[commandchars=\\\{\}]
[ 0.   0.5  1.   1.5  2.   2.5  3.   3.5  4.   4.5  5.   5.5  6.   6.5
  7.   7.5  8.   8.5  9.   9.5 10. ]
    \end{Verbatim}

    Another function in \texttt{numpy} is the \texttt{roll} function. This
treats the array like a closed loop and lets you move all elements some
number of spaces down, looping end elements back to the start.

    \begin{tcolorbox}[breakable, size=fbox, boxrule=1pt, pad at break*=1mm,colback=cellbackground, colframe=cellborder]
\prompt{In}{incolor}{20}{\boxspacing}
\begin{Verbatim}[commandchars=\\\{\}]
\PY{n}{np}\PY{o}{.}\PY{n}{roll}\PY{p}{(}\PY{n}{my\PYZus{}array}\PY{p}{,}\PY{l+m+mi}{1}\PY{p}{)} \PY{c+c1}{\PYZsh{} shift everything 1 space over, brings the last element to the front.}
\end{Verbatim}
\end{tcolorbox}

            \begin{tcolorbox}[breakable, size=fbox, boxrule=.5pt, pad at break*=1mm, opacityfill=0]
\prompt{Out}{outcolor}{20}{\boxspacing}
\begin{Verbatim}[commandchars=\\\{\}]
array([10. ,  0. ,  0.5,  1. ,  1.5,  2. ,  2.5,  3. ,  3.5,  4. ,  4.5,
        5. ,  5.5,  6. ,  6.5,  7. ,  7.5,  8. ,  8.5,  9. ,  9.5])
\end{Verbatim}
\end{tcolorbox}
        
    \begin{tcolorbox}[breakable, size=fbox, boxrule=1pt, pad at break*=1mm,colback=cellbackground, colframe=cellborder]
\prompt{In}{incolor}{19}{\boxspacing}
\begin{Verbatim}[commandchars=\\\{\}]
\PY{n}{np}\PY{o}{.}\PY{n}{roll}\PY{p}{(}\PY{n}{my\PYZus{}array}\PY{p}{,}\PY{l+m+mi}{3}\PY{p}{)} \PY{c+c1}{\PYZsh{} Shift everything 3 spaces over.}
\end{Verbatim}
\end{tcolorbox}

            \begin{tcolorbox}[breakable, size=fbox, boxrule=.5pt, pad at break*=1mm, opacityfill=0]
\prompt{Out}{outcolor}{19}{\boxspacing}
\begin{Verbatim}[commandchars=\\\{\}]
array([ 9. ,  9.5, 10. ,  0. ,  0.5,  1. ,  1.5,  2. ,  2.5,  3. ,  3.5,
        4. ,  4.5,  5. ,  5.5,  6. ,  6.5,  7. ,  7.5,  8. ,  8.5])
\end{Verbatim}
\end{tcolorbox}
        
    \hypertarget{numpy.linalg}{%
\subsubsection{numpy.linalg}\label{numpy.linalg}}

NumPy has a number of submodules that are useful for certain tasks. The
\texttt{linalg} submodule is short for Linear Algebra, which is focused
on vector-, matrix-, and tensor-math functions. Most of the things you
learn to calculate by hand in a multivariable calculus course can be
done with simple functions in \texttt{numpy.linalg}

For example, if you have a vector \([1,2,3]\), you can get the magnitude
of the vector by using \texttt{linalg.norm}.

    \begin{tcolorbox}[breakable, size=fbox, boxrule=1pt, pad at break*=1mm,colback=cellbackground, colframe=cellborder]
\prompt{In}{incolor}{21}{\boxspacing}
\begin{Verbatim}[commandchars=\\\{\}]
\PY{k+kn}{import} \PY{n+nn}{numpy}\PY{n+nn}{.}\PY{n+nn}{linalg} \PY{k}{as} \PY{n+nn}{la}

\PY{n}{my\PYZus{}vector}\PY{o}{=}\PY{p}{[}\PY{l+m+mi}{1}\PY{p}{,}\PY{l+m+mi}{2}\PY{p}{,}\PY{l+m+mi}{3}\PY{p}{]}
\PY{n+nb}{print}\PY{p}{(}\PY{n}{la}\PY{o}{.}\PY{n}{norm}\PY{p}{(}\PY{n}{my\PYZus{}vector}\PY{p}{)}\PY{p}{)}
\end{Verbatim}
\end{tcolorbox}

    \begin{Verbatim}[commandchars=\\\{\}]
3.7416573867739413
    \end{Verbatim}

    You can also do more complicated operations like the dot-product or
cross-products

\hypertarget{np.dot}{%
\paragraph{np.dot()}\label{np.dot}}

    \begin{tcolorbox}[breakable, size=fbox, boxrule=1pt, pad at break*=1mm,colback=cellbackground, colframe=cellborder]
\prompt{In}{incolor}{25}{\boxspacing}
\begin{Verbatim}[commandchars=\\\{\}]
\PY{n}{vec1} \PY{o}{=} \PY{p}{[}\PY{l+m+mi}{1}\PY{p}{,}\PY{l+m+mi}{2}\PY{p}{,}\PY{l+m+mi}{3}\PY{p}{]}
\PY{n}{vec2} \PY{o}{=} \PY{p}{[}\PY{l+m+mi}{4}\PY{p}{,}\PY{l+m+mi}{5}\PY{p}{,}\PY{l+m+mi}{6}\PY{p}{]}
\PY{n}{dot\PYZus{}product} \PY{o}{=} \PY{n}{np}\PY{o}{.}\PY{n}{dot}\PY{p}{(}\PY{n}{vec1}\PY{p}{,}\PY{n}{vec2}\PY{p}{)}
\PY{n+nb}{print}\PY{p}{(}\PY{n}{dot\PYZus{}product}\PY{p}{)}
\end{Verbatim}
\end{tcolorbox}

    \begin{Verbatim}[commandchars=\\\{\}]
32
    \end{Verbatim}

    \hypertarget{np.cross}{%
\paragraph{np.cross()}\label{np.cross}}

    \begin{tcolorbox}[breakable, size=fbox, boxrule=1pt, pad at break*=1mm,colback=cellbackground, colframe=cellborder]
\prompt{In}{incolor}{36}{\boxspacing}
\begin{Verbatim}[commandchars=\\\{\}]
\PY{n}{mat1} \PY{o}{=} \PY{n}{np}\PY{o}{.}\PY{n}{array}\PY{p}{(}\PY{p}{[}\PY{p}{[}\PY{l+m+mi}{1}\PY{p}{,}\PY{l+m+mi}{2}\PY{p}{,}\PY{l+m+mi}{3}\PY{p}{]}\PY{p}{,}\PY{p}{[}\PY{l+m+mi}{4}\PY{p}{,}\PY{l+m+mi}{5}\PY{p}{,}\PY{l+m+mi}{6}\PY{p}{]}\PY{p}{,}\PY{p}{[}\PY{l+m+mi}{7}\PY{p}{,}\PY{l+m+mi}{8}\PY{p}{,}\PY{l+m+mi}{9}\PY{p}{]}\PY{p}{]}\PY{p}{)}
\PY{n}{mat2} \PY{o}{=} \PY{n}{np}\PY{o}{.}\PY{n}{array}\PY{p}{(}\PY{p}{[}\PY{p}{[}\PY{l+m+mi}{1}\PY{p}{,}\PY{l+m+mi}{2}\PY{p}{]}\PY{p}{,}\PY{p}{[}\PY{l+m+mi}{3}\PY{p}{,}\PY{l+m+mi}{4}\PY{p}{]}\PY{p}{,}\PY{p}{[}\PY{l+m+mi}{5}\PY{p}{,}\PY{l+m+mi}{6}\PY{p}{]}\PY{p}{]}\PY{p}{)}
\PY{n}{cross\PYZus{}product} \PY{o}{=} \PY{n}{np}\PY{o}{.}\PY{n}{cross}\PY{p}{(}\PY{n}{mat1}\PY{p}{,}\PY{n}{mat2}\PY{p}{)}

\PY{n+nb}{print}\PY{p}{(}\PY{n}{mat1}\PY{p}{)}
\PY{n+nb}{print}\PY{p}{(}\PY{n}{mat2}\PY{p}{)}
\PY{n+nb}{print}\PY{p}{(}\PY{n}{cross\PYZus{}product}\PY{p}{)}
\end{Verbatim}
\end{tcolorbox}

    \begin{Verbatim}[commandchars=\\\{\}]
[[1 2 3]
 [4 5 6]
 [7 8 9]]
[[1 2]
 [3 4]
 [5 6]]
[[ -6   3   0]
 [-24  18   1]
 [-54  45   2]]
    \end{Verbatim}

    \hypertarget{random-number-generators}{%
\subsubsection{Random Number
Generators}\label{random-number-generators}}

Numpy is also home to a good selection of random number generators that
can provide different random distributions depending on what kind of
random number you need. - uniform - normal (Gaussian) - exponential -
binomial - etc.

\hypertarget{polynomial-fits}{%
\subsubsection{Polynomial Fits}\label{polynomial-fits}}

\texttt{numpy.polynomial} can provide you with data fitting tools to
obtain \(N\)-degree polynomials for your data.
\section{Library - Pandas}
Pandas is a wonderful data organization and processing library. It works
off of a single major underlying structure, the \texttt{DataFrame}. The
\texttt{DataFrame} is an object that can be described as a ``dictionary
of dictionaries''. Similar to how previous examples have shown that you
can have nested dictionary datatypes, Pandas takes it a step further and
provides tools to organize, manipulate, combine, compare, and store
these multi-layer dictionaries. Pandas also includes built-in Excel
Spreadsheet creation functionality, meaning that you can transform your
data into more complex structures in Excel, making it easier to share
with collaborators.

Let's look at an example of a simple dataframe with just a few rows and
columns.

    \begin{tcolorbox}[breakable, size=fbox, boxrule=1pt, pad at break*=1mm,colback=cellbackground, colframe=cellborder]
\prompt{In}{incolor}{2}{\boxspacing}
\begin{Verbatim}[commandchars=\\\{\}]
\PY{k+kn}{import} \PY{n+nn}{pandas} \PY{k}{as} \PY{n+nn}{pd}

\PY{n}{df} \PY{o}{=} \PY{n}{pd}\PY{o}{.}\PY{n}{DataFrame}\PY{p}{(}\PY{p}{[}\PY{p}{\PYZob{}}\PY{l+s+s2}{\PYZdq{}}\PY{l+s+s2}{First Name}\PY{l+s+s2}{\PYZdq{}}\PY{p}{:}\PY{l+s+s2}{\PYZdq{}}\PY{l+s+s2}{Zee}\PY{l+s+s2}{\PYZdq{}}\PY{p}{,}\PY{l+s+s2}{\PYZdq{}}\PY{l+s+s2}{Age}\PY{l+s+s2}{\PYZdq{}}\PY{p}{:}\PY{l+m+mi}{40}\PY{p}{,}\PY{l+s+s2}{\PYZdq{}}\PY{l+s+s2}{Team}\PY{l+s+s2}{\PYZdq{}}\PY{p}{:}\PY{l+s+s2}{\PYZdq{}}\PY{l+s+s2}{Blue}\PY{l+s+s2}{\PYZdq{}}\PY{p}{\PYZcb{}}\PY{p}{,}
                   \PY{p}{\PYZob{}}\PY{l+s+s2}{\PYZdq{}}\PY{l+s+s2}{First Name}\PY{l+s+s2}{\PYZdq{}}\PY{p}{:}\PY{l+s+s2}{\PYZdq{}}\PY{l+s+s2}{Charlotte}\PY{l+s+s2}{\PYZdq{}}\PY{p}{,}\PY{l+s+s2}{\PYZdq{}}\PY{l+s+s2}{Age}\PY{l+s+s2}{\PYZdq{}}\PY{p}{:}\PY{l+m+mi}{45}\PY{p}{,}\PY{l+s+s2}{\PYZdq{}}\PY{l+s+s2}{Team}\PY{l+s+s2}{\PYZdq{}}\PY{p}{:}\PY{l+s+s2}{\PYZdq{}}\PY{l+s+s2}{Red}\PY{l+s+s2}{\PYZdq{}}\PY{p}{\PYZcb{}}\PY{p}{,}
                   \PY{p}{\PYZob{}}\PY{l+s+s2}{\PYZdq{}}\PY{l+s+s2}{First Name}\PY{l+s+s2}{\PYZdq{}}\PY{p}{:}\PY{l+s+s2}{\PYZdq{}}\PY{l+s+s2}{Wilbur}\PY{l+s+s2}{\PYZdq{}}\PY{p}{,}\PY{l+s+s2}{\PYZdq{}}\PY{l+s+s2}{Age}\PY{l+s+s2}{\PYZdq{}}\PY{p}{:}\PY{l+m+mi}{50}\PY{p}{,}\PY{l+s+s2}{\PYZdq{}}\PY{l+s+s2}{Team}\PY{l+s+s2}{\PYZdq{}}\PY{p}{:}\PY{l+s+s2}{\PYZdq{}}\PY{l+s+s2}{Green}\PY{l+s+s2}{\PYZdq{}}\PY{p}{\PYZcb{}}\PY{p}{]}\PY{p}{)}
\end{Verbatim}
\end{tcolorbox}

    \begin{tcolorbox}[breakable, size=fbox, boxrule=1pt, pad at break*=1mm,colback=cellbackground, colframe=cellborder]
\prompt{In}{incolor}{3}{\boxspacing}
\begin{Verbatim}[commandchars=\\\{\}]
\PY{n}{display}\PY{p}{(}\PY{n}{df}\PY{p}{)}
\end{Verbatim}
\end{tcolorbox}

    
    \begin{Verbatim}[commandchars=\\\{\}]
  First Name  Age   Team
0        Zee   40   Blue
1  Charlotte   45    Red
2     Wilbur   50  Green
    \end{Verbatim}

    
    \begin{tcolorbox}[breakable, size=fbox, boxrule=1pt, pad at break*=1mm,colback=cellbackground, colframe=cellborder]
\prompt{In}{incolor}{4}{\boxspacing}
\begin{Verbatim}[commandchars=\\\{\}]
\PY{n+nb}{print}\PY{p}{(}\PY{n}{df}\PY{p}{)}
\end{Verbatim}
\end{tcolorbox}

    \begin{Verbatim}[commandchars=\\\{\}]
  First Name  Age   Team
0        Zee   40   Blue
1  Charlotte   45    Red
2     Wilbur   50  Green
    \end{Verbatim}

    Notice the difference between \texttt{display} and \texttt{print} in the
example above. It should be pointed out that \texttt{display} is
functional in notebook environments only, not in terminals. Trying to
use \texttt{display} in a terminal will result in an error.

You can add to existing dataframes by appending new rows. Each row will
be made as a dictionary beforehand.

    \begin{tcolorbox}[breakable, size=fbox, boxrule=1pt, pad at break*=1mm,colback=cellbackground, colframe=cellborder]
\prompt{In}{incolor}{5}{\boxspacing}
\begin{Verbatim}[commandchars=\\\{\}]
\PY{n}{newrow} \PY{o}{=} \PY{p}{\PYZob{}}\PY{l+s+s2}{\PYZdq{}}\PY{l+s+s2}{First Name}\PY{l+s+s2}{\PYZdq{}}\PY{p}{:}\PY{l+s+s2}{\PYZdq{}}\PY{l+s+s2}{Iroh}\PY{l+s+s2}{\PYZdq{}}\PY{p}{,}\PY{l+s+s2}{\PYZdq{}}\PY{l+s+s2}{Age}\PY{l+s+s2}{\PYZdq{}}\PY{p}{:}\PY{l+m+mi}{99}\PY{p}{,}\PY{l+s+s2}{\PYZdq{}}\PY{l+s+s2}{Team}\PY{l+s+s2}{\PYZdq{}}\PY{p}{:}\PY{l+s+s2}{\PYZdq{}}\PY{l+s+s2}{Red}\PY{l+s+s2}{\PYZdq{}}\PY{p}{\PYZcb{}}
\PY{n}{df} \PY{o}{=} \PY{n}{df}\PY{o}{.}\PY{n}{append}\PY{p}{(}\PY{n}{newrow}\PY{p}{,}\PY{n}{ignore\PYZus{}index}\PY{o}{=}\PY{k+kc}{True}\PY{p}{)}
\end{Verbatim}
\end{tcolorbox}

    \begin{tcolorbox}[breakable, size=fbox, boxrule=1pt, pad at break*=1mm,colback=cellbackground, colframe=cellborder]
\prompt{In}{incolor}{6}{\boxspacing}
\begin{Verbatim}[commandchars=\\\{\}]
\PY{n}{display}\PY{p}{(}\PY{n}{df}\PY{p}{)}
\end{Verbatim}
\end{tcolorbox}

    
    \begin{Verbatim}[commandchars=\\\{\}]
  First Name  Age   Team
0        Zee   40   Blue
1  Charlotte   45    Red
2     Wilbur   50  Green
3       Iroh   99    Red
    \end{Verbatim}

    
    Notice that we had to reassign the dataframe when we added the new row.
This is because all of the internal dataframe functions return a
\emph{new} dataframe object. This method ensures that dataframes are not
overwritten on accident, and that data is not lost without intent. Also,
we now have two rows that have a shared value in a column. We can sort
the rows by specific columns, which can be used to group things
together. We can also group by multiple columns to have improved
organization.

    \begin{tcolorbox}[breakable, size=fbox, boxrule=1pt, pad at break*=1mm,colback=cellbackground, colframe=cellborder]
\prompt{In}{incolor}{7}{\boxspacing}
\begin{Verbatim}[commandchars=\\\{\}]
\PY{n}{df2} \PY{o}{=} \PY{n}{df}\PY{o}{.}\PY{n}{sort\PYZus{}values}\PY{p}{(}\PY{l+s+s2}{\PYZdq{}}\PY{l+s+s2}{Team}\PY{l+s+s2}{\PYZdq{}}\PY{p}{,}\PY{n}{ascending}\PY{o}{=}\PY{k+kc}{True}\PY{p}{)}
\PY{n}{display}\PY{p}{(}\PY{n}{df2}\PY{p}{)}
\end{Verbatim}
\end{tcolorbox}

    
    \begin{Verbatim}[commandchars=\\\{\}]
  First Name  Age   Team
0        Zee   40   Blue
2     Wilbur   50  Green
1  Charlotte   45    Red
3       Iroh   99    Red
    \end{Verbatim}

    
    \begin{tcolorbox}[breakable, size=fbox, boxrule=1pt, pad at break*=1mm,colback=cellbackground, colframe=cellborder]
\prompt{In}{incolor}{8}{\boxspacing}
\begin{Verbatim}[commandchars=\\\{\}]
\PY{n}{df3} \PY{o}{=} \PY{n}{df}\PY{o}{.}\PY{n}{sort\PYZus{}values}\PY{p}{(}\PY{p}{[}\PY{l+s+s2}{\PYZdq{}}\PY{l+s+s2}{Team}\PY{l+s+s2}{\PYZdq{}}\PY{p}{,}\PY{l+s+s2}{\PYZdq{}}\PY{l+s+s2}{Age}\PY{l+s+s2}{\PYZdq{}}\PY{p}{]}\PY{p}{,}\PY{n}{ascending}\PY{o}{=}\PY{p}{[}\PY{k+kc}{True}\PY{p}{,}\PY{k+kc}{False}\PY{p}{]}\PY{p}{)}
\PY{n}{display}\PY{p}{(}\PY{n}{df3}\PY{p}{)}
\end{Verbatim}
\end{tcolorbox}

    
    \begin{Verbatim}[commandchars=\\\{\}]
  First Name  Age   Team
0        Zee   40   Blue
2     Wilbur   50  Green
3       Iroh   99    Red
1  Charlotte   45    Red
    \end{Verbatim}

    
    In the first cell, we sort only by the column ``Team'', and put the
results in ascending order. Pandas uses alphabetical sorting unless all
cells in a column are \emph{entirely} numerical.

In the second cell, we sorted by a list, which means the whole dataframe
is sorted by the first element, and any rows that have the same value in
that first element are then sorted by the second element, and so on.
Additionally, we use a second list for the ascending values to set the
ascending/descending state for each element being sorted.

What if we wanted just the values in one column? We can call the
dataframe like we would a dictionary, using the square brackets and a
column name.

    \begin{tcolorbox}[breakable, size=fbox, boxrule=1pt, pad at break*=1mm,colback=cellbackground, colframe=cellborder]
\prompt{In}{incolor}{9}{\boxspacing}
\begin{Verbatim}[commandchars=\\\{\}]
\PY{n+nb}{print}\PY{p}{(}\PY{n}{df}\PY{p}{[}\PY{l+s+s2}{\PYZdq{}}\PY{l+s+s2}{Age}\PY{l+s+s2}{\PYZdq{}}\PY{p}{]}\PY{p}{)}
\end{Verbatim}
\end{tcolorbox}

    \begin{Verbatim}[commandchars=\\\{\}]
0    40
1    45
2    50
3    99
Name: Age, dtype: int64
    \end{Verbatim}

    \begin{tcolorbox}[breakable, size=fbox, boxrule=1pt, pad at break*=1mm,colback=cellbackground, colframe=cellborder]
\prompt{In}{incolor}{10}{\boxspacing}
\begin{Verbatim}[commandchars=\\\{\}]
\PY{n+nb}{print}\PY{p}{(}\PY{n}{df}\PY{p}{[}\PY{l+s+s2}{\PYZdq{}}\PY{l+s+s2}{Age}\PY{l+s+s2}{\PYZdq{}}\PY{p}{]}\PY{o}{.}\PY{n}{values}\PY{p}{)}
\end{Verbatim}
\end{tcolorbox}

    \begin{Verbatim}[commandchars=\\\{\}]
[40 45 50 99]
    \end{Verbatim}

    We can also call individual rows with the \texttt{.iloc} function. In
the example below, \texttt{.iloc(0)} indicates we're iterating on the
0th axis, which is down the rows. The \texttt{{[}2{]}} indicates we want
the data at index 2 in that list of rows. In the original \texttt{df}
dataframe object, that is the third row.

    \begin{tcolorbox}[breakable, size=fbox, boxrule=1pt, pad at break*=1mm,colback=cellbackground, colframe=cellborder]
\prompt{In}{incolor}{12}{\boxspacing}
\begin{Verbatim}[commandchars=\\\{\}]
\PY{n+nb}{print}\PY{p}{(}\PY{n}{df}\PY{o}{.}\PY{n}{iloc}\PY{p}{(}\PY{l+m+mi}{0}\PY{p}{)}\PY{p}{[}\PY{l+m+mi}{2}\PY{p}{]}\PY{p}{)}
\end{Verbatim}
\end{tcolorbox}

    \begin{Verbatim}[commandchars=\\\{\}]
First Name    Wilbur
Age               50
Team           Green
Name: 2, dtype: object
    \end{Verbatim}

    Let's consider an example that might be more relevant to the lab. Mark
is currently working on a Automated Fluorescent Nucleotide Workflow
(that needs a better name with a fun acronym). Over time, this workflow
is intended to produce large amounts of data for hundreds - even
thousands - of molecules. We want to use this data in a machine learning
model to predict possible new fluorescent nucleotides.

We can use Pandas to organize this data into a more easily managed form.

What are the different things to consider for each of the molecules? -
Sugar type (Ribose / Deoxyribose) - Nucleobase (A/C/G/T/U) - Connection
point on the base (C6,C5, etc.) - Tag structure and connection point. -
Absorption Wavelength - Emission Wavelength - Quantum Yield

Let's assume that we don't have all that information for every single
molecule just yet. In fact, that's kind of the point of machine
learning, filling in the gaps of data.

    \begin{tcolorbox}[breakable, size=fbox, boxrule=1pt, pad at break*=1mm,colback=cellbackground, colframe=cellborder]
\prompt{In}{incolor}{54}{\boxspacing}
\begin{Verbatim}[commandchars=\\\{\}]
\PY{c+c1}{\PYZsh{}\PYZsh{} A list of all the columns we want in our dataframe, corresponding to values we want to keep track of for each system.}
\PY{n}{df\PYZus{}columns} \PY{o}{=} \PY{p}{[}\PY{l+s+s2}{\PYZdq{}}\PY{l+s+s2}{Sugar}\PY{l+s+s2}{\PYZdq{}}\PY{p}{,}\PY{l+s+s2}{\PYZdq{}}\PY{l+s+s2}{Base}\PY{l+s+s2}{\PYZdq{}}\PY{p}{,}\PY{l+s+s2}{\PYZdq{}}\PY{l+s+s2}{ConnectionPoint}\PY{l+s+s2}{\PYZdq{}}\PY{p}{,}\PY{l+s+s2}{\PYZdq{}}\PY{l+s+s2}{FluorescentTag}\PY{l+s+s2}{\PYZdq{}}\PY{p}{, }\PY{l+s+s2}{\PYZdq{}}\PY{l+s+s2}{AbsorptionWavelength}\PY{l+s+s2}{\PYZdq{}}\PY{p}{,}\PY{l+s+s2}{\PYZdq{}}\PY{l+s+s2}{EmissionWavelength}\PY{l+s+s2}{\PYZdq{}}\PY{p}{,}\PY{l+s+s2}{\PYZdq{}}\PY{l+s+s2}{QuantumYield}\PY{l+s+s2}{\PYZdq{}}\PY{p}{]}

\PY{c+c1}{\PYZsh{}\PYZsh{} Initialize the dataframe without data, just column names}
\PY{n}{nucleotides} \PY{o}{=} \PY{n}{pd}\PY{o}{.}\PY{n}{DataFrame}\PY{p}{(}\PY{n}{columns}\PY{o}{=}\PY{n}{df\PYZus{}columns}\PY{p}{)}

\PY{n}{display}\PY{p}{(}\PY{n}{nucleotides}\PY{p}{)}
\end{Verbatim}
\end{tcolorbox}

    
    \begin{Verbatim}[commandchars=\\\{\}]
Empty DataFrame
Columns: [Sugar, Base, ConnectionPoint, FluorescentTag, AbsorptionWavelength, EmissionWavelength, QuantumYield]
Index: []
    \end{Verbatim}

    
    I'm including a function below to make my life easier for the example
process, but if we haven't talked about functions yet, you can ignore it
for now.

    \begin{tcolorbox}[breakable, size=fbox, boxrule=1pt, pad at break*=1mm,colback=cellbackground, colframe=cellborder]
\prompt{In}{incolor}{55}{\boxspacing}
\begin{Verbatim}[commandchars=\\\{\}]
\PY{k}{def} \PY{n+nf}{add\PYZus{}molecule}\PY{p}{(}\PY{n}{df}\PY{p}{,}\PY{o}{*}\PY{o}{*}\PY{n}{kwargs}\PY{p}{)}\PY{p}{:}
    \PY{n}{new\PYZus{}molecule} \PY{o}{=} \PY{p}{\PYZob{}}\PY{p}{\PYZcb{}}
    \PY{k}{for} \PY{n}{key}\PY{p}{,}\PY{n}{val} \PY{o+ow}{in} \PY{n}{kwargs}\PY{o}{.}\PY{n}{items}\PY{p}{(}\PY{p}{)}\PY{p}{:}
        \PY{n}{new\PYZus{}molecule}\PY{p}{[}\PY{n}{key}\PY{p}{]} \PY{o}{=} \PY{n}{val}
    \PY{n}{df} \PY{o}{=} \PY{n}{df}\PY{o}{.}\PY{n}{append}\PY{p}{(}\PY{n}{new\PYZus{}molecule}\PY{p}{,}\PY{n}{ignore\PYZus{}index}\PY{o}{=}\PY{k+kc}{True}\PY{p}{)}
    \PY{k}{return} \PY{n}{df}
\end{Verbatim}
\end{tcolorbox}

    \begin{tcolorbox}[breakable, size=fbox, boxrule=1pt, pad at break*=1mm,colback=cellbackground, colframe=cellborder]
\prompt{In}{incolor}{56}{\boxspacing}
\begin{Verbatim}[commandchars=\\\{\}]
\PY{n}{nucleotides} \PY{o}{=} \PY{n}{add\PYZus{}molecule}\PY{p}{(}\PY{n}{nucleotides}\PY{p}{,}\PY{n}{Sugar}\PY{o}{=}\PY{l+s+s2}{\PYZdq{}}\PY{l+s+s2}{Ribose}\PY{l+s+s2}{\PYZdq{}}\PY{p}{,}\PY{n}{Base}\PY{o}{=}\PY{l+s+s2}{\PYZdq{}}\PY{l+s+s2}{C}\PY{l+s+s2}{\PYZdq{}}\PY{p}{,}\PY{n}{ConnectionPoint}\PY{o}{=}\PY{l+s+s2}{\PYZdq{}}\PY{l+s+s2}{C5}\PY{l+s+s2}{\PYZdq{}}\PY{p}{, }\PY{n}{AbsorptionWavelength}\PY{o}{=}\PY{l+m+mi}{380}\PY{p}{,}\PY{n}{EmissionWavelength}\PY{o}{=}\PY{l+m+mi}{415}\PY{p}{,}\PY{n}{QuantumYield}\PY{o}{=}\PY{l+m+mf}{.93}\PY{p}{)}
\PY{n}{nucleotides} \PY{o}{=} \PY{n}{add\PYZus{}molecule}\PY{p}{(}\PY{n}{nucleotides}\PY{p}{,}\PY{n}{Sugar}\PY{o}{=}\PY{l+s+s2}{\PYZdq{}}\PY{l+s+s2}{Ribose}\PY{l+s+s2}{\PYZdq{}}\PY{p}{,}\PY{n}{Base}\PY{o}{=}\PY{l+s+s2}{\PYZdq{}}\PY{l+s+s2}{C}\PY{l+s+s2}{\PYZdq{}}\PY{p}{,}\PY{n}{FluorescentTag}\PY{o}{=}\PY{l+s+s2}{\PYZdq{}}\PY{l+s+s2}{Perylene}\PY{l+s+s2}{\PYZdq{}}\PY{p}{, }\PY{n}{ConnectionPoint}\PY{o}{=}\PY{l+s+s2}{\PYZdq{}}\PY{l+s+s2}{C6}\PY{l+s+s2}{\PYZdq{}}\PY{p}{,}\PY{n}{EmissionWavelength}\PY{o}{=}\PY{l+m+mi}{465}\PY{p}{,}\PY{n}{QuantumYield}\PY{o}{=}\PY{l+m+mf}{.96}\PY{p}{)}
\PY{n}{nucleotides} \PY{o}{=} \PY{n}{add\PYZus{}molecule}\PY{p}{(}\PY{n}{nucleotides}\PY{p}{,}\PY{n}{Sugar}\PY{o}{=}\PY{l+s+s2}{\PYZdq{}}\PY{l+s+s2}{Deoxyribose}\PY{l+s+s2}{\PYZdq{}}\PY{p}{,}\PY{n}{Base}\PY{o}{=}\PY{l+s+s2}{\PYZdq{}}\PY{l+s+s2}{C}\PY{l+s+s2}{\PYZdq{}}\PY{p}{,}\PY{n}{FluorescentTag}\PY{o}{=}\PY{l+s+s2}{\PYZdq{}}\PY{l+s+s2}{Benzopyrene}\PY{l+s+s2}{\PYZdq{}}\PY{p}{, }\PY{n}{ConnectionPoint}\PY{o}{=}\PY{l+s+s2}{\PYZdq{}}\PY{l+s+s2}{C5}\PY{l+s+s2}{\PYZdq{}}\PY{p}{,}\PY{n}{AbsorptionWavelength}\PY{o}{=}\PY{l+m+mi}{360}\PY{p}{,}\PY{n}{EmissionWavelength}\PY{o}{=}\PY{l+m+mi}{395}\PY{p}{)}
\PY{n}{nucleotides} \PY{o}{=} \PY{n}{add\PYZus{}molecule}\PY{p}{(}\PY{n}{nucleotides}\PY{p}{,}\PY{n}{Sugar}\PY{o}{=}\PY{l+s+s2}{\PYZdq{}}\PY{l+s+s2}{Deoxyribose}\PY{l+s+s2}{\PYZdq{}}\PY{p}{,}\PY{n}{Base}\PY{o}{=}\PY{l+s+s2}{\PYZdq{}}\PY{l+s+s2}{C}\PY{l+s+s2}{\PYZdq{}}\PY{p}{,}\PY{n}{FluorescentTag}\PY{o}{=}\PY{l+s+s2}{\PYZdq{}}\PY{l+s+s2}{Furan}\PY{l+s+s2}{\PYZdq{}}\PY{p}{, }\PY{n}{ConnectionPoint}\PY{o}{=}\PY{l+s+s2}{\PYZdq{}}\PY{l+s+s2}{C6}\PY{l+s+s2}{\PYZdq{}}\PY{p}{,}\PY{n}{AbsorptionWavelength}\PY{o}{=}\PY{l+m+mi}{390}\PY{p}{,}\PY{n}{QuantumYield}\PY{o}{=}\PY{l+m+mf}{.75}\PY{p}{)}

\PY{n}{display}\PY{p}{(}\PY{n}{nucleotides}\PY{p}{)}
\end{Verbatim}
\end{tcolorbox}

    
    \begin{Verbatim}[commandchars=\\\{\}]
         Sugar Base ConnectionPoint FluorescentTag AbsorptionWavelength  \textbackslash{}
0       Ribose    C              C5            NaN                  380   
1       Ribose    C              C6       Perylene                  NaN   
2  Deoxyribose    C              C5    Benzopyrene                  360   
3  Deoxyribose    C              C6          Furan                  390   

  EmissionWavelength  QuantumYield  
0                415          0.93  
1                465          0.96  
2                395           NaN  
3                NaN          0.75  
    \end{Verbatim}

    
    In the above cell, we can see that empty values are printed as
\texttt{NaN}. We can clear those out and replace them with empty cells
to make it easier to see where the data is missing.

    \begin{tcolorbox}[breakable, size=fbox, boxrule=1pt, pad at break*=1mm,colback=cellbackground, colframe=cellborder]
\prompt{In}{incolor}{57}{\boxspacing}
\begin{Verbatim}[commandchars=\\\{\}]
\PY{n}{display}\PY{p}{(}\PY{n}{nucleotides}\PY{o}{.}\PY{n}{fillna}\PY{p}{(}\PY{l+s+s1}{\PYZsq{}}\PY{l+s+s1}{\PYZsq{}}\PY{p}{)}\PY{p}{)}
\end{Verbatim}
\end{tcolorbox}

    
    \begin{Verbatim}[commandchars=\\\{\}]
         Sugar Base ConnectionPoint FluorescentTag AbsorptionWavelength  \textbackslash{}
0       Ribose    C              C5                                 380   
1       Ribose    C              C6       Perylene                        
2  Deoxyribose    C              C5    Benzopyrene                  360   
3  Deoxyribose    C              C6          Furan                  390   

  EmissionWavelength QuantumYield  
0                415         0.93  
1                465         0.96  
2                395               
3                            0.75  
    \end{Verbatim}

    
    \begin{tcolorbox}[breakable, size=fbox, boxrule=1pt, pad at break*=1mm,colback=cellbackground, colframe=cellborder]
\prompt{In}{incolor}{58}{\boxspacing}
\begin{Verbatim}[commandchars=\\\{\}]
\PY{n}{display}\PY{p}{(}\PY{n}{nucleotides}\PY{o}{.}\PY{n}{sort\PYZus{}values}\PY{p}{(}\PY{p}{[}\PY{l+s+s2}{\PYZdq{}}\PY{l+s+s2}{AbsorptionWavelength}\PY{l+s+s2}{\PYZdq{}}\PY{p}{,}\PY{l+s+s2}{\PYZdq{}}\PY{l+s+s2}{EmissionWavelength}\PY{l+s+s2}{\PYZdq{}}\PY{p}{,}\PY{l+s+s2}{\PYZdq{}}\PY{l+s+s2}{QuantumYield}\PY{l+s+s2}{\PYZdq{}}\PY{p}{]}\PY{p}{)}\PY{o}{.}\PY{n}{fillna}\PY{p}{(}\PY{l+s+s2}{\PYZdq{}}\PY{l+s+s2}{\PYZdq{}}\PY{p}{)}\PY{p}{)}
\end{Verbatim}
\end{tcolorbox}

    
    \begin{Verbatim}[commandchars=\\\{\}]
         Sugar Base ConnectionPoint FluorescentTag AbsorptionWavelength  \textbackslash{}
2  Deoxyribose    C              C5    Benzopyrene                  360   
0       Ribose    C              C5                                 380   
3  Deoxyribose    C              C6          Furan                  390   
1       Ribose    C              C6       Perylene                        

  EmissionWavelength QuantumYield  
2                395               
0                415         0.93  
3                            0.75  
1                465         0.96  
    \end{Verbatim}

    
    Let's generate some random data to mess with.

    \begin{tcolorbox}[breakable, size=fbox, boxrule=1pt, pad at break*=1mm,colback=cellbackground, colframe=cellborder]
\prompt{In}{incolor}{71}{\boxspacing}
\begin{Verbatim}[commandchars=\\\{\}]
\PY{k+kn}{import} \PY{n+nn}{numpy} \PY{k}{as} \PY{n+nn}{np}
\PY{n}{base\PYZus{}list}\PY{o}{=}\PY{p}{[}\PY{l+s+s2}{\PYZdq{}}\PY{l+s+s2}{A\PYZus{}C2}\PY{l+s+s2}{\PYZdq{}}\PY{p}{,}\PY{l+s+s2}{\PYZdq{}}\PY{l+s+s2}{A\PYZus{}C8}\PY{l+s+s2}{\PYZdq{}}\PY{p}{,}\PY{l+s+s2}{\PYZdq{}}\PY{l+s+s2}{C\PYZus{}C5}\PY{l+s+s2}{\PYZdq{}}\PY{p}{,}\PY{l+s+s2}{\PYZdq{}}\PY{l+s+s2}{C\PYZus{}C6}\PY{l+s+s2}{\PYZdq{}}\PY{p}{,}\PY{l+s+s2}{\PYZdq{}}\PY{l+s+s2}{G\PYZus{}C2}\PY{l+s+s2}{\PYZdq{}}\PY{p}{,}\PY{l+s+s2}{\PYZdq{}}\PY{l+s+s2}{G\PYZus{}C8}\PY{l+s+s2}{\PYZdq{}}\PY{p}{,}\PY{l+s+s2}{\PYZdq{}}\PY{l+s+s2}{G\PYZus{}N7}\PY{l+s+s2}{\PYZdq{}}\PY{p}{,}\PY{l+s+s2}{\PYZdq{}}\PY{l+s+s2}{U\PYZus{}C6}\PY{l+s+s2}{\PYZdq{}}\PY{p}{,}\PY{l+s+s2}{\PYZdq{}}\PY{l+s+s2}{U\PYZus{}C5}\PY{l+s+s2}{\PYZdq{}}\PY{p}{,}\PY{l+s+s2}{\PYZdq{}}\PY{l+s+s2}{T\PYZus{}C6}\PY{l+s+s2}{\PYZdq{}}\PY{p}{]}
\PY{n}{sugar\PYZus{}list}\PY{o}{=}\PY{p}{[}\PY{l+s+s2}{\PYZdq{}}\PY{l+s+s2}{Ribose}\PY{l+s+s2}{\PYZdq{}}\PY{p}{,}\PY{l+s+s2}{\PYZdq{}}\PY{l+s+s2}{Deoxyribose}\PY{l+s+s2}{\PYZdq{}}\PY{p}{]}
\PY{n}{tag\PYZus{}list} \PY{o}{=} \PY{p}{[}\PY{l+s+s2}{\PYZdq{}}\PY{l+s+s2}{Perylene}\PY{l+s+s2}{\PYZdq{}}\PY{p}{,}\PY{l+s+s2}{\PYZdq{}}\PY{l+s+s2}{Benzopyrene}\PY{l+s+s2}{\PYZdq{}}\PY{p}{,}\PY{l+s+s2}{\PYZdq{}}\PY{l+s+s2}{Furan}\PY{l+s+s2}{\PYZdq{}}\PY{p}{,}\PY{l+s+s2}{\PYZdq{}}\PY{l+s+s2}{Naphthalene}\PY{l+s+s2}{\PYZdq{}}\PY{p}{,}\PY{l+s+s2}{\PYZdq{}}\PY{l+s+s2}{Beta Carotene}\PY{l+s+s2}{\PYZdq{}}\PY{p}{,}\PY{l+s+s2}{\PYZdq{}}\PY{l+s+s2}{Imidazole}\PY{l+s+s2}{\PYZdq{}}\PY{p}{]}
\PY{n}{nucleotides} \PY{o}{=} \PY{n}{pd}\PY{o}{.}\PY{n}{DataFrame}\PY{p}{(}\PY{n}{columns}\PY{o}{=}\PY{n}{df\PYZus{}columns}\PY{p}{)}
\PY{k}{for} \PY{n}{base} \PY{o+ow}{in} \PY{n}{base\PYZus{}list}\PY{p}{:}
    \PY{k}{for} \PY{n}{sugar} \PY{o+ow}{in} \PY{n}{sugar\PYZus{}list}\PY{p}{:}
        \PY{k}{for} \PY{n}{tag} \PY{o+ow}{in} \PY{n}{tag\PYZus{}list}\PY{p}{:}
            \PY{n}{abs\PYZus{}wl} \PY{o}{=} \PY{n}{np}\PY{o}{.}\PY{n}{random}\PY{o}{.}\PY{n}{randint}\PY{p}{(}\PY{l+m+mi}{355}\PY{p}{,}\PY{l+m+mi}{800}\PY{p}{)}
            \PY{n}{emi\PYZus{}wl} \PY{o}{=} \PY{n}{np}\PY{o}{.}\PY{n}{random}\PY{o}{.}\PY{n}{randint}\PY{p}{(}\PY{n}{abs\PYZus{}wl}\PY{p}{,}\PY{l+m+mi}{850}\PY{p}{)}
            \PY{n}{nucleotides} \PY{o}{=} \PY{n}{add\PYZus{}molecule}\PY{p}{(}\PY{n}{nucleotides}\PY{p}{,}\PY{n}{Sugar}\PY{o}{=}\PY{n}{sugar}\PY{p}{,}\PY{n}{Base}\PY{o}{=}\PY{n}{base}\PY{o}{.}\PY{n}{split}\PY{p}{(}\PY{l+s+s2}{\PYZdq{}}\PY{l+s+s2}{\PYZus{}}\PY{l+s+s2}{\PYZdq{}}\PY{p}{)}\PY{p}{[}\PY{l+m+mi}{0}\PY{p}{]}\PY{p}{,}\PY{n}{ConnectionPoint}\PY{o}{=}\PY{n}{base}\PY{o}{.}\PY{n}{split}\PY{p}{(}\PY{l+s+s2}{\PYZdq{}}\PY{l+s+s2}{\PYZus{}}\PY{l+s+s2}{\PYZdq{}}\PY{p}{)}\PY{p}{[}\PY{l+m+mi}{1}\PY{p}{]}\PY{p}{,}\PY{n}{FluorescentTag}\PY{o}{=}\PY{n}{tag}\PY{p}{,}\PY{n}{AbsorptionWavelength}\PY{o}{=}\PY{n}{abs\PYZus{}wl}\PY{p}{,}\PY{n}{EmissionWavelength}\PY{o}{=}\PY{n}{emi\PYZus{}wl}\PY{p}{,}\PY{n}{QuantumYield}\PY{o}{=}\PY{n+nb}{round}\PY{p}{(}\PY{n}{np}\PY{o}{.}\PY{n}{random}\PY{o}{.}\PY{n}{rand}\PY{p}{(}\PY{p}{)}\PY{p}{,}\PY{l+m+mi}{2}\PY{p}{)}\PY{p}{)}
\end{Verbatim}
\end{tcolorbox}

    \begin{tcolorbox}[breakable, size=fbox, boxrule=1pt, pad at break*=1mm,colback=cellbackground, colframe=cellborder]
\prompt{In}{incolor}{72}{\boxspacing}
\begin{Verbatim}[commandchars=\\\{\}]
\PY{n}{display}\PY{p}{(}\PY{n}{nucleotides}\PY{o}{.}\PY{n}{fillna}\PY{p}{(}\PY{l+s+s2}{\PYZdq{}}\PY{l+s+s2}{\PYZdq{}}\PY{p}{)}\PY{p}{)}
\end{Verbatim}
\end{tcolorbox}

    
    \begin{Verbatim}[commandchars=\\\{\}]
           Sugar Base ConnectionPoint FluorescentTag  AbsorptionWavelength  \textbackslash{}
0         Ribose    A              C2       Perylene                   561   
1         Ribose    A              C2    Benzopyrene                   664   
2         Ribose    A              C2          Furan                   604   
3         Ribose    A              C2    Naphthalene                   424   
4         Ribose    A              C2  Beta Carotene                   774   
..           {\ldots}  {\ldots}             {\ldots}            {\ldots}                   {\ldots}   
115  Deoxyribose    T              C6    Benzopyrene                   602   
116  Deoxyribose    T              C6          Furan                   531   
117  Deoxyribose    T              C6    Naphthalene                   489   
118  Deoxyribose    T              C6  Beta Carotene                   697   
119  Deoxyribose    T              C6      Imidazole                   389   

     EmissionWavelength  QuantumYield  
0                   828          0.65  
1                   771          0.55  
2                   737          0.09  
3                   734          0.61  
4                   791          0.01  
..                  {\ldots}           {\ldots}  
115                 808          0.38  
116                 776          0.91  
117                 639          0.68  
118                 757          0.11  
119                 529          0.09  

[120 rows x 7 columns]
    \end{Verbatim}

    
    Note here that the full dataframe is not shown, but rather displays only
the first five and last five rows. The dimensions of the dataframe are
given below it. In this case, it's 120 rows and 7 columns.

What if we wanted to sort the data by Quantum Yield?

    \begin{tcolorbox}[breakable, size=fbox, boxrule=1pt, pad at break*=1mm,colback=cellbackground, colframe=cellborder]
\prompt{In}{incolor}{73}{\boxspacing}
\begin{Verbatim}[commandchars=\\\{\}]
\PY{n}{nucleotides}\PY{o}{.}\PY{n}{sort\PYZus{}values}\PY{p}{(}\PY{l+s+s2}{\PYZdq{}}\PY{l+s+s2}{QuantumYield}\PY{l+s+s2}{\PYZdq{}}\PY{p}{,}\PY{n}{ascending}\PY{o}{=}\PY{k+kc}{False}\PY{p}{)}
\end{Verbatim}
\end{tcolorbox}

            \begin{tcolorbox}[breakable, size=fbox, boxrule=.5pt, pad at break*=1mm, opacityfill=0]
\prompt{Out}{outcolor}{73}{\boxspacing}
\begin{Verbatim}[commandchars=\\\{\}]
           Sugar Base ConnectionPoint FluorescentTag AbsorptionWavelength  \textbackslash{}
83   Deoxyribose    G              N7      Imidazole                  587
92   Deoxyribose    U              C6          Furan                  724
21   Deoxyribose    A              C8    Naphthalene                  513
70   Deoxyribose    G              C8  Beta Carotene                  643
29        Ribose    C              C5      Imidazole                  450
..           {\ldots}  {\ldots}             {\ldots}            {\ldots}                  {\ldots}
44   Deoxyribose    C              C6          Furan                  588
89        Ribose    U              C6      Imidazole                  415
26        Ribose    C              C5          Furan                  761
110       Ribose    T              C6          Furan                  419
4         Ribose    A              C2  Beta Carotene                  774

    EmissionWavelength  QuantumYield
83                 740          1.00
92                 846          0.99
21                 765          0.98
70                 758          0.96
29                 500          0.95
..                 {\ldots}           {\ldots}
44                 828          0.06
89                 707          0.03
26                 797          0.01
110                472          0.01
4                  791          0.01

[120 rows x 7 columns]
\end{Verbatim}
\end{tcolorbox}
        
    Keep in mind that the above result did not sort the dataframe in itself,
it just presented the results of a sorting algorithm applied to it. Note
the indices on the left side and how they're out of order.

Now, what if we wanted to plot some of the data from our dataframe?

If our dataframe is simple enough, we can just call the \texttt{plot()}
function directly from it, which returns a \texttt{matplotlib\ Axes}
object.

    \begin{tcolorbox}[breakable, size=fbox, boxrule=1pt, pad at break*=1mm,colback=cellbackground, colframe=cellborder]
\prompt{In}{incolor}{83}{\boxspacing}
\begin{Verbatim}[commandchars=\\\{\}]
\PY{k+kn}{import} \PY{n+nn}{matplotlib}\PY{n+nn}{.}\PY{n+nn}{pyplot} \PY{k}{as} \PY{n+nn}{plt}
\PY{n}{fig} \PY{o}{=} \PY{n}{plt}\PY{o}{.}\PY{n}{figure}\PY{p}{(}\PY{n}{figsize}\PY{o}{=}\PY{p}{[}\PY{l+m+mi}{5}\PY{p}{,}\PY{l+m+mi}{4}\PY{p}{]}\PY{p}{,}\PY{n}{dpi}\PY{o}{=}\PY{l+m+mi}{300}\PY{p}{)}
\PY{n}{ax} \PY{o}{=} \PY{n}{nucleotides}\PY{o}{.}\PY{n}{plot}\PY{p}{(}\PY{p}{)}
\end{Verbatim}
\end{tcolorbox}

    
    \begin{Verbatim}[commandchars=\\\{\}]
<Figure size 1500x1200 with 0 Axes>
    \end{Verbatim}

    
    \begin{center}
    \adjustimage{max size={0.9\linewidth}{0.9\paperheight}}{output_29_1.png}
    \end{center}
    { \hspace*{\fill} \\}
    
    We can also simply call columns of our data as though they were any
other array of numbers (assuming the column is, in fact, strictly
numerical).

    \begin{tcolorbox}[breakable, size=fbox, boxrule=1pt, pad at break*=1mm,colback=cellbackground, colframe=cellborder]
\prompt{In}{incolor}{85}{\boxspacing}
\begin{Verbatim}[commandchars=\\\{\}]
\PY{k+kn}{import} \PY{n+nn}{matplotlib}\PY{n+nn}{.}\PY{n+nn}{pyplot} \PY{k}{as} \PY{n+nn}{plt}
\PY{n}{fig} \PY{o}{=} \PY{n}{plt}\PY{o}{.}\PY{n}{figure}\PY{p}{(}\PY{n}{figsize}\PY{o}{=}\PY{p}{[}\PY{l+m+mi}{5}\PY{p}{,}\PY{l+m+mi}{4}\PY{p}{]}\PY{p}{,}\PY{n}{dpi}\PY{o}{=}\PY{l+m+mi}{300}\PY{p}{)}
\PY{n}{ax} \PY{o}{=} \PY{n}{fig}\PY{o}{.}\PY{n}{add\PYZus{}subplot}\PY{p}{(}\PY{l+m+mi}{1}\PY{p}{,}\PY{l+m+mi}{1}\PY{p}{,}\PY{l+m+mi}{1}\PY{p}{)}
\PY{n}{ax}\PY{o}{.}\PY{n}{scatter}\PY{p}{(}\PY{n}{nucleotides}\PY{p}{[}\PY{l+s+s2}{\PYZdq{}}\PY{l+s+s2}{AbsorptionWavelength}\PY{l+s+s2}{\PYZdq{}}\PY{p}{]}\PY{p}{,}\PY{n}{nucleotides}\PY{p}{[}\PY{l+s+s2}{\PYZdq{}}\PY{l+s+s2}{EmissionWavelength}\PY{l+s+s2}{\PYZdq{}}\PY{p}{]}\PY{p}{,}\PY{n}{s}\PY{o}{=}\PY{n}{ nucleotides}\PY{p}{[}\PY{l+s+s2}{\PYZdq{}}\PY{l+s+s2}{QuantumYield}\PY{l+s+s2}{\PYZdq{}}\PY{p}{]}\PY{o}{*}\PY{l+m+mi}{5}\PY{p}{)}

\PY{c+c1}{\PYZsh{}\PYZsh{}\PYZsh{} I have included the QuantumYield column values as the marker size variable so that each point gives us the absorption/emission wavelengths and the quantum yields.}
\end{Verbatim}
\end{tcolorbox}

            \begin{tcolorbox}[breakable, size=fbox, boxrule=.5pt, pad at break*=1mm, opacityfill=0]
\prompt{Out}{outcolor}{85}{\boxspacing}
\begin{Verbatim}[commandchars=\\\{\}]
<matplotlib.collections.PathCollection at 0x7f45f705cfd0>
\end{Verbatim}
\end{tcolorbox}
        
    \begin{center}
    \adjustimage{max size={0.9\linewidth}{0.9\paperheight}}{output_31_1.png}
    \end{center}
    { \hspace*{\fill} \\}